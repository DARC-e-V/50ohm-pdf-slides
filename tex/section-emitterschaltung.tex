
\section{Emitterschaltung}
\label{section:emitterschaltung}
\begin{frame}%STARTCONTENT

\only<1>{
\begin{PQuestion}{AD409}{Bei dieser Schaltung handelt es sich um~...}{einen Verstärker als Emitterfolger.}
{einen Verstärker in Emitterschaltung.}
{einen Verstärker in Kollektorschaltung.}
{einen Verstärker für Gleichspannung.}
{\DARCimage{1.0\linewidth}{136include}}\end{PQuestion}

}
\only<2>{
\begin{PQuestion}{AD409}{Bei dieser Schaltung handelt es sich um~...}{einen Verstärker als Emitterfolger.}
{\textbf{\textcolor{DARCgreen}{einen Verstärker in Emitterschaltung.}}}
{einen Verstärker in Kollektorschaltung.}
{einen Verstärker für Gleichspannung.}
{\DARCimage{1.0\linewidth}{136include}}\end{PQuestion}

}
\end{frame}

\begin{frame}
\only<1>{
\begin{PQuestion}{AD411}{Welche Funktion haben die Widerstände $R_1$ und $R_2$ in der folgenden Schaltung? Sie dienen zur~...}{Einstellung der Gegenkopplung.}
{Verhinderung von Phasendrehungen.}
{Verhinderung von Eigenschwingungen.}
{Einstellung der Basisvorspannung.}
{\DARCimage{1.0\linewidth}{137include}}\end{PQuestion}

}
\only<2>{
\begin{PQuestion}{AD411}{Welche Funktion haben die Widerstände $R_1$ und $R_2$ in der folgenden Schaltung? Sie dienen zur~...}{Einstellung der Gegenkopplung.}
{Verhinderung von Phasendrehungen.}
{Verhinderung von Eigenschwingungen.}
{\textbf{\textcolor{DARCgreen}{Einstellung der Basisvorspannung.}}}
{\DARCimage{1.0\linewidth}{137include}}\end{PQuestion}

}
\end{frame}

\begin{frame}
\only<1>{
\begin{PQuestion}{AD413}{Welche Funktion hat der Kondensator $C_1$ in der folgenden Schaltung? Er dient zur~...}{Einstellung der Vorspannung am Emitter.}
{Verringerung der Wechselspannungsverstärkung.}
{Stabilisierung des Arbeitspunktes des Transistors.}
{Maximierung der Wechselspannungsverstärkung.}
{\DARCimage{1.0\linewidth}{138include}}\end{PQuestion}

}
\only<2>{
\begin{PQuestion}{AD413}{Welche Funktion hat der Kondensator $C_1$ in der folgenden Schaltung? Er dient zur~...}{Einstellung der Vorspannung am Emitter.}
{Verringerung der Wechselspannungsverstärkung.}
{Stabilisierung des Arbeitspunktes des Transistors.}
{\textbf{\textcolor{DARCgreen}{Maximierung der Wechselspannungsverstärkung.}}}
{\DARCimage{1.0\linewidth}{138include}}\end{PQuestion}

}
\end{frame}

\begin{frame}
\only<1>{
\begin{PQuestion}{AD412}{Welche Funktion haben die Kondensatoren $C_1$ und $C_2$ in der folgenden Schaltung? Sie dienen zur~...}{Festlegung der oberen Grenzfrequenz.}
{Wechselstromkopplung und Gleichspannungsentkopplung.}
{Erzeugung der erforderlichen Phasenverschiebung.}
{Anhebung niederfrequenter Signalanteile.}
{\DARCimage{1.0\linewidth}{139include}}\end{PQuestion}

}
\only<2>{
\begin{PQuestion}{AD412}{Welche Funktion haben die Kondensatoren $C_1$ und $C_2$ in der folgenden Schaltung? Sie dienen zur~...}{Festlegung der oberen Grenzfrequenz.}
{\textbf{\textcolor{DARCgreen}{Wechselstromkopplung und Gleichspannungsentkopplung.}}}
{Erzeugung der erforderlichen Phasenverschiebung.}
{Anhebung niederfrequenter Signalanteile.}
{\DARCimage{1.0\linewidth}{139include}}\end{PQuestion}

}
\end{frame}

\begin{frame}
\only<1>{
\begin{QQuestion}{AD407}{Welche Phasenverschiebung tritt zwischen den sinusförmigen Ein- und Ausgangsspannungen eines Transistorverstärkers in Emitterschaltung auf?}{\qty{0}{\degree}}
{\qty{90}{\degree}}
{\qty{180}{\degree}}
{\qty{270}{\degree}}
\end{QQuestion}

}
\only<2>{
\begin{QQuestion}{AD407}{Welche Phasenverschiebung tritt zwischen den sinusförmigen Ein- und Ausgangsspannungen eines Transistorverstärkers in Emitterschaltung auf?}{\qty{0}{\degree}}
{\qty{90}{\degree}}
{\textbf{\textcolor{DARCgreen}{\qty{180}{\degree}}}}
{\qty{270}{\degree}}
\end{QQuestion}

}
\end{frame}

\begin{frame}
\only<1>{
\begin{PQuestion}{AD408}{Das Signal $U_{\symup{E}}$ wird auf den Eingang folgender Schaltung gegeben. In welcher Antwort sind alle dargestellten Signale phasenrichtig zugeordnet?}{\DARCimage{1.0\linewidth}{218include}}
{\DARCimage{1.0\linewidth}{219include}}
{\DARCimage{1.0\linewidth}{220include}}
{\DARCimage{1.0\linewidth}{221include}}
{\DARCimage{1.0\linewidth}{222include}}\end{PQuestion}

}
\only<2>{
\begin{PQuestion}{AD408}{Das Signal $U_{\symup{E}}$ wird auf den Eingang folgender Schaltung gegeben. In welcher Antwort sind alle dargestellten Signale phasenrichtig zugeordnet?}{\textbf{\textcolor{DARCgreen}{\DARCimage{1.0\linewidth}{218include}}}}
{\DARCimage{1.0\linewidth}{219include}}
{\DARCimage{1.0\linewidth}{220include}}
{\DARCimage{1.0\linewidth}{221include}}
{\DARCimage{1.0\linewidth}{222include}}\end{PQuestion}

}
\end{frame}

\begin{frame}
\only<1>{
\begin{PQuestion}{AD406}{An den Eingang dieser Schaltung wird das folgende Signal gelegt. Welches ist ein mögliches Ausgangssignal $U_{\symup{A}}$?}{\DARCimage{1.0\linewidth}{223include}}
{\DARCimage{1.0\linewidth}{228include}}
{\DARCimage{1.0\linewidth}{225include}}
{\DARCimage{1.0\linewidth}{226include}}
{\DARCimage{1.0\linewidth}{227include}}\end{PQuestion}

}
\only<2>{
\begin{PQuestion}{AD406}{An den Eingang dieser Schaltung wird das folgende Signal gelegt. Welches ist ein mögliches Ausgangssignal $U_{\symup{A}}$?}{\textbf{\textcolor{DARCgreen}{\DARCimage{1.0\linewidth}{223include}}}}
{\DARCimage{1.0\linewidth}{228include}}
{\DARCimage{1.0\linewidth}{225include}}
{\DARCimage{1.0\linewidth}{226include}}
{\DARCimage{1.0\linewidth}{227include}}\end{PQuestion}

}
\end{frame}

\begin{frame}
\only<1>{
\begin{PQuestion}{AD414}{Wie verhält sich die Spannungsverstärkung bei der folgenden Schaltung, wenn der Kondensator $C_1$ entfernt wird?}{Sie nimmt ab.}
{Sie bleibt konstant.}
{Sie nimmt zu.}
{Sie fällt auf Null ab.}
{\DARCimage{1.0\linewidth}{138include}}\end{PQuestion}

}
\only<2>{
\begin{PQuestion}{AD414}{Wie verhält sich die Spannungsverstärkung bei der folgenden Schaltung, wenn der Kondensator $C_1$ entfernt wird?}{\textbf{\textcolor{DARCgreen}{Sie nimmt ab.}}}
{Sie bleibt konstant.}
{Sie nimmt zu.}
{Sie fällt auf Null ab.}
{\DARCimage{1.0\linewidth}{138include}}\end{PQuestion}

}
\end{frame}

\begin{frame}
\only<1>{
\begin{PQuestion}{AD415}{Bei folgender Emitterschaltung wird die Schaltung ohne den Emitterkondensator betrieben. Auf welchen Betrag sinkt die Spannungsverstärkung ungefähr?}{1}
{1/10}
{10}
{0}
{\DARCimage{1.0\linewidth}{366include}}\end{PQuestion}

}
\only<2>{
\begin{PQuestion}{AD415}{Bei folgender Emitterschaltung wird die Schaltung ohne den Emitterkondensator betrieben. Auf welchen Betrag sinkt die Spannungsverstärkung ungefähr?}{1}
{1/10}
{\textbf{\textcolor{DARCgreen}{10}}}
{0}
{\DARCimage{1.0\linewidth}{366include}}\end{PQuestion}

}
\end{frame}

\begin{frame}
\only<1>{
\begin{PQuestion}{AD410}{Was lässt sich über die Wechselspannungsverstärkung $v_U$ und die Phasenverschiebung $\varphi$ zwischen Ausgangs- und Eingangsspannung dieser Schaltung aussagen?}{$v_U$ ist klein (z.~B. 0,9~... 0,98) und $\varphi = \qty{0}{\degree}$.}
{$v_U$ ist groß (z.~B. 100~... 300) und $\varphi = \qty{0}{\degree}$}
{$v_U$ ist klein (z.~B. 0,9~... 0,98) und $\varphi = \qty{180}{\degree}$.}
{$v_U$ ist groß (z.~B. 100~... 300) und $\varphi = \qty{180}{\degree}$.}
{\DARCimage{1.0\linewidth}{136include}}\end{PQuestion}

}
\only<2>{
\begin{PQuestion}{AD410}{Was lässt sich über die Wechselspannungsverstärkung $v_U$ und die Phasenverschiebung $\varphi$ zwischen Ausgangs- und Eingangsspannung dieser Schaltung aussagen?}{$v_U$ ist klein (z.~B. 0,9~... 0,98) und $\varphi = \qty{0}{\degree}$.}
{$v_U$ ist groß (z.~B. 100~... 300) und $\varphi = \qty{0}{\degree}$}
{$v_U$ ist klein (z.~B. 0,9~... 0,98) und $\varphi = \qty{180}{\degree}$.}
{\textbf{\textcolor{DARCgreen}{$v_U$ ist groß (z.~B. 100~... 300) und $\varphi = \qty{180}{\degree}$.}}}
{\DARCimage{1.0\linewidth}{136include}}\end{PQuestion}

}
\end{frame}%ENDCONTENT
