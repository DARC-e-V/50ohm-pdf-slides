
\section{Elektromagnetisches Feld}
\label{section:em_feld}
\begin{frame}%STARTCONTENT

\begin{columns}
    \begin{column}{0.48\textwidth}
    \begin{itemize}
  \item Fließt ein zeitlich veränderlicher Strom durch einen Leiter, z.B. eine Antenne, entsteht sowohl ein elektrisches als auch ein magnetisches Feld
  \item Dieses wird als elektromagnetisches Feld bezeichnet
  \item Die beiden Felder stehen in einem 90º-Winkel zueinander
  \end{itemize}

    \end{column}
   \begin{column}{0.48\textwidth}
       
\begin{figure}
    \DARCimage{0.85\linewidth}{192include}
    \caption{\scriptsize Elektrisches und magnetisches Feld an einer Antenne}
    \label{e_vertikalantenne_magnetfeld}
\end{figure}


   \end{column}
\end{columns}

\end{frame}

\begin{frame}
\only<1>{
\begin{QQuestion}{EB301}{Wodurch entsteht ein elektromagnetisches Feld beispielsweise?}{Ein elektromagnetisches Feld entsteht, wenn ein zeitlich konstanter Strom durch einen elektrischen Leiter fließt.}
{Ein elektromagnetisches Feld entsteht, wenn ein zeitlich veränderlicher Strom durch einen elektrischen Leiter fließt.}
{Ein elektromagnetisches Feld entsteht, wenn eine zeitlich konstante Spannung an einem elektrischen Leiter anliegt.}
{Ein elektromagnetisches Feld entsteht, wenn eine zeitlich konstante Spannung an einem elektrischen Isolator anliegt.}
\end{QQuestion}

}
\only<2>{
\begin{QQuestion}{EB301}{Wodurch entsteht ein elektromagnetisches Feld beispielsweise?}{Ein elektromagnetisches Feld entsteht, wenn ein zeitlich konstanter Strom durch einen elektrischen Leiter fließt.}
{\textbf{\textcolor{DARCgreen}{Ein elektromagnetisches Feld entsteht, wenn ein zeitlich veränderlicher Strom durch einen elektrischen Leiter fließt.}}}
{Ein elektromagnetisches Feld entsteht, wenn eine zeitlich konstante Spannung an einem elektrischen Leiter anliegt.}
{Ein elektromagnetisches Feld entsteht, wenn eine zeitlich konstante Spannung an einem elektrischen Isolator anliegt.}
\end{QQuestion}

}
\end{frame}

\begin{frame}
\only<1>{
\begin{QQuestion}{EB302}{Wie erfolgt die Ausbreitung einer elektromagnetischen Welle? Die Ausbreitung erfolgt~...}{nur über das magnetische Feld. Das elektrische Feld wirkt sich nur im Nahfeld aus.}
{nur über das elektrische Feld. Das magnetische Feld wirkt sich nur im Nahfeld aus.}
{durch eine Wechselwirkung zwischen elektrischem und magnetischem Feld.}
{durch die unabhängige Ausbreitung von elektrischem und magnetischem Feld.}
\end{QQuestion}

}
\only<2>{
\begin{QQuestion}{EB302}{Wie erfolgt die Ausbreitung einer elektromagnetischen Welle? Die Ausbreitung erfolgt~...}{nur über das magnetische Feld. Das elektrische Feld wirkt sich nur im Nahfeld aus.}
{nur über das elektrische Feld. Das magnetische Feld wirkt sich nur im Nahfeld aus.}
{\textbf{\textcolor{DARCgreen}{durch eine Wechselwirkung zwischen elektrischem und magnetischem Feld.}}}
{durch die unabhängige Ausbreitung von elektrischem und magnetischem Feld.}
\end{QQuestion}

}
\end{frame}

\begin{frame}
\only<1>{
\begin{QQuestion}{EB303}{Der Winkel zwischen den elektrischen und magnetischen Feldkomponenten eines elektromagnetischen Feldes beträgt bei Freiraumausbreitung im Fernfeld~...}{\qty{90}{\degree}.}
{\qty{45}{\degree}.}
{\qty{180}{\degree}.}
{\qty{360}{\degree}.}
\end{QQuestion}

}
\only<2>{
\begin{QQuestion}{EB303}{Der Winkel zwischen den elektrischen und magnetischen Feldkomponenten eines elektromagnetischen Feldes beträgt bei Freiraumausbreitung im Fernfeld~...}{\textbf{\textcolor{DARCgreen}{\qty{90}{\degree}.}}}
{\qty{45}{\degree}.}
{\qty{180}{\degree}.}
{\qty{360}{\degree}.}
\end{QQuestion}

}
\end{frame}

\begin{frame}
\only<1>{
\begin{QQuestion}{EB304}{Welche Aussage trifft auf die elektromagnetische Ausstrahlung im ungestörten Fernfeld zu?}{Die E-Feldkomponente und die H-Feldkomponente sind phasengleich und sind parallel zueinander. Die Ausbreitungsrichtung verläuft dazu in einem rechten Winkel.}
{Die E-Feldkomponente und die H-Feldkomponente stehen in einem rechten Winkel zueinander. Die Ausbreitungsrichtung hat keine feste Beziehung dazu.}
{Die E-Feldkomponente, die H-Feldkomponente und die Ausbreitungsrichtung stehen in einem rechten Winkel zueinander.}
{Die Ausbreitungsrichtung befindet sich parallel zur E-Feldkomponente und verläuft senkrecht zur H-Feldkomponente.}
\end{QQuestion}

}
\only<2>{
\begin{QQuestion}{EB304}{Welche Aussage trifft auf die elektromagnetische Ausstrahlung im ungestörten Fernfeld zu?}{Die E-Feldkomponente und die H-Feldkomponente sind phasengleich und sind parallel zueinander. Die Ausbreitungsrichtung verläuft dazu in einem rechten Winkel.}
{Die E-Feldkomponente und die H-Feldkomponente stehen in einem rechten Winkel zueinander. Die Ausbreitungsrichtung hat keine feste Beziehung dazu.}
{\textbf{\textcolor{DARCgreen}{Die E-Feldkomponente, die H-Feldkomponente und die Ausbreitungsrichtung stehen in einem rechten Winkel zueinander.}}}
{Die Ausbreitungsrichtung befindet sich parallel zur E-Feldkomponente und verläuft senkrecht zur H-Feldkomponente.}
\end{QQuestion}

}
\end{frame}%ENDCONTENT
