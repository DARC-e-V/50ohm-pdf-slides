
\section{Widerstandsnetzwerke I}
\label{section:reihe_parallel_widerstandsnetz_1}
\begin{frame}%STARTCONTENT

\begin{columns}
    \begin{column}{0.48\textwidth}
    \begin{itemize}
  \item Bei einer komplexeren Schaltung geht man wie folgt vor: In kleinere Teile auflösen und diese berechnen, danach die Schaltung neu zeichnen und überlegen wie es weitergeht
  \item Schauen wir uns die Beispielschaltung mal genauer an
  \end{itemize}

    \end{column}
   \begin{column}{0.48\textwidth}
       
\begin{figure}
    \DARCimage{0.85\linewidth}{815include}
    \caption{\scriptsize Widerstandsnetzwerk}
    \label{e_widerstandsnetzwerk_1}
\end{figure}


   \end{column}
\end{columns}

\end{frame}

\begin{frame}
\begin{columns}
    \begin{column}{0.48\textwidth}
    \begin{itemize}
  \item $R_5$ und $R_7$ liegen in Reihe und dazu ist $R_8$ parallel geschaltet. Wir berechen diese und nennen den Wert dann $R_{ 5,7,8 }$
  \item $R_3$ und $R_6$ liegen in Reihe und dazu ist $R_2$ parallel geschaltet. Wir berechen diese und nennen den Wert dann $R_{ 2,3,6 }$
  \end{itemize}

    \end{column}
   \begin{column}{0.48\textwidth}
       
\begin{figure}
    \DARCimage{0.85\linewidth}{815include}
    \caption{\scriptsize Widerstandsnetzwerk}
    \label{e_widerstandsnetzwerk_1}
\end{figure}


   \end{column}
\end{columns}

\end{frame}

\begin{frame}
\begin{columns}
    \begin{column}{0.48\textwidth}
    \begin{itemize}
  \item Dann schauen wir uns an, was von der Schaltung übrig geblieben ist.
  \item Wir sehen eine Reihenschaltung von 4 Widerständen, die sich leicht berechnen lässt.
  \item Damit können wir dann auch die folgenden Prüfungsfragen leicht beantworten.
  \end{itemize}

    \end{column}
   \begin{column}{0.48\textwidth}
       
\begin{figure}
    \DARCimage{0.85\linewidth}{817include}
    \caption{\scriptsize Widerstandsnetzwerk in der Auflösung}
    \label{e_widerstandsnetzwerk_2}
\end{figure}


   \end{column}
\end{columns}

\end{frame}

\begin{frame}
\only<1>{
\begin{PQuestion}{ED115}{Wie groß ist der Gesamtwiderstand der dargestellten Schaltung?}{\qty{383}{\ohm}}
{\qty{360}{\ohm}}
{\qty{1150}{\ohm}}
{\qty{550}{\ohm}}
{\DARCimage{1.0\linewidth}{306include}}\end{PQuestion}

}
\only<2>{
\begin{PQuestion}{ED115}{Wie groß ist der Gesamtwiderstand der dargestellten Schaltung?}{\qty{383}{\ohm}}
{\qty{360}{\ohm}}
{\qty{1150}{\ohm}}
{\textbf{\textcolor{DARCgreen}{\qty{550}{\ohm}}}}
{\DARCimage{1.0\linewidth}{306include}}\end{PQuestion}

}
\end{frame}

\begin{frame}
\only<1>{
\begin{PQuestion}{ED116}{Wie groß ist der Gesamtwiderstand der dargestellten Schaltung?}{\qty{950}{\ohm}}
{\qty{120}{\ohm}}
{\qty{2950}{\ohm}}
{\qty{750}{\ohm}}
{\DARCimage{1.0\linewidth}{395include}}\end{PQuestion}

}
\only<2>{
\begin{PQuestion}{ED116}{Wie groß ist der Gesamtwiderstand der dargestellten Schaltung?}{\textbf{\textcolor{DARCgreen}{\qty{950}{\ohm}}}}
{\qty{120}{\ohm}}
{\qty{2950}{\ohm}}
{\qty{750}{\ohm}}
{\DARCimage{1.0\linewidth}{395include}}\end{PQuestion}

}
\end{frame}%ENDCONTENT
