
\section{Effektive Strahlungsleistung (ERP)}
\label{section:effektive_strahlungsleistung_erp_1}
\begin{frame}%STARTCONTENT
Kurze Wiederholung zu Antennen:
\begin{columns}
    \begin{column}{0.48\textwidth}
    Groundplane-Antenne strahlt in alle Himmelsrichtungen nahezu gleichmäßig ab, aber nicht nach oben oder unten


    \end{column}
   \begin{column}{0.48\textwidth}
       Yagi-Uda-Antenne bündelt die Funkstrahlen nach vorn und reduziert in alle anderen Richtungen


   \end{column}
\end{columns}
    \pause
    Bei der Berechnung der Grenzwerte für den Schutzabstand wird die \emph{Hauptstrahlrichtung} verwendet



\end{frame}

\begin{frame}
\frametitle{Gewinnfaktor}
\begin{itemize}
  \item Wie viel besser eine Antenne in Hauptstrahlrichtung im Vergleich zu einem Halbwellendipol abstrahlt
  \item Gewinnfaktor 2: Antenne strahlt in Hauptstrahlrichtung doppelt so stark wie ein Halbwellendipol in seine Hauptstrahlrichtung
  \end{itemize}

\end{frame}

\begin{frame}
\frametitle{Effektive Strahlungsleistung (ERP)}
Sendeleistung zur Antenne multipliziert mit Gewinnfaktor
    \pause
    Beispiel: 5~W auf eine Antenne mit Gewinnfaktor 2 ergibt die effektive Strahlungsleistung von 10~W

\end{frame}

\begin{frame}
\only<1>{
\begin{QQuestion}{NG401}{Die effektive Strahlungsleistung ERP (Effective Radiated Power) ist die von~...}{einem isotropen Strahler abgestrahlte Leistung, bezogen auf eine Antenne.}
{einer Antenne abgestrahlte Leistung, bezogen auf einen isotropen Strahler.}
{einem Halbwellendipol abgestrahlte Leistung, bezogen auf eine Antenne.}
{einer Antenne abgestrahlte Leistung, bezogen auf einen Halbwellendipol.}
\end{QQuestion}

}
\only<2>{
\begin{QQuestion}{NG401}{Die effektive Strahlungsleistung ERP (Effective Radiated Power) ist die von~...}{einem isotropen Strahler abgestrahlte Leistung, bezogen auf eine Antenne.}
{einer Antenne abgestrahlte Leistung, bezogen auf einen isotropen Strahler.}
{einem Halbwellendipol abgestrahlte Leistung, bezogen auf eine Antenne.}
{\textbf{\textcolor{DARCgreen}{einer Antenne abgestrahlte Leistung, bezogen auf einen Halbwellendipol.}}}
\end{QQuestion}

}
\end{frame}%ENDCONTENT
