
\section{Strom- und Spannungsspeisung I}
\label{section:strom_spannung_speisung_1}
\begin{frame}%STARTCONTENT

\frametitle{Speisewiderstand}
\begin{itemize}
  \item Eine Antenne wird mit Spannung und Strom gespeist
  \item Deren Verhältnis zueinander ergibt den Speisewiderstand
  \item Für Leistung müssen immer Spannung \underline{und} Strom vorhanden sein
  \item Wäre eines von beiden 0, erfolgt keine Leistungsabgabe
  \item Speisewiderstand hängt vom Ort der Einspeisung ab
  \end{itemize}

\end{frame}

\begin{frame}
\frametitle{Stromgespeiste Antennen}
\begin{itemize}
  \item Hoher Strom bei vergleichsweise geringer Spannung am Speisepunkt
  \item Niedriger Speisewiderstand
  \item ca. 36 Ω bis 100 Ω
  \item Niederohmiges Verhalten
  \end{itemize}
\end{frame}

\begin{frame}
\frametitle{Spannungsgespeiste Antenne}
\begin{itemize}
  \item Hohe Spannung bei vergleichsweise geringem Strom am Speisepunkt
  \item Hoher Speisewiderstand
  \item ca. 1500 Ω bis 4000 Ω
  \item Hochohmiges Verhalten
  \end{itemize}
\end{frame}

\begin{frame}
\frametitle{Einspeisung am Halbwellendipol}
\begin{itemize}
  \item Ladungsträger schwingen hin und her
  \item In der Mitte werden besonders viele Ladungsträger bewegt $\rightarrow$ Strombauch
  \item An den Enden entstehen besonders hohe Spannungen $\rightarrow$ Spannungsbauch
  \item Wenige Ladungsträger $\rightarrow$ Stromknoten
  \item Keine Spannung $\rightarrow$ Spannungsknoten
  \end{itemize}

\end{frame}

\begin{frame}
\begin{columns}
    \begin{column}{0.48\textwidth}
    \begin{itemize}
  \item Strombauch in der Mitte
  \item Spannungsbauch an den Enden
  \item Stromknoten an den Enden
  \item Spannungsknoten in der Mitte
  \end{itemize}

    \end{column}
   \begin{column}{0.48\textwidth}
       
\begin{figure}
    \DARCimage{0.85\linewidth}{787include}
    \caption{\scriptsize Halbwellendipol mit Spannungs- und Stromverteilung}
    \label{e_strom_spannung_speisung_dipol}
\end{figure}


   \end{column}
\end{columns}

\end{frame}

\begin{frame}
\only<1>{
\begin{QQuestion}{EG203}{Welche Aussage zur Strom- und Spannungsverteilung auf einem Dipol ist richtig?}{Am Einspeisepunkt eines Dipols entsteht immer ein Spannungsknoten und ein Strombauch.}
{An den Enden eines Dipols entsteht immer ein Spannungsknoten und ein Strombauch.}
{An den Enden eines Dipols entsteht immer ein Stromknoten und ein Spannungsbauch.}
{Am Einspeisepunkt eines Dipols entsteht immer ein Spannungsbauch und ein Stromknoten.}
\end{QQuestion}

}
\only<2>{
\begin{QQuestion}{EG203}{Welche Aussage zur Strom- und Spannungsverteilung auf einem Dipol ist richtig?}{Am Einspeisepunkt eines Dipols entsteht immer ein Spannungsknoten und ein Strombauch.}
{An den Enden eines Dipols entsteht immer ein Spannungsknoten und ein Strombauch.}
{\textbf{\textcolor{DARCgreen}{An den Enden eines Dipols entsteht immer ein Stromknoten und ein Spannungsbauch.}}}
{Am Einspeisepunkt eines Dipols entsteht immer ein Spannungsbauch und ein Stromknoten.}
\end{QQuestion}

}
\end{frame}

\begin{frame}
\only<1>{
\begin{QQuestion}{EG204}{Ein Dipol wird stromgespeist, wenn an seinem Einspeisepunkt~...}{ein Spannungsbauch und ein Stromknoten vorhanden sind. Er ist dann hochohmig.}
{ein Spannungsknoten und ein Strombauch vorhanden sind. Er ist dann niederohmig.}
{ein Spannungs- und ein Strombauch vorhanden sind. Er ist dann niederohmig.}
{ein Spannungs- und ein Stromknoten vorhanden sind. Er ist dann hochohmig.}
\end{QQuestion}

}
\only<2>{
\begin{QQuestion}{EG204}{Ein Dipol wird stromgespeist, wenn an seinem Einspeisepunkt~...}{ein Spannungsbauch und ein Stromknoten vorhanden sind. Er ist dann hochohmig.}
{\textbf{\textcolor{DARCgreen}{ein Spannungsknoten und ein Strombauch vorhanden sind. Er ist dann niederohmig.}}}
{ein Spannungs- und ein Strombauch vorhanden sind. Er ist dann niederohmig.}
{ein Spannungs- und ein Stromknoten vorhanden sind. Er ist dann hochohmig.}
\end{QQuestion}

}
\end{frame}

\begin{frame}
\only<1>{
\begin{QQuestion}{EG206}{Ein Halbwellendipol wird auf der Grundfrequenz in der Mitte~...}{spannungsgespeist.}
{stromgespeist.}
{endgespeist.}
{parallel gespeist.}
\end{QQuestion}

}
\only<2>{
\begin{QQuestion}{EG206}{Ein Halbwellendipol wird auf der Grundfrequenz in der Mitte~...}{spannungsgespeist.}
{\textbf{\textcolor{DARCgreen}{stromgespeist.}}}
{endgespeist.}
{parallel gespeist.}
\end{QQuestion}

}
\end{frame}

\begin{frame}
\frametitle{Endgespeister Halbwellendipol}
\begin{itemize}
  \item Spannungsgespeiste Antenne
  \item Hoher Speisewiderstand
  \end{itemize}
\end{frame}

\begin{frame}
\only<1>{
\begin{QQuestion}{EG205}{Ein Dipol wird spannungsgespeist, wenn an seinem Einspeisepunkt~...}{ein Spannungsknoten und ein Strombauch liegt. Er ist dann niederohmig.}
{ein Spannungsbauch und ein Stromknoten liegt. Er ist dann hochohmig.}
{ein Spannungs- und ein Strombauch liegt. Er ist dann niederohmig.}
{ein Spannungs- und ein Stromknoten liegt. Er ist dann hochohmig.}
\end{QQuestion}

}
\only<2>{
\begin{QQuestion}{EG205}{Ein Dipol wird spannungsgespeist, wenn an seinem Einspeisepunkt~...}{ein Spannungsknoten und ein Strombauch liegt. Er ist dann niederohmig.}
{\textbf{\textcolor{DARCgreen}{ein Spannungsbauch und ein Stromknoten liegt. Er ist dann hochohmig.}}}
{ein Spannungs- und ein Strombauch liegt. Er ist dann niederohmig.}
{ein Spannungs- und ein Stromknoten liegt. Er ist dann hochohmig.}
\end{QQuestion}

}
\end{frame}%ENDCONTENT
