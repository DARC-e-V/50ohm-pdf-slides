
\section{Schutzerdung und Potentialausgleich II}
\label{section:schutzerdung_2}
\begin{frame}%STARTCONTENT

\only<1>{
\begin{QQuestion}{AK202}{Warum ist eine möglichst niederohmige Verbindung aller Potentialausgleichsanschlüsse der Geräte einer Amateurfunkstelle anzustreben?}{Zur Vermeidung von Geräteschäden bei Überspannungen}
{Zur Begrenzung von Kurzschlussströmen bei Gerätefehlern}
{Zum Schutz von Personen}
{Zur Symmetrierung bei paralleldrahtgespeisten Antennen}
\end{QQuestion}

}
\only<2>{
\begin{QQuestion}{AK202}{Warum ist eine möglichst niederohmige Verbindung aller Potentialausgleichsanschlüsse der Geräte einer Amateurfunkstelle anzustreben?}{Zur Vermeidung von Geräteschäden bei Überspannungen}
{Zur Begrenzung von Kurzschlussströmen bei Gerätefehlern}
{\textbf{\textcolor{DARCgreen}{Zum Schutz von Personen}}}
{Zur Symmetrierung bei paralleldrahtgespeisten Antennen}
\end{QQuestion}

}
\end{frame}

\begin{frame}
\only<1>{
\begin{QQuestion}{AK203}{Ihr \qty{400}{\W}-Kurzwellensender ist über eine separate Erdungsleitung mit dem Potentialausgleich Ihres Hauses verbunden. Im Sendebetrieb stellen Sie fest, dass auf bestimmten Bändern das Gehäuse des Senders \glqq heiß\grqq{} ist, d. h. Hochfrequenzspannung merklicher Amplitude auf dem Gerätegehäuse liegt. Was kann die Ursache hierfür sein?}{Die Länge der Erdleitung entspricht annähernd einem Viertel der Wellenlänge der
Sendefrequenz oder einem ungeraden Vielfachen davon.}
{Die verwendete Kupfer-Erdleitung ist nicht versilbert und somit zur guten Ableitung von Hochfrequenz nicht geeignet.}
{Die Länge der Erdleitung entspricht annähernd einer halben Wellenlänge der Sendefrequenz oder Vielfachen davon.}
{Für die verwendete Erdleitung wurde ein massiver Leiter anstatt einer für Hochfrequenz besser geeigneten mehradrigen Litze verwendet.}
\end{QQuestion}

}
\only<2>{
\begin{QQuestion}{AK203}{Ihr \qty{400}{\W}-Kurzwellensender ist über eine separate Erdungsleitung mit dem Potentialausgleich Ihres Hauses verbunden. Im Sendebetrieb stellen Sie fest, dass auf bestimmten Bändern das Gehäuse des Senders \glqq heiß\grqq{} ist, d. h. Hochfrequenzspannung merklicher Amplitude auf dem Gerätegehäuse liegt. Was kann die Ursache hierfür sein?}{\textbf{\textcolor{DARCgreen}{Die Länge der Erdleitung entspricht annähernd einem Viertel der Wellenlänge der
Sendefrequenz oder einem ungeraden Vielfachen davon.}}}
{Die verwendete Kupfer-Erdleitung ist nicht versilbert und somit zur guten Ableitung von Hochfrequenz nicht geeignet.}
{Die Länge der Erdleitung entspricht annähernd einer halben Wellenlänge der Sendefrequenz oder Vielfachen davon.}
{Für die verwendete Erdleitung wurde ein massiver Leiter anstatt einer für Hochfrequenz besser geeigneten mehradrigen Litze verwendet.}
\end{QQuestion}

}
\end{frame}%ENDCONTENT
