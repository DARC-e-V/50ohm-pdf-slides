
\section{Rundstrahlantennen}
\label{section:rundstrahler}
\begin{frame}%STARTCONTENT
Ein Dipolschenkel wird durch eine Erdung (Ground) oder große Metallfläche (Fahrzeug) ersetzt
\begin{columns}
    \begin{column}{0.48\textwidth}
    
\begin{figure}
    \DARCimage{0.85\linewidth}{669include}
    \caption{\scriptsize Marconi-Antenne}
    \label{n_marconi_antenne}
\end{figure}


    \end{column}
   \begin{column}{0.48\textwidth}
       Erdung kann durch \emph{Radials} ersetzt werden, die eine \emph{Groundplane} bilden


   \end{column}
\end{columns}

\end{frame}

\begin{frame}
\only<1>{
\begin{PQuestion}{NG105}{Wie wird die dargestellte Antenne bezeichnet?}{Groundplane-Antenne}
{Yagi-Uda-Antenne}
{Dipol-Antenne}
{Endgespeiste Antenne}
{\DARCimage{0.5\linewidth}{614include}}\end{PQuestion}

}
\only<2>{
\begin{PQuestion}{NG105}{Wie wird die dargestellte Antenne bezeichnet?}{\textbf{\textcolor{DARCgreen}{Groundplane-Antenne}}}
{Yagi-Uda-Antenne}
{Dipol-Antenne}
{Endgespeiste Antenne}
{\DARCimage{0.5\linewidth}{614include}}\end{PQuestion}

}
\end{frame}

\begin{frame}
\only<1>{
\begin{QQuestion}{NG106}{Die elektrischen Gegengewichte einer Groundplane-Antenne bezeichnet man auch als~...}{Erdelemente.}
{Reflektoren.}
{Direktoren.}
{Radials.}
\end{QQuestion}

}
\only<2>{
\begin{QQuestion}{NG106}{Die elektrischen Gegengewichte einer Groundplane-Antenne bezeichnet man auch als~...}{Erdelemente.}
{Reflektoren.}
{Direktoren.}
{\textbf{\textcolor{DARCgreen}{Radials.}}}
\end{QQuestion}

}
\end{frame}

\begin{frame}
\only<1>{
\begin{QQuestion}{NG104}{Eine Marconi-Antenne ist~...}{eine vertikale Halbwellenantenne.}
{eine 5/8-$\lambda$-Antenne mit abgestimmten Radials.}
{eine horizontale $\lambda$/2-Langdrahtantenne.}
{eine gegen Erde erregte $\lambda$/4-Vertikalantenne.}
\end{QQuestion}

}
\only<2>{
\begin{QQuestion}{NG104}{Eine Marconi-Antenne ist~...}{eine vertikale Halbwellenantenne.}
{eine 5/8-$\lambda$-Antenne mit abgestimmten Radials.}
{eine horizontale $\lambda$/2-Langdrahtantenne.}
{\textbf{\textcolor{DARCgreen}{eine gegen Erde erregte $\lambda$/4-Vertikalantenne.}}}
\end{QQuestion}

}
\end{frame}

\begin{frame}
\only<1>{
\begin{PQuestion}{NG102}{Was wird durch dieses Schaltzeichen symbolisiert?}{Erde}
{Antenne}
{Diode}
{Batterie}
{\DARCimage{0.1\linewidth}{544include}}\end{PQuestion}

}
\only<2>{
\begin{PQuestion}{NG102}{Was wird durch dieses Schaltzeichen symbolisiert?}{\textbf{\textcolor{DARCgreen}{Erde}}}
{Antenne}
{Diode}
{Batterie}
{\DARCimage{0.1\linewidth}{544include}}\end{PQuestion}

}
\end{frame}

\begin{frame}
\only<1>{
\begin{QQuestion}{NG110}{Welche Antenne ist für eine \qty{2}{\m}-QSO-Runde mit im Umkreis verteilten Funkamateuren am besten geeignet?}{Ferritantenne}
{Yagi-Uda-Antenne}
{Rundstrahlantenne}
{Langdrahtantenne}
\end{QQuestion}

}
\only<2>{
\begin{QQuestion}{NG110}{Welche Antenne ist für eine \qty{2}{\m}-QSO-Runde mit im Umkreis verteilten Funkamateuren am besten geeignet?}{Ferritantenne}
{Yagi-Uda-Antenne}
{\textbf{\textcolor{DARCgreen}{Rundstrahlantenne}}}
{Langdrahtantenne}
\end{QQuestion}

}
\end{frame}

\begin{frame}
\only<1>{
\begin{QQuestion}{NG111}{Welche Antennenkonfiguration ist zu wählen, wenn möglichst viele umliegende Relaisstationen im \qty{2}{\m}- oder im \qty{70}{\cm}-Band erreicht werden sollen?}{Eine Ferritantenne auf der Fensterbank.}
{Ein Rundstrahler auf dem Hausdach.}
{Eine in einer Richtung fest montierte horizontale Richtantenne.}
{Eine Magnetfußantenne auf dem Dachboden.}
\end{QQuestion}

}
\only<2>{
\begin{QQuestion}{NG111}{Welche Antennenkonfiguration ist zu wählen, wenn möglichst viele umliegende Relaisstationen im \qty{2}{\m}- oder im \qty{70}{\cm}-Band erreicht werden sollen?}{Eine Ferritantenne auf der Fensterbank.}
{\textbf{\textcolor{DARCgreen}{Ein Rundstrahler auf dem Hausdach.}}}
{Eine in einer Richtung fest montierte horizontale Richtantenne.}
{Eine Magnetfußantenne auf dem Dachboden.}
\end{QQuestion}

}
\end{frame}%ENDCONTENT
