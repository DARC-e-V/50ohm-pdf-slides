
\section{Aufbau eines Senders}
\label{section:aufbau_sender}
\begin{frame}%STARTCONTENT

\frametitle{1. Mikrofon}
\begin{columns}
    \begin{column}{0.48\textwidth}
    
\begin{figure}
    \DARCimage{0.85\linewidth}{735include}
    \caption{\scriptsize Blockdiagramm eines einfachen Senders}
    \label{aufbau_sender}
\end{figure}


    \end{column}
   \begin{column}{0.48\textwidth}
       \begin{itemize}
  \item Wandelt Schallwellen in NF-Signal um
  \end{itemize}

   \end{column}
\end{columns}

\end{frame}

\begin{frame}
\frametitle{2. Niederfrequenz-Verstärker}
\begin{columns}
    \begin{column}{0.48\textwidth}
    
\begin{figure}
    \DARCimage{0.85\linewidth}{735include}
    \caption{\scriptsize Blockdiagramm eines einfachen Senders}
    \label{aufbau_sender}
\end{figure}


    \end{column}
   \begin{column}{0.48\textwidth}
       \begin{itemize}
  \item Verstärkt das NF-Signal vom Mikrofon
  \end{itemize}

   \end{column}
\end{columns}

\end{frame}

\begin{frame}
\frametitle{3. Mischer}
\begin{columns}
    \begin{column}{0.48\textwidth}
    
\begin{figure}
    \DARCimage{0.85\linewidth}{735include}
    \caption{\scriptsize Blockdiagramm eines einfachen Senders}
    \label{aufbau_sender}
\end{figure}


    \end{column}
   \begin{column}{0.48\textwidth}
       \begin{itemize}
  \item Mischt das NF-Signal mit dem HF-Träger vom Oszillator (4)
  \end{itemize}

   \end{column}
\end{columns}

\end{frame}

\begin{frame}
\frametitle{4. Oszillator}
\begin{columns}
    \begin{column}{0.48\textwidth}
    
\begin{figure}
    \DARCimage{0.85\linewidth}{735include}
    \caption{\scriptsize Blockdiagramm eines einfachen Senders}
    \label{aufbau_sender}
\end{figure}


    \end{column}
   \begin{column}{0.48\textwidth}
       \begin{itemize}
  \item Erzeugt hochfrequente Schwingung der Sendefrequenz
  \end{itemize}

   \end{column}
\end{columns}

\end{frame}

\begin{frame}
\frametitle{5. Bandfilter}
\begin{columns}
    \begin{column}{0.48\textwidth}
    
\begin{figure}
    \DARCimage{0.85\linewidth}{735include}
    \caption{\scriptsize Blockdiagramm eines einfachen Senders}
    \label{aufbau_sender}
\end{figure}


    \end{column}
   \begin{column}{0.48\textwidth}
       \begin{itemize}
  \item Mischer erzeugt unerwünschte Frequenzen
  \item Mit dem Bandfilter werden nur die gewünschten Frequenzen durchgelassen
  \end{itemize}

   \end{column}
\end{columns}

\end{frame}

\begin{frame}
\frametitle{6. Verstärker}
\begin{columns}
    \begin{column}{0.48\textwidth}
    
\begin{figure}
    \DARCimage{0.85\linewidth}{735include}
    \caption{\scriptsize Blockdiagramm eines einfachen Senders}
    \label{aufbau_sender}
\end{figure}


    \end{column}
   \begin{column}{0.48\textwidth}
       \begin{itemize}
  \item Verstärkt das HF-Signal auf gewünschte Sendeleistung
  \end{itemize}

   \end{column}
\end{columns}

\end{frame}

\begin{frame}
\frametitle{7. Bandfilter}
\begin{columns}
    \begin{column}{0.48\textwidth}
    
\begin{figure}
    \DARCimage{0.85\linewidth}{735include}
    \caption{\scriptsize Blockdiagramm eines einfachen Senders}
    \label{aufbau_sender}
\end{figure}


    \end{column}
   \begin{column}{0.48\textwidth}
       \begin{itemize}
  \item Verstärker kann unerwünschte Frequenzen erzeugen
  \item Nur die gewünschten Frequenzen werden durchgelassen
  \end{itemize}

   \end{column}
\end{columns}

\end{frame}

\begin{frame}
\frametitle{8. Antenne}
\begin{columns}
    \begin{column}{0.48\textwidth}
    
\begin{figure}
    \DARCimage{0.85\linewidth}{735include}
    \caption{\scriptsize Blockdiagramm eines einfachen Senders}
    \label{aufbau_sender}
\end{figure}


    \end{column}
   \begin{column}{0.48\textwidth}
       \begin{itemize}
  \item HF-Signal wird auf Antenne gegeben
  \item Antenne strahlt es als Funkwelle ab
  \end{itemize}

   \end{column}
\end{columns}

\end{frame}

\begin{frame}
\only<1>{
\begin{PQuestion}{NF401}{Was stellt folgendes Blockdiagramm dar?}{Sender}
{Empfänger}
{Relaisfunkstelle}
{Antennenvorverstärker}
{\DARCimage{1.0\linewidth}{524include}}\end{PQuestion}

}
\only<2>{
\begin{PQuestion}{NF401}{Was stellt folgendes Blockdiagramm dar?}{\textbf{\textcolor{DARCgreen}{Sender}}}
{Empfänger}
{Relaisfunkstelle}
{Antennenvorverstärker}
{\DARCimage{1.0\linewidth}{524include}}\end{PQuestion}

}
\end{frame}

\begin{frame}
\only<1>{
\begin{PQuestion}{NF403}{Das nachfolgende Blockschaltbild zeigt einen einfachen Sender. An welcher Stelle befindet sich welche Stufe?}{1 HF-Verstärker; 
2 Filter; 
3 HF-Oszillator;  
4 NF-Verstärker; 
5 Mischer;
6 NF-Verstärker
}
{1 HF-Verstärker; 
2 Mischer; 
3 HF-Oszillator;  
4 Filter; 
5 NF-Verstärker; 
6 Filter
}
{1 NF-Verstärker; 
2 Filter; 
3 HF-Oszillator;  
4 Mischer; 
5 HF-Verstärker;
6 Mischer}
{1 NF-Verstärker; 
2 Mischer; 
3 HF-Oszillator;  
4 Filter; 
5 HF-Verstärker; 
6 Filter}
{\DARCimage{1.0\linewidth}{495include}}\end{PQuestion}

}
\only<2>{
\begin{PQuestion}{NF403}{Das nachfolgende Blockschaltbild zeigt einen einfachen Sender. An welcher Stelle befindet sich welche Stufe?}{1 HF-Verstärker; 
2 Filter; 
3 HF-Oszillator;  
4 NF-Verstärker; 
5 Mischer;
6 NF-Verstärker
}
{1 HF-Verstärker; 
2 Mischer; 
3 HF-Oszillator;  
4 Filter; 
5 NF-Verstärker; 
6 Filter
}
{1 NF-Verstärker; 
2 Filter; 
3 HF-Oszillator;  
4 Mischer; 
5 HF-Verstärker;
6 Mischer}
{\textbf{\textcolor{DARCgreen}{1 NF-Verstärker; 
2 Mischer; 
3 HF-Oszillator;  
4 Filter; 
5 HF-Verstärker; 
6 Filter}}}
{\DARCimage{1.0\linewidth}{495include}}\end{PQuestion}

}
\end{frame}

\begin{frame}
\only<1>{
\begin{QQuestion}{NF402}{Aus welchen Stufen besteht ein einfacher Sender?}{Vorverstärker, Filter, NF-Verstärker, Antenne}
{Vorverstärker, Filter, Demodulator, NF-Verstärker}
{Oszillator, Mischer, Filter, Leistungsverstärker}
{NF-Verstärker, Filter, Leistungsverstärker, Antenne}
\end{QQuestion}

}
\only<2>{
\begin{QQuestion}{NF402}{Aus welchen Stufen besteht ein einfacher Sender?}{Vorverstärker, Filter, NF-Verstärker, Antenne}
{Vorverstärker, Filter, Demodulator, NF-Verstärker}
{\textbf{\textcolor{DARCgreen}{Oszillator, Mischer, Filter, Leistungsverstärker}}}
{NF-Verstärker, Filter, Leistungsverstärker, Antenne}
\end{QQuestion}

}
\end{frame}

\begin{frame}Eine Amateurfunkanlage muss nach den allgemein anerkannten Regeln der Technik aufgebaut und betrieben werden. Das gilt natürlich auch ganz besonders für Sender.

\end{frame}

\begin{frame}
\only<1>{
\begin{QQuestion}{VD106}{Welche technischen Anforderungen stellt die Amateurfunkverordnung u.~a. an eine Amateurfunkstelle?}{Alle für den Sendebetrieb notwendigen Geräte müssen über ein CE-Zeichen verfügen.}
{Sie ist nach den allgemein anerkannten Regeln der Technik einzurichten und zu unterhalten.}
{Das Sendesignal muss über ein Koaxialkabel der Antenne zugeführt werden.}
{Sie darf bauartbedingt keine höhere Leistung erzeugen, als der Besitzer verwenden darf.}
\end{QQuestion}

}
\only<2>{
\begin{QQuestion}{VD106}{Welche technischen Anforderungen stellt die Amateurfunkverordnung u.~a. an eine Amateurfunkstelle?}{Alle für den Sendebetrieb notwendigen Geräte müssen über ein CE-Zeichen verfügen.}
{\textbf{\textcolor{DARCgreen}{Sie ist nach den allgemein anerkannten Regeln der Technik einzurichten und zu unterhalten.}}}
{Das Sendesignal muss über ein Koaxialkabel der Antenne zugeführt werden.}
{Sie darf bauartbedingt keine höhere Leistung erzeugen, als der Besitzer verwenden darf.}
\end{QQuestion}

}
\end{frame}%ENDCONTENT
