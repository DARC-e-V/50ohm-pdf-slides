
\section{Kanalcodierung}
\label{section:kanalcodierung}
\begin{frame}%STARTCONTENT
\begin{itemize}
  \item Die Abbildung zeigt einen Sender und einen Empfänger, welche über einen Kanal miteinander verbunden sind.
  \item Durch atmosphärische Einflüsse oder Aussendungen anderer Stationen kann es zu Störungen auf dem Kanal kommen, welche zu Fehlern bei der Übertragung führen.
  \end{itemize}

\begin{figure}
    \DARCimage{0.85\linewidth}{674include}
    \caption{\scriptsize Kanal}
    \label{kanal}
\end{figure}

\end{frame}

\begin{frame}Die Kanalcodierung fügt der zu übertragenden Information gezielt Redundanz hinzu, beispielsweise Wiederholungen oder Prüfsummen.

\end{frame}

\begin{frame}Wir unterscheiden zwei Arten der Kanalcodierung:

\begin{itemize}
  \item Fehlererkennung: Man kann erkennen, dass bei der Übertragung ein Fehler aufgetreten ist, und dann z. B. eine erneute Übertragung anfordern.
  \item Vorwärtsfehlerkorrektur: Fehler, die bei der Übertragung entstehen, werden mit Hilfe der Redundanz beim Empfänger korrigiert.
  \end{itemize}

\begin{figure}
    \DARCimage{0.85\linewidth}{676include}
    \caption{\scriptsize Kanalcodierer}
    \label{kanalcodierer}
\end{figure}

\end{frame}

\begin{frame}
\only<1>{
\begin{QQuestion}{AE409}{Was wird unter Kanalcodierung verstanden?}{Verschlüsselung des Kanals zum Schutz gegen unbefugtes Abhören}
{Kompression von Daten vor der Übertragung zur Reduktion der Datenmenge}
{Hinzufügen von Redundanz vor der Übertragung zum Schutz vor Übertragungsfehlern}
{Zuordnung von Frequenzen zu Sende- bzw. Empfangskanälen zur häufigen Verwendung}
\end{QQuestion}

}
\only<2>{
\begin{QQuestion}{AE409}{Was wird unter Kanalcodierung verstanden?}{Verschlüsselung des Kanals zum Schutz gegen unbefugtes Abhören}
{Kompression von Daten vor der Übertragung zur Reduktion der Datenmenge}
{\textbf{\textcolor{DARCgreen}{Hinzufügen von Redundanz vor der Übertragung zum Schutz vor Übertragungsfehlern}}}
{Zuordnung von Frequenzen zu Sende- bzw. Empfangskanälen zur häufigen Verwendung}
\end{QQuestion}

}
\end{frame}%ENDCONTENT
