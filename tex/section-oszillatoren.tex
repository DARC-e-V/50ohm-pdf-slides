
\section{Oszillatoren}
\label{section:oszillatoren}
\begin{frame}%STARTCONTENT
\begin{itemize}
  \item Oszillatoren erzeugen eine Wechselspannung
  \item Es gibt verschiedene Methoden
  \end{itemize}
\end{frame}

\begin{frame}
\frametitle{LC-Oszillator}
\begin{columns}
    \begin{column}{0.48\textwidth}
    \begin{itemize}
  \item Schwingungserzeugung mit Spule und Kondensator als Schwingkreis
  \item Ein aufgeladener Kondensator entlädt sich an der Spule
  \item Eine aufgeladene Spule entlädt sich am Kondensator
  \item Je nach Wert der Bauteile in einer bestimmten Frequenz
  \end{itemize}

    \end{column}
   \begin{column}{0.48\textwidth}
       
\begin{figure}
    \DARCimage{0.85\linewidth}{755include}
    \caption{\scriptsize Parallelschwingkreis aus Kondensator und Spule}
    \label{e_parallelschwingkreis_cl}
\end{figure}


   \end{column}
\end{columns}

\end{frame}

\begin{frame}
\only<1>{
\begin{QQuestion}{ED501}{Was ist ein LC-Oszillator? Es ist ein Schwingungserzeuger, wobei die Frequenz~...}{mittels LC-Tiefpass gefiltert wird.}
{durch einen hochstabilen Quarz bestimmt wird.}
{von einer Spule und einem Kondensator als Schwingkreis bestimmt wird.}
{mittels LC-Hochpass gefiltert wird.}
\end{QQuestion}

}
\only<2>{
\begin{QQuestion}{ED501}{Was ist ein LC-Oszillator? Es ist ein Schwingungserzeuger, wobei die Frequenz~...}{mittels LC-Tiefpass gefiltert wird.}
{durch einen hochstabilen Quarz bestimmt wird.}
{\textbf{\textcolor{DARCgreen}{von einer Spule und einem Kondensator als Schwingkreis bestimmt wird.}}}
{mittels LC-Hochpass gefiltert wird.}
\end{QQuestion}

}
\end{frame}

\begin{frame}
\frametitle{Temperaturstabilität}
\begin{columns}
    \begin{column}{0.48\textwidth}
    \begin{itemize}
  \item Die passiven Bauelemente haben bei veränderlicher Temperatur unterschiedliche Werte
  \end{itemize}

    \end{column}
   \begin{column}{0.48\textwidth}
       \begin{itemize}
  \item Höhere Frequenz bei \emph{kleinerer} Kapazität oder Induktivität
  \item Niedrigere Frequenz bei \emph{höherer} Kapazität oder Induktivität
  \end{itemize}

   \end{column}
\end{columns}

\end{frame}

\begin{frame}
\only<1>{
\begin{QQuestion}{ED502}{Wie verhält sich die Frequenz eines LC-Oszillators, wenn bei zunehmender Temperatur die Kapazität des Kondensators größer wird?}{Die Frequenz wird höher.}
{Die Schwingungen reißen sofort ab.}
{Die Frequenz wird niedriger.}
{Die Frequenz bleibt stabil.}
\end{QQuestion}

}
\only<2>{
\begin{QQuestion}{ED502}{Wie verhält sich die Frequenz eines LC-Oszillators, wenn bei zunehmender Temperatur die Kapazität des Kondensators größer wird?}{Die Frequenz wird höher.}
{Die Schwingungen reißen sofort ab.}
{\textbf{\textcolor{DARCgreen}{Die Frequenz wird niedriger.}}}
{Die Frequenz bleibt stabil.}
\end{QQuestion}

}
\end{frame}

\begin{frame}
\only<1>{
\begin{QQuestion}{ED503}{Wie verhält sich die Frequenz eines LC-Oszillators, wenn bei zunehmender Temperatur die Kapazität des Kondensators kleiner wird?}{Die Frequenz wird höher.}
{Die Schwingungen reißen sofort ab.}
{Die Frequenz wird niedriger.}
{Die Frequenz bleibt stabil.}
\end{QQuestion}

}
\only<2>{
\begin{QQuestion}{ED503}{Wie verhält sich die Frequenz eines LC-Oszillators, wenn bei zunehmender Temperatur die Kapazität des Kondensators kleiner wird?}{\textbf{\textcolor{DARCgreen}{Die Frequenz wird höher.}}}
{Die Schwingungen reißen sofort ab.}
{Die Frequenz wird niedriger.}
{Die Frequenz bleibt stabil.}
\end{QQuestion}

}
\end{frame}

\begin{frame}
\only<1>{
\begin{QQuestion}{ED504}{Wie verhält sich die Frequenz eines LC-Oszillators, wenn bei zunehmender Temperatur die Induktivität der Spule größer wird?}{Die Frequenz bleibt stabil.}
{Die Frequenz wird höher.}
{Die Frequenz wird niedriger.}
{Die Schwingungen reißen sofort ab.}
\end{QQuestion}

}
\only<2>{
\begin{QQuestion}{ED504}{Wie verhält sich die Frequenz eines LC-Oszillators, wenn bei zunehmender Temperatur die Induktivität der Spule größer wird?}{Die Frequenz bleibt stabil.}
{Die Frequenz wird höher.}
{\textbf{\textcolor{DARCgreen}{Die Frequenz wird niedriger.}}}
{Die Schwingungen reißen sofort ab.}
\end{QQuestion}

}
\end{frame}

\begin{frame}
\only<1>{
\begin{QQuestion}{ED505}{Wie verhält sich die Frequenz eines LC-Oszillators, wenn bei zunehmender Temperatur die Induktivität der Spule kleiner wird?}{Die Schwingungen reißen sofort ab.}
{Die Frequenz wird niedriger.}
{Die Frequenz wird höher.}
{Die Frequenz bleibt stabil.}
\end{QQuestion}

}
\only<2>{
\begin{QQuestion}{ED505}{Wie verhält sich die Frequenz eines LC-Oszillators, wenn bei zunehmender Temperatur die Induktivität der Spule kleiner wird?}{Die Schwingungen reißen sofort ab.}
{Die Frequenz wird niedriger.}
{\textbf{\textcolor{DARCgreen}{Die Frequenz wird höher.}}}
{Die Frequenz bleibt stabil.}
\end{QQuestion}

}
\end{frame}

\begin{frame}
\only<1>{
\begin{QQuestion}{EF304}{Der VFO eines Senders ist schwankenden Temperaturen unterworfen. Welche wesentliche Auswirkung könnte dies haben?}{Die Frequenz des Oszillators springt schnell zwischen verschiedenen Werten. }
{Die Frequenz des Oszillators ändert sich langsam.}
{Die Amplitude der Oszillatorfrequenz schwankt langsam.}
{Die Amplitude des Oszillators springt schnell zwischen verschiedenen Werten. }
\end{QQuestion}

}
\only<2>{
\begin{QQuestion}{EF304}{Der VFO eines Senders ist schwankenden Temperaturen unterworfen. Welche wesentliche Auswirkung könnte dies haben?}{Die Frequenz des Oszillators springt schnell zwischen verschiedenen Werten. }
{\textbf{\textcolor{DARCgreen}{Die Frequenz des Oszillators ändert sich langsam.}}}
{Die Amplitude der Oszillatorfrequenz schwankt langsam.}
{Die Amplitude des Oszillators springt schnell zwischen verschiedenen Werten. }
\end{QQuestion}

}
\end{frame}

\begin{frame}
\frametitle{Quarz-Oszillator}
\begin{itemize}
  \item Schwingungserzeugung mit Quarz (Siliziumdioxid SiO<sub>2</sub>)
  \item Umgekehrter Piezoelektrischer Effekt an einem Quarzkristall
  \item Quarz wird mit einem (schlechten) LC-Oszillator zum stabilen Schwingen angeregt
  \item Bessere Frequenzstabilität
  \end{itemize}

\end{frame}

\begin{frame}
\only<1>{
\begin{QQuestion}{ED506}{Bei einem Quarz-Oszillator handelt es sich um einen Schwingungserzeuger, bei dem die Frequenz~...}{durch einen Quarz verstärkt wird.}
{durch einen Quarz bestimmt wird.}
{mittels Quarz-Tiefpass gefiltert wird.}
{mittels Quarz-Hochpass gefiltert wird.}
\end{QQuestion}

}
\only<2>{
\begin{QQuestion}{ED506}{Bei einem Quarz-Oszillator handelt es sich um einen Schwingungserzeuger, bei dem die Frequenz~...}{durch einen Quarz verstärkt wird.}
{\textbf{\textcolor{DARCgreen}{durch einen Quarz bestimmt wird.}}}
{mittels Quarz-Tiefpass gefiltert wird.}
{mittels Quarz-Hochpass gefiltert wird.}
\end{QQuestion}

}
\end{frame}

\begin{frame}
\only<1>{
\begin{QQuestion}{ED507}{Der Vorteil von Quarzoszillatoren gegenüber LC-Oszillatoren liegt darin, dass sie~...}{eine bessere Frequenzstabilität aufweisen.}
{eine breitere Resonanzkurve haben.}
{einen größeren Abstimmbereich aufweisen.}
{keine Oberschwingungen erzeugen.}
\end{QQuestion}

}
\only<2>{
\begin{QQuestion}{ED507}{Der Vorteil von Quarzoszillatoren gegenüber LC-Oszillatoren liegt darin, dass sie~...}{\textbf{\textcolor{DARCgreen}{eine bessere Frequenzstabilität aufweisen.}}}
{eine breitere Resonanzkurve haben.}
{einen größeren Abstimmbereich aufweisen.}
{keine Oberschwingungen erzeugen.}
\end{QQuestion}

}
\end{frame}

\begin{frame}
\frametitle{Abstrahlung}
\begin{itemize}
  \item Vermeiden
  \item Abschirmung durch Metallgehäuse
  \end{itemize}
\end{frame}

\begin{frame}
\only<1>{
\begin{QQuestion}{EF207}{Wie sollte ein Oszillator aufgebaut werden, um unerwünschte Abstrahlungen zu vermeiden?}{Die Speisespannung sollte ungesiebt sein. }
{Er sollte nicht abgeschirmt werden.}
{Er sollte niederohmig HF-entkoppelt sein.}
{Er sollte durch ein Metallgehäuse abgeschirmt werden.}
\end{QQuestion}

}
\only<2>{
\begin{QQuestion}{EF207}{Wie sollte ein Oszillator aufgebaut werden, um unerwünschte Abstrahlungen zu vermeiden?}{Die Speisespannung sollte ungesiebt sein. }
{Er sollte nicht abgeschirmt werden.}
{Er sollte niederohmig HF-entkoppelt sein.}
{\textbf{\textcolor{DARCgreen}{Er sollte durch ein Metallgehäuse abgeschirmt werden.}}}
\end{QQuestion}

}
\end{frame}%ENDCONTENT
