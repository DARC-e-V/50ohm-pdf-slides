
\section{Kondensator in Reihen- und Parallelschaltung}
\label{section:reihe_parallel_kondensator}
\begin{frame}%STARTCONTENT

\frametitle{Reihenschaltung}
\begin{itemize}
  \item Da die Spannung entscheidend für das Entstehen des elektrischen Feldes ist (und diese sich bei der Reihenschaltung aufteilt), ist die Berechung der Kapazität genau umgekehrt wie bei Widerständen.
  \item Anwendungsfall: Bei hohen Spannungen werden mehrere Kondensatoren in Reihe geschaltet, um die Gefahr eines Durchschlags zu verhindern. Dabei ist hilfreich, dass sich die Gesamtspannung an den Kondensatoren aufteilt.
  \end{itemize}
\end{frame}

\begin{frame}\begin{itemize}
  \item Bei einer Reihenschaltung von Kondensatoren ist die Gesamtkapazität kleiner als der Wert des kleinsten Kondensators
  \end{itemize}

\begin{figure}
    \DARCimage{0.85\linewidth}{823include}
    \caption{\scriptsize Reihenschaltung von 3 Kondensatoren}
    \label{e_reihenschaltung_kondensatoren}
\end{figure}

$\frac{ 1 }{ C_{ G } } = \frac{ 1 }{ C_{ 1 } } + \frac{ 1 }{ C_{ 2 } } + \frac{ 1 }{ C_{ 3 } }$

\end{frame}

\begin{frame}\begin{itemize}
  \item Vereinfachung für zwei Kondensatoren:
  \end{itemize}
$C_{ G } = \dfrac{ C_{ 1 } \cdot C_{ 2 } }{ C_{ 1 } + C_{ 2 }}$

\end{frame}

\begin{frame}\begin{itemize}
  \item Vereinfachung für gleiche Kondensatoren:
  \end{itemize}
$C_{ G } = \dfrac{ C }{ n }$

$n$ steht für die Anzahl der Kondensatoren

\end{frame}

\begin{frame}
\only<1>{
\begin{QQuestion}{ED119}{Eine Reihenschaltung besteht aus drei Kondensatoren von je \qty{0,33}{\micro\F}. Wie groß ist die Gesamtkapazität dieser Schaltung?}{\qty{0,110}{\micro\F}}
{\qty{0,990}{\micro\F}}
{\qty{0,011}{\micro\F}}
{\qty{0,099}{\micro\F}}
\end{QQuestion}

}
\only<2>{
\begin{QQuestion}{ED119}{Eine Reihenschaltung besteht aus drei Kondensatoren von je \qty{0,33}{\micro\F}. Wie groß ist die Gesamtkapazität dieser Schaltung?}{\textbf{\textcolor{DARCgreen}{\qty{0,110}{\micro\F}}}}
{\qty{0,990}{\micro\F}}
{\qty{0,011}{\micro\F}}
{\qty{0,099}{\micro\F}}
\end{QQuestion}

}
\end{frame}

\begin{frame}
\only<1>{
\begin{QQuestion}{ED120}{Welche Gesamtkapazität ergibt sich bei einer Reihenschaltung der Kondensatoren \qty{100}{\micro\F}, \qty{200000}{\nF} und \qty{200}{\micro\F}?}{\qty{102}{\micro\F}}
{\qty{320}{\nF}}
{\qty{300,2}{\micro\F}}
{\qty{50}{\micro\F}}
\end{QQuestion}

}
\only<2>{
\begin{QQuestion}{ED120}{Welche Gesamtkapazität ergibt sich bei einer Reihenschaltung der Kondensatoren \qty{100}{\micro\F}, \qty{200000}{\nF} und \qty{200}{\micro\F}?}{\qty{102}{\micro\F}}
{\qty{320}{\nF}}
{\qty{300,2}{\micro\F}}
{\textbf{\textcolor{DARCgreen}{\qty{50}{\micro\F}}}}
\end{QQuestion}

}
\end{frame}

\begin{frame}
\frametitle{Parallelschaltung}
\begin{itemize}
  \item Hier ist es genau umgekehrt wie bei Widerständen, weil an allen Kondensatoren die gleiche Spannung anliegt, welche ja entscheidend für die Entstehung des elektrischen Feldes ist.
  \item Anwendungsfall: Kondensatoren werden parallel geschaltet, um aus der Normreihe auf den Wert zu kommen, den man benötigt.
  \end{itemize}

\end{frame}

\begin{frame}\begin{itemize}
  \item Bei einer Parallelschaltung addieren sich die Kapazitäten
  \end{itemize}

\begin{figure}
    \DARCimage{0.85\linewidth}{822include}
    \caption{\scriptsize Parallelschaltung von 3 Kondensatoren}
    \label{e_parallelschaltung_kondensatoren}
\end{figure}

$C_{ G } = C_{ 1 } + C_{ 2 } + C_{ 3 }$

\end{frame}

\begin{frame}
\only<1>{
\begin{QQuestion}{ED117}{Drei Kondensatoren mit den Kapazitäten $C_1$ = \qty{0,1}{\micro\F}, $C_2$ = \qty{150}{\nF} und $C_3$ = \qty{50000}{\pF} werden parallel geschaltet. Wie groß ist die Gesamtkapazität?}{\qty{0,3}{\micro\F}}
{\qty{0,2}{\micro\F}}
{\qty{0,027}{\micro\F}}
{\qty{0,255}{\micro\F}}
\end{QQuestion}

}
\only<2>{
\begin{QQuestion}{ED117}{Drei Kondensatoren mit den Kapazitäten $C_1$ = \qty{0,1}{\micro\F}, $C_2$ = \qty{150}{\nF} und $C_3$ = \qty{50000}{\pF} werden parallel geschaltet. Wie groß ist die Gesamtkapazität?}{\textbf{\textcolor{DARCgreen}{\qty{0,3}{\micro\F}}}}
{\qty{0,2}{\micro\F}}
{\qty{0,027}{\micro\F}}
{\qty{0,255}{\micro\F}}
\end{QQuestion}

}
\end{frame}

\begin{frame}
\only<1>{
\begin{QQuestion}{ED118}{Wie groß ist die Gesamtkapazität von drei parallel geschalteten Kondensatoren von \qty{22}{\nF}, \qty{0,033}{\micro\F} und \qty{15000}{\pF}?}{\qty{40,3}{\nF}}
{\qty{700}{\nF}}
{\qty{0,070}{\micro\F}}
{\qty{7021}{\pF}}
\end{QQuestion}

}
\only<2>{
\begin{QQuestion}{ED118}{Wie groß ist die Gesamtkapazität von drei parallel geschalteten Kondensatoren von \qty{22}{\nF}, \qty{0,033}{\micro\F} und \qty{15000}{\pF}?}{\qty{40,3}{\nF}}
{\qty{700}{\nF}}
{\textbf{\textcolor{DARCgreen}{\qty{0,070}{\micro\F}}}}
{\qty{7021}{\pF}}
\end{QQuestion}

}
\end{frame}

\begin{frame}
\frametitle{Gemischte Schaltungen}
\end{frame}

\begin{frame}
\frametitle{Variante 1: Zwei Parallel und dazu einer in Reihe}
\begin{columns}
    \begin{column}{0.48\textwidth}
    \begin{itemize}
  \item Hier berechnet man zuerst die Parallelschaltung von $C_{ 2 }$ und $C_{ 3 }$
  \end{itemize}
$C_{ Gp } = C_{ 2 } + C_{ 3 }$

\begin{itemize}
  \item Danach berechnet man die Reihenschaltung von $C_{ 1 }$ und $C_{ Gp }$
  \end{itemize}
$C_{ G } = \frac{ C_{ 1 } \cdot C_{ Gp } }{ C_{ 1 } + C_{ Gp }}$


    \end{column}
   \begin{column}{0.48\textwidth}
       
\begin{figure}
    \DARCimage{0.85\linewidth}{820include}
    \caption{\scriptsize Gemischte Schaltung -- Variante 1}
    \label{e_gemischt_variante_1}
\end{figure}


   \end{column}
\end{columns}

\end{frame}

\begin{frame}
\only<1>{
\begin{PQuestion}{ED123}{Welche Gesamtkapazität hat die folgende Schaltung? Gegeben: $C_1$ = \qty{8}{\nF}; $C_2$ = \qty{4}{\nF}; $C_3$ = \qty{4}{\nF}}{\qty{4}{\nF}}
{\qty{16}{\nF}}
{\qty{1}{\nF}}
{\qty{9}{\nF}}
{\DARCimage{1.0\linewidth}{440include}}\end{PQuestion}

}
\only<2>{
\begin{PQuestion}{ED123}{Welche Gesamtkapazität hat die folgende Schaltung? Gegeben: $C_1$ = \qty{8}{\nF}; $C_2$ = \qty{4}{\nF}; $C_3$ = \qty{4}{\nF}}{\textbf{\textcolor{DARCgreen}{\qty{4}{\nF}}}}
{\qty{16}{\nF}}
{\qty{1}{\nF}}
{\qty{9}{\nF}}
{\DARCimage{1.0\linewidth}{440include}}\end{PQuestion}

}
\end{frame}

\begin{frame}
\only<1>{
\begin{PQuestion}{ED124}{Welche Gesamtkapazität hat diese Schaltung, wenn $C_1$ = \qty{200}{\nF}, $C_2$ = \qty{100}{\nF} und $C_3$ = \qty{100000}{\pF} betragen?}{\qty{400}{\nF}}
{\qty{250}{\nF}}
{\qty{100}{\nF}}
{\qty{200}{\nF}}
{\DARCimage{1.0\linewidth}{440include}}\end{PQuestion}

}
\only<2>{
\begin{PQuestion}{ED124}{Welche Gesamtkapazität hat diese Schaltung, wenn $C_1$ = \qty{200}{\nF}, $C_2$ = \qty{100}{\nF} und $C_3$ = \qty{100000}{\pF} betragen?}{\qty{400}{\nF}}
{\qty{250}{\nF}}
{\textbf{\textcolor{DARCgreen}{\qty{100}{\nF}}}}
{\qty{200}{\nF}}
{\DARCimage{1.0\linewidth}{440include}}\end{PQuestion}

}
\end{frame}

\begin{frame}
\only<1>{
\begin{PQuestion}{ED122}{Welche Gesamtkapazität hat diese Schaltung, wenn $C_1$~=~\qty{2}{\micro\F}, $C_2$~=~\qty{1}{\micro\F} und $C_3$~=~\qty{1}{\micro\F} betragen? }{\qty{1,0}{\micro\F}}
{\qty{4400}{\nF}}
{\qty{2,5}{\micro\F}}
{\qty{4,0}{\micro\F}}
{\DARCimage{1.0\linewidth}{440include}}\end{PQuestion}

}
\only<2>{
\begin{PQuestion}{ED122}{Welche Gesamtkapazität hat diese Schaltung, wenn $C_1$~=~\qty{2}{\micro\F}, $C_2$~=~\qty{1}{\micro\F} und $C_3$~=~\qty{1}{\micro\F} betragen? }{\textbf{\textcolor{DARCgreen}{\qty{1,0}{\micro\F}}}}
{\qty{4400}{\nF}}
{\qty{2,5}{\micro\F}}
{\qty{4,0}{\micro\F}}
{\DARCimage{1.0\linewidth}{440include}}\end{PQuestion}

}
\end{frame}

\begin{frame}
\frametitle{Variante 2: Zwei in Reihe und dazu einer Parallel}
\begin{columns}
    \begin{column}{0.48\textwidth}
    \begin{itemize}
  \item Hier berechnet man zuerst die Reihenschaltung von $C_{ 1 }$ und $C_{ 2 }$
  \end{itemize}
$C_{ Gr } = \frac{ C_{ 1 } \cdot C_{ 2 } }{ C_{ 1 } + C_{ 2 }}$

\begin{itemize}
  \item Danach berechnet man die Parallelschaltung von $C_{ 3 }$ und $C_{ Gr }$
  \end{itemize}
$C_{ G } = \frac{ C_{ 3 } \cdot C_{ Gr } }{ C_{ 3 } + C_{ Gr }}$


    \end{column}
   \begin{column}{0.48\textwidth}
       
\begin{figure}
    \DARCimage{0.85\linewidth}{457include}
    \caption{\scriptsize Gemischte Schaltung -- Variante 2}
    \label{e_gemischt_variante_2}
\end{figure}


   \end{column}
\end{columns}

\end{frame}

\begin{frame}
\only<1>{
\begin{PQuestion}{ED121}{Welche Gesamtkapazität hat die folgende Schaltung? Gegeben: $C_1$~=~\qty{10}{\nF}; $C_2$~=~\qty{10}{\nF}; $C_3$~=~\qty{5}{\nF}}{\qty{25}{\nF}}
{\qty{5}{\nF}}
{\qty{20}{\nF}}
{\qty{10}{\nF}}
{\DARCimage{1.0\linewidth}{457include}}\end{PQuestion}

}
\only<2>{
\begin{PQuestion}{ED121}{Welche Gesamtkapazität hat die folgende Schaltung? Gegeben: $C_1$~=~\qty{10}{\nF}; $C_2$~=~\qty{10}{\nF}; $C_3$~=~\qty{5}{\nF}}{\qty{25}{\nF}}
{\qty{5}{\nF}}
{\qty{20}{\nF}}
{\textbf{\textcolor{DARCgreen}{\qty{10}{\nF}}}}
{\DARCimage{1.0\linewidth}{457include}}\end{PQuestion}

}
\end{frame}%ENDCONTENT
