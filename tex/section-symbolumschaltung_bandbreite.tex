
\section{Symbolumschaltung und Bandbreite}
\label{section:symbolumschaltung_bandbreite}
\begin{frame}%STARTCONTENT
\begin{itemize}
  \item Als Symbol werden in der Digitaltechnik die verschiedenen Zeicheneinheiten zur Übertragung des Informationsgehaltes bezeichnet.
  \item Die Anzahl der pro Zeitspanne übertragenen Symbole ist die Symbolrate und wird in der Einheit Baud ausgedrückt.
  \item Bei jeder Umschaltung zwischen zwei Symbolen wird die Amplitude, Frequenz oder Phase eines Trägers geändert.
  \item Je schneller Amplitude, Frequenz oder Phase verändert werden, umso breitbandiger wird das erzeugte Signal.
  \end{itemize}
\end{frame}

\begin{frame}
\only<1>{
\begin{QQuestion}{AE415}{Welche Auswirkung hat eine Erhöhung der Umschaltgeschwindigkeit zwischen verschiedenen Symbolen bei digitalen Übertragungsverfahren auf die benötigte Bandbreite? Die Bandbreite~...}{bleibt gleich.}
{sinkt.}
{steigt.}
{steigt im oberen und sinkt im unteren Seitenband.}
\end{QQuestion}

}
\only<2>{
\begin{QQuestion}{AE415}{Welche Auswirkung hat eine Erhöhung der Umschaltgeschwindigkeit zwischen verschiedenen Symbolen bei digitalen Übertragungsverfahren auf die benötigte Bandbreite? Die Bandbreite~...}{bleibt gleich.}
{sinkt.}
{\textbf{\textcolor{DARCgreen}{steigt.}}}
{steigt im oberen und sinkt im unteren Seitenband.}
\end{QQuestion}

}
\end{frame}

\begin{frame}
\only<1>{
\begin{question2x2}{AE214}{Welches dieser amplitudenmodulierten Signale belegt die geringste Bandbreite?}{\DARCimage{1.0\linewidth}{599include}}
{\DARCimage{1.0\linewidth}{598include}}
{\DARCimage{1.0\linewidth}{597include}}
{\DARCimage{1.0\linewidth}{601include}}
\end{question2x2}

}
\only<2>{
\begin{question2x2}{AE214}{Welches dieser amplitudenmodulierten Signale belegt die geringste Bandbreite?}{\DARCimage{1.0\linewidth}{599include}}
{\DARCimage{1.0\linewidth}{598include}}
{\textbf{\textcolor{DARCgreen}{\DARCimage{1.0\linewidth}{597include}}}}
{\DARCimage{1.0\linewidth}{601include}}
\end{question2x2}

}
\end{frame}

\begin{frame}\begin{itemize}
  \item Von der Morsetelegrafie kennen wir bereits Tastklicks, die breitbandige Störungen darstellen.
  \item Sie entstehen, wenn beim Drücken bzw. Loslassen der Morsetaste der Träger plötzlich ein- bzw. ausgeschaltet wird.
  \end{itemize}
\end{frame}

\begin{frame}
\only<1>{
\begin{question2x2}{AJ221}{In den nachfolgenden Bildern sind mögliche Signalverläufe des Senderausgangssignals bei der CW-Tastung dargestellt. Welcher Signalverlauf führt zu den geringsten Störungen?}{\DARCimage{1.0\linewidth}{21include}}
{\DARCimage{1.0\linewidth}{20include}}
{\DARCimage{1.0\linewidth}{19include}}
{\DARCimage{1.0\linewidth}{22include}}
\end{question2x2}

}
\only<2>{
\begin{question2x2}{AJ221}{In den nachfolgenden Bildern sind mögliche Signalverläufe des Senderausgangssignals bei der CW-Tastung dargestellt. Welcher Signalverlauf führt zu den geringsten Störungen?}{\DARCimage{1.0\linewidth}{21include}}
{\DARCimage{1.0\linewidth}{20include}}
{\textbf{\textcolor{DARCgreen}{\DARCimage{1.0\linewidth}{19include}}}}
{\DARCimage{1.0\linewidth}{22include}}
\end{question2x2}

}
\end{frame}

\begin{frame}
\only<1>{
\begin{PQuestion}{AJ220}{Diese Modulationshüllkurve eines CW-Senders sollte vermieden werden, da~...}{wahrscheinlich Tastklicks erzeugt werden.}
{während der Aussetzer Probleme im Leistungsverstärker entstehen könnten.}
{die ausgesendeten Signale schwierig zu lesen sind.}
{die Stromversorgung überlastet wird.}
{\DARCimage{1.0\linewidth}{12include}}\end{PQuestion}

}
\only<2>{
\begin{PQuestion}{AJ220}{Diese Modulationshüllkurve eines CW-Senders sollte vermieden werden, da~...}{\textbf{\textcolor{DARCgreen}{wahrscheinlich Tastklicks erzeugt werden.}}}
{während der Aussetzer Probleme im Leistungsverstärker entstehen könnten.}
{die ausgesendeten Signale schwierig zu lesen sind.}
{die Stromversorgung überlastet wird.}
{\DARCimage{1.0\linewidth}{12include}}\end{PQuestion}

}
\end{frame}%ENDCONTENT
