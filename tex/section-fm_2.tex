
\section{Frequenzmodulation (FM) II}
\label{section:fm_2}
\begin{frame}%STARTCONTENT

\frametitle{Frequenzmodulation}
\begin{columns}
    \begin{column}{0.48\textwidth}
    \begin{itemize}
  \item Konstante Amplitude
  \item Veränderliche Frequenz
  \item Relativ unempfindlich gegenüber Amplitudenstörungen (z.B. Kfz, Blitze)
  \end{itemize}

    \end{column}
   \begin{column}{0.48\textwidth}
       
\begin{figure}
    \DARCimage{0.85\linewidth}{301include}
    \caption{\scriptsize Frequenzmodulation}
    \label{e_fm}
\end{figure}


   \end{column}
\end{columns}

\end{frame}

\begin{frame}
\only<1>{
\begin{PQuestion}{EE301}{Welches Modulationsverfahren zeigt das Bild?}{AM}
{FM}
{USB}
{LSB}
{\DARCimage{1.0\linewidth}{301include}}\end{PQuestion}

}
\only<2>{
\begin{PQuestion}{EE301}{Welches Modulationsverfahren zeigt das Bild?}{AM}
{\textbf{\textcolor{DARCgreen}{FM}}}
{USB}
{LSB}
{\DARCimage{1.0\linewidth}{301include}}\end{PQuestion}

}
\end{frame}

\begin{frame}
\only<1>{
\begin{QQuestion}{EE302}{FM hat gegenüber SSB den Vorteil der~...}{geringeren Leistungsaufnahme bei fehlender Modulation.}
{geringen Anforderungen an die Bandbreite.}
{größeren Entfernungsüberbrückung.}
{geringeren Beeinflussung durch Amplitudenstörungen.}
\end{QQuestion}

}
\only<2>{
\begin{QQuestion}{EE302}{FM hat gegenüber SSB den Vorteil der~...}{geringeren Leistungsaufnahme bei fehlender Modulation.}
{geringen Anforderungen an die Bandbreite.}
{größeren Entfernungsüberbrückung.}
{\textbf{\textcolor{DARCgreen}{geringeren Beeinflussung durch Amplitudenstörungen.}}}
\end{QQuestion}

}
\end{frame}

\begin{frame}
\only<1>{
\begin{QQuestion}{EE303}{Welches der nachfolgenden Modulationsverfahren wird am wenigsten durch Amplitudenstörungen in Kraftfahrzeugen beeinträchtigt?}{FM}
{SSB}
{DSB}
{AM}
\end{QQuestion}

}
\only<2>{
\begin{QQuestion}{EE303}{Welches der nachfolgenden Modulationsverfahren wird am wenigsten durch Amplitudenstörungen in Kraftfahrzeugen beeinträchtigt?}{\textbf{\textcolor{DARCgreen}{FM}}}
{SSB}
{DSB}
{AM}
\end{QQuestion}

}
\end{frame}

\begin{frame}
\frametitle{Frequenzhub}
\begin{columns}
    \begin{column}{0.48\textwidth}
    \begin{itemize}
  \item Lautstärkeinformation wird bei FM durch \emph{Trägerfrequenzauslenkung} (Frequenzhub) übertragen
  \item Lautes NF-Signal $\rightarrow$ größerer Hub $\rightarrow$ höhere Bandbreite
  \end{itemize}

    \end{column}
   \begin{column}{0.48\textwidth}
       
\begin{figure}
    \DARCimage{0.85\linewidth}{827include}
    \caption{\scriptsize Trägerauslenkung bei Frequenzmodulation}
    \label{e_frequenzmodulation_frequenzhub}
\end{figure}


   \end{column}
\end{columns}

\end{frame}

\begin{frame}
\only<1>{
\begin{QQuestion}{EE306}{Wodurch wird bei Frequenzmodulation die Lautstärke-Information übertragen?}{Durch die Häufigkeit der Trägerfrequenzänderung.}
{Durch die Trägerfrequenzauslenkung.}
{Durch die Häufigkeit des Frequenzhubes.}
{Durch die Größe der Amplitude des HF-Signals.}
\end{QQuestion}

}
\only<2>{
\begin{QQuestion}{EE306}{Wodurch wird bei Frequenzmodulation die Lautstärke-Information übertragen?}{Durch die Häufigkeit der Trägerfrequenzänderung.}
{\textbf{\textcolor{DARCgreen}{Durch die Trägerfrequenzauslenkung.}}}
{Durch die Häufigkeit des Frequenzhubes.}
{Durch die Größe der Amplitude des HF-Signals.}
\end{QQuestion}

}
\end{frame}

\begin{frame}
\only<1>{
\begin{QQuestion}{EE304}{Größerer Frequenzhub führt bei einem FM-Sender zu~...}{einer Erhöhung der Senderausgangsleistung.}
{einer größeren HF-Bandbreite.}
{einer Erhöhung der Amplitude der Trägerfrequenz.}
{einer Reduktion der Amplituden der Seitenbänder.}
\end{QQuestion}

}
\only<2>{
\begin{QQuestion}{EE304}{Größerer Frequenzhub führt bei einem FM-Sender zu~...}{einer Erhöhung der Senderausgangsleistung.}
{\textbf{\textcolor{DARCgreen}{einer größeren HF-Bandbreite.}}}
{einer Erhöhung der Amplitude der Trägerfrequenz.}
{einer Reduktion der Amplituden der Seitenbänder.}
\end{QQuestion}

}
\end{frame}

\begin{frame}
\frametitle{Modulation}
\begin{itemize}
  \item Zur Einschränkung der Bandbreite wird das Mikrofonsignal in der Amplitude begrenzt
  \item Dieses Signal wird auf den Träger mittels FM aufmoduliert
  \item Der Frequenzhub kann dabei fest sein oder einstellbar mittels eines Hub-Reglers
  \end{itemize}
\end{frame}

\begin{frame}
\only<1>{
\begin{QQuestion}{EE305}{Durch welche Maßnahme kann eine zu große Bandbreite einer FM-Aussendung verringert werden? Durch die Verringerung der~...}{Vorspannungsreglereinstellung.}
{HF-Begrenzung.}
{Hubeinstellung.}
{Trägerfrequenz.}
\end{QQuestion}

}
\only<2>{
\begin{QQuestion}{EE305}{Durch welche Maßnahme kann eine zu große Bandbreite einer FM-Aussendung verringert werden? Durch die Verringerung der~...}{Vorspannungsreglereinstellung.}
{HF-Begrenzung.}
{\textbf{\textcolor{DARCgreen}{Hubeinstellung.}}}
{Trägerfrequenz.}
\end{QQuestion}

}
\end{frame}%ENDCONTENT
