
\section{Abtasttheorem}
\label{section:abtasttheorem}
\begin{frame}%STARTCONTENT

\only<1>{
\begin{QQuestion}{AF616}{Welche Aussage trifft auf das Abtasttheorem zu? Das Theorem~...}{bestimmt die für eine fehlerfreie Rekonstruktion eines Signals theoretisch notwendige minimale Abtastrate.}
{besagt, dass theoretisch eine unendliche Abtastrate erforderlich ist, um ein bandbegrenztes Signal fehlerfrei zu rekonstruieren.}
{bestimmt die maximale Bandbreite, die durch eine Übertragung mit einer bestimmten Datenübertragungsrate theoretisch belegt werden kann.}
{besagt, dass unabhängig von der Art der vorherrschenden Störungen eines Übertragungskanals theoretisch eine unbegrenzte Datenübertragungsrate erzielt werden kann.}
\end{QQuestion}

}
\only<2>{
\begin{QQuestion}{AF616}{Welche Aussage trifft auf das Abtasttheorem zu? Das Theorem~...}{\textbf{\textcolor{DARCgreen}{bestimmt die für eine fehlerfreie Rekonstruktion eines Signals theoretisch notwendige minimale Abtastrate.}}}
{besagt, dass theoretisch eine unendliche Abtastrate erforderlich ist, um ein bandbegrenztes Signal fehlerfrei zu rekonstruieren.}
{bestimmt die maximale Bandbreite, die durch eine Übertragung mit einer bestimmten Datenübertragungsrate theoretisch belegt werden kann.}
{besagt, dass unabhängig von der Art der vorherrschenden Störungen eines Übertragungskanals theoretisch eine unbegrenzte Datenübertragungsrate erzielt werden kann.}
\end{QQuestion}

}
\end{frame}

\begin{frame}
\only<1>{
\begin{QQuestion}{AF618}{Ein analoges Signal mit einer Bandbreite von $f_{\symup{max}}$ soll digital verarbeitet werden. Welche der folgenden Abtastraten ist die kleinste, die Alias-Effekte vermeidet?}{knapp über $2 \cdot f_{\symup{max}}$}
{knapp über $f_{\symup{max}}$}
{knapp unter $\dfrac{f_{\mathrm{max}}}{2}$}
{knapp unter $f_{\symup{max}}$}
\end{QQuestion}

}
\only<2>{
\begin{QQuestion}{AF618}{Ein analoges Signal mit einer Bandbreite von $f_{\symup{max}}$ soll digital verarbeitet werden. Welche der folgenden Abtastraten ist die kleinste, die Alias-Effekte vermeidet?}{\textbf{\textcolor{DARCgreen}{knapp über $2 \cdot f_{\symup{max}}$}}}
{knapp über $f_{\symup{max}}$}
{knapp unter $\dfrac{f_{\mathrm{max}}}{2}$}
{knapp unter $f_{\symup{max}}$}
\end{QQuestion}

}
\end{frame}

\begin{frame}
\only<1>{
\begin{QQuestion}{AF619}{Ein analoges Sprachsignal mit \qty{4}{\kHz} Bandbreite soll digital verarbeitet werden. Welche der folgenden Abtastraten ist die kleinste, die Alias-Effekte vermeidet?}{9600 Samples/s}
{4800 Samples/s}
{4000 Samples/s}
{2400 Samples/s}
\end{QQuestion}

}
\only<2>{
\begin{QQuestion}{AF619}{Ein analoges Sprachsignal mit \qty{4}{\kHz} Bandbreite soll digital verarbeitet werden. Welche der folgenden Abtastraten ist die kleinste, die Alias-Effekte vermeidet?}{\textbf{\textcolor{DARCgreen}{9600 Samples/s}}}
{4800 Samples/s}
{4000 Samples/s}
{2400 Samples/s}
\end{QQuestion}

}
\end{frame}%ENDCONTENT
