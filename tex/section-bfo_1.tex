
\section{BFO I}
\label{section:bfo_1}
\begin{frame}%STARTCONTENT

\begin{columns}
    \begin{column}{0.48\textwidth}
    \begin{itemize}
  \item Beat-Frequency-Oszillator (BFO)
  \item ZF-Signal wird mittels Überlagerungsmischung durch den BFO als Hilfsträger demoduliert
  \item Angewandt bei Signalen ohne Hilfsträger (SSB, CW)
  \end{itemize}

    \end{column}
   \begin{column}{0.48\textwidth}
       
\begin{figure}
    \DARCimage{0.85\linewidth}{838include}
    \caption{\scriptsize BFO im Überlagerungsempfänger}
    \label{bfo_1_bfo}
\end{figure}


   \end{column}
\end{columns}

\end{frame}

\begin{frame}
\only<1>{
\begin{PQuestion}{EF209}{Welchem Zweck dient ein BFO in einem Empfänger?}{Zur Hilfsträgererzeugung, um CW- oder SSB-Signale hörbar zu machen}
{Zur Mischung mit einem Empfangssignal zur Erzeugung der ZF}
{Zur Unterdrückung der Amplitudenüberlagerung}
{Um FM-Signale zu unterdrücken}
{\DARCimage{1.0\linewidth}{590include}}\end{PQuestion}

}
\only<2>{
\begin{PQuestion}{EF209}{Welchem Zweck dient ein BFO in einem Empfänger?}{\textbf{\textcolor{DARCgreen}{Zur Hilfsträgererzeugung, um CW- oder SSB-Signale hörbar zu machen}}}
{Zur Mischung mit einem Empfangssignal zur Erzeugung der ZF}
{Zur Unterdrückung der Amplitudenüberlagerung}
{Um FM-Signale zu unterdrücken}
{\DARCimage{1.0\linewidth}{590include}}\end{PQuestion}

}
\end{frame}%ENDCONTENT
