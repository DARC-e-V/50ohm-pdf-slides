
\section{Polarisation elektromagnetischer Wellen}
\label{section:em_feld_polarisation}
\begin{frame}%STARTCONTENT

\frametitle{Horizontale Polarisation}
\begin{columns}
    \begin{column}{0.48\textwidth}
    \begin{itemize}
  \item Die Lage des E-Feldes gibt die Polarisation an
  \item Breitet sich das E-Feld horizontal aus, wird von horizontaler Polarisation gesprochen
  \item Ist von Bauform der Antenne abhängig
  \end{itemize}

    \end{column}
   \begin{column}{0.48\textwidth}
       
\begin{figure}
    \DARCimage{0.85\linewidth}{194include}
    \caption{\scriptsize Horizontale Polarisation in einem Feld}
    \label{e_horizontale_polarisation}
\end{figure}


   \end{column}
\end{columns}

\end{frame}

\begin{frame}
\only<1>{
\begin{QQuestion}{EB305}{Die Polarisation einer elektromagnetischen Welle ist durch die Richtung~...}{der Ausbreitung (S-Vektor/Poynting-Vektor) bestimmt.}
{des magnetischen Nordpols (relativ zur Antenne) bestimmt.}
{des elektrischen Feldes (Vektor des E-Feldes) bestimmt.}
{des unmittelbaren Nahfeldes ($\lambda/4$-Bereich) bestimmt.}
\end{QQuestion}

}
\only<2>{
\begin{QQuestion}{EB305}{Die Polarisation einer elektromagnetischen Welle ist durch die Richtung~...}{der Ausbreitung (S-Vektor/Poynting-Vektor) bestimmt.}
{des magnetischen Nordpols (relativ zur Antenne) bestimmt.}
{\textbf{\textcolor{DARCgreen}{des elektrischen Feldes (Vektor des E-Feldes) bestimmt.}}}
{des unmittelbaren Nahfeldes ($\lambda/4$-Bereich) bestimmt.}
\end{QQuestion}

}
\end{frame}

\begin{frame}
\only<1>{
\begin{PQuestion}{EB306}{Das folgende Bild zeigt eine Momentaufnahme eines elektromagnetischen Feldes. Welche Polarisation hat die skizzierte Welle?}{Rechtszirkulare Polarisation}
{Vertikale Polarisation}
{Horizontale Polarisation}
{Linkszirkulare Polarisation}
{\DARCimage{1.0\linewidth}{194include}}\end{PQuestion}

}
\only<2>{
\begin{PQuestion}{EB306}{Das folgende Bild zeigt eine Momentaufnahme eines elektromagnetischen Feldes. Welche Polarisation hat die skizzierte Welle?}{Rechtszirkulare Polarisation}
{Vertikale Polarisation}
{\textbf{\textcolor{DARCgreen}{Horizontale Polarisation}}}
{Linkszirkulare Polarisation}
{\DARCimage{1.0\linewidth}{194include}}\end{PQuestion}

}
\end{frame}

\begin{frame}
\only<1>{
\begin{PQuestion}{EB309}{Die Polarisation des Sendesignals in der Hauptstrahlrichtung dieser Richtantenne ist~...}{linksdrehend.}
{vertikal.}
{rechtsdrehend.}
{horizontal.}
{\DARCimage{1.0\linewidth}{326include}}\end{PQuestion}

}
\only<2>{
\begin{PQuestion}{EB309}{Die Polarisation des Sendesignals in der Hauptstrahlrichtung dieser Richtantenne ist~...}{linksdrehend.}
{vertikal.}
{rechtsdrehend.}
{\textbf{\textcolor{DARCgreen}{horizontal.}}}
{\DARCimage{1.0\linewidth}{326include}}\end{PQuestion}

}
\end{frame}

\begin{frame}
\frametitle{Vertikale Polarisation}
\begin{columns}
    \begin{column}{0.48\textwidth}
    \begin{itemize}
  \item Die Lage des E-Feldes gibt die Polarisation an
  \item Breitet sich das E-Feld vertikal aus, wird von vertikaler Polarisation gesprochen
  \item Ist von Bauform der Antenne abhängig
  \end{itemize}

    \end{column}
   \begin{column}{0.48\textwidth}
       
\begin{figure}
    \DARCimage{0.85\linewidth}{203include}
    \caption{\scriptsize Vertikale Polarisation in einem Feld}
    \label{e_vertikale_polarisation}
\end{figure}


   \end{column}
\end{columns}

\end{frame}

\begin{frame}
\only<1>{
\begin{PQuestion}{EB307}{Das folgende Bild zeigt eine Momentaufnahme eines elektromagnetischen Feldes. Welche Polarisation hat die skizzierte Welle?}{Vertikale Polarisation}
{Horizontale Polarisation}
{Linkszirkulare Polarisation}
{Rechtszirkulare Polarisation}
{\DARCimage{1.0\linewidth}{203include}}\end{PQuestion}

}
\only<2>{
\begin{PQuestion}{EB307}{Das folgende Bild zeigt eine Momentaufnahme eines elektromagnetischen Feldes. Welche Polarisation hat die skizzierte Welle?}{\textbf{\textcolor{DARCgreen}{Vertikale Polarisation}}}
{Horizontale Polarisation}
{Linkszirkulare Polarisation}
{Rechtszirkulare Polarisation}
{\DARCimage{1.0\linewidth}{203include}}\end{PQuestion}

}
\end{frame}

\begin{frame}
\only<1>{
\begin{PQuestion}{EB310}{Die Polarisation des Sendesignals in der Hauptstrahlrichtung dieser Richtantenne ist~...}{horizontal.}
{vertikal.}
{rechtsdrehend.}
{linksdrehend.}
{\DARCimage{1.0\linewidth}{51include}}\end{PQuestion}

}
\only<2>{
\begin{PQuestion}{EB310}{Die Polarisation des Sendesignals in der Hauptstrahlrichtung dieser Richtantenne ist~...}{horizontal.}
{\textbf{\textcolor{DARCgreen}{vertikal.}}}
{rechtsdrehend.}
{linksdrehend.}
{\DARCimage{1.0\linewidth}{51include}}\end{PQuestion}

}
\end{frame}

\begin{frame}
\frametitle{Zirkulare Polarisation}
\begin{columns}
    \begin{column}{0.48\textwidth}
    \begin{itemize}
  \item Die Lage des E-Feldes gibt die Polarisation an
  \item Breitet sich das E-Feld zirkular aus, wird von zirkularer Polarisation gesprochen
  \item Es ist rechts- und linksdrehend möglich
  \item Ist von Bauform der Antenne abhängig
  \end{itemize}

    \end{column}
   \begin{column}{0.48\textwidth}
       
\begin{figure}
    \DARCimage{0.85\linewidth}{204include}
    \caption{\scriptsize Zirkulare Polarisation in einem Feld}
    \label{e_zirkulare_polarisation}
\end{figure}


   \end{column}
\end{columns}

\end{frame}

\begin{frame}
\only<1>{
\begin{PQuestion}{EB308}{Das folgende Bild zeigt eine Momentaufnahme eines elektromagnetischen Feldes. Welche Polarisation hat die skizzierte Welle?}{Zirkulare Polarisation}
{Horizontale Polarisation}
{Vertikale Polarisation}
{Diagonale Polarisation}
{\DARCimage{1.0\linewidth}{204include}}\end{PQuestion}

}
\only<2>{
\begin{PQuestion}{EB308}{Das folgende Bild zeigt eine Momentaufnahme eines elektromagnetischen Feldes. Welche Polarisation hat die skizzierte Welle?}{\textbf{\textcolor{DARCgreen}{Zirkulare Polarisation}}}
{Horizontale Polarisation}
{Vertikale Polarisation}
{Diagonale Polarisation}
{\DARCimage{1.0\linewidth}{204include}}\end{PQuestion}

}
\end{frame}%ENDCONTENT
