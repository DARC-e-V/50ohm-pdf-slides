
\section{Polarisation II}
\label{section:polarisation_2}
\begin{frame}%STARTCONTENT
\begin{itemize}
  \item Polarisation einer Antenne bezieht sich auf die Ausrichtung des elektrischen Feldes
  \item In Hauptstrahlrichtung
  \item In Bezug zur Erdoberfläche
  \item Die Polarisationsrichtung kann nicht immer an der Bauform der Antenne erkannt werden
  \end{itemize}

\end{frame}

\begin{frame}
\only<1>{
\begin{QQuestion}{EG222}{Die Polarisation einer Antenne~...}{wird nach der Ausrichtung der magnetischen Feldkomponente in der Hauptstrahlrichtung in Bezug zur Erdoberfläche angegeben.}
{wird nach der Ausrichtung der elektrischen Feldkomponente in der Hauptstrahlrichtung in Bezug zur Erdoberfläche angegeben.}
{entspricht der Richtung der magnetischen Feldkomponente des empfangenen oder ausgesendeten Feldes in Bezug auf die Nordrichtung (Azimut).}
{entspricht der Richtung der elektrischen Feldkomponente des empfangenen oder ausgesendeten Feldes in Bezug auf die Nordrichtung (Azimut).}
\end{QQuestion}

}
\only<2>{
\begin{QQuestion}{EG222}{Die Polarisation einer Antenne~...}{wird nach der Ausrichtung der magnetischen Feldkomponente in der Hauptstrahlrichtung in Bezug zur Erdoberfläche angegeben.}
{\textbf{\textcolor{DARCgreen}{wird nach der Ausrichtung der elektrischen Feldkomponente in der Hauptstrahlrichtung in Bezug zur Erdoberfläche angegeben.}}}
{entspricht der Richtung der magnetischen Feldkomponente des empfangenen oder ausgesendeten Feldes in Bezug auf die Nordrichtung (Azimut).}
{entspricht der Richtung der elektrischen Feldkomponente des empfangenen oder ausgesendeten Feldes in Bezug auf die Nordrichtung (Azimut).}
\end{QQuestion}

}
\end{frame}%ENDCONTENT
