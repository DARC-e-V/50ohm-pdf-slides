
\section{Sprungdistanz II}
\label{section:sprungdistanz_2}
\begin{frame}%STARTCONTENT

\begin{columns}
    \begin{column}{0.48\textwidth}
    \begin{itemize}
  \item Bisher: Sprungdistanz durch Abstrahlwinkel verändern
  \item Auch zu beachten:
  \item \emph{Höhe der ionisierten Region}
  \item \emph{die Tageszeit} wegen der unterschiedlichen Schichten
  \item \emph{genutzte Frequenz} wegen unterschiedlicher Refraktionseigenschaften an den Schichten
  \end{itemize}

    \end{column}
   \begin{column}{0.48\textwidth}
       
   \end{column}
\end{columns}

\end{frame}

\begin{frame}
\only<1>{
\begin{QQuestion}{AH212}{Was hat \underline{keine} Auswirkungen auf die Sprungentfernung?}{Die Änderung der Frequenz des ausgesendeten Signals.}
{Die Änderung der Strahlungsleistung.}
{Die Tageszeit.}
{Die aktuelle Höhe der ionisierten Regionen.}
\end{QQuestion}

}
\only<2>{
\begin{QQuestion}{AH212}{Was hat \underline{keine} Auswirkungen auf die Sprungentfernung?}{Die Änderung der Frequenz des ausgesendeten Signals.}
{\textbf{\textcolor{DARCgreen}{Die Änderung der Strahlungsleistung.}}}
{Die Tageszeit.}
{Die aktuelle Höhe der ionisierten Regionen.}
\end{QQuestion}

}
\end{frame}

\begin{frame}
\only<1>{
\begin{QQuestion}{AH213}{Wie groß ist in etwa die maximale Entfernung, die ein KW-Signal bei Refraktion (Brechung) an der F2-Region auf der Erdoberfläche mit einem Sprung (Hop) überbrücken kann?}{Etwa \qty{8000}{\km}.}
{Etwa \qty{2000}{\km}.}
{Etwa \qty{12000}{\km}.}
{Etwa \qty{4000}{\km}.}
\end{QQuestion}

}
\only<2>{
\begin{QQuestion}{AH213}{Wie groß ist in etwa die maximale Entfernung, die ein KW-Signal bei Refraktion (Brechung) an der F2-Region auf der Erdoberfläche mit einem Sprung (Hop) überbrücken kann?}{Etwa \qty{8000}{\km}.}
{Etwa \qty{2000}{\km}.}
{Etwa \qty{12000}{\km}.}
{\textbf{\textcolor{DARCgreen}{Etwa \qty{4000}{\km}.}}}
\end{QQuestion}

}
\end{frame}%ENDCONTENT
