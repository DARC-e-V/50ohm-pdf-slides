
\section{Skineffekt}
\label{section:skineffekt}
\begin{frame}%STARTCONTENT

\only<1>{
\begin{QQuestion}{AG318}{Wie bezeichnet man den Effekt, dass sich mit steigender Frequenz der Elektronenstrom mehr und mehr zur Oberfläche eines Leiters hin verlagert, so dass sich mit steigender Frequenz der ohmsche Verlustwiderstand des Leiters erhöht?}{Als Dunning-Kruger-Effekt}
{Als Mögel-Dellinger-Effekt}
{Als Doppler-Effekt}
{Als Skin-Effekt}
\end{QQuestion}

}
\only<2>{
\begin{QQuestion}{AG318}{Wie bezeichnet man den Effekt, dass sich mit steigender Frequenz der Elektronenstrom mehr und mehr zur Oberfläche eines Leiters hin verlagert, so dass sich mit steigender Frequenz der ohmsche Verlustwiderstand des Leiters erhöht?}{Als Dunning-Kruger-Effekt}
{Als Mögel-Dellinger-Effekt}
{Als Doppler-Effekt}
{\textbf{\textcolor{DARCgreen}{Als Skin-Effekt}}}
\end{QQuestion}

}
\end{frame}

\begin{frame}
\only<1>{
\begin{QQuestion}{AG319}{Welche Folgen hat der Skin-Effekt bei steigender Frequenz? Der stromdurchflossene Querschnitt des Leiters~...}{sinkt und dadurch sinkt der effektive Widerstand des Leiters.}
{steigt und dadurch sinkt der effektive Widerstand des Leiters.}
{sinkt und dadurch steigt der effektive Widerstand des Leiters.}
{steigt und dadurch steigt der effektive Widerstand des Leiters.}
\end{QQuestion}

}
\only<2>{
\begin{QQuestion}{AG319}{Welche Folgen hat der Skin-Effekt bei steigender Frequenz? Der stromdurchflossene Querschnitt des Leiters~...}{sinkt und dadurch sinkt der effektive Widerstand des Leiters.}
{steigt und dadurch sinkt der effektive Widerstand des Leiters.}
{\textbf{\textcolor{DARCgreen}{sinkt und dadurch steigt der effektive Widerstand des Leiters.}}}
{steigt und dadurch steigt der effektive Widerstand des Leiters.}
\end{QQuestion}

}
\end{frame}%ENDCONTENT
