
\section{Personenschutz}
\label{section:personenschutzabstand}
\begin{frame}%STARTCONTENT
\begin{itemize}
  \item Elektromagnetische Felder können eine Auswirkung auf Menschen haben, die sich darin aufhalten
  \item Es darf zu keiner Gefährdung von Menschen durch Amateurfunkanlagen kommen
  \item \emph{Jeder Funkamateur} muss sich mit dem \emph{Personenschutz in elektromagnetischen Feldern} auskennen
  \end{itemize}

\end{frame}

\begin{frame}
\only<1>{
\begin{QQuestion}{NK201}{Warum muss ein Funkamateur Kenntnisse zum Personenschutz in elektromagnetischen Feldern haben?}{Damit er seinen Sender optimal an die Antenne anpassen kann.}
{Weil zu hohe Feldstärken in Antennennähe schädigend auf den menschlichen Körper wirken können.}
{Damit er bei einem Stromunfall als Ersthelfer tätig werden kann.}
{Weil eine Standortbescheinigung der Bundesnetzagentur hierfür nicht gültig wäre.}
\end{QQuestion}

}
\only<2>{
\begin{QQuestion}{NK201}{Warum muss ein Funkamateur Kenntnisse zum Personenschutz in elektromagnetischen Feldern haben?}{Damit er seinen Sender optimal an die Antenne anpassen kann.}
{\textbf{\textcolor{DARCgreen}{Weil zu hohe Feldstärken in Antennennähe schädigend auf den menschlichen Körper wirken können.}}}
{Damit er bei einem Stromunfall als Ersthelfer tätig werden kann.}
{Weil eine Standortbescheinigung der Bundesnetzagentur hierfür nicht gültig wäre.}
\end{QQuestion}

}
\end{frame}

\begin{frame}
\frametitle{EMVU}
Der Betreiber der ortsfesten Amateurfunkstelle ist für die Sicherstellung der \enquote{elektromagnetischen Verträglichkeit in der Umwelt} (EMVU) verantwortlich.

\end{frame}

\begin{frame}
\only<1>{
\begin{QQuestion}{VE501}{Was bedeutet die Abkürzung EMVU?}{Elektromagnetische Verträglichkeit in der Umwelt}
{Elektromagnetische Verträglichkeit von Geräten}
{Elektronische Messung von elektromagnetischen Unverträglichkeiten}
{Eine Bürgerinitiative zum Schutz vor elektromagnetischen Unverträglichkeiten}
\end{QQuestion}

}
\only<2>{
\begin{QQuestion}{VE501}{Was bedeutet die Abkürzung EMVU?}{\textbf{\textcolor{DARCgreen}{Elektromagnetische Verträglichkeit in der Umwelt}}}
{Elektromagnetische Verträglichkeit von Geräten}
{Elektronische Messung von elektromagnetischen Unverträglichkeiten}
{Eine Bürgerinitiative zum Schutz vor elektromagnetischen Unverträglichkeiten}
\end{QQuestion}

}
\end{frame}

\begin{frame}
\only<1>{
\begin{QQuestion}{VE502}{Wer ist für die Sicherstellung der elektromagnetischen Umweltverträglichkeit verantwortlich?}{Der Hersteller des Amateurfunkgerätes}
{Die Bundesnetzagentur}
{Der Betreiber der ortsfesten Amateurfunkstelle}
{Der Erbauer der Antennenanlage}
\end{QQuestion}

}
\only<2>{
\begin{QQuestion}{VE502}{Wer ist für die Sicherstellung der elektromagnetischen Umweltverträglichkeit verantwortlich?}{Der Hersteller des Amateurfunkgerätes}
{Die Bundesnetzagentur}
{\textbf{\textcolor{DARCgreen}{Der Betreiber der ortsfesten Amateurfunkstelle}}}
{Der Erbauer der Antennenanlage}
\end{QQuestion}

}
\end{frame}

\begin{frame}
\frametitle{BIm-SchV und BEMFV}
\begin{itemize}
  \item Grenzwerte finden sich in der \enquote{26. Verordnung zur Durchführung des Bundes-Immissionsschutzgesetzes} (26. BIm-SchV) und in der \enquote{Verordnung über das Nachweisverfahren zur Begrenzung elektromagnetischer Felder} (BEMFV)
  \item In der Verordnung über das \enquote{Nachweisverfahren zur Begrenzung elektromagnetischer Felder} (BEMFV) ist das Anzeigeverfahren beschrieben
  \item \emph{Funkamateur stellt vor Inbetriebnahme eigenständig sicher und dokumentiert, dass keine Gefährdung für Personen besteht}
  \end{itemize}

\end{frame}

\begin{frame}
\only<1>{
\begin{QQuestion}{VE505}{Wo sind die Grenzwerte zum Schutz von Personen und aktiven Körperhilfen in elektromagnetischen Feldern festgelegt?}{Im Gesetz über die Bereitstellung von Funkanlagen auf dem Markt (Funkanlagengesetz - FuAG) und im Telekommunikationsgesetz (TKG)}
{In der Richtlinie Elektromagnetische Verträglichkeit von Elektro- und Elektronikprodukten (EMV-Richtlinie) und im Gesetz über die elektromagnetische Verträglichkeit von Betriebsmitteln (EMVG)}
{In der 26. Verordnung zur Durchführung des Bundes-Immissionsschutzgesetzes (Verordnung über elektromagnetische Felder - 26. BImSchV) und in der Verordnung über das Nachweisverfahren zur Begrenzung elektromagnetischer Felder (BEMFV)}
{Im Gesetz über den Amateurfunk (Amateurfunkgesetz - AFuG) und in der Verordnung zum Gesetz über den Amateurfunk (Amateurfunkverordnung - AFuV)}
\end{QQuestion}

}
\only<2>{
\begin{QQuestion}{VE505}{Wo sind die Grenzwerte zum Schutz von Personen und aktiven Körperhilfen in elektromagnetischen Feldern festgelegt?}{Im Gesetz über die Bereitstellung von Funkanlagen auf dem Markt (Funkanlagengesetz - FuAG) und im Telekommunikationsgesetz (TKG)}
{In der Richtlinie Elektromagnetische Verträglichkeit von Elektro- und Elektronikprodukten (EMV-Richtlinie) und im Gesetz über die elektromagnetische Verträglichkeit von Betriebsmitteln (EMVG)}
{\textbf{\textcolor{DARCgreen}{In der 26. Verordnung zur Durchführung des Bundes-Immissionsschutzgesetzes (Verordnung über elektromagnetische Felder - 26. BImSchV) und in der Verordnung über das Nachweisverfahren zur Begrenzung elektromagnetischer Felder (BEMFV)}}}
{Im Gesetz über den Amateurfunk (Amateurfunkgesetz - AFuG) und in der Verordnung zum Gesetz über den Amateurfunk (Amateurfunkverordnung - AFuV)}
\end{QQuestion}

}
\end{frame}

\begin{frame}
\only<1>{
\begin{QQuestion}{VE503}{In welcher gesetzlichen Regelung ist das Verfahren zum Schutz von Personen in elektromagnetischen Feldern ortsfester Amateurfunkstellen festgelegt?}{Im Bundesimmissionsschutzgesetz (BImSchG)}
{Im Amateurfunkgesetz (AfuG)}
{In den Radio Regulations (RR)}
{In der Verordnung über das Nachweisverfahren zur Begrenzung elektromagnetischer Felder (BEMFV)}
\end{QQuestion}

}
\only<2>{
\begin{QQuestion}{VE503}{In welcher gesetzlichen Regelung ist das Verfahren zum Schutz von Personen in elektromagnetischen Feldern ortsfester Amateurfunkstellen festgelegt?}{Im Bundesimmissionsschutzgesetz (BImSchG)}
{Im Amateurfunkgesetz (AfuG)}
{In den Radio Regulations (RR)}
{\textbf{\textcolor{DARCgreen}{In der Verordnung über das Nachweisverfahren zur Begrenzung elektromagnetischer Felder (BEMFV)}}}
\end{QQuestion}

}
\end{frame}

\begin{frame}
\only<1>{
\begin{QQuestion}{VE504}{Was versteht man nach der Verordnung über das Nachweisverfahren zur Begrenzung elektromagnetischer Felder (BEMFV) unter dem \glqq Anzeigeverfahren ortsfester Amateurfunkanlagen\grqq{}?}{Die Erklärung des Funkamateurs, dass er den Grenzwert von \qty{10}{\W} EIRP unterschreitet}
{Ein Verfahren, das ein zertifiziertes Messlabor durchführen muss, um sicherzustellen, dass keine Gefährdung für Personen besteht}
{Ein Verfahren, das es dem Funkamateur ermöglicht, eigenständig sicherzustellen und zu dokumentieren, dass keine Gefährdung für Personen besteht}
{Die Erklärung des Funkamateurs, dass er den Grenzwert von \qty{750}{\W} PEP nicht überschreitet}
\end{QQuestion}

}
\only<2>{
\begin{QQuestion}{VE504}{Was versteht man nach der Verordnung über das Nachweisverfahren zur Begrenzung elektromagnetischer Felder (BEMFV) unter dem \glqq Anzeigeverfahren ortsfester Amateurfunkanlagen\grqq{}?}{Die Erklärung des Funkamateurs, dass er den Grenzwert von \qty{10}{\W} EIRP unterschreitet}
{Ein Verfahren, das ein zertifiziertes Messlabor durchführen muss, um sicherzustellen, dass keine Gefährdung für Personen besteht}
{\textbf{\textcolor{DARCgreen}{Ein Verfahren, das es dem Funkamateur ermöglicht, eigenständig sicherzustellen und zu dokumentieren, dass keine Gefährdung für Personen besteht}}}
{Die Erklärung des Funkamateurs, dass er den Grenzwert von \qty{750}{\W} PEP nicht überschreitet}
\end{QQuestion}

}
\end{frame}

\begin{frame}
\only<1>{
\begin{QQuestion}{VE511}{Welchen Status hat im Rahmen der EMVU die Anzeige einer ortsfesten Amateurfunkanlage?}{Die Anzeige ist die verbindliche Erklärung eines Funkamateurs über die eigenverantwortliche Einhaltung der gesetzlichen Grenzwerte zum Schutz von Personen in elektromagnetischen Feldern.}
{Die Anzeige ist eine unverbindliche Erklärung darüber, dass Funkamateure eigenverantwortlich handeln.}
{Die Anzeige hat den gleichen rechtlichen Status wie eine Standortbescheinigung, gilt aber nur für nichtkommerzielle Anlagen.}
{Die Anzeige ist die verbindliche Erklärung eines Funkamateurs über die eigenverantwortliche Einhaltung des Bundesimmissionsschutzgesetzes.}
\end{QQuestion}

}
\only<2>{
\begin{QQuestion}{VE511}{Welchen Status hat im Rahmen der EMVU die Anzeige einer ortsfesten Amateurfunkanlage?}{\textbf{\textcolor{DARCgreen}{Die Anzeige ist die verbindliche Erklärung eines Funkamateurs über die eigenverantwortliche Einhaltung der gesetzlichen Grenzwerte zum Schutz von Personen in elektromagnetischen Feldern.}}}
{Die Anzeige ist eine unverbindliche Erklärung darüber, dass Funkamateure eigenverantwortlich handeln.}
{Die Anzeige hat den gleichen rechtlichen Status wie eine Standortbescheinigung, gilt aber nur für nichtkommerzielle Anlagen.}
{Die Anzeige ist die verbindliche Erklärung eines Funkamateurs über die eigenverantwortliche Einhaltung des Bundesimmissionsschutzgesetzes.}
\end{QQuestion}

}
\end{frame}%ENDCONTENT
