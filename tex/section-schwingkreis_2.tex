
\section{Schwingkreis II}
\label{section:schwingkreis_2}
\begin{frame}%STARTCONTENT

\only<1>{
\begin{PQuestion}{AD201}{Welche Grenzfrequenz ergibt sich bei einem Hochpass mit einem Widerstand von \qty{4,7}{\kohm} und einem Kondensator von \qty{2,2}{\nF}?}{\qty{154}{\Hz}}
{\qty{1,54}{\kHz}}
{\qty{154}{\kHz}}
{\qty{15,4}{\kHz}}
{\DARCimage{1.0\linewidth}{195include}}\end{PQuestion}

}
\only<2>{
\begin{PQuestion}{AD201}{Welche Grenzfrequenz ergibt sich bei einem Hochpass mit einem Widerstand von \qty{4,7}{\kohm} und einem Kondensator von \qty{2,2}{\nF}?}{\qty{154}{\Hz}}
{\qty{1,54}{\kHz}}
{\qty{154}{\kHz}}
{\textbf{\textcolor{DARCgreen}{\qty{15,4}{\kHz}}}}
{\DARCimage{1.0\linewidth}{195include}}\end{PQuestion}

}
\end{frame}

\begin{frame}
\frametitle{Lösungsweg}
\begin{itemize}
  \item gegeben: $R = 4,7kΩ$
  \item gegeben: $C = 2,2nF$
  \item gesucht: $f_g$
  \end{itemize}
    \pause
    $f_g = \frac{1}{2 \cdot \pi \cdot R \cdot C} = \frac{1}{2 \cdot \pi \cdot 4,7kΩ \cdot 2,2nF} = 15,4kHz$



\end{frame}

\begin{frame}
\only<1>{
\begin{PQuestion}{AD202}{Welche Grenzfrequenz ergibt sich bei einem Tiefpass mit einem Widerstand von \qty{10}{\kohm} und einem Kondensator von \qty{47}{\nF}?}{\qty{339}{\kHz}}
{\qty{3,39}{\kHz}}
{\qty{339}{\Hz}}
{\qty{33,9}{\Hz}}
{\DARCimage{1.0\linewidth}{175include}}\end{PQuestion}

}
\only<2>{
\begin{PQuestion}{AD202}{Welche Grenzfrequenz ergibt sich bei einem Tiefpass mit einem Widerstand von \qty{10}{\kohm} und einem Kondensator von \qty{47}{\nF}?}{\qty{339}{\kHz}}
{\qty{3,39}{\kHz}}
{\textbf{\textcolor{DARCgreen}{\qty{339}{\Hz}}}}
{\qty{33,9}{\Hz}}
{\DARCimage{1.0\linewidth}{175include}}\end{PQuestion}

}
\end{frame}

\begin{frame}
\frametitle{Lösungsweg}
\begin{itemize}
  \item gegeben: $R = 10kΩ$
  \item gegeben: $C = 47nF$
  \item gesucht: $f_g$
  \end{itemize}
    \pause
    $f_g = \frac{1}{2 \cdot \pi \cdot R \cdot C} = \frac{1}{2 \cdot \pi \cdot 10kΩ \cdot 47nF} = 339Hz$



\end{frame}

\begin{frame}
\only<1>{
\begin{PQuestion}{AD203}{Wo liegt die Grenzfrequenz des Audio-Verstärkers, wenn $R_{1}$ = \qty{4,7}{\kilo\ohm}, $C_1$ = \qty{6,8}{\nF} und $C_2$ = \qty{47}{\nF} betragen? Der Verstärker hat eine Grenzfrequenz von \qty{1}{\MHz} und die Impedanz des Eingangs PIN 2 ist mit \qty{1}{\Mohm} sehr hochohmig.}{ca. 5~kHz}
{ca. 720~Hz}
{ca. 2,7~kHz}
{ca. 294~Hz}
{\DARCimage{1.0\linewidth}{488include}}\end{PQuestion}

}
\only<2>{
\begin{PQuestion}{AD203}{Wo liegt die Grenzfrequenz des Audio-Verstärkers, wenn $R_{1}$ = \qty{4,7}{\kilo\ohm}, $C_1$ = \qty{6,8}{\nF} und $C_2$ = \qty{47}{\nF} betragen? Der Verstärker hat eine Grenzfrequenz von \qty{1}{\MHz} und die Impedanz des Eingangs PIN 2 ist mit \qty{1}{\Mohm} sehr hochohmig.}{\textbf{\textcolor{DARCgreen}{ca. 5~kHz}}}
{ca. 720~Hz}
{ca. 2,7~kHz}
{ca. 294~Hz}
{\DARCimage{1.0\linewidth}{488include}}\end{PQuestion}

}
\end{frame}

\begin{frame}
\frametitle{Lösungsweg}
\begin{itemize}
  \item gegeben: $R_1 = 4,7kΩ$
  \item gegeben: $C_1 = 6,8nF$
  \item gesucht: $f_g$
  \end{itemize}
    \pause
    $C_2$ und alle weiteren Angaben sind für den Tiefpass uninteressant.
    \pause
    $f_g = \frac{1}{2 \cdot \pi \cdot R_1 \cdot C_1} = \frac{1}{2 \cdot \pi \cdot 4,7kΩ \cdot 6,8nF} \approx 5kHz$



\end{frame}

\begin{frame}
\only<1>{
\begin{QQuestion}{AD206}{Was ist im Resonanzfall bei der Reihenschaltung einer Induktivität mit einer Kapazität erfüllt?}{Der Betrag des elektrischen Feldes in der Spule ist dann gleich dem Betrag des elektrischen Feldes im Kondensator.}
{Der Betrag des Verlustwiderstandes der Spule ist dann gleich dem Betrag des Verlustwiderstandes des Kondensators.}
{Der Betrag des induktiven Widerstands ist dann gleich dem Betrag des kapazitiven Widerstands.}
{Der Betrag des magnetischen Feldes in der Spule ist dann gleich dem Betrag des magnetischen Feldes im Kondensator.}
\end{QQuestion}

}
\only<2>{
\begin{QQuestion}{AD206}{Was ist im Resonanzfall bei der Reihenschaltung einer Induktivität mit einer Kapazität erfüllt?}{Der Betrag des elektrischen Feldes in der Spule ist dann gleich dem Betrag des elektrischen Feldes im Kondensator.}
{Der Betrag des Verlustwiderstandes der Spule ist dann gleich dem Betrag des Verlustwiderstandes des Kondensators.}
{\textbf{\textcolor{DARCgreen}{Der Betrag des induktiven Widerstands ist dann gleich dem Betrag des kapazitiven Widerstands.}}}
{Der Betrag des magnetischen Feldes in der Spule ist dann gleich dem Betrag des magnetischen Feldes im Kondensator.}
\end{QQuestion}

}
\end{frame}

\begin{frame}
\only<1>{
\begin{PQuestion}{AD207}{Bei der Resonanzfrequenz ist die Impedanz dieser Schaltung~...}{gleich dem Wirkwiderstand $R$.}
{unendlich hoch.}
{gleich dem kapazitiven Widerstand $X_{\symup{C}}$.}
{gleich dem induktiven Widerstand $X_{\symup{L}}$.}
{\DARCimage{1.0\linewidth}{181include}}\end{PQuestion}

}
\only<2>{
\begin{PQuestion}{AD207}{Bei der Resonanzfrequenz ist die Impedanz dieser Schaltung~...}{\textbf{\textcolor{DARCgreen}{gleich dem Wirkwiderstand $R$.}}}
{unendlich hoch.}
{gleich dem kapazitiven Widerstand $X_{\symup{C}}$.}
{gleich dem induktiven Widerstand $X_{\symup{L}}$.}
{\DARCimage{1.0\linewidth}{181include}}\end{PQuestion}

}
\end{frame}

\begin{frame}
\only<1>{
\begin{question2x2}{AD204}{Welcher Schwingkreis passt zu dem neben der jeweiligen Schaltung dargestellten Verlauf der Impedanz?}{\DARCimage{1.0\linewidth}{230include}}
{\DARCimage{1.0\linewidth}{231include}}
{\DARCimage{1.0\linewidth}{232include}}
{\DARCimage{1.0\linewidth}{826include}}
\end{question2x2}

}
\only<2>{
\begin{question2x2}{AD204}{Welcher Schwingkreis passt zu dem neben der jeweiligen Schaltung dargestellten Verlauf der Impedanz?}{\textbf{\textcolor{DARCgreen}{\DARCimage{1.0\linewidth}{230include}}}}
{\DARCimage{1.0\linewidth}{231include}}
{\DARCimage{1.0\linewidth}{232include}}
{\DARCimage{1.0\linewidth}{826include}}
\end{question2x2}

}
\end{frame}

\begin{frame}
\only<1>{
\begin{PQuestion}{AD208}{Welche Resonanzfrequenz $f_{\symup{res}}$ hat die Reihenschaltung einer Spule von \qty{1,2}{\micro\H} mit einem Kondensator von \qty{6,8}{\pF} und einem Widerstand von \qty{10}{\ohm}?}{\qty{55,7}{\MHz}}
{\qty{5,57}{\MHz}}
{\qty{557}{\MHz}}
{\qty{557}{\kHz}}
{\DARCimage{1.0\linewidth}{181include}}\end{PQuestion}

}
\only<2>{
\begin{PQuestion}{AD208}{Welche Resonanzfrequenz $f_{\symup{res}}$ hat die Reihenschaltung einer Spule von \qty{1,2}{\micro\H} mit einem Kondensator von \qty{6,8}{\pF} und einem Widerstand von \qty{10}{\ohm}?}{\textbf{\textcolor{DARCgreen}{\qty{55,7}{\MHz}}}}
{\qty{5,57}{\MHz}}
{\qty{557}{\MHz}}
{\qty{557}{\kHz}}
{\DARCimage{1.0\linewidth}{181include}}\end{PQuestion}

}
\end{frame}

\begin{frame}
\frametitle{Lösungsweg}
\begin{itemize}
  \item gegeben: $L = 1,2µH$
  \item gegeben: $C = 6,8pF$
  \item gegeben: $R = 10Ω$
  \item gesucht: $f_0$
  \end{itemize}
    \pause
    $f_0 = \frac{1}{2 \cdot \pi \cdot \sqrt{L \cdot C}} = \frac{1}{2 \cdot \pi \cdot \sqrt{1,2µH \cdot 6,8pF}} = 55,7MHz$
    \pause
    Widerstand $R$ wird zur Berechnung nicht benötigt.



\end{frame}

\begin{frame}
\only<1>{
\begin{PQuestion}{AD209}{Welche Resonanzfrequenz $f_{\symup{res}}$ hat die Reihenschaltung einer Spule von \qty{10}{\micro\H} mit einem Kondensator von \qty{1}{\nF} und einem Widerstand von \qty{0,1}{\kohm}?}{\qty{15,92}{\MHz}}
{\qty{159,2}{\kHz}}
{\qty{1,592}{\MHz}}
{\qty{15,92}{\kHz}}
{\DARCimage{1.0\linewidth}{181include}}\end{PQuestion}

}
\only<2>{
\begin{PQuestion}{AD209}{Welche Resonanzfrequenz $f_{\symup{res}}$ hat die Reihenschaltung einer Spule von \qty{10}{\micro\H} mit einem Kondensator von \qty{1}{\nF} und einem Widerstand von \qty{0,1}{\kohm}?}{\qty{15,92}{\MHz}}
{\qty{159,2}{\kHz}}
{\textbf{\textcolor{DARCgreen}{\qty{1,592}{\MHz}}}}
{\qty{15,92}{\kHz}}
{\DARCimage{1.0\linewidth}{181include}}\end{PQuestion}

}
\end{frame}

\begin{frame}
\frametitle{Lösungsweg}
\begin{itemize}
  \item gegeben: $L = 10µH$
  \item gegeben: $C = 1nF$
  \item gesucht: $f_0$
  \end{itemize}
    \pause
    $f_0 = \frac{1}{2 \cdot \pi \cdot \sqrt{L \cdot C}} = \frac{1}{2 \cdot \pi \cdot \sqrt{10µH \cdot 1nF}} = 1,592MHz$



\end{frame}

\begin{frame}
\only<1>{
\begin{PQuestion}{AD210}{Welche Resonanzfrequenz $f_{\symup{res}}$ hat die Reihenschaltung einer Spule von \qty{100}{\micro\H} mit einem Kondensator von \qty{0,01}{\micro\F} und einem Widerstand von \qty{100}{\ohm}?}{\qty{1,59}{\kHz}}
{\qty{15,9}{\kHz}}
{\qty{159}{\kHz}}
{\qty{1590}{\kHz}}
{\DARCimage{1.0\linewidth}{181include}}\end{PQuestion}

}
\only<2>{
\begin{PQuestion}{AD210}{Welche Resonanzfrequenz $f_{\symup{res}}$ hat die Reihenschaltung einer Spule von \qty{100}{\micro\H} mit einem Kondensator von \qty{0,01}{\micro\F} und einem Widerstand von \qty{100}{\ohm}?}{\qty{1,59}{\kHz}}
{\qty{15,9}{\kHz}}
{\textbf{\textcolor{DARCgreen}{\qty{159}{\kHz}}}}
{\qty{1590}{\kHz}}
{\DARCimage{1.0\linewidth}{181include}}\end{PQuestion}

}
\end{frame}

\begin{frame}
\frametitle{Lösungsweg}
\begin{itemize}
  \item gegeben: $L = 100µH$
  \item gegeben: $C = 0,01µF$
  \item gesucht: $f_0$
  \end{itemize}
    \pause
    $f_0 = \frac{1}{2 \cdot \pi \cdot \sqrt{L \cdot C}} = \frac{1}{2 \cdot \pi \cdot \sqrt{100µH \cdot 0,01µF}} = 159kHz$



\end{frame}

\begin{frame}
\only<1>{
\begin{PQuestion}{AD211}{Welche Resonanzfrequenz $f_{\symup{res}}$ hat die Parallelschaltung einer Spule von \qty{2,2}{\micro\H} mit einem Kondensator von \qty{56}{\pF} und einem Widerstand von \qty{10}{\kohm}?}{\qty{14,34}{\MHz}}
{\qty{143,4}{\MHz}}
{\qty{1,434}{\MHz}}
{\qty{143,4}{\kHz}}
{\DARCimage{1.0\linewidth}{233include}}\end{PQuestion}

}
\only<2>{
\begin{PQuestion}{AD211}{Welche Resonanzfrequenz $f_{\symup{res}}$ hat die Parallelschaltung einer Spule von \qty{2,2}{\micro\H} mit einem Kondensator von \qty{56}{\pF} und einem Widerstand von \qty{10}{\kohm}?}{\textbf{\textcolor{DARCgreen}{\qty{14,34}{\MHz}}}}
{\qty{143,4}{\MHz}}
{\qty{1,434}{\MHz}}
{\qty{143,4}{\kHz}}
{\DARCimage{1.0\linewidth}{233include}}\end{PQuestion}

}
\end{frame}

\begin{frame}
\frametitle{Lösungsweg}
\begin{itemize}
  \item gegeben: $L = 2,2µH$
  \item gegeben: $C = 56pF$
  \item gesucht: $f_0$
  \end{itemize}
    \pause
    $f_0 = \frac{1}{2 \cdot \pi \cdot \sqrt{L \cdot C}} = \frac{1}{2 \cdot \pi \cdot \sqrt{2,2µH \cdot 56pF}} = 14,34MHz$



\end{frame}

\begin{frame}
\only<1>{
\begin{PQuestion}{AD212}{Wie groß ist die Resonanzfrequenz dieser Schaltung, wenn die Kapazitäten $C_1$ = \qty{0,1}{\nF}, $C_2$ = \qty{1,5}{\nF}, $C_3$ = \qty{220}{\pF} und die Induktivität der Spule \qty{1,2}{\mH} betragen?}{\qty{10,77}{\kHz}}
{\qty{107,7}{\kHz}}
{\qty{1,077}{\kHz}}
{\qty{1,077}{\MHz}}
{\DARCimage{1.0\linewidth}{776include}}\end{PQuestion}

}
\only<2>{
\begin{PQuestion}{AD212}{Wie groß ist die Resonanzfrequenz dieser Schaltung, wenn die Kapazitäten $C_1$ = \qty{0,1}{\nF}, $C_2$ = \qty{1,5}{\nF}, $C_3$ = \qty{220}{\pF} und die Induktivität der Spule \qty{1,2}{\mH} betragen?}{\qty{10,77}{\kHz}}
{\textbf{\textcolor{DARCgreen}{\qty{107,7}{\kHz}}}}
{\qty{1,077}{\kHz}}
{\qty{1,077}{\MHz}}
{\DARCimage{1.0\linewidth}{776include}}\end{PQuestion}

}
\end{frame}

\begin{frame}
\frametitle{Lösungsweg}
\begin{itemize}
  \item gegeben: $C_1 = 0,1nF$
  \item gegeben: $C_2 = 1,5nF$
  \item gegeben: $C_3 = 220pF$
  \item gegeben: $L = 1,2mH$
  \item gesucht: $f_0$
  \end{itemize}
    \pause
    $C = C_1 + C_2 + C_3 = 0,1nF + 1,5nF + 220pF = 1,82nF$
    \pause
    $f_0 = \frac{1}{2 \cdot \pi \cdot \sqrt{L \cdot C}} = \frac{1}{2 \cdot \pi \cdot \sqrt{1,2mH \cdot 1,82nF}} = 107,7kHz$



\end{frame}

\begin{frame}
\only<1>{
\begin{QQuestion}{AD213}{Sie wollen die Resonanzfrequenz eines Schwingkreises vergrößern. Welche der folgenden Maßnahmen ist geeignet?}{Kleineren Spulenwert verwenden}
{Spule zusammenschieben}
{Ferritkern in die Spule einführen}
{Anzahl der Spulenwindungen erhöhen}
\end{QQuestion}

}
\only<2>{
\begin{QQuestion}{AD213}{Sie wollen die Resonanzfrequenz eines Schwingkreises vergrößern. Welche der folgenden Maßnahmen ist geeignet?}{\textbf{\textcolor{DARCgreen}{Kleineren Spulenwert verwenden}}}
{Spule zusammenschieben}
{Ferritkern in die Spule einführen}
{Anzahl der Spulenwindungen erhöhen}
\end{QQuestion}

}
\end{frame}

\begin{frame}
\only<1>{
\begin{QQuestion}{AD214}{Sie wollen die Resonanzfrequenz eines Schwingkreises vergrößern. Welche der folgenden Maßnahmen ist geeignet?}{Spule zusammenschieben}
{Anzahl der Spulenwindungen verringern}
{Größeren Spulenwert verwenden}
{Größeren Kondensatorwert verwenden}
\end{QQuestion}

}
\only<2>{
\begin{QQuestion}{AD214}{Sie wollen die Resonanzfrequenz eines Schwingkreises vergrößern. Welche der folgenden Maßnahmen ist geeignet?}{Spule zusammenschieben}
{\textbf{\textcolor{DARCgreen}{Anzahl der Spulenwindungen verringern}}}
{Größeren Spulenwert verwenden}
{Größeren Kondensatorwert verwenden}
\end{QQuestion}

}
\end{frame}

\begin{frame}
\only<1>{
\begin{QQuestion}{AD215}{Sie wollen die Resonanzfrequenz eines Schwingkreises verringern. Welche der folgenden Maßnahmen ist geeignet?}{Anzahl der Spulenwindungen verringern}
{Kleineren Spulenwert verwenden}
{Größeren Kondensatorwert verwenden}
{Spule auseinanderziehen}
\end{QQuestion}

}
\only<2>{
\begin{QQuestion}{AD215}{Sie wollen die Resonanzfrequenz eines Schwingkreises verringern. Welche der folgenden Maßnahmen ist geeignet?}{Anzahl der Spulenwindungen verringern}
{Kleineren Spulenwert verwenden}
{\textbf{\textcolor{DARCgreen}{Größeren Kondensatorwert verwenden}}}
{Spule auseinanderziehen}
\end{QQuestion}

}
\end{frame}

\begin{frame}
\only<1>{
\begin{QQuestion}{AD216}{Sie wollen die Resonanzfrequenz eines Schwingkreises verringern. Welche der folgenden Maßnahmen ist geeignet?}{Kleineren Spulenwert verwenden}
{Spule auseinanderziehen}
{Kleineren Kondensatorwert verwenden}
{Spule zusammenschieben}
\end{QQuestion}

}
\only<2>{
\begin{QQuestion}{AD216}{Sie wollen die Resonanzfrequenz eines Schwingkreises verringern. Welche der folgenden Maßnahmen ist geeignet?}{Kleineren Spulenwert verwenden}
{Spule auseinanderziehen}
{Kleineren Kondensatorwert verwenden}
{\textbf{\textcolor{DARCgreen}{Spule zusammenschieben}}}
\end{QQuestion}

}
\end{frame}

\begin{frame}
\only<1>{
\begin{QQuestion}{AD217}{Sie wollen die Resonanzfrequenz eines Schwingkreises verringern. Welche der folgenden Maßnahmen ist geeignet?}{Kleineren Spulenwert verwenden}
{Spule auseinanderziehen}
{Kleineren Kondensatorwert verwenden}
{Ferritkern in die Spule einführen}
\end{QQuestion}

}
\only<2>{
\begin{QQuestion}{AD217}{Sie wollen die Resonanzfrequenz eines Schwingkreises verringern. Welche der folgenden Maßnahmen ist geeignet?}{Kleineren Spulenwert verwenden}
{Spule auseinanderziehen}
{Kleineren Kondensatorwert verwenden}
{\textbf{\textcolor{DARCgreen}{Ferritkern in die Spule einführen}}}
\end{QQuestion}

}
\end{frame}

\begin{frame}
\only<1>{
\begin{PQuestion}{AD218}{Wie verändert sich die Frequenz des Schwingkreises in der folgenden Schaltung, wenn das Potentiometer mehr in Richtung X gedreht wird?}{Die Frequenz des Schwingkreises ändert sich nicht.}
{Die Frequenz des Schwingkreises sinkt.}
{Die Frequenz des Schwingkreises steigt.}
{Die Frequenz sinkt zunächst und steigt dann stark an.}
{\DARCimage{1.0\linewidth}{752include}}\end{PQuestion}

}
\only<2>{
\begin{PQuestion}{AD218}{Wie verändert sich die Frequenz des Schwingkreises in der folgenden Schaltung, wenn das Potentiometer mehr in Richtung X gedreht wird?}{Die Frequenz des Schwingkreises ändert sich nicht.}
{Die Frequenz des Schwingkreises sinkt.}
{\textbf{\textcolor{DARCgreen}{Die Frequenz des Schwingkreises steigt.}}}
{Die Frequenz sinkt zunächst und steigt dann stark an.}
{\DARCimage{1.0\linewidth}{752include}}\end{PQuestion}

}
\end{frame}

\begin{frame}
\only<1>{
\begin{PQuestion}{AD205}{Welche der nachfolgenden Beschreibungen trifft auf diese Schaltung zu und wie nennt man sie? }{Es handelt sich um eine Bandsperre. Frequenzen oberhalb der oberen Grenzfrequenz und Frequenzen unterhalb der unteren Grenzfrequenz werden durchgelassen. Sie bedämpft nur einen bestimmten Frequenzbereich.}
{Es handelt sich um einen Hochpass. Frequenzen unterhalb der Grenzfrequenz werden bedämpft, oberhalb der Grenzfrequenz durchgelassen.}
{Es handelt sich um einen Tiefpass. Frequenzen oberhalb der Grenzfrequenz werden bedämpft, unterhalb der Grenzfrequenz durchgelassen.}
{Es handelt sich um einen Bandpass. Frequenzen oberhalb der oberen Grenzfrequenz und Frequenzen unterhalb der unteren Grenzfrequenz werden bedämpft. Er lässt nur einen bestimmten Frequenzbereich passieren.}
{\DARCimage{1.0\linewidth}{785include}}\end{PQuestion}

}
\only<2>{
\begin{PQuestion}{AD205}{Welche der nachfolgenden Beschreibungen trifft auf diese Schaltung zu und wie nennt man sie? }{Es handelt sich um eine Bandsperre. Frequenzen oberhalb der oberen Grenzfrequenz und Frequenzen unterhalb der unteren Grenzfrequenz werden durchgelassen. Sie bedämpft nur einen bestimmten Frequenzbereich.}
{Es handelt sich um einen Hochpass. Frequenzen unterhalb der Grenzfrequenz werden bedämpft, oberhalb der Grenzfrequenz durchgelassen.}
{Es handelt sich um einen Tiefpass. Frequenzen oberhalb der Grenzfrequenz werden bedämpft, unterhalb der Grenzfrequenz durchgelassen.}
{\textbf{\textcolor{DARCgreen}{Es handelt sich um einen Bandpass. Frequenzen oberhalb der oberen Grenzfrequenz und Frequenzen unterhalb der unteren Grenzfrequenz werden bedämpft. Er lässt nur einen bestimmten Frequenzbereich passieren.}}}
{\DARCimage{1.0\linewidth}{785include}}\end{PQuestion}

}
\end{frame}

\begin{frame}
\only<1>{
\begin{PQuestion}{AD219}{Wie groß ist die Bandbreite in dem dargestellten Diagramm bei \qty{-60}{\decibel}?}{Etwa \qty{6,0}{\kHz}}
{Etwa \qty{6,5}{\kHz}}
{Etwa \qty{4,0}{\kHz}}
{Etwa \qty{2,5}{\kHz}}
{\DARCimage{1.0\linewidth}{38include}}\end{PQuestion}

}
\only<2>{
\begin{PQuestion}{AD219}{Wie groß ist die Bandbreite in dem dargestellten Diagramm bei \qty{-60}{\decibel}?}{Etwa \qty{6,0}{\kHz}}
{Etwa \qty{6,5}{\kHz}}
{\textbf{\textcolor{DARCgreen}{Etwa \qty{4,0}{\kHz}}}}
{Etwa \qty{2,5}{\kHz}}
{\DARCimage{1.0\linewidth}{38include}}\end{PQuestion}

}
\end{frame}

\begin{frame}
\only<1>{
\begin{QQuestion}{AD221}{Ein Quarzfilter mit einer \qty{3}{\decibel}-Bandbreite von \qty{2,7}{\kHz} eignet sich besonders zur Verwendung in einem Sendeempfänger für~...}{FM.}
{AM.}
{SSB.}
{CW.}
\end{QQuestion}

}
\only<2>{
\begin{QQuestion}{AD221}{Ein Quarzfilter mit einer \qty{3}{\decibel}-Bandbreite von \qty{2,7}{\kHz} eignet sich besonders zur Verwendung in einem Sendeempfänger für~...}{FM.}
{AM.}
{\textbf{\textcolor{DARCgreen}{SSB.}}}
{CW.}
\end{QQuestion}

}
\end{frame}

\begin{frame}
\only<1>{
\begin{QQuestion}{AD222}{Ein Quarzfilter mit einer \qty{3}{\decibel}-Bandbreite von \qty{500}{\Hz} eignet sich besonders zur Verwendung in einem Sendeempfänger für~...}{SSB.}
{CW.}
{AM.}
{FM.}
\end{QQuestion}

}
\only<2>{
\begin{QQuestion}{AD222}{Ein Quarzfilter mit einer \qty{3}{\decibel}-Bandbreite von \qty{500}{\Hz} eignet sich besonders zur Verwendung in einem Sendeempfänger für~...}{SSB.}
{\textbf{\textcolor{DARCgreen}{CW.}}}
{AM.}
{FM.}
\end{QQuestion}

}
\end{frame}

\begin{frame}
\only<1>{
\begin{QQuestion}{AD220}{Wie ergibt sich die Bandbreite $B$ eines Parallelschwingkreises aus der Resonanzkurve?}{Die Bandbreite ergibt sich aus der Differenz der beiden Frequenzen, bei denen die Spannung auf den 0,7-fachen Wert gegenüber der maximalen Spannung bei der Resonanzfrequenz abgesunken ist.}
{Die Bandbreite ergibt sich aus der Differenz der beiden Frequenzen, bei denen die Spannung auf den 0,5-fachen Wert gegenüber der maximalen Spannung bei der Resonanzfrequenz abgesunken ist.}
{Die Bandbreite ergibt sich aus der Multiplikation der Resonanzfrequenz mit dem Faktor 0,5.}
{Die Bandbreite ergibt sich aus der Multiplikation der Resonanzfrequenz mit dem Faktor 0,7.}
\end{QQuestion}

}
\only<2>{
\begin{QQuestion}{AD220}{Wie ergibt sich die Bandbreite $B$ eines Parallelschwingkreises aus der Resonanzkurve?}{\textbf{\textcolor{DARCgreen}{Die Bandbreite ergibt sich aus der Differenz der beiden Frequenzen, bei denen die Spannung auf den 0,7-fachen Wert gegenüber der maximalen Spannung bei der Resonanzfrequenz abgesunken ist.}}}
{Die Bandbreite ergibt sich aus der Differenz der beiden Frequenzen, bei denen die Spannung auf den 0,5-fachen Wert gegenüber der maximalen Spannung bei der Resonanzfrequenz abgesunken ist.}
{Die Bandbreite ergibt sich aus der Multiplikation der Resonanzfrequenz mit dem Faktor 0,5.}
{Die Bandbreite ergibt sich aus der Multiplikation der Resonanzfrequenz mit dem Faktor 0,7.}
\end{QQuestion}

}
\end{frame}

\begin{frame}
\only<1>{
\begin{QQuestion}{AD223}{Welche Bandbreite $B$ hat die Reihenschaltung einer Spule von \qty{100}{\micro\H} mit einem Kondensator von \qty{0,01}{\micro\F} und einem Widerstand von \qty{10}{\ohm}?}{\qty{1,59}{\kHz}}
{\qty{159}{\kHz}}
{\qty{15,9}{\kHz}}
{\qty{159}{\Hz}}
\end{QQuestion}

}
\only<2>{
\begin{QQuestion}{AD223}{Welche Bandbreite $B$ hat die Reihenschaltung einer Spule von \qty{100}{\micro\H} mit einem Kondensator von \qty{0,01}{\micro\F} und einem Widerstand von \qty{10}{\ohm}?}{\qty{1,59}{\kHz}}
{\qty{159}{\kHz}}
{\textbf{\textcolor{DARCgreen}{\qty{15,9}{\kHz}}}}
{\qty{159}{\Hz}}
\end{QQuestion}

}
\end{frame}

\begin{frame}
\frametitle{Lösungsweg}
\begin{itemize}
  \item gegeben: $L = 100µH$
  \item gegeben: $C = 0,01µF$
  \item gegeben: $R_S = 10Ω$
  \item gesucht: $B$
  \end{itemize}
    \pause
    $B = \frac{R_S}{2\cdot \pi \cdot L} = \frac{10Ω}{2\cdot \pi \cdot 100µH} = 15,9kHz$



\end{frame}

\begin{frame}
\only<1>{
\begin{QQuestion}{AD225}{Welchen Gütefaktor $Q$ hat die Reihenschaltung einer Spule von \qty{100}{\micro\H} mit einem Kondensator von \qty{0,01}{\micro\F} und einem Widerstand von \qty{10}{\ohm}?}{10}
{1}
{0,1}
{100}
\end{QQuestion}

}
\only<2>{
\begin{QQuestion}{AD225}{Welchen Gütefaktor $Q$ hat die Reihenschaltung einer Spule von \qty{100}{\micro\H} mit einem Kondensator von \qty{0,01}{\micro\F} und einem Widerstand von \qty{10}{\ohm}?}{\textbf{\textcolor{DARCgreen}{10}}}
{1}
{0,1}
{100}
\end{QQuestion}

}
\end{frame}

\begin{frame}
\frametitle{Lösungsweg}
\begin{itemize}
  \item gegeben: $L = 100µH$
  \item gegeben: $C = 0,01µF$
  \item gegeben: $R_S = 10Ω$
  \item gesucht: $Q$
  \end{itemize}
    \pause
    $f_0 = \frac{1}{2 \cdot \pi \cdot \sqrt{L \cdot C}} = \frac{1}{2 \cdot \pi \cdot \sqrt{100µH \cdot 0,01µF}} = 159,2kHz$
    \pause
    $B$ oder $X_L$ ausrechnen – $B$ haben wir schon vorher ausgerechnet

$B = \frac{R_S}{2\cdot \pi \cdot L} = \frac{10Ω}{2\cdot \pi \cdot 100µH} = 15,92kHz$
    \pause
    $Q = \frac{f_0}{B} = \frac{159,2kHz}{15,92kHz} = 10$



\end{frame}

\begin{frame}
\only<1>{
\begin{QQuestion}{AD224}{Welche Bandbreite $B$ hat die Parallelschaltung einer Spule von \qty{2,2}{\micro\H} mit einem Kondensator von \qty{56}{\pF} und einem Widerstand von \qty{1}{\kohm}?}{\qty{28,4}{\kHz}}
{\qty{28,4}{\MHz}}
{\qty{284}{\kHz}}
{\qty{2,84}{\MHz}}
\end{QQuestion}

}
\only<2>{
\begin{QQuestion}{AD224}{Welche Bandbreite $B$ hat die Parallelschaltung einer Spule von \qty{2,2}{\micro\H} mit einem Kondensator von \qty{56}{\pF} und einem Widerstand von \qty{1}{\kohm}?}{\qty{28,4}{\kHz}}
{\qty{28,4}{\MHz}}
{\qty{284}{\kHz}}
{\textbf{\textcolor{DARCgreen}{\qty{2,84}{\MHz}}}}
\end{QQuestion}

}
\end{frame}

\begin{frame}
\frametitle{Lösungsweg}
\begin{itemize}
  \item gegeben: $L = 2,2µH$
  \item gegeben: $C = 56pF$
  \item gegeben: $R_P = 1kΩ$
  \item gesucht: $B$
  \end{itemize}
    \pause
    $B = \frac{1}{2\cdot \pi \cdot R_P \cdot C} = \frac{1}{2\cdot \pi \cdot 1kΩ \cdot 56pF} = 2,84MHz$



\end{frame}

\begin{frame}
\only<1>{
\begin{QQuestion}{AD226}{Welchen Gütefaktor $Q$ hat die Parallelschaltung einer Spule von \qty{2,2}{\micro\H} mit einem Kondensator von \qty{56}{\pF} und einem Widerstand von \qty{1}{\kohm}?}{15}
{50}
{5}
{0,2}
\end{QQuestion}

}
\only<2>{
\begin{QQuestion}{AD226}{Welchen Gütefaktor $Q$ hat die Parallelschaltung einer Spule von \qty{2,2}{\micro\H} mit einem Kondensator von \qty{56}{\pF} und einem Widerstand von \qty{1}{\kohm}?}{15}
{50}
{\textbf{\textcolor{DARCgreen}{5}}}
{0,2}
\end{QQuestion}

}
\end{frame}

\begin{frame}
\frametitle{Lösungsweg}
\begin{itemize}
  \item gegeben: $L = 2,2µH$
  \item gegeben: $C = 56pF$
  \item gegeben: $R_P = 1kΩ$
  \item gesucht: $Q$
  \end{itemize}
    \pause
    $f_0 = \frac{1}{2 \cdot \pi \cdot \sqrt{L \cdot C}} = \frac{1}{2 \cdot \pi \cdot \sqrt{2,2µH \cdot 56pF}} = 14,34MHz$
    \pause
    $B$ oder $X_L$ ausrechnen – $B$ haben wir schon vorher ausgerechnet

$B = \frac{1}{2\cdot \pi \cdot R_P \cdot C} = \frac{1}{2\cdot \pi \cdot 1kΩ \cdot 56pF} = 2,842MHz$
    \pause
    $Q = \frac{f_0}{B} = \frac{14,34MHz}{2,842MHz} = 5$



\end{frame}

\begin{frame}
\only<1>{
\begin{QQuestion}{AD229}{Welche Kopplung eines Bandfilters wird \glqq kritische Kopplung\grqq{} genannt?}{Die Kopplung, bei der die Ausgangsspannung des Bandfilters das 0{,}707-fache der Eingangsspannung erreicht.}
{Die Kopplung, bei der die Resonanzkurve des Bandfilters ihre größtmögliche Breite hat.}
{Die Kopplung, bei der die Resonanzkurve des Bandfilters eine Welligkeit von \qty{3}{\decibel} (Höcker- zu Sattelspannung) zeigt.}
{Die Kopplung, bei der die Resonanzkurve ihre größte Breite hat und dabei am Resonanzmaximum noch völlig eben ist.}
\end{QQuestion}

}
\only<2>{
\begin{QQuestion}{AD229}{Welche Kopplung eines Bandfilters wird \glqq kritische Kopplung\grqq{} genannt?}{Die Kopplung, bei der die Ausgangsspannung des Bandfilters das 0{,}707-fache der Eingangsspannung erreicht.}
{Die Kopplung, bei der die Resonanzkurve des Bandfilters ihre größtmögliche Breite hat.}
{Die Kopplung, bei der die Resonanzkurve des Bandfilters eine Welligkeit von \qty{3}{\decibel} (Höcker- zu Sattelspannung) zeigt.}
{\textbf{\textcolor{DARCgreen}{Die Kopplung, bei der die Resonanzkurve ihre größte Breite hat und dabei am Resonanzmaximum noch völlig eben ist.}}}
\end{QQuestion}

}
\end{frame}

\begin{frame}
\only<1>{
\begin{PQuestion}{AD227}{Das folgende Bild zeigt ein induktiv gekoppeltes Bandfilter und vier seiner möglichen Übertragungskurven (a bis d). Welche der folgenden Aussagen ist richtig?}{Bei der Kurve c ist die Kopplung loser als bei der Kurve a.}
{Bei der Kurve b ist die Kopplung loser als bei der Kurve c.}
{Bei der Kurve a ist die Kopplung loser als bei der Kurve c.}
{Bei der Kurve b ist die Kopplung loser als bei der Kurve d.}
{\DARCimage{1.0\linewidth}{184include}}\end{PQuestion}

}
\only<2>{
\begin{PQuestion}{AD227}{Das folgende Bild zeigt ein induktiv gekoppeltes Bandfilter und vier seiner möglichen Übertragungskurven (a bis d). Welche der folgenden Aussagen ist richtig?}{\textbf{\textcolor{DARCgreen}{Bei der Kurve c ist die Kopplung loser als bei der Kurve a.}}}
{Bei der Kurve b ist die Kopplung loser als bei der Kurve c.}
{Bei der Kurve a ist die Kopplung loser als bei der Kurve c.}
{Bei der Kurve b ist die Kopplung loser als bei der Kurve d.}
{\DARCimage{1.0\linewidth}{184include}}\end{PQuestion}

}
\end{frame}

\begin{frame}
\only<1>{
\begin{PQuestion}{AD228}{Das folgende Bild zeigt ein typisches ZF-Filter und vier seiner möglichen Übertragungskurven (a bis d). Welche Kurve ergibt sich bei kritischer Kopplung und welche bei überkritischer Kopplung?}{Die Kurve c zeigt kritische, die Kurve b zeigt überkritische Kopplung.}
{Die Kurve a zeigt kritische, die Kurve b zeigt überkritische Kopplung.}
{Die Kurve b zeigt kritische, die Kurve a zeigt überkritische Kopplung.}
{Die Kurve d zeigt kritische, die Kurve c zeigt überkritische Kopplung.}
{\DARCimage{1.0\linewidth}{184include}}\end{PQuestion}

}
\only<2>{
\begin{PQuestion}{AD228}{Das folgende Bild zeigt ein typisches ZF-Filter und vier seiner möglichen Übertragungskurven (a bis d). Welche Kurve ergibt sich bei kritischer Kopplung und welche bei überkritischer Kopplung?}{Die Kurve c zeigt kritische, die Kurve b zeigt überkritische Kopplung.}
{Die Kurve a zeigt kritische, die Kurve b zeigt überkritische Kopplung.}
{\textbf{\textcolor{DARCgreen}{Die Kurve b zeigt kritische, die Kurve a zeigt überkritische Kopplung.}}}
{Die Kurve d zeigt kritische, die Kurve c zeigt überkritische Kopplung.}
{\DARCimage{1.0\linewidth}{184include}}\end{PQuestion}

}
\end{frame}%ENDCONTENT
