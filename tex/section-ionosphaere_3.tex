
\section{Ionosphäre III}
\label{section:ionosphaere_3}
\begin{frame}%STARTCONTENT

\frametitle{Foliensatz in Arbeit}
2024-04-28: Die Inhalte werden noch aufbereitet.

Derzeit sind in diesem Abschnitt nur die Fragen sortiert enthalten.

Für das Selbststudium verweisen wir aktuell auf den Abschnitt Wellenausbreitung im DARC Online Lehrgang (\textcolor{DARCblue}{\faLink~\href{https://www.darc.de/der-club/referate/ajw/lehrgang-te/e09/}{www.darc.de/der-club/referate/ajw/lehrgang-te/e09/}}) für die Prüfung bis Juni 2024. Bis auf die Fragen hat sich an der Thematik nichts geändert. Das Thema war bisher Stoff der Klasse~E und wurde mit der neuen Prüfungsordnung auf alle drei Klassen aufgeteilt.

\end{frame}

\begin{frame}
\frametitle{Solarer Flux}

\only<1>{
\begin{QQuestion}{AH102}{Der solare Flux~F~...}{ist die gemessene Energieausstrahlung der Sonne im GHz-Bereich. Fluxwerte über 100 führen zu einem stark erhöhten Ionisationsgrad in der Ionosphäre und zu einer erheblich verbesserten Fernausbreitung auf den höheren Kurzwellenbändern.}
{ist die gemessene Energieausstrahlung der Sonne im Kurzwellenbereich. Fluxwerte über 60 führen zu einem stark erhöhten Ionisationsgrad in der Ionosphäre und zu einer erheblich verbesserten Fernausbreitung auf den höheren Kurzwellenbändern.}
{wird aus der Sonnenfleckenrelativzahl R abgeleitet und ist ein Indikator für die Aktivität der Sonne. Fluxwerte über 100 führen zu einem stark erhöhten Ionisationsgrad der D-Region und damit zu einer erheblichen Verschlechterung der Fernausbreitung auf den Kurzwellenbändern.}
{wird aus der Sonnenfleckenrelativzahl R abgeleitet und ist ein Indikator für die Aktivität der Sonne. Fluxwerte über 60 führen zu einem stark erhöhten Ionisationsgrad in der Ionosphäre und zu einer erheblich verbesserten Fernausbreitung auf den höheren Kurzwellenbändern.}
\end{QQuestion}

}
\only<2>{
\begin{QQuestion}{AH102}{Der solare Flux~F~...}{\textbf{\textcolor{DARCgreen}{ist die gemessene Energieausstrahlung der Sonne im GHz-Bereich. Fluxwerte über 100 führen zu einem stark erhöhten Ionisationsgrad in der Ionosphäre und zu einer erheblich verbesserten Fernausbreitung auf den höheren Kurzwellenbändern.}}}
{ist die gemessene Energieausstrahlung der Sonne im Kurzwellenbereich. Fluxwerte über 60 führen zu einem stark erhöhten Ionisationsgrad in der Ionosphäre und zu einer erheblich verbesserten Fernausbreitung auf den höheren Kurzwellenbändern.}
{wird aus der Sonnenfleckenrelativzahl R abgeleitet und ist ein Indikator für die Aktivität der Sonne. Fluxwerte über 100 führen zu einem stark erhöhten Ionisationsgrad der D-Region und damit zu einer erheblichen Verschlechterung der Fernausbreitung auf den Kurzwellenbändern.}
{wird aus der Sonnenfleckenrelativzahl R abgeleitet und ist ein Indikator für die Aktivität der Sonne. Fluxwerte über 60 führen zu einem stark erhöhten Ionisationsgrad in der Ionosphäre und zu einer erheblich verbesserten Fernausbreitung auf den höheren Kurzwellenbändern.}
\end{QQuestion}

}
\end{frame}

\begin{frame}
\frametitle{Wellenausbreitung an D- und E-Schicht}
\end{frame}

\begin{frame}
\only<1>{
\begin{QQuestion}{AH103}{In welcher Höhe befindet sich die für die Fernausbreitung wichtige D-Region? Sie befindet sich in ungefähr~...}{\qtyrange{130}{200}{\km} Höhe.}
{\qtyrange{9}{130}{\km} Höhe.}
{\qtyrange{50}{90}{\km} Höhe.}
{\qtyrange{250}{450}{\km} Höhe.}
\end{QQuestion}

}
\only<2>{
\begin{QQuestion}{AH103}{In welcher Höhe befindet sich die für die Fernausbreitung wichtige D-Region? Sie befindet sich in ungefähr~...}{\qtyrange{130}{200}{\km} Höhe.}
{\qtyrange{9}{130}{\km} Höhe.}
{\textbf{\textcolor{DARCgreen}{\qtyrange{50}{90}{\km} Höhe.}}}
{\qtyrange{250}{450}{\km} Höhe.}
\end{QQuestion}

}
\end{frame}

\begin{frame}
\only<1>{
\begin{QQuestion}{AH104}{In welcher Höhe befindet sich die für die Fernausbreitung wichtige E-Region? Sie befindet sich in ungefähr~...}{\qtyrange{90}{130}{\km} Höhe.}
{\qtyrange{50}{90}{\km} Höhe.}
{\qtyrange{130}{200}{\km} Höhe.}
{\qtyrange{250}{450}{\km} Höhe.}
\end{QQuestion}

}
\only<2>{
\begin{QQuestion}{AH104}{In welcher Höhe befindet sich die für die Fernausbreitung wichtige E-Region? Sie befindet sich in ungefähr~...}{\textbf{\textcolor{DARCgreen}{\qtyrange{90}{130}{\km} Höhe.}}}
{\qtyrange{50}{90}{\km} Höhe.}
{\qtyrange{130}{200}{\km} Höhe.}
{\qtyrange{250}{450}{\km} Höhe.}
\end{QQuestion}

}
\end{frame}

\begin{frame}
\only<1>{
\begin{QQuestion}{AH202}{Welches dieser Frequenzbänder kann im Sonnenfleckenminimum am ehesten für tägliche Weitverkehrsverbindungen verwendet werden?}{\qty{3,5}{\MHz}~(\qty{80}{\m}-Band)}
{\qty{14}{\MHz}~(\qty{20}{\m}-Band)}
{\qty{28}{\MHz}~(\qty{10}{\m}-Band)}
{\qty{1,8}{\MHz}~(\qty{160}{\m}-Band)}
\end{QQuestion}

}
\only<2>{
\begin{QQuestion}{AH202}{Welches dieser Frequenzbänder kann im Sonnenfleckenminimum am ehesten für tägliche Weitverkehrsverbindungen verwendet werden?}{\qty{3,5}{\MHz}~(\qty{80}{\m}-Band)}
{\textbf{\textcolor{DARCgreen}{\qty{14}{\MHz}~(\qty{20}{\m}-Band)}}}
{\qty{28}{\MHz}~(\qty{10}{\m}-Band)}
{\qty{1,8}{\MHz}~(\qty{160}{\m}-Band)}
\end{QQuestion}

}
\end{frame}%ENDCONTENT
