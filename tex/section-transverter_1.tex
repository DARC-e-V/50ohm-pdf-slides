
\section{Konverter und Transverter}
\label{section:transverter_1}
\begin{frame}%STARTCONTENT

\frametitle{Konverter}
\begin{itemize}
  \item Signale auf einem Frequenzband werden in ein anderes Frequenzband umgesetzt
  \item z.B. wird ein 2m-Signal im Empfang als ein 70cm-Signal ausgesendet
  \item Signal wird nur in eine Richtung umgewandelt
  \item Im Grunde ein einfacher Mischer
  \end{itemize}
\end{frame}

\begin{frame}
\only<1>{
\begin{PQuestion}{EF504}{Was stellt die nachfolgende Schaltung dar?}{Einen \qty{13}{\cm}-Transverter zur Vorschaltung vor einen VHF-Sender}
{Einen \qty{13}{\cm}-Konverter für einen VHF-Sender}
{Einen \qty{13}{\cm}-Transverter zur Vorschaltung vor einen VHF-Empfänger}
{Teile eines I/Q-Mischers für das \qty{13}{\cm}-Band}
{\DARCimage{1.0\linewidth}{651include}}\end{PQuestion}

}
\only<2>{
\begin{PQuestion}{EF504}{Was stellt die nachfolgende Schaltung dar?}{Einen \qty{13}{\cm}-Transverter zur Vorschaltung vor einen VHF-Sender}
{\textbf{\textcolor{DARCgreen}{Einen \qty{13}{\cm}-Konverter für einen VHF-Sender}}}
{Einen \qty{13}{\cm}-Transverter zur Vorschaltung vor einen VHF-Empfänger}
{Teile eines I/Q-Mischers für das \qty{13}{\cm}-Band}
{\DARCimage{1.0\linewidth}{651include}}\end{PQuestion}

}

\end{frame}

\begin{frame}
\only<1>{
\begin{QQuestion}{EF505}{Warum soll der Lokaloszillator (XO) in einem Transverter für Satellitenbetrieb mit einer Uplinkfrequenz von \qty{2,4}{\GHz} temperaturstabilisiert oder durch ein höherwertiges Frequenznormal synchronisiert sein?}{Da die Frequenz des Oszillators für die Sendefrequenz heruntergemischt wird, verringert sich dadurch die Abweichung. }
{Da die Frequenz des Oszillators für die Sendefrequenz vervielfacht wird, vervielfacht sich auch die Abweichung, die für SSB-Betrieb zu groß wäre. }
{Da die Frequenz des Oszillators für die Sendefrequenz vervielfacht wird, nehmen die Nebenaussendungen mit zunehmender Frequenzabweichung zu. }
{Da die Frequenz des Oszillators für die Sendefrequenz heruntergemischt wird, verringert sich bei zunehmender Frequenzabweichung der Modulationsgrad. }
\end{QQuestion}

}
\only<2>{
\begin{QQuestion}{EF505}{Warum soll der Lokaloszillator (XO) in einem Transverter für Satellitenbetrieb mit einer Uplinkfrequenz von \qty{2,4}{\GHz} temperaturstabilisiert oder durch ein höherwertiges Frequenznormal synchronisiert sein?}{Da die Frequenz des Oszillators für die Sendefrequenz heruntergemischt wird, verringert sich dadurch die Abweichung. }
{\textbf{\textcolor{DARCgreen}{Da die Frequenz des Oszillators für die Sendefrequenz vervielfacht wird, vervielfacht sich auch die Abweichung, die für SSB-Betrieb zu groß wäre. }}}
{Da die Frequenz des Oszillators für die Sendefrequenz vervielfacht wird, nehmen die Nebenaussendungen mit zunehmender Frequenzabweichung zu. }
{Da die Frequenz des Oszillators für die Sendefrequenz heruntergemischt wird, verringert sich bei zunehmender Frequenzabweichung der Modulationsgrad. }
\end{QQuestion}

}
\end{frame}

\begin{frame}
\frametitle{Transverter}
\begin{itemize}
  \item Beim Transverter funktioniert die Umsetzung in beide Richtungen
  \item Die Umsetzung erfolgt auch hier durch Mischung
  \end{itemize}
\end{frame}

\begin{frame}
\only<1>{
\begin{QQuestion}{EF501}{Welche der nachfolgenden Antworten trifft für die Wirkungsweise eines Transverters zu? Ein Transverter setzt...}{beim Empfangen z.~B. ein \qty{70}{\cm}-Signal in das \qty{10}{\m}-Band und beim Senden das \qty{10}{\m}-Sendesignal auf das \qty{70}{\cm}-Band um.}
{sowohl beim Senden als auch beim Empfangen z.~B. ein \qty{70}{\cm}-Signal in das \qty{10}{\m}-Band um.}
{sowohl beim Senden als auch beim Empfangen z.~B. ein frequenzmoduliertes Signal in ein amplitudenmoduliertes Signal um.}
{sowohl beim Senden als auch beim Empfangen z.~B. ein DMR-Signal in ein D-Star-Signal um.}
\end{QQuestion}

}
\only<2>{
\begin{QQuestion}{EF501}{Welche der nachfolgenden Antworten trifft für die Wirkungsweise eines Transverters zu? Ein Transverter setzt...}{\textbf{\textcolor{DARCgreen}{beim Empfangen z.~B. ein \qty{70}{\cm}-Signal in das \qty{10}{\m}-Band und beim Senden das \qty{10}{\m}-Sendesignal auf das \qty{70}{\cm}-Band um.}}}
{sowohl beim Senden als auch beim Empfangen z.~B. ein \qty{70}{\cm}-Signal in das \qty{10}{\m}-Band um.}
{sowohl beim Senden als auch beim Empfangen z.~B. ein frequenzmoduliertes Signal in ein amplitudenmoduliertes Signal um.}
{sowohl beim Senden als auch beim Empfangen z.~B. ein DMR-Signal in ein D-Star-Signal um.}
\end{QQuestion}

}
\end{frame}

\begin{frame}
\only<1>{
\begin{QQuestion}{EF502}{Durch welchen Vorgang setzt ein Transverter einen Frequenzbereich in einen anderen um?}{Durch Rückkopplung}
{Durch Vervielfachung}
{Durch Frequenzteilung}
{Durch Mischung}
\end{QQuestion}

}
\only<2>{
\begin{QQuestion}{EF502}{Durch welchen Vorgang setzt ein Transverter einen Frequenzbereich in einen anderen um?}{Durch Rückkopplung}
{Durch Vervielfachung}
{Durch Frequenzteilung}
{\textbf{\textcolor{DARCgreen}{Durch Mischung}}}
\end{QQuestion}

}
\end{frame}

\begin{frame}
\only<1>{
\begin{PQuestion}{EF503}{Was stellt folgendes Blockschaltbild dar?}{Einen Transverter für das \qty{2}{\m}-Band}
{Einen Empfangskonverter für das \qty{2}{\m}-Band}
{Einen Vorverstärker für das \qty{10}{\m}-Band}
{Einen Transceiver für das \qty{10}{\m}-Band}
{\DARCimage{1.0\linewidth}{94include}}\end{PQuestion}

}
\only<2>{
\begin{PQuestion}{EF503}{Was stellt folgendes Blockschaltbild dar?}{\textbf{\textcolor{DARCgreen}{Einen Transverter für das \qty{2}{\m}-Band}}}
{Einen Empfangskonverter für das \qty{2}{\m}-Band}
{Einen Vorverstärker für das \qty{10}{\m}-Band}
{Einen Transceiver für das \qty{10}{\m}-Band}
{\DARCimage{1.0\linewidth}{94include}}\end{PQuestion}

}

\end{frame}

\begin{frame}
\frametitle{Lösungsweg}
Frequenz des Generators wird ver-3-facht: $38,666MHz \cdot 3 = 116MHz$
\begin{columns}
    \begin{column}{0.48\textwidth}
    \emph{TX Weg}

\begin{itemize}
  \item Die \qtyrange{28}{30}{\mega\hertz} vom TRX werden mit \qty{116}{\mega\hertz} gemischt
  \item Das Signal kann 80-90MHz oder \qtyrange{144}{146}{\mega\hertz} sein
  \end{itemize}

    \end{column}
   \begin{column}{0.48\textwidth}
       
\begin{figure}
    \DARCimage{0.85\linewidth}{843include}
    \caption{\scriptsize Transverter im TX-Pfad}
    \label{e_transverter_tx}
\end{figure}


   \end{column}
\end{columns}

\end{frame}

\begin{frame}
\begin{columns}
    \begin{column}{0.48\textwidth}
    \emph{RX Weg}

\begin{itemize}
  \item Das Antennensignal wird mit \qty{116}{\mega\hertz} gemischt und es kommen \qtyrange{28}{30}{\mega\hertz} raus
  \item Das Antennensignal liegt somit u.a. bei \qtyrange{144}{146}{\mega\hertz}
  \item $\rightarrow$ Es ist nur die Antwort mit \qty{2}{\metre} und der Transverter richtig
  \end{itemize}

    \end{column}
   \begin{column}{0.48\textwidth}
       
\begin{figure}
    \DARCimage{0.85\linewidth}{842include}
    \caption{\scriptsize Transverter im RX-Pfad}
    \label{e_transverter_rx}
\end{figure}


   \end{column}
\end{columns}

\end{frame}

\begin{frame}
\frametitle{Frequenzstabilität}
\begin{itemize}
  \item Konverter und Transverter sollten mit frequenzstabilen Oszillatoren gebaut werden
  \item Weicht die Frequenz ab, ist die Ausgangsfrequenz auch abweichend
  \end{itemize}
\end{frame}

\begin{frame}
\begin{columns}
    \begin{column}{0.48\textwidth}
    \begin{itemize}
  \item Grafik aus vorheriger Frage
  \item Aus \qty{10}{\mega\hertz} werden \qty{2,256}{\giga\hertz}, also 225,6 Vervielfachung
  \item Statt \qty{10}{\mega\hertz} erzeugt der Oszillator aufgrund eines Fehlers \qty{10,01}{\mega\hertz}
  \item \qty{10,01}{\mega\hertz} $\cdot$ 225,6 = \qty{2,258256}{\giga\hertz}
  \item Mischer: \qty{144}{\mega\hertz} + \qty{2,258256}{\giga\hertz} = \qty{2,402256}{\giga\hertz} $\rightarrow$ \qty{2,256}{\mega\hertz} daneben
  \end{itemize}

    \end{column}
   \begin{column}{0.48\textwidth}
       
\begin{figure}
    \DARCimage{0.85\linewidth}{651include}
    \caption{\scriptsize Konverter für das 13cm-Band}
    \label{e_konverter_13cm}
\end{figure}


   \end{column}
\end{columns}

\end{frame}%ENDCONTENT
