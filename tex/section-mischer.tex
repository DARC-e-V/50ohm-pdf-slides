
\section{Mischer}
\label{section:mischer}
\begin{frame}%STARTCONTENT

\begin{columns}
    \begin{column}{0.48\textwidth}
    \begin{itemize}
  \item Beim Mischen von zwei Eingangs-Frequenzen entstehen immer zwei Ausgangs-Frequenzen
  \end{itemize}
\begin{equation}f_\text{A1} = f_\text{E1} + f_\text{E2}\end{equation}

\begin{equation}f_\text{A2} = |f_\text{E1} - f_\text{E2}|\end{equation}


    \end{column}
   \begin{column}{0.48\textwidth}
       
\begin{figure}
    \DARCimage{0.85\linewidth}{102include}
    \caption{\scriptsize Mischer mit zwei Eingangsfrequenzen}
    \label{e_mischer}
\end{figure}


   \end{column}
\end{columns}

\end{frame}

\begin{frame}
\only<1>{
\begin{PQuestion}{EF201}{Welche wesentlichen Ausgangsfrequenzen erzeugt die in der Abbildung dargestellte Stufe?}{\qty{42}{\MHz} und \qty{63,4}{\MHz}}
{\qty{10,7}{\MHz} und \qty{52,7}{\MHz}}
{\qty{21}{\MHz} und \qty{63,4}{\MHz}}
{\qty{21,4}{\MHz} und \qty{105,4}{\MHz}}
{\DARCimage{0.75\linewidth}{102include}}\end{PQuestion}

}
\only<2>{
\begin{PQuestion}{EF201}{Welche wesentlichen Ausgangsfrequenzen erzeugt die in der Abbildung dargestellte Stufe?}{\qty{42}{\MHz} und \qty{63,4}{\MHz}}
{\textbf{\textcolor{DARCgreen}{\qty{10,7}{\MHz} und \qty{52,7}{\MHz}}}}
{\qty{21}{\MHz} und \qty{63,4}{\MHz}}
{\qty{21,4}{\MHz} und \qty{105,4}{\MHz}}
{\DARCimage{0.75\linewidth}{102include}}\end{PQuestion}

}
\end{frame}

\begin{frame}
\frametitle{Lösungsweg}
\begin{itemize}
  \item Gegeben: $f_{E1} = 21MHz$, $f_{E2} = 31,7MHz$
  \item Lösung:
  \end{itemize}
\begin{equation}\begin{split}f_{A1} &= 21MHz + 31,7MHz\\ &= 52,7MHz\end{split}\end{equation}
\begin{equation}\begin{split}f_{A2} &= |21MHz - 31,7MHz|\\ &= |-10,7MHz|\\ &= 10,7MHz\end{split}\end{equation}

\end{frame}

\begin{frame}
\only<1>{
\begin{QQuestion}{EF202}{Einem Mischer werden die Frequenzen \qty{28}{\MHz} und \qty{38,7}{\MHz} zugeführt. Welche Mischfrequenzen werden hauptsächlich erzeugt?}{\qty{45,3}{\MHz} und \qty{88,1}{\MHz}}
{\qty{17,3}{\MHz} und \qty{49,4}{\MHz}}
{\qty{56}{\MHz} und \qty{77,4}{\MHz}}
{\qty{10,7}{\MHz} und \qty{66,7}{\MHz}}
\end{QQuestion}

}
\only<2>{
\begin{QQuestion}{EF202}{Einem Mischer werden die Frequenzen \qty{28}{\MHz} und \qty{38,7}{\MHz} zugeführt. Welche Mischfrequenzen werden hauptsächlich erzeugt?}{\qty{45,3}{\MHz} und \qty{88,1}{\MHz}}
{\qty{17,3}{\MHz} und \qty{49,4}{\MHz}}
{\qty{56}{\MHz} und \qty{77,4}{\MHz}}
{\textbf{\textcolor{DARCgreen}{\qty{10,7}{\MHz} und \qty{66,7}{\MHz}}}}
\end{QQuestion}

}
\end{frame}

\begin{frame}
\only<1>{
\begin{QQuestion}{EF203}{Welches sind die erwünschten Produkte, die bei der Mischung der Frequenzen \qty{30}{\MHz} und \qty{39}{\MHz} am Ausgang des Mischers entstehen?}{\qty{9}{\MHz} und \qty{69}{\MHz}}
{\qty{9}{\MHz} und \qty{39}{\MHz}}
{\qty{30}{\MHz} und \qty{39}{\MHz}}
{\qty{39}{\MHz} und \qty{69}{\MHz}}
\end{QQuestion}

}
\only<2>{
\begin{QQuestion}{EF203}{Welches sind die erwünschten Produkte, die bei der Mischung der Frequenzen \qty{30}{\MHz} und \qty{39}{\MHz} am Ausgang des Mischers entstehen?}{\textbf{\textcolor{DARCgreen}{\qty{9}{\MHz} und \qty{69}{\MHz}}}}
{\qty{9}{\MHz} und \qty{39}{\MHz}}
{\qty{30}{\MHz} und \qty{39}{\MHz}}
{\qty{39}{\MHz} und \qty{69}{\MHz}}
\end{QQuestion}

}
\end{frame}

\begin{frame}
\only<1>{
\begin{QQuestion}{EF204}{Einem Mischer werden die Frequenzen \qty{136}{\MHz} und \qty{145}{\MHz} zugeführt. Welche Mischfrequenzen werden hauptsächlich erzeugt?}{\qty{9}{\MHz} und \qty{281}{\MHz}}
{\qty{127}{\MHz} und \qty{154}{\MHz}}
{\qty{272}{\MHz} und \qty{290}{\MHz}}
{\qty{118}{\MHz} und \qty{163}{\MHz}}
\end{QQuestion}

}
\only<2>{
\begin{QQuestion}{EF204}{Einem Mischer werden die Frequenzen \qty{136}{\MHz} und \qty{145}{\MHz} zugeführt. Welche Mischfrequenzen werden hauptsächlich erzeugt?}{\textbf{\textcolor{DARCgreen}{\qty{9}{\MHz} und \qty{281}{\MHz}}}}
{\qty{127}{\MHz} und \qty{154}{\MHz}}
{\qty{272}{\MHz} und \qty{290}{\MHz}}
{\qty{118}{\MHz} und \qty{163}{\MHz}}
\end{QQuestion}

}
\end{frame}

\begin{frame}
\only<1>{
\begin{QQuestion}{EF205}{Welches sind die erwünschten Produkte, die bei der Mischung der Frequenzen \qty{136}{\MHz} und \qty{145}{\MHz} am Ausgang des Mischers entstehen?}{\qty{9}{\MHz} und \qty{281}{\MHz}}
{\qty{127}{\MHz} und \qty{154}{\MHz}}
{\qty{272}{\MHz} und \qty{290}{\MHz}}
{\qty{154}{\MHz} und \qty{281}{\MHz}}
\end{QQuestion}

}
\only<2>{
\begin{QQuestion}{EF205}{Welches sind die erwünschten Produkte, die bei der Mischung der Frequenzen \qty{136}{\MHz} und \qty{145}{\MHz} am Ausgang des Mischers entstehen?}{\textbf{\textcolor{DARCgreen}{\qty{9}{\MHz} und \qty{281}{\MHz}}}}
{\qty{127}{\MHz} und \qty{154}{\MHz}}
{\qty{272}{\MHz} und \qty{290}{\MHz}}
{\qty{154}{\MHz} und \qty{281}{\MHz}}
\end{QQuestion}

}
\end{frame}

\begin{frame}
\frametitle{Schirmung}
\begin{itemize}
  \item In der Regel ist nur eine von den beiden Frequenzen erwünscht
  \item Die unerwünschte Frequenz wird durch Filter beseitigt
  \item Bis dahin sollte diese Frequenz nicht außerhalb der Mischerstufe zu detektieren sein
  \item Deshalb wird die Mischerstufe vor Abstrahlungen gut geschirmt, z.B. mit einem Metallgehäuse
  \end{itemize}
\end{frame}

\begin{frame}
\only<1>{
\begin{QQuestion}{EF206}{Wie sollte eine Mischstufe beschaffen sein, um unerwünschte Abstrahlungen zu vermeiden?}{Sie sollte möglichst lose mit dem VFO gekoppelt sein. }
{Sie sollte niederfrequent entkoppelt werden.}
{Sie sollte nicht geerdet werden.}
{Sie sollte gut abgeschirmt sein.}
\end{QQuestion}

}
\only<2>{
\begin{QQuestion}{EF206}{Wie sollte eine Mischstufe beschaffen sein, um unerwünschte Abstrahlungen zu vermeiden?}{Sie sollte möglichst lose mit dem VFO gekoppelt sein. }
{Sie sollte niederfrequent entkoppelt werden.}
{Sie sollte nicht geerdet werden.}
{\textbf{\textcolor{DARCgreen}{Sie sollte gut abgeschirmt sein.}}}
\end{QQuestion}

}
\end{frame}%ENDCONTENT
