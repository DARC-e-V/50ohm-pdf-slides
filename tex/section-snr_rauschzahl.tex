
\section{SNR und Rauschzahl}
\label{section:snr_rauschzahl}
\begin{frame}%STARTCONTENT

\only<1>{
\begin{QQuestion}{AF227}{Was bedeutet Signal-Rausch-Abstand (SNR) bei einem Empfänger?}{Er gibt an, in welchem Verhältnis das Rauschsignal stärker ist als das Nutzsignal.}
{Er gibt an, in welchem Verhältnis das Nutzsignal stärker ist als das Rauschsignal.}
{Es ist der Frequenzabstand zwischen Empfangssignal und Störsignal.}
{Es ist der Abstand zwischen Empfangsfrequenz und Spiegelfrequenz.}
\end{QQuestion}

}
\only<2>{
\begin{QQuestion}{AF227}{Was bedeutet Signal-Rausch-Abstand (SNR) bei einem Empfänger?}{Er gibt an, in welchem Verhältnis das Rauschsignal stärker ist als das Nutzsignal.}
{\textbf{\textcolor{DARCgreen}{Er gibt an, in welchem Verhältnis das Nutzsignal stärker ist als das Rauschsignal.}}}
{Es ist der Frequenzabstand zwischen Empfangssignal und Störsignal.}
{Es ist der Abstand zwischen Empfangsfrequenz und Spiegelfrequenz.}
\end{QQuestion}

}
\end{frame}

\begin{frame}
\only<1>{
\begin{QQuestion}{AF228}{Was bedeutet die Rauschzahl von \qty{1,8}{\decibel} bei einem UHF-Vorverstärker?}{Das Rauschen des Ausgangssignals ist um \qty{1,8}{\decibel} niedriger als das Rauschen des Eingangssignals.}
{Das Ausgangssignal des Vorverstärkers hat ein um \qty{1,8}{\decibel} höheres Signal-Rausch-Verhältnis als das Eingangssignal.}
{Das Ausgangssignal des Vorverstärkers hat ein um \qty{1,8}{\decibel} geringeres Signal-Rausch-Verhältnis als das Eingangssignal.}
{Die Verstärkung des Nutzsignals beträgt \qty{1,8}{\decibel}, die Rauschleistung bleibt unverändert.}
\end{QQuestion}

}
\only<2>{
\begin{QQuestion}{AF228}{Was bedeutet die Rauschzahl von \qty{1,8}{\decibel} bei einem UHF-Vorverstärker?}{Das Rauschen des Ausgangssignals ist um \qty{1,8}{\decibel} niedriger als das Rauschen des Eingangssignals.}
{Das Ausgangssignal des Vorverstärkers hat ein um \qty{1,8}{\decibel} höheres Signal-Rausch-Verhältnis als das Eingangssignal.}
{\textbf{\textcolor{DARCgreen}{Das Ausgangssignal des Vorverstärkers hat ein um \qty{1,8}{\decibel} geringeres Signal-Rausch-Verhältnis als das Eingangssignal.}}}
{Die Verstärkung des Nutzsignals beträgt \qty{1,8}{\decibel}, die Rauschleistung bleibt unverändert.}
\end{QQuestion}

}
\end{frame}

\begin{frame}
\only<1>{
\begin{QQuestion}{AF229}{Was bedeutet die Rauschzahl F~=~2 bei einem UHF-Vorverstärker?}{Das Ausgangssignal des Verstärkers hat ein um \qty{6}{\decibel} höheres Signal-Rausch-Verhältnis als das Eingangssignal.}
{Das Ausgangssignal des Verstärkers hat ein um \qty{3}{\decibel} höheres Signal-Rausch-Verhältnis als das Eingangssignal.}
{Das Ausgangssignal des Verstärkers hat ein um \qty{6}{\decibel} geringeres Signal-Rausch-Verhältnis als das Eingangssignal.}
{Das Ausgangssignal des Verstärkers hat ein um \qty{3}{\decibel} geringeres Signal-Rausch-Verhältnis als das Eingangssignal.}
\end{QQuestion}

}
\only<2>{
\begin{QQuestion}{AF229}{Was bedeutet die Rauschzahl F~=~2 bei einem UHF-Vorverstärker?}{Das Ausgangssignal des Verstärkers hat ein um \qty{6}{\decibel} höheres Signal-Rausch-Verhältnis als das Eingangssignal.}
{Das Ausgangssignal des Verstärkers hat ein um \qty{3}{\decibel} höheres Signal-Rausch-Verhältnis als das Eingangssignal.}
{Das Ausgangssignal des Verstärkers hat ein um \qty{6}{\decibel} geringeres Signal-Rausch-Verhältnis als das Eingangssignal.}
{\textbf{\textcolor{DARCgreen}{Das Ausgangssignal des Verstärkers hat ein um \qty{3}{\decibel} geringeres Signal-Rausch-Verhältnis als das Eingangssignal.}}}
\end{QQuestion}

}
\end{frame}%ENDCONTENT
