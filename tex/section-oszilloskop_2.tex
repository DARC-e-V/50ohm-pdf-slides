
\section{Oszilloskop II}
\label{section:oszilloskop_2}
\begin{frame}%STARTCONTENT

\frametitle{Foliensatz in Arbeit}
2024-04-28: Die Inhalte werden noch aufbereitet.

Derzeit sind in diesem Abschnitt nur die Fragen sortiert enthalten.

Für das Selbststudium verweisen wir aktuell auf den Abschnitt Wellenausbreitung im DARC Online Lehrgang (\textcolor{DARCblue}{\faLink~\href{https://www.darc.de/der-club/referate/ajw/lehrgang-te/e09/}{www.darc.de/der-club/referate/ajw/lehrgang-te/e09/}}) für die Prüfung bis Juni 2024. Bis auf die Fragen hat sich an der Thematik nichts geändert. Das Thema war bisher Stoff der Klasse~E und wurde mit der neuen Prüfungsordnung auf alle drei Klassen aufgeteilt.

\end{frame}

\begin{frame}
\only<1>{
\begin{QQuestion}{AI301}{Welches Gerät kann für die Prüfung von Signalverläufen verwendet werden?}{Absorptionsfrequenzmesser}
{Oszilloskop}
{Frequenzzähler}
{Dipmeter}
\end{QQuestion}

}
\only<2>{
\begin{QQuestion}{AI301}{Welches Gerät kann für die Prüfung von Signalverläufen verwendet werden?}{Absorptionsfrequenzmesser}
{\textbf{\textcolor{DARCgreen}{Oszilloskop}}}
{Frequenzzähler}
{Dipmeter}
\end{QQuestion}

}
\end{frame}

\begin{frame}
\only<1>{
\begin{QQuestion}{AI302}{Was benötigt ein Oszilloskop zur Darstellung stehender Bilder?}{X-Vorteiler}
{Triggereinrichtung}
{Y-Vorteiler}
{Frequenzmarken-Generator}
\end{QQuestion}

}
\only<2>{
\begin{QQuestion}{AI302}{Was benötigt ein Oszilloskop zur Darstellung stehender Bilder?}{X-Vorteiler}
{\textbf{\textcolor{DARCgreen}{Triggereinrichtung}}}
{Y-Vorteiler}
{Frequenzmarken-Generator}
\end{QQuestion}

}
\end{frame}

\begin{frame}
\only<1>{
\begin{QQuestion}{AI303}{Die Pulsbreite wird mit einem Oszilloskop bei~...}{\qty{90}{\percent} des Spitzenwertes gemessen.}
{\qty{50}{\percent} des Spitzenwertes gemessen.}
{\qty{70}{\percent} des Spitzenwertes gemessen.}
{\qty{10}{\percent} des Spitzenwertes gemessen.}
\end{QQuestion}

}
\only<2>{
\begin{QQuestion}{AI303}{Die Pulsbreite wird mit einem Oszilloskop bei~...}{\qty{90}{\percent} des Spitzenwertes gemessen.}
{\textbf{\textcolor{DARCgreen}{\qty{50}{\percent} des Spitzenwertes gemessen.}}}
{\qty{70}{\percent} des Spitzenwertes gemessen.}
{\qty{10}{\percent} des Spitzenwertes gemessen.}
\end{QQuestion}

}
\end{frame}

\begin{frame}
\only<1>{
\begin{QQuestion}{AI304}{Womit misst man am einfachsten die Hüllkurvenform eines HF-Signals? Mit einem~...}{breitbandigen Detektor und Kopfhörer.}
{hochohmigen Vielfachinstrument in Stellung AC.}
{empfindlichen SWR-Meter in Stellung Wellenmessung.}
{breitbandigen Oszilloskop.}
\end{QQuestion}

}
\only<2>{
\begin{QQuestion}{AI304}{Womit misst man am einfachsten die Hüllkurvenform eines HF-Signals? Mit einem~...}{breitbandigen Detektor und Kopfhörer.}
{hochohmigen Vielfachinstrument in Stellung AC.}
{empfindlichen SWR-Meter in Stellung Wellenmessung.}
{\textbf{\textcolor{DARCgreen}{breitbandigen Oszilloskop.}}}
\end{QQuestion}

}
\end{frame}

\begin{frame}
\only<1>{
\begin{PQuestion}{AI305}{Das folgende Bild zeigt das Zweiton-SSB-Ausgangssignal eines KW-Senders, das mit einem Oszilloskop ausreichender Bandbreite über einen 1:1-Tastkopf direkt an der angeschlossenen künstlichen \qty{50}{\ohm}-Antenne gemessen wurde. Welche Ausgangsleistung (PEP) liefert der Sender?}{\qty{144}{\W}}
{\qty{36}{\W}}
{\qty{100}{\W}}
{\qty{1600}{\W}}
{\DARCimage{1.0\linewidth}{108include}}\end{PQuestion}

}
\only<2>{
\begin{PQuestion}{AI305}{Das folgende Bild zeigt das Zweiton-SSB-Ausgangssignal eines KW-Senders, das mit einem Oszilloskop ausreichender Bandbreite über einen 1:1-Tastkopf direkt an der angeschlossenen künstlichen \qty{50}{\ohm}-Antenne gemessen wurde. Welche Ausgangsleistung (PEP) liefert der Sender?}{\qty{144}{\W}}
{\qty{36}{\W}}
{\textbf{\textcolor{DARCgreen}{\qty{100}{\W}}}}
{\qty{1600}{\W}}
{\DARCimage{1.0\linewidth}{108include}}\end{PQuestion}

}
\end{frame}

\begin{frame}
\frametitle{Lösungsweg}
\begin{itemize}
  \item gegeben: $R=50\Omega$
  \item gegeben: (aus Darstellung) $\^{U} = 100V$
  \item gesucht: $P_{\textrm{PEP}}$
  \end{itemize}
    \pause
    \begin{equation}\begin{split} \nonumber P_{\textrm{PEP}} &= \frac{U_{\textrm{eff}}^2}{R} = \frac{(\frac{100V}{\sqrt{2}})^2}{50\Omega}\\ &=\frac{\frac{(100V)^2}{2}}{50\Omega} = \frac{5000V^2}{50\Omega} = 100W \end{split}\end{equation}



\end{frame}

\begin{frame}
\only<1>{
\begin{PQuestion}{AI306}{Das folgende Bild zeigt das Zweiton-SSB-Ausgangssignal eines KW-Senders, das mit einem Oszilloskop ausreichender Bandbreite über einen 10:1-Tastkopf direkt an der angeschlossenen künstlichen 50 Ohm-Antenne gemessen wurde. Welche Ausgangsleistung (PEP) liefert der Sender?}{\qty{36}{\W}}
{\qty{72}{\W}}
{\qty{144}{\W}}
{\qty{400}{\W}}
{\DARCimage{1.0\linewidth}{43include}}\end{PQuestion}

}
\only<2>{
\begin{PQuestion}{AI306}{Das folgende Bild zeigt das Zweiton-SSB-Ausgangssignal eines KW-Senders, das mit einem Oszilloskop ausreichender Bandbreite über einen 10:1-Tastkopf direkt an der angeschlossenen künstlichen 50 Ohm-Antenne gemessen wurde. Welche Ausgangsleistung (PEP) liefert der Sender?}{\textbf{\textcolor{DARCgreen}{\qty{36}{\W}}}}
{\qty{72}{\W}}
{\qty{144}{\W}}
{\qty{400}{\W}}
{\DARCimage{1.0\linewidth}{43include}}\end{PQuestion}

}
\end{frame}

\begin{frame}
\frametitle{Lösungsweg}
\begin{itemize}
  \item gegeben: $R=50\Omega$
  \item gegeben: (aus Darstellung mit 10:1-Tastkopf) $\^{U} = 6V\cdot 10$
  \item gesucht: $P_{\textrm{PEP}}$
  \end{itemize}
    \pause
    \begin{equation}\begin{split} \nonumber P_{\textrm{PEP}} &= \frac{U_{\textrm{eff}}^2}{R} = \frac{(\frac{6V\cdot 10}{\sqrt{2}})^2}{50\Omega}\\ &=\frac{\frac{(60V)^2}{2}}{50\Omega} = 36W \end{split}\end{equation}



\end{frame}%ENDCONTENT
