
\section{Paketvermittelte Netzwerke}
\label{section:paketvermittelte_netzwerke}
\begin{frame}%STARTCONTENT
\begin{itemize}
  \item Das HAMNET, das Netzwerk nur für Funkamateure, basiert auf dem Internet-Protokoll (IP).
  \item Deswegen kann man das Hamnet mit der gleichen Software, die auch für das Internet verwendet wird, nutzen.
  \item Im einfachsten Fall ist das ein Webbrowser.
  \end{itemize}
\end{frame}

\begin{frame}\begin{itemize}
  \item Das Internet-Protokoll (IP) weist den beteiligten Computern IP-Adressen zu, damit sie sich gegenseitig erreichen können.
  \item IP-Adressen werden als vier Dezimalzahlen mit einem Punkt dazwischen geschrieben. Beispiel: 141.17.5.18
  \item Jede Dezimalzahl hat eine Länge von 8 Bit, deswegen ist die größtmögliche Zahl 255 (binär: 11111111).
  \end{itemize}

\end{frame}

\begin{frame}\begin{itemize}
  \item IP-Adressen sind in einen Netz- und einen Hostanteil aufgeteilt.
  \item Bei allen Computern, die sich im selben Netzwerk befinden, ist der Anfang der IP-Adressen gleich, diesen Anfang nennt man Netzanteil.
  \item Der Netzanteil ist unterschiedlich groß, je nachdem wie viele Computer (Hosts) im Netzwerk verwaltet werden sollen.
  \end{itemize}
\end{frame}

\begin{frame}Beispiele:

\emph{10}.100.234.22 (kleiner Netzanteil, großer Hostanteil)

\emph{192.168.1}.252 (großer Netzanteil, kleiner Hostanteil)

Dieses Prinzip kennt man vom Telefonnetz. Die großen Städte haben kürzere Vorwahlen als kleine Städte.

\end{frame}

\begin{frame}
\begin{figure}
    \DARCimage{0.85\linewidth}{699include}
    \caption{\scriptsize IPv4-Adresse und Netzmaske in Dezimal- und Dualschreibweise}
    \label{netzmaske}
\end{figure}

\begin{itemize}
  \item Eine Subnetzmaske gibt die Aufteilung einer IP-Adresse in Netz- und Hostanteil an, indem sie alle Bits des Netzanteils als 1 darstellt.
  \end{itemize}
\end{frame}

\begin{frame}\begin{itemize}
  \item Es zwei Möglichkeiten dieses niederzuschreiben, Beispiel für einen Netzanteil von 24:
  \item 255.255.255.0, was binär 11111111.11111111.11111111.00000000 ist.
  \item Die Schreibweise mit dem Schrägstrich, zum Beispiel 192.168.111.90/24
  \end{itemize}

\end{frame}

\begin{frame}
\begin{figure}
    \DARCimage{0.85\linewidth}{706include}
    \caption{\scriptsize Ausschnitt aus einer Netzwerk-Infrastruktur}
    \label{netzwerk}
\end{figure}

\begin{itemize}
  \item Netzwerkgeräte können nur innerhalb ihres eigenen lokalen Netzwerks direkt miteinander kommunizieren.
  \end{itemize}
\end{frame}

\begin{frame}
\begin{figure}
    \DARCimage{0.85\linewidth}{706include}
    \caption{\scriptsize Ausschnitt aus einer Netzwerk-Infrastruktur}
    \label{netzwerk}
\end{figure}

\begin{itemize}
  \item Man erkennt sie daran, dass sich aus ihrer eigenen IP-Adresse und Subnetzmaske derselbe Netzanteil ergibt wie beim Partner.
  \end{itemize}
\end{frame}

\begin{frame}
\begin{figure}
    \DARCimage{0.85\linewidth}{706include}
    \caption{\scriptsize Ausschnitt aus einer Netzwerk-Infrastruktur}
    \label{netzwerk}
\end{figure}

\begin{itemize}
  \item In allen anderen Fällen schicken sie die Daten an einen Router. Das ist eine Zwischenstation, die zwei oder mehr Netzwerke miteinander verbindet, um die Datenpakete weiterzuleiten.
  \end{itemize}
\end{frame}

\begin{frame}
\only<1>{
\begin{QQuestion}{EE412}{Wie können Informationen innerhalb eines paketvermittelten Netzes zwischen zwei Stationen ausgetauscht werden, die sich nicht direkt erreichen können?}{Durch wiederholte Aussendung (Paketwiederholung)}
{Durch Weiterleitung über Zwischenstationen (Paketweiterleitung)}
{Durch Entpacken vor der Sendung (Paketdekompression)}
{Durch Zusammenfassung von Übertragungen (Paketdefragmentierung)}
\end{QQuestion}

}
\only<2>{
\begin{QQuestion}{EE412}{Wie können Informationen innerhalb eines paketvermittelten Netzes zwischen zwei Stationen ausgetauscht werden, die sich nicht direkt erreichen können?}{Durch wiederholte Aussendung (Paketwiederholung)}
{\textbf{\textcolor{DARCgreen}{Durch Weiterleitung über Zwischenstationen (Paketweiterleitung)}}}
{Durch Entpacken vor der Sendung (Paketdekompression)}
{Durch Zusammenfassung von Übertragungen (Paketdefragmentierung)}
\end{QQuestion}

}
\end{frame}

\begin{frame}
\only<1>{
\begin{QQuestion}{EE414}{Kann das Internetprotokoll (IP) im Amateurfunk verwendet werden?}{Nein, Internetnutzern würde so Zugang zum Amateurfunkband ermöglicht.}
{Ja, die Kodierung des Amateurfunkrufzeichens erfolgt in der Subnetzmaske.}
{Ja, es ist nicht auf das Internet beschränkt.}
{Nein, die benötigte Bandbreite steht im Amateurfunk nicht zur Verfügung.}
\end{QQuestion}

}
\only<2>{
\begin{QQuestion}{EE414}{Kann das Internetprotokoll (IP) im Amateurfunk verwendet werden?}{Nein, Internetnutzern würde so Zugang zum Amateurfunkband ermöglicht.}
{Ja, die Kodierung des Amateurfunkrufzeichens erfolgt in der Subnetzmaske.}
{\textbf{\textcolor{DARCgreen}{Ja, es ist nicht auf das Internet beschränkt.}}}
{Nein, die benötigte Bandbreite steht im Amateurfunk nicht zur Verfügung.}
\end{QQuestion}

}
\end{frame}

\begin{frame}
\only<1>{
\begin{QQuestion}{EE413}{Was ergibt sich aus der eingestellten IP-Adresse und Subnetzmaske einer Kommunikationsschnittstelle beim Internetprotokoll (IP)?}{Das Standardgateway und die maximale Anzahl der Zwischenstationen (Hops)}
{Die Protokoll- und Portnummer des über die Schnittstelle verwendeten Protokolls}
{Die Gegenstelle und die durch das Teilnetz verwendete Bandbreite}
{Der direkt (d.~h. ohne Router) über die Schnittstelle erreichbare Adressbereich}
\end{QQuestion}

}
\only<2>{
\begin{QQuestion}{EE413}{Was ergibt sich aus der eingestellten IP-Adresse und Subnetzmaske einer Kommunikationsschnittstelle beim Internetprotokoll (IP)?}{Das Standardgateway und die maximale Anzahl der Zwischenstationen (Hops)}
{Die Protokoll- und Portnummer des über die Schnittstelle verwendeten Protokolls}
{Die Gegenstelle und die durch das Teilnetz verwendete Bandbreite}
{\textbf{\textcolor{DARCgreen}{Der direkt (d.~h. ohne Router) über die Schnittstelle erreichbare Adressbereich}}}
\end{QQuestion}

}
\end{frame}%ENDCONTENT
