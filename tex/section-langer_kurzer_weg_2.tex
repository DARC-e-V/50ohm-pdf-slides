
\section{Langer und kurzer Weg II}
\label{section:langer_kurzer_weg_2}
\begin{frame}%STARTCONTENT
\begin{itemize}
  \item Bei einer Richtantenne ist der Drehwinkel der Hauptstrahlrichtung entscheidend für das zu erreichende Funkziel
  \item Eine geradlinige Verbindung zwischen zwei Orten auf einer Kugel verläuft immer entlang des Großkreises
  \item Ein anderer Ort kann somit über zwei Drehrichtungen erreicht werden
  \item Die Strecke ist dabei unterschiedlich lang
  \item Der Drehwinkel unterscheidet sich dabei um \qty{180}{\degree}
  \item Beispiel: von Berlin nach Sidney/Australien ist der kurze Weg bei \qty{315}{\degree}, der lange Weg bei \qty{75}{\degree}
  \end{itemize}
\end{frame}

\begin{frame}
\only<1>{
\begin{QQuestion}{AH216}{Wie erkennt ein Funkamateur in der Regel, dass er mit \glqq PY\grqq{} auf dem indirekten und somit längeren Weg gearbeitet hat?}{Aus der Stellung seiner Richtantenne erkennt er, dass diese der Richtung des kürzesten Weges nach Brasilien um \qty{180}{\degree} entgegengesetzt ist. Das heißt, er hat \glqq PY\grqq{} auf dem \glqq langen Weg\grqq{} gearbeitet.}
{Durch die verhallte Tonlage der Verbindung erkennt er, dass diese in zwei Richtungen nach Brasilien stattgefunden hat. Das heißt, er hat \glqq PY\grqq{} nicht nur direkt, sondern auf einem längeren Weg gearbeitet.}
{Aus der Stellung seiner Richtantenne erkennt er, dass diese in Richtung des längeren Weges nach Brasilien eingesetzt ist. Das heißt, er hat \glqq PY\grqq{} auf dem direkten Weg gearbeitet.}
{Durch die verhallte Tonlage der Verbindung nach Brasilien, Ausbreitung der Funkwellen über zwei entgegengesetzte Wege.}
\end{QQuestion}

}
\only<2>{
\begin{QQuestion}{AH216}{Wie erkennt ein Funkamateur in der Regel, dass er mit \glqq PY\grqq{} auf dem indirekten und somit längeren Weg gearbeitet hat?}{\textbf{\textcolor{DARCgreen}{Aus der Stellung seiner Richtantenne erkennt er, dass diese der Richtung des kürzesten Weges nach Brasilien um \qty{180}{\degree} entgegengesetzt ist. Das heißt, er hat \glqq PY\grqq{} auf dem \glqq langen Weg\grqq{} gearbeitet.}}}
{Durch die verhallte Tonlage der Verbindung erkennt er, dass diese in zwei Richtungen nach Brasilien stattgefunden hat. Das heißt, er hat \glqq PY\grqq{} nicht nur direkt, sondern auf einem längeren Weg gearbeitet.}
{Aus der Stellung seiner Richtantenne erkennt er, dass diese in Richtung des längeren Weges nach Brasilien eingesetzt ist. Das heißt, er hat \glqq PY\grqq{} auf dem direkten Weg gearbeitet.}
{Durch die verhallte Tonlage der Verbindung nach Brasilien, Ausbreitung der Funkwellen über zwei entgegengesetzte Wege.}
\end{QQuestion}

}
\end{frame}

\begin{frame}
\frametitle{Rechnung}
Für den langen Weg

\begin{itemize}
  \item Bei Drehwinkel zwischen \qty{0}{\degree} und \qty{180}{\degree}: Drehwinkel + \qty{180}{\degree}
  \item Bei Drehwinkel zwischen \qty{180}{\degree} und \qty{360}{\degree}: Drehwinkel -- \qty{180}{\degree}
  \end{itemize}

\end{frame}

\begin{frame}
\only<1>{
\begin{QQuestion}{AH217}{Eine Amateurfunkstation in Frankfurt/Main will eine Verbindung nach Tokio auf dem langen Weg herstellen. Auf welchen Winkel gegen Nord (Azimut) muss der Funkamateur seinen Kurzwellenbeam drehen, wenn die Beamrichtung für den kurzen Weg \qty{38}{\degree} beträgt? Er muss die Antenne drehen auf~...}{\qty{122}{\degree}}
{\qty{322}{\degree}}
{\qty{218}{\degree}}
{\qty{308}{\degree}}
\end{QQuestion}

}
\only<2>{
\begin{QQuestion}{AH217}{Eine Amateurfunkstation in Frankfurt/Main will eine Verbindung nach Tokio auf dem langen Weg herstellen. Auf welchen Winkel gegen Nord (Azimut) muss der Funkamateur seinen Kurzwellenbeam drehen, wenn die Beamrichtung für den kurzen Weg \qty{38}{\degree} beträgt? Er muss die Antenne drehen auf~...}{\qty{122}{\degree}}
{\qty{322}{\degree}}
{\textbf{\textcolor{DARCgreen}{\qty{218}{\degree}}}}
{\qty{308}{\degree}}
\end{QQuestion}

}
\end{frame}

\begin{frame}
\only<1>{
\begin{QQuestion}{AH218}{Eine Amateurfunkstation in Frankfurt/Main will eine Verbindung nach Buenos Aires auf dem langen Weg herstellen. Auf welchen Winkel gegen Nord (Azimut) muss der Funkamateur seinen Kurzwellenbeam drehen, wenn die Beamrichtung für den kurzen Weg \qty{231}{\degree} beträgt? Er muss die Antenne drehen auf~...}{\qty{51}{\degree}}
{\qty{321}{\degree}}
{\qty{141}{\degree}}
{\qty{129}{\degree}}
\end{QQuestion}

}
\only<2>{
\begin{QQuestion}{AH218}{Eine Amateurfunkstation in Frankfurt/Main will eine Verbindung nach Buenos Aires auf dem langen Weg herstellen. Auf welchen Winkel gegen Nord (Azimut) muss der Funkamateur seinen Kurzwellenbeam drehen, wenn die Beamrichtung für den kurzen Weg \qty{231}{\degree} beträgt? Er muss die Antenne drehen auf~...}{\textbf{\textcolor{DARCgreen}{\qty{51}{\degree}}}}
{\qty{321}{\degree}}
{\qty{141}{\degree}}
{\qty{129}{\degree}}
\end{QQuestion}

}
\end{frame}%ENDCONTENT
