
\section{Näherungsformel I}
\label{section:naeherungsformel_1}
\begin{frame}%STARTCONTENT

\frametitle{Näherungsformel für Feldstärke}
\begin{columns}
    \begin{column}{0.48\textwidth}
    \begin{itemize}
  \item Berechnung der elektrischen Feldstärke
  \item Im Abstand zu einem Strahler
  \item Bei gegebener Leistung und Gewinn
  \item Gilt nur im Freiraum <br/> ($d > \frac{\lambda}{2\pi}$)
  \end{itemize}

    \end{column}
   \begin{column}{0.48\textwidth}
       \begin{equation}\begin{split} E &= \dfrac{\sqrt{30\Omega \cdot P_A \cdot G_i}}{d}\\ &= \dfrac{\sqrt{30\Omega \cdot P_{\textrm{EIRP}}}}{d} \end{split}\end{equation}


   \end{column}
\end{columns}

\end{frame}

\begin{frame}
\frametitle{Näherungsformel für Abstand}
\begin{columns}
    \begin{column}{0.48\textwidth}
    \begin{itemize}
  \item Bei gegebener Feldstärke
  \item Umstellen nach $d$
  \end{itemize}

    \end{column}
   \begin{column}{0.48\textwidth}
       \begin{equation}\begin{split} d &= \dfrac{\sqrt{30\Omega \cdot P_A \cdot G_i}}{E}\\ &= \dfrac{\sqrt{30\Omega \cdot P_{\textrm{EIRP}}}}{E} \end{split}\end{equation}


   \end{column}
\end{columns}

\end{frame}

\begin{frame}
\only<1>{
\begin{QQuestion}{EK108}{Sie möchten den Personenschutz-Sicherheitsabstand für die Antenne Ihrer Amateurfunkstelle für das \qty{10}{\m}-Band und das Modulationsverfahren FM berechnen. Der Grenzwert im Fall des Personenschutzes beträgt \qty{28}{\V}/m. Sie betreiben eine Yagi-Uda-Antenne mit einem Gewinn von $7,5~$dBd. Die Antenne wird von einem Sender mit einer Leistung von \qty{100}{\W} über ein langes Koaxialkabel gespeist. Die Kabeldämpfung beträgt \qty{1,5}{\decibel}. Wie groß muss der Sicherheitsabstand sein?}{\qty{3,9}{\m}}
{\qty{5,0}{\m}}
{\qty{2,5}{\m}}
{\qty{20,7}{\m}}
\end{QQuestion}

}
\only<2>{
\begin{QQuestion}{EK108}{Sie möchten den Personenschutz-Sicherheitsabstand für die Antenne Ihrer Amateurfunkstelle für das \qty{10}{\m}-Band und das Modulationsverfahren FM berechnen. Der Grenzwert im Fall des Personenschutzes beträgt \qty{28}{\V}/m. Sie betreiben eine Yagi-Uda-Antenne mit einem Gewinn von $7,5~$dBd. Die Antenne wird von einem Sender mit einer Leistung von \qty{100}{\W} über ein langes Koaxialkabel gespeist. Die Kabeldämpfung beträgt \qty{1,5}{\decibel}. Wie groß muss der Sicherheitsabstand sein?}{\qty{3,9}{\m}}
{\textbf{\textcolor{DARCgreen}{\qty{5,0}{\m}}}}
{\qty{2,5}{\m}}
{\qty{20,7}{\m}}
\end{QQuestion}

}
\end{frame}

\begin{frame}
\frametitle{Lösungsweg}
\begin{columns}
    \begin{column}{0.48\textwidth}
    \begin{itemize}
  \item gegeben: $E = 28\frac{V}{m}$
  \item gegeben: $g_d = 7,5dBd$
  \item gegeben: $P_S = 100W$
  \end{itemize}

    \end{column}
   \begin{column}{0.48\textwidth}
       \begin{itemize}
  \item gegeben: $a_{\textrm{Kabel}} = 1,5dB$
  \item gesucht: $P_{\textrm{EIRP}}$
  \item gesucht: $d$
  \end{itemize}

   \end{column}
\end{columns}
\begin{columns}
    \begin{column}{0.48\textwidth}
    
    \pause
    \begin{equation}\begin{split} \nonumber P_{\textrm{EIRP}} &= P_S \cdot 10^{\frac{g_d -- a + 2,15dB}{10dB}}\\ &= 100W \cdot 10^{\frac{7,5dB -- 1,5dB + 2,15dB}{10dB}}\\ &\approx 100W \cdot 6,5\\ &= 650W \end{split}\end{equation}




    \end{column}
   \begin{column}{0.48\textwidth}
       
    \pause
    \begin{equation}\begin{split} \nonumber d &= \dfrac{\sqrt{30\Omega \cdot P_{\textrm{EIRP}}}}{E}\\ &= \dfrac{\sqrt{30\Omega \cdot 650W}}{28\frac{V}{m}}\\ &\approx 5m \end{split}\end{equation}




   \end{column}
\end{columns}

\end{frame}

\begin{frame}
\frametitle{Bonusfrage}
Liegen die errechneten 5m nicht im Nahfeld für das 10m-Band aus der Frage?
    \pause
    \begin{equation}\nonumber \begin{split} \nonumber d &> \frac{\lambda}{2\pi}\\ \nonumber 5m &> \frac{10m}{2\pi}\\ \nonumber 5m &\gtrapprox 1,6m \end{split}\end{equation}



\end{frame}

\begin{frame}
\only<1>{
\begin{QQuestion}{EK106}{Wann ist die Berechnung des Personenschutz-Sicherheitsabstands mit der Näherungsformel für die Fernfeldberechnung auf den Bändern \qty{160}{\m} und \qty{80}{\m} ungültig? Die Berechnung ist ungültig, wenn das Ergebnis kleiner ist als~...}{\qty{160}{\m}-Band:~\qty{12,8}{\m}, \qty{80}{\m}-Band:~\qty{6,4}{\m}}
{\qty{160}{\m}-Band:~\qty{51,0}{\m}, \qty{80}{\m}-Band:~\qty{25,4}{\m}}
{\qty{160}{\m}-Band:~\qty{25,5}{\m}, \qty{80}{\m}-Band:~\qty{12,7}{\m}}
{\qty{160}{\m}-Band:~\qty{640}{\m}, \qty{80}{\m}-Band: \qty{320}{\m}}
\end{QQuestion}

}
\only<2>{
\begin{QQuestion}{EK106}{Wann ist die Berechnung des Personenschutz-Sicherheitsabstands mit der Näherungsformel für die Fernfeldberechnung auf den Bändern \qty{160}{\m} und \qty{80}{\m} ungültig? Die Berechnung ist ungültig, wenn das Ergebnis kleiner ist als~...}{\qty{160}{\m}-Band:~\qty{12,8}{\m}, \qty{80}{\m}-Band:~\qty{6,4}{\m}}
{\qty{160}{\m}-Band:~\qty{51,0}{\m}, \qty{80}{\m}-Band:~\qty{25,4}{\m}}
{\textbf{\textcolor{DARCgreen}{\qty{160}{\m}-Band:~\qty{25,5}{\m}, \qty{80}{\m}-Band:~\qty{12,7}{\m}}}}
{\qty{160}{\m}-Band:~\qty{640}{\m}, \qty{80}{\m}-Band: \qty{320}{\m}}
\end{QQuestion}

}
\end{frame}

\begin{frame}
\frametitle{Lösung}
\begin{itemize}
  \item Personenschutz-Sicherheitsabstand gilt nur im Freiraum
  \item $d > \frac{\lambda}{2\pi}$
  \item 160m-Band: \qty{25,5}{\metre}
  \item 80m-Band: \qty{12,7}{\metre}
  \end{itemize}
\end{frame}

\begin{frame}
\only<1>{
\begin{QQuestion}{EK105}{Sie möchten den Personenschutz-Sicherheitsabstand für ihren neuen, fest aufgebauten Halbwellendipol für das \qty{80}{\m}-Band (3,5 - \qty{3,8}{\MHz}) bestimmen. Bei \qty{100}{\W} Sendeleistung errechnen Sie mit Hilfe der Näherungsformel für die Fernfeldberechnung einen erforderlichen Abstand von \qty{3,65}{\m}. Ist dieser Sicherheitsabstand gültig?}{Der errechnete Personenschutz-Sicherheitsabstand muss erst noch mit einem Sicherheitszuschlag ($\sqrt{2}$) multipliziert werden.}
{Der errechnete Personenschutz-Sicherheitsabstand ist gültig, da Berechnungen mit der Näherungsformel für die Fernfeldberechnung im Amateurfunk hinreichend genau sind.}
{Der errechnete Abstand ist ungültig, da er im reaktiven Nahfeld der Antenne liegt, und muss deshalb durch andere Methoden wie z. B. Messungen der E- und H-Feldanteile, Simulations- oder Nahfeldberechnungen bestimmt werden.}
{Der errechnete Personenschutz-Sicherheitsabstand ist akzeptiert, sofern die vor Inbetriebnahme einzureichende \glqq Anzeige ortsfester Amateurfunkanlagen\grqq{} gemäß \S 9 BEMFV von der Bundesnetzagentur nicht beanstandet wird.}
\end{QQuestion}

}
\only<2>{
\begin{QQuestion}{EK105}{Sie möchten den Personenschutz-Sicherheitsabstand für ihren neuen, fest aufgebauten Halbwellendipol für das \qty{80}{\m}-Band (3,5 - \qty{3,8}{\MHz}) bestimmen. Bei \qty{100}{\W} Sendeleistung errechnen Sie mit Hilfe der Näherungsformel für die Fernfeldberechnung einen erforderlichen Abstand von \qty{3,65}{\m}. Ist dieser Sicherheitsabstand gültig?}{Der errechnete Personenschutz-Sicherheitsabstand muss erst noch mit einem Sicherheitszuschlag ($\sqrt{2}$) multipliziert werden.}
{Der errechnete Personenschutz-Sicherheitsabstand ist gültig, da Berechnungen mit der Näherungsformel für die Fernfeldberechnung im Amateurfunk hinreichend genau sind.}
{\textbf{\textcolor{DARCgreen}{Der errechnete Abstand ist ungültig, da er im reaktiven Nahfeld der Antenne liegt, und muss deshalb durch andere Methoden wie z. B. Messungen der E- und H-Feldanteile, Simulations- oder Nahfeldberechnungen bestimmt werden.}}}
{Der errechnete Personenschutz-Sicherheitsabstand ist akzeptiert, sofern die vor Inbetriebnahme einzureichende \glqq Anzeige ortsfester Amateurfunkanlagen\grqq{} gemäß \S 9 BEMFV von der Bundesnetzagentur nicht beanstandet wird.}
\end{QQuestion}

}

\end{frame}%ENDCONTENT
