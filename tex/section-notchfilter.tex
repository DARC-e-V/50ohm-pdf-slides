
\section{Notch-Filter}
\label{section:notchfilter}
\begin{frame}%STARTCONTENT

\begin{columns}
    \begin{column}{0.48\textwidth}
    \begin{itemize}
  \item \emph{Notch-Filter} oder Kerbfilter
  \item Schmalbandiges Filter
  \item Unterdrückt eine bestimmte NF-Frequenz
  \item Realisierbar im NF-Bereich oder ZF-Bereich
  \end{itemize}

    \end{column}
   \begin{column}{0.48\textwidth}
       
\begin{figure}
    \DARCimage{0.85\linewidth}{242include}
    \caption{\scriptsize Filtercharakteristik eines Notch-Filters}
    \label{frequenzverlauf_notchfilter}
\end{figure}


   \end{column}
\end{columns}

\end{frame}

\begin{frame}
\only<1>{
\begin{question2x2}{EF216}{Welches Diagramm stellt den Frequenzverlauf eines Empfänger-Notchfilters dar?}{\DARCimage{0.75\linewidth}{242include}}
{\DARCimage{0.75\linewidth}{243include}}
{\DARCimage{0.75\linewidth}{244include}}
{\DARCimage{0.75\linewidth}{245include}}
\end{question2x2}

}
\only<2>{
\begin{question2x2}{EF216}{Welches Diagramm stellt den Frequenzverlauf eines Empfänger-Notchfilters dar?}{\textbf{\textcolor{DARCgreen}{\DARCimage{0.75\linewidth}{242include}}}}
{\DARCimage{0.75\linewidth}{243include}}
{\DARCimage{0.75\linewidth}{244include}}
{\DARCimage{0.75\linewidth}{245include}}
\end{question2x2}

}
\end{frame}%ENDCONTENT
