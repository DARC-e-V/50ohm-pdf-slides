
\section{ITU-Regionen}
\label{section:itu_regionen}
\begin{frame}%STARTCONTENT

\begin{columns}
    \begin{column}{0.48\textwidth}
    
\begin{figure}
    \DARCimage{0.85\linewidth}{658include}
    \caption{\scriptsize Kartendarstellung der ITU-Regionen}
    \label{n_itu_regionen_karte}
\end{figure}


    \end{column}
   \begin{column}{0.48\textwidth}
       \begin{itemize}
  \item Weltweite Koordinierung von Funkfrequenzen getrennt nach 3 Regionen
  \item Durch ITU geregelt
  \item Nötig, um unterschiedliche Zuweisungen von Frequenzbereichen zu Funkdiensten vorzunehmen
  \end{itemize}

   \end{column}
\end{columns}

\end{frame}

\begin{frame}
\only<1>{
\begin{QQuestion}{VA402}{Nach den Radio Regulations (RR) ist die Welt für die Zuweisung von Frequenzbereichen an Funkdienste in Regionen aufgeteilt. Wie viele Regionen gibt es?}{Zwei}
{Fünf}
{Vier}
{Drei}
\end{QQuestion}

}
\only<2>{
\begin{QQuestion}{VA402}{Nach den Radio Regulations (RR) ist die Welt für die Zuweisung von Frequenzbereichen an Funkdienste in Regionen aufgeteilt. Wie viele Regionen gibt es?}{Zwei}
{Fünf}
{Vier}
{\textbf{\textcolor{DARCgreen}{Drei}}}
\end{QQuestion}

}
\end{frame}

\begin{frame}
\only<1>{
\begin{QQuestion}{VA401}{Weshalb wird in den Radio Regulations (RR) die Erde in verschiedene Regionen eingeteilt?}{Weil es sich um unterschiedliche Zeitzonen handelt und es so den Funkverkehr vereinfacht}
{Weil der Amateurfunkverkehr nur innerhalb einer Region zulässig ist}
{Um für die einzelnen Regionen Regelungen für Gastlizenzen einführen zu können}
{Um in den Regionen unterschiedliche Frequenzbereichszuweisungen für die Funkdienste vornehmen zu können}
\end{QQuestion}

}
\only<2>{
\begin{QQuestion}{VA401}{Weshalb wird in den Radio Regulations (RR) die Erde in verschiedene Regionen eingeteilt?}{Weil es sich um unterschiedliche Zeitzonen handelt und es so den Funkverkehr vereinfacht}
{Weil der Amateurfunkverkehr nur innerhalb einer Region zulässig ist}
{Um für die einzelnen Regionen Regelungen für Gastlizenzen einführen zu können}
{\textbf{\textcolor{DARCgreen}{Um in den Regionen unterschiedliche Frequenzbereichszuweisungen für die Funkdienste vornehmen zu können}}}
\end{QQuestion}

}
\end{frame}

\begin{frame}
\only<1>{
\begin{QQuestion}{VA403}{Zu welcher Region nach den Radio Regulations (RR) gehört Deutschland?}{Region 2}
{Region 1}
{Region 3}
{Region 4}
\end{QQuestion}

}
\only<2>{
\begin{QQuestion}{VA403}{Zu welcher Region nach den Radio Regulations (RR) gehört Deutschland?}{Region 2}
{\textbf{\textcolor{DARCgreen}{Region 1}}}
{Region 3}
{Region 4}
\end{QQuestion}

}
\end{frame}

\begin{frame}
\only<1>{
\begin{QQuestion}{VA404}{Zu welcher Region nach den Radio Regulations (RR) gehört Kanada?}{Region 3}
{Region 2}
{Region 4}
{Region 1}
\end{QQuestion}

}
\only<2>{
\begin{QQuestion}{VA404}{Zu welcher Region nach den Radio Regulations (RR) gehört Kanada?}{Region 3}
{\textbf{\textcolor{DARCgreen}{Region 2}}}
{Region 4}
{Region 1}
\end{QQuestion}

}
\end{frame}

\begin{frame}
\only<1>{
\begin{QQuestion}{VA405}{Zu welcher Region nach den Radio Regulations (RR) gehört Australien?}{Region 3}
{Region 1}
{Region 2}
{Region 4}
\end{QQuestion}

}
\only<2>{
\begin{QQuestion}{VA405}{Zu welcher Region nach den Radio Regulations (RR) gehört Australien?}{\textbf{\textcolor{DARCgreen}{Region 3}}}
{Region 1}
{Region 2}
{Region 4}
\end{QQuestion}

}
\end{frame}%ENDCONTENT
