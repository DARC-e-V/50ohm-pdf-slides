
\section{Antennenformen II}
\label{section:antennenformen_2}
\begin{frame}%STARTCONTENT

\frametitle{Symmetrie}
\begin{itemize}
  \item Mittengespeiste Dipole sind \emph{symmetrische Antennen}
  \item Weist an beiden Polen (z.B. den Einspeisepunkten) bis auf das Vorzeichen die gleiche Spannung gegenüber Erde auf
  \item Bei Dipolen und darauf basierenden Yagi-Uda-Antennen der Fall
  \item Die Groundplane-Antenne ist \emph{unsymmetrisch}, da sie am Anschlusspunkt der Radiale Erdpotential hat
  \end{itemize}

\end{frame}

\begin{frame}
\only<1>{
\begin{QQuestion}{EG213}{Welche Antenne gehört \underline{nicht} zu den symmetrischen Antennen?}{mittengespeister $\lambda$/2-Dipol}
{Faltdipol}
{Lang-Yagi-Uda}
{Groundplane}
\end{QQuestion}

}
\only<2>{
\begin{QQuestion}{EG213}{Welche Antenne gehört \underline{nicht} zu den symmetrischen Antennen?}{mittengespeister $\lambda$/2-Dipol}
{Faltdipol}
{Lang-Yagi-Uda}
{\textbf{\textcolor{DARCgreen}{Groundplane}}}
\end{QQuestion}

}
\end{frame}

\begin{frame}
\frametitle{Schleifenantennen}
\begin{itemize}
  \item Draht von insgesamt etwa einer Wellenlänge
  \item In Form eines Kreises, Quadrats, Dreiecks, …
  \item Beliebt: Delta-Loop-Antenne in Form eines Delta (Δ), da nur ein Mast benötigt wird
  \end{itemize}
\end{frame}

\begin{frame}
\only<1>{
\begin{QQuestion}{EG101}{Wie nennt man eine Schleifenantenne, die aus drei gleich langen Drahtstücken besteht?}{3-Element-Beam}
{3-Element-Quad-Loop-Antenne}
{W3DZZ-Antenne}
{Delta-Loop-Antenne}
\end{QQuestion}

}
\only<2>{
\begin{QQuestion}{EG101}{Wie nennt man eine Schleifenantenne, die aus drei gleich langen Drahtstücken besteht?}{3-Element-Beam}
{3-Element-Quad-Loop-Antenne}
{W3DZZ-Antenne}
{\textbf{\textcolor{DARCgreen}{Delta-Loop-Antenne}}}
\end{QQuestion}

}
\end{frame}

\begin{frame}
\frametitle{Magnetic-Loop}
\begin{itemize}
  \item Magnetische Ringantenne, da Abstrahlung im Nahfeld über das Magnetfeld erfolgt
  \item Ca. $\frac{\lambda}{10}$ Umfang
  \item Wirkungsgrad bei \qty{1}{\percent}-\qty{10}{\percent} im Sendebetrieb
  \item Weniger Störungen bei elektrisch leitfähigen oder dämpfenden Gegenständen im Nahfeld
  \end{itemize}

\end{frame}

\begin{frame}
\only<1>{
\begin{QQuestion}{EG105}{Welche Antennenform eignet sich für Sendebetrieb und weist dabei im Nahfeld ein starkes magnetisches Feld auf?}{Eine Cubical-Quad-Antenne}
{Eine Ferritstabantenne}
{Ein Faltdipol}
{Eine magnetische Ringantenne mit einem Umfang von etwa $\lambda$/10}
\end{QQuestion}

}
\only<2>{
\begin{QQuestion}{EG105}{Welche Antennenform eignet sich für Sendebetrieb und weist dabei im Nahfeld ein starkes magnetisches Feld auf?}{Eine Cubical-Quad-Antenne}
{Eine Ferritstabantenne}
{Ein Faltdipol}
{\textbf{\textcolor{DARCgreen}{Eine magnetische Ringantenne mit einem Umfang von etwa $\lambda$/10}}}
\end{QQuestion}

}
\end{frame}

\begin{frame}
\frametitle{Endgespeiste Antennen}
\begin{itemize}
  \item Speisung vom Ende her
  \item Länge häufig $\frac{\lambda}{2}$
  \item Benötigt eine höhere Spannung
  \end{itemize}
\end{frame}

\begin{frame}
\frametitle{Fuchs-Antenne}
\begin{columns}
    \begin{column}{0.48\textwidth}
    \begin{itemize}
  \item Verwendung eines Anpassglieds (Transformator)
  \item Oft verwendet: Fuchskreis
  \end{itemize}

    \end{column}
   \begin{column}{0.48\textwidth}
       
\begin{figure}
    \DARCimage{0.85\linewidth}{310include}
    \caption{\scriptsize Schematische Darstellung einer Fuchs-Antenne mit Fuchskreis}
    \label{e_antennenformen_fuchskreis}
\end{figure}


   \end{column}
\end{columns}

\end{frame}

\begin{frame}
\only<1>{
\begin{PQuestion}{EG104}{Welche Antennenart ist hier dargestellt?  }{Dipol-Antenne}
{Windom-Antenne}
{Fuchs-Antenne}
{Groundplane-Antenne}
{\DARCimage{1.0\linewidth}{310include}}\end{PQuestion}

}
\only<2>{
\begin{PQuestion}{EG104}{Welche Antennenart ist hier dargestellt?  }{Dipol-Antenne}
{Windom-Antenne}
{\textbf{\textcolor{DARCgreen}{Fuchs-Antenne}}}
{Groundplane-Antenne}
{\DARCimage{1.0\linewidth}{310include}}\end{PQuestion}

}
\end{frame}

\begin{frame}
\only<1>{
\begin{PQuestion}{EG103}{Welche Antenne ist hier dargestellt?}{Einband-Drahtantenne mit Preselektor}
{Einseitig geerdeter Winkeldipol mit Oberwellenfilter}
{Endgespeiste Antenne mit Collins-Filter zur Anpassung}
{Endgespeiste Antenne mit einfachem Anpassglied}
{\DARCimage{1.0\linewidth}{310include}}\end{PQuestion}

}
\only<2>{
\begin{PQuestion}{EG103}{Welche Antenne ist hier dargestellt?}{Einband-Drahtantenne mit Preselektor}
{Einseitig geerdeter Winkeldipol mit Oberwellenfilter}
{Endgespeiste Antenne mit Collins-Filter zur Anpassung}
{\textbf{\textcolor{DARCgreen}{Endgespeiste Antenne mit einfachem Anpassglied}}}
{\DARCimage{1.0\linewidth}{310include}}\end{PQuestion}

}
\end{frame}

\begin{frame}
\frametitle{Richtwirkung}
\begin{itemize}
  \item Darstellung als \emph{Strahlungsdiagramm}
  \item Für eine Ebene wird in jede Richtung der Gewinn bzw. Feldstärke oder Strahlungsleistung aufgetragen
  \item Je weiter der Graphenverlauf vom Mittelpunkt entfernt ist, umso größer der Gewinn bzw. umso höher die Feldstärke und Strahlungsleistung im Fernfeld
  \item Oft wird Antenne mit darin dargestellt
  \end{itemize}
\end{frame}

\begin{frame}
\frametitle{Richtwirkung eines Dipols}
\begin{columns}
    \begin{column}{0.48\textwidth}
    \begin{itemize}
  \item Strahlt rechtwinklig vom Draht ab
  \item In einer Ebene betrachtet ergeben sich Keulen neben dem Dipol
  \item Ein vertikaler Dipol strahlt rund herum ab
  \end{itemize}

    \end{column}
   \begin{column}{0.48\textwidth}
       
\begin{figure}
    \DARCimage{0.85\linewidth}{261include}
    \caption{\scriptsize Strahlungsdiagramm eines Dipols}
    \label{e_antennenformen_strahlungsdiagramm_dipol}
\end{figure}


   \end{column}
\end{columns}

\end{frame}

\begin{frame}
\only<1>{
\begin{PQuestion}{EG215}{Für welche Antenne ist dieses Strahlungsdiagramm typisch?}{Yagi-Uda-Antenne}
{Halbwellendipol}
{Groundplane}
{Kugelstrahler}
{\DARCimage{0.5\linewidth}{261include}}\end{PQuestion}

}
\only<2>{
\begin{PQuestion}{EG215}{Für welche Antenne ist dieses Strahlungsdiagramm typisch?}{Yagi-Uda-Antenne}
{\textbf{\textcolor{DARCgreen}{Halbwellendipol}}}
{Groundplane}
{Kugelstrahler}
{\DARCimage{0.5\linewidth}{261include}}\end{PQuestion}

}
\end{frame}

\begin{frame}
\only<1>{
\begin{question2x2}{EG214}{Welches der Bilder zeigt das Strahlungsdiagramm eines Halbwellendipols?}{\DARCimage{1.0\linewidth}{450include}}
{\DARCimage{1.0\linewidth}{262include}}
{\DARCimage{1.0\linewidth}{268include}}
{\DARCimage{1.0\linewidth}{261include}}
\end{question2x2}

}
\only<2>{
\begin{question2x2}{EG214}{Welches der Bilder zeigt das Strahlungsdiagramm eines Halbwellendipols?}{\DARCimage{1.0\linewidth}{450include}}
{\DARCimage{1.0\linewidth}{262include}}
{\DARCimage{1.0\linewidth}{268include}}
{\textbf{\textcolor{DARCgreen}{\DARCimage{1.0\linewidth}{261include}}}}
\end{question2x2}

}
\end{frame}

\begin{frame}
\frametitle{Vertikaler Halbwellendipol}
\begin{itemize}
  \item Ein vertikal montierter Halbwellendipol hat eine flache Abstrahlung
  \item Beliebt im DX-Betrieb oder Kontakten über Direkt- oder Bodenwelle
  \end{itemize}
\end{frame}

\begin{frame}
\only<1>{
\begin{QQuestion}{EG219}{Eine $\lambda$/2-Vertikalantenne erzeugt~...}{elliptische Polarisation.}
{zirkulare Polarisation.}
{einen hohen Abstrahlwinkel.}
{einen flachen Abstrahlwinkel.}
\end{QQuestion}

}
\only<2>{
\begin{QQuestion}{EG219}{Eine $\lambda$/2-Vertikalantenne erzeugt~...}{elliptische Polarisation.}
{zirkulare Polarisation.}
{einen hohen Abstrahlwinkel.}
{\textbf{\textcolor{DARCgreen}{einen flachen Abstrahlwinkel.}}}
\end{QQuestion}

}
\end{frame}

\begin{frame}
\frametitle{5/8&lambda;-Antenne}
\begin{itemize}
  \item Gegen Erde oder Fahrzeugkarosserie erregte 5/8$\lambda$-Antenne
  \item Spezialfall einer Vertikalantenne
  \item Die Länge ist so gewählt, damit sich ein optimaler Gewinn ergibt
  \end{itemize}
\end{frame}

\begin{frame}
\only<1>{
\begin{QQuestion}{EG108}{Warum ist eine 5/8-$\lambda$-Antenne besser als eine $\lambda$/4-Antenne für VHF-UHF-Mobilbetrieb geeignet? Sie~...}{ist weniger störanfällig.}
{verträgt mehr Leistung.}
{ist leichter zu montieren.}
{hat mehr Gewinn.}
\end{QQuestion}

}
\only<2>{
\begin{QQuestion}{EG108}{Warum ist eine 5/8-$\lambda$-Antenne besser als eine $\lambda$/4-Antenne für VHF-UHF-Mobilbetrieb geeignet? Sie~...}{ist weniger störanfällig.}
{verträgt mehr Leistung.}
{ist leichter zu montieren.}
{\textbf{\textcolor{DARCgreen}{hat mehr Gewinn.}}}
\end{QQuestion}

}
\end{frame}

\begin{frame}
\frametitle{Groundplane-Antenne}
\begin{columns}
    \begin{column}{0.48\textwidth}
    \begin{itemize}
  \item Strahlt rechtwinklig zum Strahler ab
  \item Strahlungsdiagramm wird von oben betrachtet
  \item Nahezu ein Rundstrahler, bis auf den Bereich der Radiale
  \end{itemize}

    \end{column}
   \begin{column}{0.48\textwidth}
       
\begin{figure}
    \DARCimage{0.85\linewidth}{268include}
    \caption{\scriptsize Strahlungsdiagramm einer Groundplane-Antenne von oben betrachtet}
    \label{e_antennenformen_strahlungsdiagramm_groundplane}
\end{figure}


   \end{column}
\end{columns}

\end{frame}

\begin{frame}
\only<1>{
\begin{PQuestion}{EG216}{Für welche Antenne ist dieses Strahlungsdiagramm typisch?}{Yagi-Uda}
{Kugelstrahler}
{Dipol}
{Groundplane}
{\DARCimage{0.5\linewidth}{268include}}\end{PQuestion}

}
\only<2>{
\begin{PQuestion}{EG216}{Für welche Antenne ist dieses Strahlungsdiagramm typisch?}{Yagi-Uda}
{Kugelstrahler}
{Dipol}
{\textbf{\textcolor{DARCgreen}{Groundplane}}}
{\DARCimage{0.5\linewidth}{268include}}\end{PQuestion}

}
\end{frame}

\begin{frame}
\frametitle{Richtantenne}
\begin{columns}
    \begin{column}{0.48\textwidth}
    \begin{itemize}
  \item Gewinn ist in eine Richtung deutlich höher als in andere Richtungen
  \end{itemize}

    \end{column}
   \begin{column}{0.48\textwidth}
       
\begin{figure}
    \DARCimage{0.85\linewidth}{262include}
    \caption{\scriptsize Strahlungsdiagramm einer Richtantenne}
    \label{e_antennenformen_strahlungsdiagramm_richtantenne}
\end{figure}


   \end{column}
\end{columns}

\end{frame}

\begin{frame}
\only<1>{
\begin{PQuestion}{EG217}{Dieses Strahlungsdiagramm ist typisch für~...}{einen Viertelwellenstrahler.}
{einen Halbwellendipol.}
{eine Richtantenne.}
{eine Marconi-Antenne.}
{\DARCimage{0.5\linewidth}{262include}}\end{PQuestion}

}
\only<2>{
\begin{PQuestion}{EG217}{Dieses Strahlungsdiagramm ist typisch für~...}{einen Viertelwellenstrahler.}
{einen Halbwellendipol.}
{\textbf{\textcolor{DARCgreen}{eine Richtantenne.}}}
{eine Marconi-Antenne.}
{\DARCimage{0.5\linewidth}{262include}}\end{PQuestion}

}
\end{frame}

\begin{frame}
\frametitle{Antennen für UHF/VHF/SHF}
\begin{columns}
    \begin{column}{0.48\textwidth}
    \begin{itemize}
  \item Nur für hohe Frequenzen geeignet
  \item Im Kurzwellenbereich unüblich, da sie unhandliche Größen erreichen würden
  \end{itemize}

    \end{column}
   \begin{column}{0.48\textwidth}
       \begin{itemize}
  \item Hornstrahler
  \item Parabolantennen
  \item Patchantennen auf Leiterplatten
  \item Sperrtopfantenne
  \end{itemize}

   \end{column}
\end{columns}

\end{frame}

\begin{frame}
\frametitle{Weitere Antennen für Kurzwelle}
\begin{itemize}
  \item Die \emph{Windom-Antenne} ist eine Mehrbandantenne, die aufgrund zwei unterschiedlich langer Schenkel eine Anpassung für mehrere Frequenzen erlaubt
  \item Die \emph{W3DZZ-Antenne} ist ein Dipol für 40m und 80m, deren Enden sich durch Sperrkreise bei 40m verkürzen
  \end{itemize}
\end{frame}

\begin{frame}
\only<1>{
\begin{QQuestion}{EG106}{Was sind gebräuchliche Kurzwellen-Amateurfunksendeantennen?}{Schlitzantenne, Groundplane-Antenne, Hornstrahler, Dipol-Antenne, Windom-Antenne}
{Langdraht-Antenne, Groundplane-Antenne, Parabolantenne, Windom-Antenne, Delta-Loop-Antenne}
{Groundplane-Antenne, Dipol-Antenne, Windom-Antenne, Delta-Loop-Antenne, Patchantenne}
{Langdraht-Antenne, Yagi-Uda-Antenne, Dipol-Antenne, Windom-Antenne, Delta-Loop-Antenne}
\end{QQuestion}

}
\only<2>{
\begin{QQuestion}{EG106}{Was sind gebräuchliche Kurzwellen-Amateurfunksendeantennen?}{Schlitzantenne, Groundplane-Antenne, Hornstrahler, Dipol-Antenne, Windom-Antenne}
{Langdraht-Antenne, Groundplane-Antenne, Parabolantenne, Windom-Antenne, Delta-Loop-Antenne}
{Groundplane-Antenne, Dipol-Antenne, Windom-Antenne, Delta-Loop-Antenne, Patchantenne}
{\textbf{\textcolor{DARCgreen}{Langdraht-Antenne, Yagi-Uda-Antenne, Dipol-Antenne, Windom-Antenne, Delta-Loop-Antenne}}}
\end{QQuestion}

}
\end{frame}

\begin{frame}
\only<1>{
\begin{QQuestion}{EG107}{Sie wollen verschiedene Antennen für den Funkbetrieb auf Kurzwelle für das \qty{80}{\m}-Band testen. Welche drei Antennen sind besonders geeignet?  }{Kreuz-Yagi-Uda, Groundplane-Antenne, Dipol}
{Dipol, Delta-Loop, W3DZZ-Antenne}
{Dipol, Sperrtopfantenne, W3DZZ-Antenne}
{Dipol, Delta-Loop, Parabolspiegel}
\end{QQuestion}

}
\only<2>{
\begin{QQuestion}{EG107}{Sie wollen verschiedene Antennen für den Funkbetrieb auf Kurzwelle für das \qty{80}{\m}-Band testen. Welche drei Antennen sind besonders geeignet?  }{Kreuz-Yagi-Uda, Groundplane-Antenne, Dipol}
{\textbf{\textcolor{DARCgreen}{Dipol, Delta-Loop, W3DZZ-Antenne}}}
{Dipol, Sperrtopfantenne, W3DZZ-Antenne}
{Dipol, Delta-Loop, Parabolspiegel}
\end{QQuestion}

}
\end{frame}%ENDCONTENT
