
\section{Primärer und sekundärer Funkdienst}
\label{section:primaerer_sekundaerer_funkdienst}
\begin{frame}%STARTCONTENT
Einige Frequenzbereiche sind uns primär und andere sekundär zugewiesen

\begin{itemize}
  \item \emph{Primär} bedeutet, dass wir vor anderen Funkdiensten Vorrang haben und von diesen keine Störungen hinnehmen müssen
  \item \emph{Sekundär} bedeutet, dass wir als Funkamateure andere Funkdienste nicht stören dürfen und Störungen durch diese hinnehmen müssen
  \end{itemize}

\end{frame}

\begin{frame}
\only<1>{
\begin{QQuestion}{VD704}{Wie ist ein primärer Funkdienst laut Amateurfunkverordnung (AFuV) definiert?}{Amateurfunkstellen sind keine Funkstellen eines primären Funkdienstes, da der Amateurfunk nach den Bestimmungen des Amateurfunkgesetzes (AFuG) kein Sicherheitsfunkdienst ist.}
{Kommerzielle Funkstellen, Funkstellen von Behörden und Organisationen mit Sicherheitsaufgaben sind immer Funkstellen des primären Funkdienstes.}
{Ein primärer Funkdienst ist ein Funkdienst, dessen Funkstellen Schutz gegen Störungen durch Funkstellen sekundärer Funkdienste verlangen können.}
{Die Unterteilung in primäre und sekundäre Funkdienste gilt nur für kommerzielle Funkstellen oder Funkstellen von Behörden und  Organisationen mit Sicherheitsaufgaben.}
\end{QQuestion}

}
\only<2>{
\begin{QQuestion}{VD704}{Wie ist ein primärer Funkdienst laut Amateurfunkverordnung (AFuV) definiert?}{Amateurfunkstellen sind keine Funkstellen eines primären Funkdienstes, da der Amateurfunk nach den Bestimmungen des Amateurfunkgesetzes (AFuG) kein Sicherheitsfunkdienst ist.}
{Kommerzielle Funkstellen, Funkstellen von Behörden und Organisationen mit Sicherheitsaufgaben sind immer Funkstellen des primären Funkdienstes.}
{\textbf{\textcolor{DARCgreen}{Ein primärer Funkdienst ist ein Funkdienst, dessen Funkstellen Schutz gegen Störungen durch Funkstellen sekundärer Funkdienste verlangen können.}}}
{Die Unterteilung in primäre und sekundäre Funkdienste gilt nur für kommerzielle Funkstellen oder Funkstellen von Behörden und  Organisationen mit Sicherheitsaufgaben.}
\end{QQuestion}

}
\end{frame}

\begin{frame}
\only<1>{
\begin{QQuestion}{VD705}{Wie ist ein sekundärer Funkdienst laut Amateurfunkverordnung (AFuV) definiert?}{Ein sekundärer Funkdienst muss Störungen durch andere hinnehmen und kann die Störungen nicht an die Funkstörungsannahme der Bundesnetzagentur melden.}
{Ein sekundärer Funkdienst ist ein Funkdienst, dessen Frequenzzuteilung zeitlich später erfolgte. Die Einteilung bedeutet nicht, dass der sekundäre Funkdienst dem primären Funkdienst nachgeordnet ist.}
{Ein sekundärer Funkdienst ist ein Funkdienst, dessen Funkstellen weder Störungen bei den Funkstellen eines primären Funkdienstes verursachen dürfen noch Schutz vor Störungen durch solche Funkstellen verlangen können.}
{Die Unterteilung in primäre und sekundäre Funkdienste gilt nur für kommerzielle Funkstellen oder Funkstellen von Behörden und Organisationen mit Sicherheitsaufgaben.}
\end{QQuestion}

}
\only<2>{
\begin{QQuestion}{VD705}{Wie ist ein sekundärer Funkdienst laut Amateurfunkverordnung (AFuV) definiert?}{Ein sekundärer Funkdienst muss Störungen durch andere hinnehmen und kann die Störungen nicht an die Funkstörungsannahme der Bundesnetzagentur melden.}
{Ein sekundärer Funkdienst ist ein Funkdienst, dessen Frequenzzuteilung zeitlich später erfolgte. Die Einteilung bedeutet nicht, dass der sekundäre Funkdienst dem primären Funkdienst nachgeordnet ist.}
{\textbf{\textcolor{DARCgreen}{Ein sekundärer Funkdienst ist ein Funkdienst, dessen Funkstellen weder Störungen bei den Funkstellen eines primären Funkdienstes verursachen dürfen noch Schutz vor Störungen durch solche Funkstellen verlangen können.}}}
{Die Unterteilung in primäre und sekundäre Funkdienste gilt nur für kommerzielle Funkstellen oder Funkstellen von Behörden und Organisationen mit Sicherheitsaufgaben.}
\end{QQuestion}

}
\end{frame}

\begin{frame}\begin{itemize}
  \item Die primären und sekundären Zuweisungen können in anderen Ländern abweichen
  \item Vor der Betriebsaufnahme über die Bestimmungen im Gastland informieren!
  \end{itemize}
\end{frame}

\begin{frame}
\frametitle{Seefunkdienst}
\begin{itemize}
  \item Das 80m-Band ist dem Amateurfunk primär zugeordnet
  \item Küstenfunkstellen des Seefunkdienstes haben dennoch Vorrang
  \item Grund: Feste Frequenz zugeteilt
  \end{itemize}

\end{frame}

\begin{frame}
\only<1>{
\begin{QQuestion}{VD707}{Das \qty{80}{\m}-Amateurfunkband ist unter anderem dem Amateurfunkdienst und dem Seefunkdienst auf primärer Basis zugewiesen. Unter welchen Umständen dürfen Sie in einer Amateurfunkverbindung fortfahren, wenn Sie erst nach Betriebsaufnahme bemerken, dass Ihre benutzte Frequenz auch von einer Küstenfunkstelle benutzt wird?}{Sie dürfen die Frequenz unter keinen Umständen weiterbenutzen (außer im echten Notfall), da der Küstenfunkstelle eine feste Frequenz zugeteilt ist, die sie nicht verändern kann.}
{Sie dürfen die Frequenz weiter benutzen, wenn aus der dauernd wiederholten, automatisch ablaufenden Morseaussendung klar hervorgeht, dass die Küstenfunkstelle keinen zweiseitigen Funkverkehr abwickelt, sondern offenbar nur die Frequenz belegt.}
{Sie dürfen die Frequenz weiter benutzen, wenn der Standort Ihrer Amateurfunkstelle mehr als \qty{200}{\km} von einer Meeresküste entfernt ist und Sie weniger als \qty{100}{\W} Sendeleistung anwenden.}
{Sie dürfen die begonnene Funkverbindung mit Ihrer Gegenfunkstelle solange fortführen, bis Sie von der Küstenfunkstelle zum Frequenzwechsel aufgefordert werden.}
\end{QQuestion}

}
\only<2>{
\begin{QQuestion}{VD707}{Das \qty{80}{\m}-Amateurfunkband ist unter anderem dem Amateurfunkdienst und dem Seefunkdienst auf primärer Basis zugewiesen. Unter welchen Umständen dürfen Sie in einer Amateurfunkverbindung fortfahren, wenn Sie erst nach Betriebsaufnahme bemerken, dass Ihre benutzte Frequenz auch von einer Küstenfunkstelle benutzt wird?}{\textbf{\textcolor{DARCgreen}{Sie dürfen die Frequenz unter keinen Umständen weiterbenutzen (außer im echten Notfall), da der Küstenfunkstelle eine feste Frequenz zugeteilt ist, die sie nicht verändern kann.}}}
{Sie dürfen die Frequenz weiter benutzen, wenn aus der dauernd wiederholten, automatisch ablaufenden Morseaussendung klar hervorgeht, dass die Küstenfunkstelle keinen zweiseitigen Funkverkehr abwickelt, sondern offenbar nur die Frequenz belegt.}
{Sie dürfen die Frequenz weiter benutzen, wenn der Standort Ihrer Amateurfunkstelle mehr als \qty{200}{\km} von einer Meeresküste entfernt ist und Sie weniger als \qty{100}{\W} Sendeleistung anwenden.}
{Sie dürfen die begonnene Funkverbindung mit Ihrer Gegenfunkstelle solange fortführen, bis Sie von der Küstenfunkstelle zum Frequenzwechsel aufgefordert werden.}
\end{QQuestion}

}
\end{frame}

\begin{frame}
\frametitle{ISM-Bereich}
\begin{itemize}
  \item \enquote{Industrial, Scientific and Medical Band}
  \item Teilbereich des 70cm-Amateurfunkbandes
  \item Viele Haushaltsgeräte nutzen dieses: Garagentoröffner, Funkwetterstationen, Autoschlüssel, Wegfahrsperren, Reifendrucksensoren, …
  \item Störungen im Amateurfunk müssen trotz primärer Zuweisung hingenommen werden
  \end{itemize}
\end{frame}

\begin{frame}
\only<1>{
\begin{QQuestion}{VD708}{Was besagt der Hinweis, dass der Frequenzbereich \qtyrange{433,05}{434,79}{\MHz} als ISM-Frequenzbereich zugewiesen ist?}{Dieser Frequenzbereich wird für internationale Satellitenmessungen verwendet; hierdurch kann es zu Störungen im normalen Funkverkehr kommen.}
{Dieser Frequenzbereich wird für industrielle, wissenschaftliche, medizinische, häusliche oder ähnliche Anwendungen mitbenutzt.}
{Dieser Frequenzbereich wird für industrielle Sender in Maschinen benutzt und ist für den Amateurfunkverkehr nur auf sekundärer Basis zugelassen.}
{Dieser Frequenzbereich wird von ISM-Geräten genutzt. Die Sendeleistungen im Amateurfunkdienst sind in diesem Frequenzbereich zu reduzieren.}
\end{QQuestion}

}
\only<2>{
\begin{QQuestion}{VD708}{Was besagt der Hinweis, dass der Frequenzbereich \qtyrange{433,05}{434,79}{\MHz} als ISM-Frequenzbereich zugewiesen ist?}{Dieser Frequenzbereich wird für internationale Satellitenmessungen verwendet; hierdurch kann es zu Störungen im normalen Funkverkehr kommen.}
{\textbf{\textcolor{DARCgreen}{Dieser Frequenzbereich wird für industrielle, wissenschaftliche, medizinische, häusliche oder ähnliche Anwendungen mitbenutzt.}}}
{Dieser Frequenzbereich wird für industrielle Sender in Maschinen benutzt und ist für den Amateurfunkverkehr nur auf sekundärer Basis zugelassen.}
{Dieser Frequenzbereich wird von ISM-Geräten genutzt. Die Sendeleistungen im Amateurfunkdienst sind in diesem Frequenzbereich zu reduzieren.}
\end{QQuestion}

}
\end{frame}%ENDCONTENT
