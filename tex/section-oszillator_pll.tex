
\section{Phasenregelschleife (PLL)}
\label{section:oszillator_pll}
\begin{frame}%STARTCONTENT

\only<1>{
\begin{QQuestion}{AD701}{Welche Baugruppen muss eine Phasenregelschleife (PLL) mindestens enthalten?}{Einen Phasenvergleicher, einen Hochpass und einen Frequenzteiler}
{Einen VCO, einen Hochpass und einen Phasenvergleicher}
{Einen Phasenvergleicher, einen Tiefpass und einen Frequenzteiler}
{Einen VCO, einen Tiefpass und einen Phasenvergleicher}
\end{QQuestion}

}
\only<2>{
\begin{QQuestion}{AD701}{Welche Baugruppen muss eine Phasenregelschleife (PLL) mindestens enthalten?}{Einen Phasenvergleicher, einen Hochpass und einen Frequenzteiler}
{Einen VCO, einen Hochpass und einen Phasenvergleicher}
{Einen Phasenvergleicher, einen Tiefpass und einen Frequenzteiler}
{\textbf{\textcolor{DARCgreen}{Einen VCO, einen Tiefpass und einen Phasenvergleicher}}}
\end{QQuestion}

}
\end{frame}

\begin{frame}
\only<1>{
\begin{PQuestion}{AD702}{Welche der nachfolgenden Aussagen ist richtig, wenn die im Bild dargestellte Regelschleife in stabilem Zustand ist?}{Die Frequenzen an den Punkten A und B sind gleich.}
{Die Frequenz an Punkt A ist höher als die Frequenz an Punkt B.}
{Die Frequenzen an den Punkten A und C sind gleich.}
{Die Frequenz an Punkt B ist höher als die Frequenz an Punkt C.}
{\DARCimage{1.0\linewidth}{45include}}\end{PQuestion}

}
\only<2>{
\begin{PQuestion}{AD702}{Welche der nachfolgenden Aussagen ist richtig, wenn die im Bild dargestellte Regelschleife in stabilem Zustand ist?}{\textbf{\textcolor{DARCgreen}{Die Frequenzen an den Punkten A und B sind gleich.}}}
{Die Frequenz an Punkt A ist höher als die Frequenz an Punkt B.}
{Die Frequenzen an den Punkten A und C sind gleich.}
{Die Frequenz an Punkt B ist höher als die Frequenz an Punkt C.}
{\DARCimage{1.0\linewidth}{45include}}\end{PQuestion}

}
\end{frame}

\begin{frame}
\only<1>{
\begin{QQuestion}{AD705}{Ein Frequenzsynthesizer soll eine einstellbare Frequenz mit hoher Frequenzgenauigkeit erzeugen. Die Genauigkeit und Stabilität der Ausgangsfrequenz eines Frequenzsynthesizers wird hauptsächlich bestimmt von~...}{den Eigenschaften des eingesetzten Quarzgenerators.}
{den Eigenschaften des spannungsgesteuerten Oszillators (VCO).}
{den Eigenschaften der eingesetzten Frequenzteiler.}
{den Eigenschaften des eingesetzten Phasenvergleichers.}
\end{QQuestion}

}
\only<2>{
\begin{QQuestion}{AD705}{Ein Frequenzsynthesizer soll eine einstellbare Frequenz mit hoher Frequenzgenauigkeit erzeugen. Die Genauigkeit und Stabilität der Ausgangsfrequenz eines Frequenzsynthesizers wird hauptsächlich bestimmt von~...}{\textbf{\textcolor{DARCgreen}{den Eigenschaften des eingesetzten Quarzgenerators.}}}
{den Eigenschaften des spannungsgesteuerten Oszillators (VCO).}
{den Eigenschaften der eingesetzten Frequenzteiler.}
{den Eigenschaften des eingesetzten Phasenvergleichers.}
\end{QQuestion}

}
\end{frame}

\begin{frame}
\only<1>{
\begin{PQuestion}{AD703}{Wie groß muss bei der folgenden Schaltung die Frequenz an Punkt A sein, wenn ein Kanalabstand von \qty{12,5}{\kHz} benötigt wird?}{\qty{11,64}{\Hz}}
{\qty{25}{\kHz}}
{\qty{1,25}{\kHz}}
{\qty{12,5}{\kHz}}
{\DARCimage{1.0\linewidth}{45include}}\end{PQuestion}

}
\only<2>{
\begin{PQuestion}{AD703}{Wie groß muss bei der folgenden Schaltung die Frequenz an Punkt A sein, wenn ein Kanalabstand von \qty{12,5}{\kHz} benötigt wird?}{\qty{11,64}{\Hz}}
{\qty{25}{\kHz}}
{\qty{1,25}{\kHz}}
{\textbf{\textcolor{DARCgreen}{\qty{12,5}{\kHz}}}}
{\DARCimage{1.0\linewidth}{45include}}\end{PQuestion}

}
\end{frame}

\begin{frame}
\only<1>{
\begin{PQuestion}{AD704}{Die Frequenz an Punkt A beträgt \qty{12,5}{\kHz}. Es sollen Ausgangsfrequenzen im Bereich von \qty{12,000}{\MHz} bis \qty{14,000}{\MHz} erzeugt werden. In welchem Bereich bewegt sich dabei das Teilerverhältnis n?}{960 bis 1120}
{300 bis 857}
{960 bis 857}
{300 bis 1120}
{\DARCimage{1.0\linewidth}{45include}}\end{PQuestion}

}
\only<2>{
\begin{PQuestion}{AD704}{Die Frequenz an Punkt A beträgt \qty{12,5}{\kHz}. Es sollen Ausgangsfrequenzen im Bereich von \qty{12,000}{\MHz} bis \qty{14,000}{\MHz} erzeugt werden. In welchem Bereich bewegt sich dabei das Teilerverhältnis n?}{\textbf{\textcolor{DARCgreen}{960 bis 1120}}}
{300 bis 857}
{960 bis 857}
{300 bis 1120}
{\DARCimage{1.0\linewidth}{45include}}\end{PQuestion}

}
\end{frame}

\begin{frame}
\frametitle{Lösungsweg}
\begin{itemize}
  \item gegeben: $f_{Osc} = 12,5kHz$
  \item gegeben: $f_{Out,low} = 12,000MHz$
  \item gegeben: $f_{Out,high} = 14,000MHz$
  \item gesucht: $:n$
  \end{itemize}
    \pause
    Bei $f_{Out,low} = 12,000MHz$:

$n = \frac{f_{Out,low}}{f_{Osc}} = \frac{12,000MHz}{12,5kHz} = 960$
    \pause
    Bei $f_{Out,high} = 14,000MHz$:

$n = \frac{f_{Out,high}}{f_{Osc}} = \frac{14,000MHz}{12,5kHz} = 1120$



\end{frame}%ENDCONTENT
