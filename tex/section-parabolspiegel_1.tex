
\section{Parabolspiegel I}
\label{section:parabolspiegel_1}
\begin{frame}%STARTCONTENT

\frametitle{Mikrowellen}
\begin{itemize}
  \item Frequenzen zwischen \qty{1}{\giga\hertz} und \qty{300}{\giga\hertz}
  \item Wellenlänge von Millimetern bis wenigen Dezimetern
  \item Können von Metallen reflektiert werden 
  \end{itemize}

\end{frame}

\begin{frame}
\frametitle{Parabolspiegel}
\begin{columns}
    \begin{column}{0.48\textwidth}
    \begin{itemize}
  \item Parabolisch geformte Metalloberfläche oder engmaschiges Gitter
  \item Parallel einfallende Wellen werden auf einem Punkt vor dem Spiegel gebündelt
  \end{itemize}

    \end{column}
   \begin{column}{0.48\textwidth}
       
\begin{figure}
    \DARCimage{0.85\linewidth}{850include}
    \caption{\scriptsize Funktionsweise eines Parabolspiegels}
    \label{e_parabolspiegel_funktion}
\end{figure}


   \end{column}
\end{columns}

\end{frame}

\begin{frame}
\only<1>{
\begin{QQuestion}{EG113}{Eine scharf bündelnde Antenne für den Mikrowellenbereich besteht häufig aus einem~...}{paraboloid geformten Spiegelkörper und einer Erregerantenne (Feed).}
{paraboloid geformten Spiegelkörper und einem isotropen Strahler.}
{zylindrisch konvex geformten Spiegelkörper und einer Erregerantenne (Feed).}
{hyperbolisch konkav geformten Spiegelkörper und einem isotropen Strahler.}
\end{QQuestion}

}
\only<2>{
\begin{QQuestion}{EG113}{Eine scharf bündelnde Antenne für den Mikrowellenbereich besteht häufig aus einem~...}{\textbf{\textcolor{DARCgreen}{paraboloid geformten Spiegelkörper und einer Erregerantenne (Feed).}}}
{paraboloid geformten Spiegelkörper und einem isotropen Strahler.}
{zylindrisch konvex geformten Spiegelkörper und einer Erregerantenne (Feed).}
{hyperbolisch konkav geformten Spiegelkörper und einem isotropen Strahler.}
\end{QQuestion}

}
\end{frame}

\begin{frame}
\frametitle{Beugungseffekt}
\begin{itemize}
  \item Durch die Welleneigenschaften kommt es zu Beugungseffekten
  \item Bündelung kommt nicht genau in einem Punkt zustande
  \item Abweichung kann durch Größe der Schüssel kompensiert werden
  \item Gewinn wird dadurch erhöht
  \item Optimal: Einige Wellenlängen oder mehr
  \end{itemize}
\end{frame}

\begin{frame}
\only<1>{
\begin{QQuestion}{EG114}{Welcher Durchmesser sollte für eine Parabolspiegelantenne im Hinblick auf möglichst hohen Gewinn gewählt werden?}{Höchstens drei Wellenlängen (Lambda) der verwendeten Frequenz.}
{Genau zwei Wellenlängen (Lambda) der verwendeten Frequenz.}
{Mindestens fünf Wellenlängen (Lambda) der verwendeten Frequenz.}
{Eine Wellenlänge (Lambda) der verwendeten Frequenz.}
\end{QQuestion}

}
\only<2>{
\begin{QQuestion}{EG114}{Welcher Durchmesser sollte für eine Parabolspiegelantenne im Hinblick auf möglichst hohen Gewinn gewählt werden?}{Höchstens drei Wellenlängen (Lambda) der verwendeten Frequenz.}
{Genau zwei Wellenlängen (Lambda) der verwendeten Frequenz.}
{\textbf{\textcolor{DARCgreen}{Mindestens fünf Wellenlängen (Lambda) der verwendeten Frequenz.}}}
{Eine Wellenlänge (Lambda) der verwendeten Frequenz.}
\end{QQuestion}

}
\end{frame}%ENDCONTENT
