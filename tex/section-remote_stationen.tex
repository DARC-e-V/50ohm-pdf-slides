
\section{Remote-Stationen}
\label{section:remote_stationen}
\begin{frame}%STARTCONTENT
\begin{itemize}
  \item Remotestationen ermöglichen einen Betrieb an einem anderen Standort
  \item z.B. wenn am Wohnort keine eigene Station realisiert werden kann
  \item Die gesamte Bedienung erfolgt ferngesteuert
  \end{itemize}
\begin{itemize}
  \item Betrieb durch Funkamateure der Klasse~A
  \item Mitbenutzung möglich
  \end{itemize}

\end{frame}

\begin{frame}
\only<1>{
\begin{QQuestion}{VD601}{Was versteht der Funkamateur unter \glqq Remote-Betrieb\grqq{}?}{Funkbetrieb bei Wettbewerben mit mehreren Funkamateuren mit verteilten Aufgaben}
{Funkbetrieb über sehr weite Entfernungen (größer \qty{500}{\km} UKW, größer \qty{2000}{\km} KW)}
{Die lokale Steuerung einer Funkstation über einen daneben stehenden Computer}
{Funkbetrieb, bei dem eine räumlich entfernte Amateurfunkstation z.~B. über das Internet betrieben wird}
\end{QQuestion}

}
\only<2>{
\begin{QQuestion}{VD601}{Was versteht der Funkamateur unter \glqq Remote-Betrieb\grqq{}?}{Funkbetrieb bei Wettbewerben mit mehreren Funkamateuren mit verteilten Aufgaben}
{Funkbetrieb über sehr weite Entfernungen (größer \qty{500}{\km} UKW, größer \qty{2000}{\km} KW)}
{Die lokale Steuerung einer Funkstation über einen daneben stehenden Computer}
{\textbf{\textcolor{DARCgreen}{Funkbetrieb, bei dem eine räumlich entfernte Amateurfunkstation z.~B. über das Internet betrieben wird}}}
\end{QQuestion}

}
\end{frame}

\begin{frame}
\only<1>{
\begin{QQuestion}{VD603}{Wer darf eine \glqq Remote-Station\grqq{} betreiben?}{Funkamateure der Klassen A und E}
{Funkamateure der Klasse A}
{Funkamateure der Klassen A, E und N}
{Funkamateure, die seit mindestens einem Jahr eine Zulassung besitzen}
\end{QQuestion}

}
\only<2>{
\begin{QQuestion}{VD603}{Wer darf eine \glqq Remote-Station\grqq{} betreiben?}{Funkamateure der Klassen A und E}
{\textbf{\textcolor{DARCgreen}{Funkamateure der Klasse A}}}
{Funkamateure der Klassen A, E und N}
{Funkamateure, die seit mindestens einem Jahr eine Zulassung besitzen}
\end{QQuestion}

}
\end{frame}

\begin{frame}
\only<1>{
\begin{QQuestion}{VD607}{Wer darf mit einer Amateurfunkstelle im \glqq Remote-Betrieb\grqq{} senden? Vom Betreiber der Amateurfunkstelle berechtigte Funkamateure, die ...}{über eine Zulassung für die Klasse~A, E oder N verfügen.}
{über eine Zulassung für die Klasse~A oder E verfügen.}
{über eine Zulassung für die Klasse~A verfügen.}
{seit mindestens einem Jahr über eine Zulassung verfügen.}
\end{QQuestion}

}
\only<2>{
\begin{QQuestion}{VD607}{Wer darf mit einer Amateurfunkstelle im \glqq Remote-Betrieb\grqq{} senden? Vom Betreiber der Amateurfunkstelle berechtigte Funkamateure, die ...}{über eine Zulassung für die Klasse~A, E oder N verfügen.}
{über eine Zulassung für die Klasse~A oder E verfügen.}
{\textbf{\textcolor{DARCgreen}{über eine Zulassung für die Klasse~A verfügen.}}}
{seit mindestens einem Jahr über eine Zulassung verfügen.}
\end{QQuestion}

}
\end{frame}

\begin{frame}
\frametitle{Betriebsmeldung}
\begin{itemize}
  \item Remotebetrieb muss durch den Betreiber angezeigt werden
  \item Mit Betriebsmeldung und Kontaktdaten an die BNetzA
  \item Erreichbarkeit während des Betriebs unter den angegebenen Kontaktdaten
  \end{itemize}

\end{frame}

\begin{frame}
\only<1>{
\begin{QQuestion}{VD602}{Ist für \glqq Remote-Betrieb\grqq{} bei der BNetzA eine Betriebsmeldung erforderlich?}{Ja, für Betreiber und Nutzer der Remote-Station}
{Ja, für den Nutzer der Remote-Station}
{Ja, für den Betreiber der Remote-Station}
{Nein, es besteht keine Anzeigepflicht.}
\end{QQuestion}

}
\only<2>{
\begin{QQuestion}{VD602}{Ist für \glqq Remote-Betrieb\grqq{} bei der BNetzA eine Betriebsmeldung erforderlich?}{Ja, für Betreiber und Nutzer der Remote-Station}
{Ja, für den Nutzer der Remote-Station}
{\textbf{\textcolor{DARCgreen}{Ja, für den Betreiber der Remote-Station}}}
{Nein, es besteht keine Anzeigepflicht.}
\end{QQuestion}

}
\end{frame}

\begin{frame}
\only<1>{
\begin{QQuestion}{VD608}{Warum muss der Betreiber der \glqq Remote-Station\grqq{} seine Kontaktdaten bei der BNetzA angeben?}{Die Bandwacht der Amateurfunkverbände nutzt die Kontaktdaten zum Datenabgleich, um im Störungsfall den Betreiber der \glqq Remote-Station\grqq{} zu ermitteln.}
{Die Kontaktdaten dienen der monatlichen Rechnungsstellung für die \glqq Remote-Station\grqq{}.}
{Der Betreiber muss für die BNetzA als Ansprechpartner erreichbar sein.}
{Die Kontaktdaten zum Remote-Betrieb werden in der Rufzeichenliste der BNetzA aufgeführt.}
\end{QQuestion}

}
\only<2>{
\begin{QQuestion}{VD608}{Warum muss der Betreiber der \glqq Remote-Station\grqq{} seine Kontaktdaten bei der BNetzA angeben?}{Die Bandwacht der Amateurfunkverbände nutzt die Kontaktdaten zum Datenabgleich, um im Störungsfall den Betreiber der \glqq Remote-Station\grqq{} zu ermitteln.}
{Die Kontaktdaten dienen der monatlichen Rechnungsstellung für die \glqq Remote-Station\grqq{}.}
{\textbf{\textcolor{DARCgreen}{Der Betreiber muss für die BNetzA als Ansprechpartner erreichbar sein.}}}
{Die Kontaktdaten zum Remote-Betrieb werden in der Rufzeichenliste der BNetzA aufgeführt.}
\end{QQuestion}

}
\end{frame}

\begin{frame}
\frametitle{Betriebssicherheit}
\begin{itemize}
  \item Ununterbrochene, mittelbare und vollständige Kontrolle der Station
  \item Kann über Hilfsmittel oder Helfer erfolgen
  \item Bei Störungen muss die Stationen in einen sicheren Zustand versetzt werden
  \end{itemize}

\end{frame}

\begin{frame}
\only<1>{
\begin{QQuestion}{VD605}{Wie muss der Betreiber die Betriebssicherheit seiner \glqq Remote-Station\grqq{} gewährleisten? Der Betreiber muss sicherstellen, dass~...}{die \glqq Remote-Station\grqq{} über eine unterbrechungsfreie Stromversorgung verfügt.}
{für die \glqq Remote-Station\grqq{} keine selbstgebauten Komponenten zum Einsatz kommen.}
{die \glqq Remote-Station\grqq{} unter seiner mittelbaren Kontrolle steht.}
{ein technisches Protokoll der Nutzung der \glqq Remote-Station\grqq{} erstellt wird.}
\end{QQuestion}

}
\only<2>{
\begin{QQuestion}{VD605}{Wie muss der Betreiber die Betriebssicherheit seiner \glqq Remote-Station\grqq{} gewährleisten? Der Betreiber muss sicherstellen, dass~...}{die \glqq Remote-Station\grqq{} über eine unterbrechungsfreie Stromversorgung verfügt.}
{für die \glqq Remote-Station\grqq{} keine selbstgebauten Komponenten zum Einsatz kommen.}
{\textbf{\textcolor{DARCgreen}{die \glqq Remote-Station\grqq{} unter seiner mittelbaren Kontrolle steht.}}}
{ein technisches Protokoll der Nutzung der \glqq Remote-Station\grqq{} erstellt wird.}
\end{QQuestion}

}
\end{frame}

\begin{frame}
\frametitle{Berechtigte Funkamateure}
\begin{itemize}
  \item Erlaubnis des Betreibers für Nutzung notwendig
  \item Betreiber darf nur berechtigte Funkamateure die Remotestation nutzen lassen
  \end{itemize}
\end{frame}

\begin{frame}
\only<1>{
\begin{QQuestion}{VD606}{Was ist bei der Übertragung des Nutzungsrechts an einer \glqq Remote-Station\grqq{} auf andere Funkamateure zu beachten?}{Der Betreiber muss sicherstellen, dass nur von ihm berechtigte Funkamateure die Station nutzen können.}
{Der Zugang für die Nutzung der \glqq Remote-Station\grqq{} muss für alle Funkamateure öffentlich sein.}
{Die Nutzer der \glqq Remote-Station\grqq{} dürfen keinen Ausbildungsfunkbetrieb durchführen.}
{Die Funkamateure müssen mindestens im Besitz einer Amateurfunkzulassung der Klasse~E sein.}
\end{QQuestion}

}
\only<2>{
\begin{QQuestion}{VD606}{Was ist bei der Übertragung des Nutzungsrechts an einer \glqq Remote-Station\grqq{} auf andere Funkamateure zu beachten?}{\textbf{\textcolor{DARCgreen}{Der Betreiber muss sicherstellen, dass nur von ihm berechtigte Funkamateure die Station nutzen können.}}}
{Der Zugang für die Nutzung der \glqq Remote-Station\grqq{} muss für alle Funkamateure öffentlich sein.}
{Die Nutzer der \glqq Remote-Station\grqq{} dürfen keinen Ausbildungsfunkbetrieb durchführen.}
{Die Funkamateure müssen mindestens im Besitz einer Amateurfunkzulassung der Klasse~E sein.}
\end{QQuestion}

}
\end{frame}

\begin{frame}
\frametitle{Klubstationen}
\begin{itemize}
  \item Klubstationen der Klasse~A dürfen als Remotestation betrieben werden
  \item Muss auf die Mitglieder der Gruppe von Funkamateuren begrenzt sein
  \end{itemize}
\end{frame}

\begin{frame}
\only<1>{
\begin{QQuestion}{VD604}{Welche der folgenden Amateurfunkstellen darf als \glqq Remote-Station\grqq{} verwendet werden?}{Amateurfunkstellen mit personengebundenem Rufzeichen der Klasse~N}
{Klubstationen der Klasse~E}
{Klubstationen der Klasse~A}
{Amateurfunkstellen mit personengebundenem Rufzeichen der Klasse~E}
\end{QQuestion}

}
\only<2>{
\begin{QQuestion}{VD604}{Welche der folgenden Amateurfunkstellen darf als \glqq Remote-Station\grqq{} verwendet werden?}{Amateurfunkstellen mit personengebundenem Rufzeichen der Klasse~N}
{Klubstationen der Klasse~E}
{\textbf{\textcolor{DARCgreen}{Klubstationen der Klasse~A}}}
{Amateurfunkstellen mit personengebundenem Rufzeichen der Klasse~E}
\end{QQuestion}

}
\end{frame}

\begin{frame}
\only<1>{
\begin{QQuestion}{VD609}{Wem darf Zugriff auf eine Klubstation im Remote-Betrieb eingeräumt werden?}{Nur Mitgliedern der Gruppe von Funkamateuren, die die Klubstation betreibt}
{Nur auf der Zuteilungsurkunde eingetragenen Mitgliedern der Gruppe von Funkamateuren}
{Nur bei der Bundesnetzagentur schriftlich oder elektronisch gemeldeten Funkamateuren}
{Nur Funkamateuren, die die Klubstation persönlich nicht aufsuchen können}
\end{QQuestion}

}
\only<2>{
\begin{QQuestion}{VD609}{Wem darf Zugriff auf eine Klubstation im Remote-Betrieb eingeräumt werden?}{\textbf{\textcolor{DARCgreen}{Nur Mitgliedern der Gruppe von Funkamateuren, die die Klubstation betreibt}}}
{Nur auf der Zuteilungsurkunde eingetragenen Mitgliedern der Gruppe von Funkamateuren}
{Nur bei der Bundesnetzagentur schriftlich oder elektronisch gemeldeten Funkamateuren}
{Nur Funkamateuren, die die Klubstation persönlich nicht aufsuchen können}
\end{QQuestion}

}
\end{frame}

\begin{frame}
\frametitle{Ausbildungsfunkbetrieb}
\begin{itemize}
  \item Ist an einer Remotestation möglich
  \item Es gelten die gleichen Regeln wie für den Ausbildungsbetrieb
  \item Auch an Remote-Klubstationen möglich
  \end{itemize}

\end{frame}%ENDCONTENT
