
\section{Troposphärische Inversionsbildung}
\label{section:troposphaere}
\begin{frame}%STARTCONTENT

\begin{columns}
    \begin{column}{0.48\textwidth}
    
\begin{figure}
    \DARCimage{0.85\linewidth}{734include}
    \caption{\scriptsize Troposhärische Inversionsbildung, Schichten unterschiedlicher Temperatur liegen aufeinander, an der Grenze der Schichten werden Funkwellen im VHF-Bereich reflektiert}
    \label{n_tropo}
\end{figure}


    \end{column}
   \begin{column}{0.48\textwidth}
       \begin{itemize}
  \item Besonderer Effekt in der Troposphäre (ca. 15km)
  \item \emph{Troposphärische Inversionsschichten} zwischen warmen und kalten Luftschichten
  \item Führt zu erheblich größeren Reichweiten im VHF-Bereich (800-1000km)
  \item Tritt hauptsächlich im Frühjahr und Herbst auf
  \end{itemize}

   \end{column}
\end{columns}

\end{frame}

\begin{frame}
\only<1>{
\begin{QQuestion}{NH304}{Welcher Effekt ist normalerweise für die Ausbreitung eines VHF-Signals über 800~bis \qty{1 000}{\km} verantwortlich?}{Atmosphärische Absorption}
{Reflexion an der Mondoberfläche}
{Bodenwellenausbreitung}
{Troposphärische Inversionsbildung}
\end{QQuestion}

}
\only<2>{
\begin{QQuestion}{NH304}{Welcher Effekt ist normalerweise für die Ausbreitung eines VHF-Signals über 800~bis \qty{1 000}{\km} verantwortlich?}{Atmosphärische Absorption}
{Reflexion an der Mondoberfläche}
{Bodenwellenausbreitung}
{\textbf{\textcolor{DARCgreen}{Troposphärische Inversionsbildung}}}
\end{QQuestion}

}
\end{frame}%ENDCONTENT
