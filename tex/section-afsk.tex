
\section{AFSK}
\label{section:afsk}
\begin{frame}%STARTCONTENT
\begin{itemize}
  \item Eine Sonderform der digitalen Modulation stellt das Audio Frequency Shift Keying (AFSK) dar.
  \item Im Gegensatz zu ASK steht hier das „A“ nicht für Amplitude, sondern für Audio, also für hörbare Frequenzen (Niederfrequenz).
  \item Es wird eine Frequenzumtastung (FSK) im Bereich deutlich unter 20 kHz durchgeführt. Oftmals wird der Bereich von ca. 300 Hz bis 2700 Hz genutzt.
  \item Für eine Aussendung per Funk muss eine weitere Modulation stattfinden, beispielsweise per FM, AM oder SSB.
  \end{itemize}

\end{frame}

\begin{frame}
\only<1>{
\begin{QQuestion}{EE408}{Was ist Audio Frequency Shift Keying (AFSK)?}{Ein unmodulierter Hochfrequenzträger, bei dem die Frequenzabweichung im hörbaren Bereich liegt
 }
{Ein hochfrequentes PSK-Signal, das mittels automatischer Umtastung auf zwei NF-Träger übertragen wird, um Bandbreite zu sparen}
{Eine Kombination aus digitaler Amplituden- und Frequenzmodulation, um zwei Informationen gleichzeitig zu übertragen}
{Ein durch Frequenzumtastung erzeugtes NF-Signal, mit dem ein Hochfrequenzträger (z.~B. mittels FM) moduliert werden kann}
\end{QQuestion}

}
\only<2>{
\begin{QQuestion}{EE408}{Was ist Audio Frequency Shift Keying (AFSK)?}{Ein unmodulierter Hochfrequenzträger, bei dem die Frequenzabweichung im hörbaren Bereich liegt
 }
{Ein hochfrequentes PSK-Signal, das mittels automatischer Umtastung auf zwei NF-Träger übertragen wird, um Bandbreite zu sparen}
{Eine Kombination aus digitaler Amplituden- und Frequenzmodulation, um zwei Informationen gleichzeitig zu übertragen}
{\textbf{\textcolor{DARCgreen}{Ein durch Frequenzumtastung erzeugtes NF-Signal, mit dem ein Hochfrequenzträger (z.~B. mittels FM) moduliert werden kann}}}
\end{QQuestion}

}
\end{frame}%ENDCONTENT
