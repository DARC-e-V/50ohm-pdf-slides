
\section{Einseitenbandmodulation (SSB)}
\label{section:ssb}
\begin{frame}%STARTCONTENT

\begin{columns}
    \begin{column}{0.48\textwidth}
    
\begin{figure}
    \DARCimage{0.85\linewidth}{482include}
    \caption{\scriptsize Amplitudenmodulation, Träger mit unterem (a) und oberen (b) Seitenband}
    \label{n_seitenband}
\end{figure}


    \end{column}
   \begin{column}{0.48\textwidth}
       \begin{itemize}
  \item Bei Amplitudenmodulation zusätzlich zum Träger zwei Seitenbänder $\rightarrow$ unteres bzw. oberes Seitenband
  \item \enquote{lower side band} (\emph{LSB}) und \enquote{upper side band} (\emph{USB})
  \item Der Träger selbst enthält gar keine Information
  \end{itemize}

   \end{column}
\end{columns}

\end{frame}

\begin{frame}
\begin{columns}
    \begin{column}{0.48\textwidth}
    \begin{itemize}
  \item Es reicht also, nur ein Seitenband auszusenden und auf den Träger und das andere Seitenband zu verzichten
  \item Gesamte Sendeleistung wird für die Übertragung der Information genutzt
  \item Belegte Bandbreite entspricht der Bandbreite des aufmodulierten Signals
  \end{itemize}

    \end{column}
   \begin{column}{0.48\textwidth}
       
\begin{figure}
    \DARCimage{0.85\linewidth}{743include}
    \caption{\scriptsize Seitenbänder bei AM, LSB und USB im Vergleich}
    \label{n_seitenband}
\end{figure}

\emph{Einseitenbandmodulation} bzw. \emph{single-sideband (SSB)}


   \end{column}
\end{columns}

\end{frame}

\begin{frame}USB steht für \emph{Upper Side Band}

(im Deutschen wird es gerne mit Unteres Seitenband verwechselt)

\end{frame}

\begin{frame}
\only<1>{
\begin{QQuestion}{NE203}{Was ist der Unterschied zwischen AM und SSB?}{AM hat keinen Träger und zwei Seitenbänder, SSB arbeitet mit Träger und nur einem Seitenband.}
{AM hat einen Träger und ein Seitenband, SSB arbeitet mit Trägerunterdrückung und hat zwei Seitenbänder.}
{AM hat keinen Träger und zwei Seitenbänder, SSB arbeitet mit Trägerunterdrückung und nur einem Seitenband.}
{AM hat einen Träger und zwei Seitenbänder, SSB arbeitet mit Trägerunterdrückung und nur einem Seitenband.}
\end{QQuestion}

}
\only<2>{
\begin{QQuestion}{NE203}{Was ist der Unterschied zwischen AM und SSB?}{AM hat keinen Träger und zwei Seitenbänder, SSB arbeitet mit Träger und nur einem Seitenband.}
{AM hat einen Träger und ein Seitenband, SSB arbeitet mit Trägerunterdrückung und hat zwei Seitenbänder.}
{AM hat keinen Träger und zwei Seitenbänder, SSB arbeitet mit Trägerunterdrückung und nur einem Seitenband.}
{\textbf{\textcolor{DARCgreen}{AM hat einen Träger und zwei Seitenbänder, SSB arbeitet mit Trägerunterdrückung und nur einem Seitenband.}}}
\end{QQuestion}

}
\end{frame}

\begin{frame}
\only<1>{
\begin{QQuestion}{NE204}{Was ist der Unterschied zwischen LSB und USB?}{LSB arbeitet mit Trägerunterdrückung und dem unteren Seitenband, USB arbeitet mit Trägerunterdrückung und dem oberen Seitenband.}
{LSB arbeitet mit Träger und zwei Seitenbändern, USB arbeitet mit Trägerunterdrückung und einem Seitenband.}
{LSB arbeitet mit Träger und einem Seitenband, USB arbeitet mit Trägerunterdrückung und beiden Seitenbändern.}
{LSB arbeitet mit Trägerunterdrückung und dem linken Seitenband, USB arbeitet mit Trägerunterdrückung und dem unteren Seitenband.}
\end{QQuestion}

}
\only<2>{
\begin{QQuestion}{NE204}{Was ist der Unterschied zwischen LSB und USB?}{\textbf{\textcolor{DARCgreen}{LSB arbeitet mit Trägerunterdrückung und dem unteren Seitenband, USB arbeitet mit Trägerunterdrückung und dem oberen Seitenband.}}}
{LSB arbeitet mit Träger und zwei Seitenbändern, USB arbeitet mit Trägerunterdrückung und einem Seitenband.}
{LSB arbeitet mit Träger und einem Seitenband, USB arbeitet mit Trägerunterdrückung und beiden Seitenbändern.}
{LSB arbeitet mit Trägerunterdrückung und dem linken Seitenband, USB arbeitet mit Trägerunterdrückung und dem unteren Seitenband.}
\end{QQuestion}

}
\end{frame}

\begin{frame}
\only<1>{
\begin{PQuestion}{NE208}{Welche spektrale Darstellung ergibt sich für Einseitenbandmodulation in LSB bei diesem Audiospektrum?}{\DARCimage{1.0\linewidth}{477include}}
{\DARCimage{1.0\linewidth}{480include}}
{\DARCimage{1.0\linewidth}{479include}}
{\DARCimage{1.0\linewidth}{481include}}
{\DARCimage{1.0\linewidth}{475include}}\end{PQuestion}

}
\only<2>{
\begin{PQuestion}{NE208}{Welche spektrale Darstellung ergibt sich für Einseitenbandmodulation in LSB bei diesem Audiospektrum?}{\DARCimage{1.0\linewidth}{477include}}
{\DARCimage{1.0\linewidth}{480include}}
{\DARCimage{1.0\linewidth}{479include}}
{\textbf{\textcolor{DARCgreen}{\DARCimage{1.0\linewidth}{481include}}}}
{\DARCimage{1.0\linewidth}{475include}}\end{PQuestion}

}
\end{frame}

\begin{frame}
\only<1>{
\begin{PQuestion}{NE207}{Welche spektrale Darstellung ergibt sich für Einseitenbandmodulation in USB bei diesem Audiospektrum?}{\DARCimage{1.0\linewidth}{481include}}
{\DARCimage{1.0\linewidth}{480include}}
{\DARCimage{1.0\linewidth}{479include}}
{\DARCimage{1.0\linewidth}{477include}}
{\DARCimage{1.0\linewidth}{475include}}\end{PQuestion}

}
\only<2>{
\begin{PQuestion}{NE207}{Welche spektrale Darstellung ergibt sich für Einseitenbandmodulation in USB bei diesem Audiospektrum?}{\DARCimage{1.0\linewidth}{481include}}
{\textbf{\textcolor{DARCgreen}{\DARCimage{1.0\linewidth}{480include}}}}
{\DARCimage{1.0\linewidth}{479include}}
{\DARCimage{1.0\linewidth}{477include}}
{\DARCimage{1.0\linewidth}{475include}}\end{PQuestion}

}
\end{frame}

\begin{frame}
\only<1>{
\begin{PQuestion}{NE205}{Welche Begriffe sind den Bereichen a und b des Modulationsverfahrens AM zuzuordnen?}{a = LSB; b = USB}
{a = USB; b = LSB}
{a = DSB; b = SSB}
{a = NF; b = HF}
{\DARCimage{1.0\linewidth}{482include}}\end{PQuestion}

}
\only<2>{
\begin{PQuestion}{NE205}{Welche Begriffe sind den Bereichen a und b des Modulationsverfahrens AM zuzuordnen?}{\textbf{\textcolor{DARCgreen}{a = LSB; b = USB}}}
{a = USB; b = LSB}
{a = DSB; b = SSB}
{a = NF; b = HF}
{\DARCimage{1.0\linewidth}{482include}}\end{PQuestion}

}
\end{frame}%ENDCONTENT
