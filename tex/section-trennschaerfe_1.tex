
\section{Trennschärfe I}
\label{section:trennschaerfe_1}
\begin{frame}%STARTCONTENT

\begin{columns}
    \begin{column}{0.48\textwidth}
    \begin{itemize}
  \item Empfang des gewünschten Signals
  \item Bei gleichzeitiger Unterdrückung von naheliegenden, unerwünschten Signalen
  \end{itemize}

    \end{column}
   \begin{column}{0.48\textwidth}
       \begin{itemize}
  \item Hohe Trennschärfe $\rightarrow$ geringe Bandbreite notwendig
  \item Idealerweise nur so breit wie das zu empfangene Signal
  \end{itemize}

   \end{column}
\end{columns}

\end{frame}

\begin{frame}
\only<1>{
\begin{QQuestion}{EF208}{Wo liegt bei einem Direktüberlagerungsempfänger üblicherweise die Oszillatorfrequenz für den Mischer?}{Sie liegt bei der Zwischenfrequenz. }
{Sie liegt sehr weit über der Empfangsfrequenz.}
{Sie liegt sehr viel tiefer als die Empfangsfrequenz.}
{Sie liegt in nächster Nähe zur Empfangsfrequenz.}
\end{QQuestion}

}
\only<2>{
\begin{QQuestion}{EF208}{Wo liegt bei einem Direktüberlagerungsempfänger üblicherweise die Oszillatorfrequenz für den Mischer?}{Sie liegt bei der Zwischenfrequenz. }
{Sie liegt sehr weit über der Empfangsfrequenz.}
{Sie liegt sehr viel tiefer als die Empfangsfrequenz.}
{\textbf{\textcolor{DARCgreen}{Sie liegt in nächster Nähe zur Empfangsfrequenz.}}}
\end{QQuestion}

}
\end{frame}%ENDCONTENT
