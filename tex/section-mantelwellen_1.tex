
\section{Mantelwellen I}
\label{section:mantelwellen_1}
\begin{frame}%STARTCONTENT
\begin{itemize}
  \item Ziel beim Aufbau einer Funkanlage: Nur die Antenne soll Signale abstrahlen bzw. aufnehmen
  \item Dazu eignen sich geschirmte Leitungen, z.B. Koaxialkabel
  \item Im Idealfall strahlen sie selbst nicht oder nehmen keine Strahlung auf
  \end{itemize}
\end{frame}

\begin{frame}
\begin{columns}
    \begin{column}{0.48\textwidth}
    \begin{itemize}
  \item Unsymmetrisches Koaxialkabel wird an symmetrischen Dipol angeschlossen
  \item Auf der Außenseite des Koaxialkabels können hochfrequente Ströme fließen
  \item Dadurch strahlt das Kabel selbst $\rightarrow$ \emph{Mantelwellen}
  \item Mantelströme fehlen, was zur Verformung der Richtcharakteristik führt
  \end{itemize}

    \end{column}
   \begin{column}{0.48\textwidth}
       
\begin{figure}
    \DARCimage{0.85\linewidth}{633include}
    \caption{\scriptsize Mantelstrom bei I<sub>3</sub>}
    \label{e_mantelwelle_effekt}
\end{figure}


   \end{column}
\end{columns}

\end{frame}

\begin{frame}
\only<1>{
\begin{QQuestion}{EG405}{Mantelwellen auf dem Koaxialkabel zur Antenne~...}{werden durch Fehlanpassung und Überlastung des Transceivers verursacht.}
{sind für die Funktionsweise jeder koaxial-gespeisten Antenne notwendig.}
{können zu Störungen anderer Geräte und Störungen des eigenen Empfangs führen.}
{werden für die Messung des Stromes beim SWR verwendet.}
\end{QQuestion}

}
\only<2>{
\begin{QQuestion}{EG405}{Mantelwellen auf dem Koaxialkabel zur Antenne~...}{werden durch Fehlanpassung und Überlastung des Transceivers verursacht.}
{sind für die Funktionsweise jeder koaxial-gespeisten Antenne notwendig.}
{\textbf{\textcolor{DARCgreen}{können zu Störungen anderer Geräte und Störungen des eigenen Empfangs führen.}}}
{werden für die Messung des Stromes beim SWR verwendet.}
\end{QQuestion}

}
\end{frame}

\begin{frame}
\only<1>{
\begin{QQuestion}{EG406}{Welche Effekte treten auf, wenn ein Halbwellendipol mit einem Koaxkabel gleicher Impedanz mittig gespeist wird?}{Es treten Polarisationsdrehungen auf, die von der Kabellänge abhängig sind.}
{Es treten keine nennenswerten Effekte auf, da die Antenne angepasst ist und die Speisung über ein Koaxkabel erfolgt, dessen Außenleiter Erdpotential hat.}
{Am Speisepunkt der Antenne treten gegenphasige Spannungen und Ströme gleicher Größe auf, die eine Fehlanpassung hervorrufen.}
{Die Richtcharakteristik der Antenne wird verformt und es treten Mantelwellen auf.}
\end{QQuestion}

}
\only<2>{
\begin{QQuestion}{EG406}{Welche Effekte treten auf, wenn ein Halbwellendipol mit einem Koaxkabel gleicher Impedanz mittig gespeist wird?}{Es treten Polarisationsdrehungen auf, die von der Kabellänge abhängig sind.}
{Es treten keine nennenswerten Effekte auf, da die Antenne angepasst ist und die Speisung über ein Koaxkabel erfolgt, dessen Außenleiter Erdpotential hat.}
{Am Speisepunkt der Antenne treten gegenphasige Spannungen und Ströme gleicher Größe auf, die eine Fehlanpassung hervorrufen.}
{\textbf{\textcolor{DARCgreen}{Die Richtcharakteristik der Antenne wird verformt und es treten Mantelwellen auf.}}}
\end{QQuestion}

}
\end{frame}

\begin{frame}
\only<1>{
\begin{PQuestion}{EG404}{Die Darstellung zeigt die bei Ankopplung eines Koaxialkabels an eine Antenne auftretenden Ströme. Wie wird der mit $I_3$ bezeichnete Strom genannt?}{Mantelstrom}
{Rückwärtsstrom}
{Potentialstrom}
{Phantomstrom}
{\DARCimage{1.0\linewidth}{633include}}\end{PQuestion}

}
\only<2>{
\begin{PQuestion}{EG404}{Die Darstellung zeigt die bei Ankopplung eines Koaxialkabels an eine Antenne auftretenden Ströme. Wie wird der mit $I_3$ bezeichnete Strom genannt?}{\textbf{\textcolor{DARCgreen}{Mantelstrom}}}
{Rückwärtsstrom}
{Potentialstrom}
{Phantomstrom}
{\DARCimage{1.0\linewidth}{633include}}\end{PQuestion}

}
\end{frame}

\begin{frame}
\frametitle{Mantelwellen verhindern}
\begin{itemize}
  \item Durch ein \emph{Symmetrierglied}, einen Balun (balanced-unbalanced)
  \item Oder zur Dämpfung Koaxialkabel auf einen Ferritkern wickeln
  \end{itemize}
\end{frame}

\begin{frame}
\only<1>{
\begin{QQuestion}{EG407}{Wozu wird ein Symmetrierglied (Balun) beispielsweise verwendet?}{Zur Nutzung einer Wechselspannungsversorgung am Gleichstromanschluss eines Transceivers}
{Zur Einstellung der Frequenzablage für Relaisbetrieb}
{Zum Anschluss eines Koaxialkabels an eine Dipol-Antenne}
{Zur Umschaltung zwischen horizontaler und vertikaler Polarisation einer Kreuz-Yagi-Uda}
\end{QQuestion}

}
\only<2>{
\begin{QQuestion}{EG407}{Wozu wird ein Symmetrierglied (Balun) beispielsweise verwendet?}{Zur Nutzung einer Wechselspannungsversorgung am Gleichstromanschluss eines Transceivers}
{Zur Einstellung der Frequenzablage für Relaisbetrieb}
{\textbf{\textcolor{DARCgreen}{Zum Anschluss eines Koaxialkabels an eine Dipol-Antenne}}}
{Zur Umschaltung zwischen horizontaler und vertikaler Polarisation einer Kreuz-Yagi-Uda}
\end{QQuestion}

}
\end{frame}

\begin{frame}
\only<1>{
\begin{PQuestion}{EG408}{Auf einem Ferritkern sind einige Windungen Koaxialkabel aufgewickelt. Mit diesem Aufbau~...}{lassen sich Oberwellen unterdrücken.}
{lassen sich statische Aufladungen verhindern.}
{lässt sich die Trennschärfe verbessern.}
{lassen sich Mantelwellen dämpfen.}
{\DARCimage{0.5\linewidth}{40include}}\end{PQuestion}

}
\only<2>{
\begin{PQuestion}{EG408}{Auf einem Ferritkern sind einige Windungen Koaxialkabel aufgewickelt. Mit diesem Aufbau~...}{lassen sich Oberwellen unterdrücken.}
{lassen sich statische Aufladungen verhindern.}
{lässt sich die Trennschärfe verbessern.}
{\textbf{\textcolor{DARCgreen}{lassen sich Mantelwellen dämpfen.}}}
{\DARCimage{0.5\linewidth}{40include}}\end{PQuestion}

}
\end{frame}%ENDCONTENT
