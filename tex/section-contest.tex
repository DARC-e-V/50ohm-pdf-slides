
\section{Contest}
\label{section:contest}
\begin{frame}%STARTCONTENT

\frametitle{Was ist ein Contest?}
\begin{itemize}
  \item Im Amateurfunk finden auch Wettbewerbe statt, die als Contest bezeichnet werden.
  \item Conteste dienen dem sportlichen Wettkampf, aber auch dazu, die eigene Amateurfunkstation und Betriebsabwicklung zu verbessern.
  \end{itemize}
\end{frame}

\begin{frame}
\only<1>{
\begin{QQuestion}{BE301}{Was ist der Zweck eines Amateurfunkwettbewerbs (Contest)? Er dient dem Wettkampf und~...}{dem Gewinnen von Preisgeldern.}
{der stetigen Verbesserung von Amateurfunkanlagen und Betriebstechnik.}
{der Erlangung eines Amateurfunkzeugnisses.}
{dem Testen der Störfestigkeit der Empfangsgeräte der Nachbarn.}
\end{QQuestion}

}
\only<2>{
\begin{QQuestion}{BE301}{Was ist der Zweck eines Amateurfunkwettbewerbs (Contest)? Er dient dem Wettkampf und~...}{dem Gewinnen von Preisgeldern.}
{\textbf{\textcolor{DARCgreen}{der stetigen Verbesserung von Amateurfunkanlagen und Betriebstechnik.}}}
{der Erlangung eines Amateurfunkzeugnisses.}
{dem Testen der Störfestigkeit der Empfangsgeräte der Nachbarn.}
\end{QQuestion}

}
\end{frame}

\begin{frame}
\frametitle{Was zeichnet Contestverbindungen aus?}
\begin{itemize}
  \item Die Verbindungen sind besonders kurz, weil man in der vorgegebenen Zeitdauer möglichst viele Verbindungen herstellen möchte.
  \end{itemize}
\end{frame}

\begin{frame}
\only<1>{
\begin{QQuestion}{BE302}{Warum ist der Informationsaustausch bei Verbindungen in einem Amateurfunkwettbewerb (Contest) besonders kurz?}{Um die vorgegebene Zeitbegrenzung für eine einzelne Verbindung einzuhalten.}
{Um in der vorgegebenen Zeitdauer möglichst viele Verbindungen herzustellen.}
{Weil sonst die Disqualifikation droht.}
{Weil alle notwendigen Informationen auch im Internet auffindbar sind.}
\end{QQuestion}

}
\only<2>{
\begin{QQuestion}{BE302}{Warum ist der Informationsaustausch bei Verbindungen in einem Amateurfunkwettbewerb (Contest) besonders kurz?}{Um die vorgegebene Zeitbegrenzung für eine einzelne Verbindung einzuhalten.}
{\textbf{\textcolor{DARCgreen}{Um in der vorgegebenen Zeitdauer möglichst viele Verbindungen herzustellen.}}}
{Weil sonst die Disqualifikation droht.}
{Weil alle notwendigen Informationen auch im Internet auffindbar sind.}
\end{QQuestion}

}
\end{frame}

\begin{frame}
\frametitle{Woran erkennt man Conteststationen?}
\begin{itemize}
  \item Oft erkennt man Conteststationen daran, dass der Namen des Wettbewerbes oder einfach nur \enquote{CONTEST} in den CQ-Ruf integriert wird.
  \item Zum Beispiel: \enquote{CQ FD} oder \enquote{CQ Ruhrgebietscontest} oder \enquote{CQ Contest}.
  \item In Telegrafie dann CQ TEST.
  \end{itemize}

\end{frame}

\begin{frame}
\only<1>{
\begin{QQuestion}{BE116}{Sie hören in Telegrafie \glqq CQ FD DD4UQ/P TEST\grqq{}. Was bedeutet das? Die Station DD4UQ~...}{sucht Verbindungen mit Stationen aus dem Raum Fulda.}
{sucht Verbindungen mit Stationen, die am Fieldday-Contest teilnehmen.}
{führt Testaussendungen im Full-Duplex-Betrieb durch.}
{führt Testaussendungen des Fernmeldedienstes durch.}
\end{QQuestion}

}
\only<2>{
\begin{QQuestion}{BE116}{Sie hören in Telegrafie \glqq CQ FD DD4UQ/P TEST\grqq{}. Was bedeutet das? Die Station DD4UQ~...}{sucht Verbindungen mit Stationen aus dem Raum Fulda.}
{\textbf{\textcolor{DARCgreen}{sucht Verbindungen mit Stationen, die am Fieldday-Contest teilnehmen.}}}
{führt Testaussendungen im Full-Duplex-Betrieb durch.}
{führt Testaussendungen des Fernmeldedienstes durch.}
\end{QQuestion}

}
\end{frame}

\begin{frame}
\frametitle{Besondere Betriebsabwicklung im Contest}
\begin{itemize}
  \item Welche Daten ausgetauscht werden, kann man der Ausschreibung entnehmen.
  \item Es gibt Wettbewerbe, bei denen nach jeder Verbindung die CQ-rufende Station die Frequenz der Gegenstation überlassen muss. Diese werden oft als \enquote{Sprint Contest} bezeichnet, ein Blick in Ausschreibung ist angeraten.
  \end{itemize}
\end{frame}

\begin{frame}
\only<1>{
\begin{QQuestion}{BE303}{Sie nehmen an einem Amateurfunkwettbewerb (Contest) teil. Welche Informationen sollten Sie in einem QSO austauschen?}{Ich beschränke mich auf das Rufzeichen, damit ich schnell möglichst viele Verbindungen erzielen kann.}
{Ich sende Rufzeichen, Signalrapport, Name, Standort und Stationsbeschreibung, damit das Logbuch der Gegenstation vollständig ist.}
{Ich entscheide für jede Verbindung einzeln, welche Daten ich sende, damit ich nicht zu viel über mich preisgebe.}
{Ich übermittle die in der Ausschreibung festgelegten Daten, damit die Verbindung gewertet wird.}
\end{QQuestion}

}
\only<2>{
\begin{QQuestion}{BE303}{Sie nehmen an einem Amateurfunkwettbewerb (Contest) teil. Welche Informationen sollten Sie in einem QSO austauschen?}{Ich beschränke mich auf das Rufzeichen, damit ich schnell möglichst viele Verbindungen erzielen kann.}
{Ich sende Rufzeichen, Signalrapport, Name, Standort und Stationsbeschreibung, damit das Logbuch der Gegenstation vollständig ist.}
{Ich entscheide für jede Verbindung einzeln, welche Daten ich sende, damit ich nicht zu viel über mich preisgebe.}
{\textbf{\textcolor{DARCgreen}{Ich übermittle die in der Ausschreibung festgelegten Daten, damit die Verbindung gewertet wird.}}}
\end{QQuestion}

}
\end{frame}

\begin{frame}
\only<1>{
\begin{QQuestion}{BE304}{Welche besondere Regelung gilt in einem \glqq Sprint-Contest\grqq{}?}{Die Teilnehmer müssen während des Funkbetriebs in ständiger Bewegung bleiben.}
{Es wird in Telegrafie gearbeitet und die Geschwindigkeit, mit der gegeben wird, fließt in die Wertung ein.}
{Das Rufzeichen darf nicht buchstabiert werden, außer am Anfang und am Ende des Wettbewerbs.}
{Nach jeder Verbindung überlässt die CQ-rufende Station die Frequenz der Gegenstation.}
\end{QQuestion}

}
\only<2>{
\begin{QQuestion}{BE304}{Welche besondere Regelung gilt in einem \glqq Sprint-Contest\grqq{}?}{Die Teilnehmer müssen während des Funkbetriebs in ständiger Bewegung bleiben.}
{Es wird in Telegrafie gearbeitet und die Geschwindigkeit, mit der gegeben wird, fließt in die Wertung ein.}
{Das Rufzeichen darf nicht buchstabiert werden, außer am Anfang und am Ende des Wettbewerbs.}
{\textbf{\textcolor{DARCgreen}{Nach jeder Verbindung überlässt die CQ-rufende Station die Frequenz der Gegenstation.}}}
\end{QQuestion}

}
\end{frame}%ENDCONTENT
