
\section{Sporadic-E}
\label{section:sporadic_e_1}
\begin{frame}%STARTCONTENT

\begin{columns}
    \begin{column}{0.48\textwidth}
    
\begin{figure}
    \DARCimage{0.85\linewidth}{733include}
    \caption{\scriptsize Refraktion (Brechung) von Funkwellen an stark ionisierten Bereichen der E-Schicht}
    \label{n_sporadic_e}
\end{figure}


    \end{column}
   \begin{column}{0.48\textwidth}
       \begin{itemize}
  \item Im Sommer mit größeren Reichweiten (1000-2000km)
  \item Refraktionen (Brechungen) an ionisierten Bereichen
  \item Treten in 100-110km in der E-Schicht auf
  \item Tritt zufällig und schwer vorhersagbar auf
  \item Sehr kleine Bereiche
  \end{itemize}

   \end{column}
\end{columns}

\end{frame}

\begin{frame}
\only<1>{
\begin{QQuestion}{NH306}{Ein Funkamateur sagt, dass auf dem \qty{2}{\m}-Band \glqq Sporadic-E-Bedingungen\grqq{} herrschen. Er meint damit, dass derzeit~...}{Stationen aus Entfernungen von \qtyrange{1000}{2000}{\km} zu hören sind, die über Reflexion an Ionisationserscheinungen des Polarkreises empfangen werden.}
{Stationen aus Nordamerika zu hören sind, die über Refraktion (Brechung) an energiereichen leuchtenden Nachtwolken (NLCs) empfangen werden.}
{Stationen aus Nordamerika zu hören sind, die über Reflexion an Ionisationserscheinungen des Polarkreises empfangen werden.}
{Stationen aus Entfernungen von \qtyrange{1000}{2000}{\km} zu hören sind, die über Refraktion (Brechung) in der sporadischen E-Region empfangen werden.}
\end{QQuestion}

}
\only<2>{
\begin{QQuestion}{NH306}{Ein Funkamateur sagt, dass auf dem \qty{2}{\m}-Band \glqq Sporadic-E-Bedingungen\grqq{} herrschen. Er meint damit, dass derzeit~...}{Stationen aus Entfernungen von \qtyrange{1000}{2000}{\km} zu hören sind, die über Reflexion an Ionisationserscheinungen des Polarkreises empfangen werden.}
{Stationen aus Nordamerika zu hören sind, die über Refraktion (Brechung) an energiereichen leuchtenden Nachtwolken (NLCs) empfangen werden.}
{Stationen aus Nordamerika zu hören sind, die über Reflexion an Ionisationserscheinungen des Polarkreises empfangen werden.}
{\textbf{\textcolor{DARCgreen}{Stationen aus Entfernungen von \qtyrange{1000}{2000}{\km} zu hören sind, die über Refraktion (Brechung) in der sporadischen E-Region empfangen werden.}}}
\end{QQuestion}

}
\end{frame}

\begin{frame}
\only<1>{
\begin{QQuestion}{NH305}{Bei welcher Ausbreitungsart wird über stark ionisierte Bereiche gearbeitet, die sich vor allem in den Sommermonaten in etwa 100 bis 110 Kilometer Höhe bilden?}{Troposphärische Ausbreitung}
{Sporadic-E}
{Reflexion an Inversionsschichten}
{Reflexion an Gewitterwolken}
\end{QQuestion}

}
\only<2>{
\begin{QQuestion}{NH305}{Bei welcher Ausbreitungsart wird über stark ionisierte Bereiche gearbeitet, die sich vor allem in den Sommermonaten in etwa 100 bis 110 Kilometer Höhe bilden?}{Troposphärische Ausbreitung}
{\textbf{\textcolor{DARCgreen}{Sporadic-E}}}
{Reflexion an Inversionsschichten}
{Reflexion an Gewitterwolken}
\end{QQuestion}

}
\end{frame}%ENDCONTENT
