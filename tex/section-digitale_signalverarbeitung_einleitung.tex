
\section{Digitale Signalverarbeitung}
\label{section:digitale_signalverarbeitung_einleitung}
\begin{frame}%STARTCONTENT
\begin{itemize}
  \item Im Bereich der Funktechnik spricht man bei Geräten, die mittels digitaler Signalverarbeitung arbeiten von sogenannten SDR-Geräten.
  \item \emph{SDR} steht dabei  für Software Defined Radio.
  \item In diesen Geräten ist zumindest ein Teil der Signalverarbeitung in Software realisiert.
  \item Dieses hat einen Kostenvorteil und bringt eine große Flexibilität mit sich.
  \end{itemize}
\end{frame}

\begin{frame}
\only<1>{
\begin{QQuestion}{EF603}{Worauf deutet die Bezeichnung SDR bei einem Transceiver oder Empfänger hin?}{Zumindest ein Teil der Signalaufbereitung ist in Software realisiert.}
{Es werden spezielle Antennenanschlüsse für digitale Signale verwendet.}
{Zumindest im NF-Bereich wird Analogtechnik eingesetzt, um besseren Klang zu erreichen.}
{Die Aussendung bzw. der Empfang erfolgt über das Internet und nicht per Funk.}
\end{QQuestion}

}
\only<2>{
\begin{QQuestion}{EF603}{Worauf deutet die Bezeichnung SDR bei einem Transceiver oder Empfänger hin?}{\textbf{\textcolor{DARCgreen}{Zumindest ein Teil der Signalaufbereitung ist in Software realisiert.}}}
{Es werden spezielle Antennenanschlüsse für digitale Signale verwendet.}
{Zumindest im NF-Bereich wird Analogtechnik eingesetzt, um besseren Klang zu erreichen.}
{Die Aussendung bzw. der Empfang erfolgt über das Internet und nicht per Funk.}
\end{QQuestion}

}
\end{frame}

\begin{frame}
\frametitle{A/D-Umsetzer}
\begin{itemize}
  \item Damit die Daten digital verarbeitet werden können, müssen sie zunächst digitalisiert werden.
  \item Dazu wird das analoge Signal mittels eines Analog-Digital Umsetzers (A/D-Umsetzer) in digitale Werte umgesetzt.
  \end{itemize}
\end{frame}

\begin{frame}
\begin{columns}
    \begin{column}{0.48\textwidth}
    \begin{itemize}
  \item Hierbei wird das analoge Signal in festen Zeitintervallen abgetastet und in einem digitalen Wertebereich (z.B. von -128 bis +127) abgebildet.
  \item Die einzelnen gemessenen Signalwerte werden als Samples (Proben) bezeichnet.
  \end{itemize}

    \end{column}
   \begin{column}{0.48\textwidth}
       
\begin{figure}
    \DARCimage{0.85\linewidth}{411include}
    \caption{\scriptsize Einfache Darstellung einer Sinuswelle aus 16 Samples und 7 Werten}
    \label{e_digitale_signalverarbeitung}
\end{figure}


   \end{column}
\end{columns}

\end{frame}

\begin{frame}
\only<1>{
\begin{QQuestion}{EF602}{Was ist die Voraussetzung, um ein analoges Signal mit digitaler Signalverarbeitung zu filtern? Das Eingangssignal muss zunächst~...}{von Oberschwingungen befreit werden.}
{demoduliert werden.}
{von Rauschen befreit werden.}
{digitalisiert werden.}
\end{QQuestion}

}
\only<2>{
\begin{QQuestion}{EF602}{Was ist die Voraussetzung, um ein analoges Signal mit digitaler Signalverarbeitung zu filtern? Das Eingangssignal muss zunächst~...}{von Oberschwingungen befreit werden.}
{demoduliert werden.}
{von Rauschen befreit werden.}
{\textbf{\textcolor{DARCgreen}{digitalisiert werden.}}}
\end{QQuestion}

}
\end{frame}

\begin{frame}
\frametitle{D/A-Umsetzer}
\begin{itemize}
  \item Nach der digitalen Verarbeitung des Signals wird es in einem Digital-Analog-Umsetzer (D/A-Umsetzer) wieder in ein analoges Signal verwandelt.
  \end{itemize}
\end{frame}

\begin{frame}
\only<1>{
\begin{PQuestion}{EF601}{Folgendes Blockschaltbild stellt das Prinzip einer digitalen Signalverarbeitung dar. Welche Aufgaben haben die beiden Blöcke 1 und 2?}{beides A/D-Umsetzer}
{1: D/A-Umsetzer, 2: A/D-Umsetzer}
{beides D/A-Umsetzer}
{1: A/D-Umsetzer, 2: D/A-Umsetzer}
{\DARCimage{1.0\linewidth}{96include}}\end{PQuestion}

}
\only<2>{
\begin{PQuestion}{EF601}{Folgendes Blockschaltbild stellt das Prinzip einer digitalen Signalverarbeitung dar. Welche Aufgaben haben die beiden Blöcke 1 und 2?}{beides A/D-Umsetzer}
{1: D/A-Umsetzer, 2: A/D-Umsetzer}
{beides D/A-Umsetzer}
{\textbf{\textcolor{DARCgreen}{1: A/D-Umsetzer, 2: D/A-Umsetzer}}}
{\DARCimage{1.0\linewidth}{96include}}\end{PQuestion}

}
\end{frame}%ENDCONTENT
