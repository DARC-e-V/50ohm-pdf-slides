
\section{Bandbreite}
\label{section:bandbreite}
\begin{frame}%STARTCONTENT

\begin{columns}
    \begin{column}{0.48\textwidth}
    \begin{itemize}
  \item Für die verschiedenen Amateurfunkbänder sind jeweils maximal zulässige Bandbreiten festgelegt
  \item Besonders aufpassen muss man bei Sendungen in der Nähe der Grenzen der Amateurfunkbänder
  \end{itemize}

    \end{column}
    \pause
    
   \begin{column}{0.48\textwidth}
       \begin{itemize}
  \item Nehmen wir an, ein FM-Signal ist \qty{15}{\kilo\hertz} breit und wir senden auf auf \qty{430}{\mega\hertz}
  \item Das Sendesignal befindet sich jeweils \qty{7,5}{\kilo\hertz} unterhalb und oberhalb
  \item Es würde sich also von \qtyrange{429,9925}{430,0075}{\mega\hertz} erstrecken
  \end{itemize}

   \end{column}
\end{columns}

\end{frame}

\begin{frame}
\only<1>{
\begin{QQuestion}{NE305}{Die gesamte Bandbreite einer FM-Übertragung beträgt \qty{15}{\kHz}. Wie weit muss die am Transceiver eingestellte Sendefrequenz von einer Bandgrenze mindestens entfernt sein, damit die Aussendung innerhalb des Bandes bleibt?}{\qty{0}{\kHz}}
{\qty{7,5}{\kHz}}
{\qty{15}{\kHz}}
{\qty{2,7}{\kHz}}
\end{QQuestion}

}
\only<2>{
\begin{QQuestion}{NE305}{Die gesamte Bandbreite einer FM-Übertragung beträgt \qty{15}{\kHz}. Wie weit muss die am Transceiver eingestellte Sendefrequenz von einer Bandgrenze mindestens entfernt sein, damit die Aussendung innerhalb des Bandes bleibt?}{\qty{0}{\kHz}}
{\textbf{\textcolor{DARCgreen}{\qty{7,5}{\kHz}}}}
{\qty{15}{\kHz}}
{\qty{2,7}{\kHz}}
\end{QQuestion}

}
\end{frame}

\begin{frame}
\begin{columns}
    \begin{column}{0.48\textwidth}
    Bei SSB  ist das Signal nur auf einer Seite der Trägerfrequenz zu finden:

\begin{itemize}
  \item Bei LSB vollständig unterhalb der Trägerfrequenz
  \item Bei USB vollständig oberhalb der Trägerfrequenz
  \end{itemize}

    \end{column}
    \pause
    
   \begin{column}{0.48\textwidth}
       Beispiel:

\begin{itemize}
  \item Am Funkgerät Sendefrequenz auf obere Bandgrenze einstellen
  \item Mit LSB darf gesendet werden
  \item Mit USB ist das Signal außerhalb des Bandes
  \end{itemize}

   \end{column}
\end{columns}



\end{frame}

\begin{frame}
\only<1>{
\begin{QQuestion}{VD738}{In welchen Amateurfunkfrequenzbereichen beträgt die maximal zulässige belegte Bandbreite einer Aussendung \qty{800}{\Hz}?}{\qtyrange{18068}{18168}{\kHz} und \qtyrange{24890}{24990}{\kHz}}
{\qtyrange{1810}{2000}{\kHz}, \qtyrange{3500}{3800}{\kHz} und \qtyrange{7000}{7200}{\kHz}}
{\qtyrange{7000}{7100}{\kHz} und \qtyrange{14000}{14350}{\kHz}}
{\qtyrange{135,7}{137,8}{\kHz}, \qtyrange{472}{479}{\kHz} und \qtyrange{10100}{10150}{\kHz}}
\end{QQuestion}

}
\only<2>{
\begin{QQuestion}{VD738}{In welchen Amateurfunkfrequenzbereichen beträgt die maximal zulässige belegte Bandbreite einer Aussendung \qty{800}{\Hz}?}{\qtyrange{18068}{18168}{\kHz} und \qtyrange{24890}{24990}{\kHz}}
{\qtyrange{1810}{2000}{\kHz}, \qtyrange{3500}{3800}{\kHz} und \qtyrange{7000}{7200}{\kHz}}
{\qtyrange{7000}{7100}{\kHz} und \qtyrange{14000}{14350}{\kHz}}
{\textbf{\textcolor{DARCgreen}{\qtyrange{135,7}{137,8}{\kHz}, \qtyrange{472}{479}{\kHz} und \qtyrange{10100}{10150}{\kHz}}}}
\end{QQuestion}

}
\end{frame}

\begin{frame}
\only<1>{
\begin{QQuestion}{VD739}{In welchem der folgenden Amateurfunkfrequenzbereiche beträgt die maximal zulässige belegte Bandbreite einer Aussendung \qty{2,7}{\kHz}?}{\qtyrange{135,7}{137,8}{\kHz}}
{\qtyrange{10100}{10150}{\kHz}}
{\qtyrange{3500}{3800}{\kHz}}
{\qtyrange{28000}{29700}{\kHz}}
\end{QQuestion}

}
\only<2>{
\begin{QQuestion}{VD739}{In welchem der folgenden Amateurfunkfrequenzbereiche beträgt die maximal zulässige belegte Bandbreite einer Aussendung \qty{2,7}{\kHz}?}{\qtyrange{135,7}{137,8}{\kHz}}
{\qtyrange{10100}{10150}{\kHz}}
{\textbf{\textcolor{DARCgreen}{\qtyrange{3500}{3800}{\kHz}}}}
{\qtyrange{28000}{29700}{\kHz}}
\end{QQuestion}

}
\end{frame}

\begin{frame}
\only<1>{
\begin{QQuestion}{VD740}{In welchem der folgenden Amateurfunkfrequenzbereiche beträgt die maximal zulässige belegte Bandbreite einer Aussendung \qty{7}{\kHz}?}{\qtyrange{21000}{21450}{\kHz}}
{\qtyrange{14000}{14350}{\kHz}}
{\qtyrange{28000}{29000}{\kHz}}
{\qtyrange{10100}{10150}{\kHz}}
\end{QQuestion}

}
\only<2>{
\begin{QQuestion}{VD740}{In welchem der folgenden Amateurfunkfrequenzbereiche beträgt die maximal zulässige belegte Bandbreite einer Aussendung \qty{7}{\kHz}?}{\qtyrange{21000}{21450}{\kHz}}
{\qtyrange{14000}{14350}{\kHz}}
{\textbf{\textcolor{DARCgreen}{\qtyrange{28000}{29000}{\kHz}}}}
{\qtyrange{10100}{10150}{\kHz}}
\end{QQuestion}

}
\end{frame}

\begin{frame}
\only<1>{
\begin{QQuestion}{VD741}{In welchem der folgenden Amateurfunkfrequenzbereiche beträgt die maximal zulässige belegte Bandbreite einer Aussendung \qty{40}{\kHz}?}{\qtyrange{144}{146}{\MHz}}
{\qtyrange{430}{440}{\MHz}}
{\qtyrange{1240}{1300}{\MHz}}
{\qtyrange{7000}{7200}{\kHz}}
\end{QQuestion}

}
\only<2>{
\begin{QQuestion}{VD741}{In welchem der folgenden Amateurfunkfrequenzbereiche beträgt die maximal zulässige belegte Bandbreite einer Aussendung \qty{40}{\kHz}?}{\textbf{\textcolor{DARCgreen}{\qtyrange{144}{146}{\MHz}}}}
{\qtyrange{430}{440}{\MHz}}
{\qtyrange{1240}{1300}{\MHz}}
{\qtyrange{7000}{7200}{\kHz}}
\end{QQuestion}

}
\end{frame}

\begin{frame}
\only<1>{
\begin{QQuestion}{VD742}{In welchem der folgenden Amateurfunkfrequenzbereiche beträgt die maximal zulässige belegte Bandbreite einer Aussendung \qty{2}{\MHz} bzw. für amplitudenmodulierte Fernsehaussendungen \qty{7}{\MHz}?}{\qtyrange{2320}{2450}{\MHz}}
{\qtyrange{430}{440}{\MHz}}
{\qtyrange{3400}{3475}{\MHz}}
{\qtyrange{10,0}{10,5}{\GHz}}
\end{QQuestion}

}
\only<2>{
\begin{QQuestion}{VD742}{In welchem der folgenden Amateurfunkfrequenzbereiche beträgt die maximal zulässige belegte Bandbreite einer Aussendung \qty{2}{\MHz} bzw. für amplitudenmodulierte Fernsehaussendungen \qty{7}{\MHz}?}{\qtyrange{2320}{2450}{\MHz}}
{\textbf{\textcolor{DARCgreen}{\qtyrange{430}{440}{\MHz}}}}
{\qtyrange{3400}{3475}{\MHz}}
{\qtyrange{10,0}{10,5}{\GHz}}
\end{QQuestion}

}
\end{frame}%ENDCONTENT
