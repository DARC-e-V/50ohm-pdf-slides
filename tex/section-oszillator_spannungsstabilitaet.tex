
\section{Spannungsstabilität von Oszillatoren}
\label{section:oszillator_spannungsstabilitaet}
\begin{frame}%STARTCONTENT

\only<1>{
\begin{QQuestion}{AD612}{Wie sollte die Gleichspannungsversorgung eines VFOs beschaffen sein, um Rückwirkungen nachfolgender HF-Leistungsverstärkerstufen zu verhindern?}{Die durch die PA hervorgerufenen HF-Überlagerungen auf der VFO-Stromversorgung müssen mit einem Hochpass gefiltert werden.}
{Sie muss möglichst direkt an die Spannungsversorgung der PA angekoppelt werden.}
{Sie darf nicht mit der Masseleitung der PA verbunden werden.}
{Sie muss gut gefiltert und von der Spannungsversorgung der PA entkoppelt werden.}
\end{QQuestion}

}
\only<2>{
\begin{QQuestion}{AD612}{Wie sollte die Gleichspannungsversorgung eines VFOs beschaffen sein, um Rückwirkungen nachfolgender HF-Leistungsverstärkerstufen zu verhindern?}{Die durch die PA hervorgerufenen HF-Überlagerungen auf der VFO-Stromversorgung müssen mit einem Hochpass gefiltert werden.}
{Sie muss möglichst direkt an die Spannungsversorgung der PA angekoppelt werden.}
{Sie darf nicht mit der Masseleitung der PA verbunden werden.}
{\textbf{\textcolor{DARCgreen}{Sie muss gut gefiltert und von der Spannungsversorgung der PA entkoppelt werden.}}}
\end{QQuestion}

}
\end{frame}

\begin{frame}
\only<1>{
\begin{QQuestion}{AD608}{Worauf ist bei der Spannungsversorgung eines VFO zu achten?}{Stromstabilisierte Gleichspannung}
{Unmittelbare Stromzufuhr vom Gleichrichter}
{Spannungsstabilisierte Gleichspannung}
{Stabilisierte Wechselspannung}
\end{QQuestion}

}
\only<2>{
\begin{QQuestion}{AD608}{Worauf ist bei der Spannungsversorgung eines VFO zu achten?}{Stromstabilisierte Gleichspannung}
{Unmittelbare Stromzufuhr vom Gleichrichter}
{\textbf{\textcolor{DARCgreen}{Spannungsstabilisierte Gleichspannung}}}
{Stabilisierte Wechselspannung}
\end{QQuestion}

}
\end{frame}

\begin{frame}
\only<1>{
\begin{QQuestion}{AD607}{Wie sollte der VFO in einem Sender betrieben werden, damit seine Frequenz stabil bleibt?}{Er sollte mit einer unstabilisierten Wechselspannung versorgt werden. }
{Er sollte in einem verlustarmen Teflongehäuse untergebracht sein.}
{Er sollte mit einer stabilisierten Gleichspannung versorgt werden.}
{Er sollte in einem Pertinaxgehäuse untergebracht sein.}
\end{QQuestion}

}
\only<2>{
\begin{QQuestion}{AD607}{Wie sollte der VFO in einem Sender betrieben werden, damit seine Frequenz stabil bleibt?}{Er sollte mit einer unstabilisierten Wechselspannung versorgt werden. }
{Er sollte in einem verlustarmen Teflongehäuse untergebracht sein.}
{\textbf{\textcolor{DARCgreen}{Er sollte mit einer stabilisierten Gleichspannung versorgt werden.}}}
{Er sollte in einem Pertinaxgehäuse untergebracht sein.}
\end{QQuestion}

}
\end{frame}

\begin{frame}
\only<1>{
\begin{QQuestion}{AD609}{Wodurch wird \glqq Chirp\grqq{} bei Morsetelegrafie hervorgerufen?}{Durch Amplitudenänderungen des Oszillators, weil die Tastung in der falschen Stufe erfolgt.}
{Durch Betriebsspannungsänderungen des Oszillators bei der Tastung.}
{Durch zu steile Flanken des Tastsignals.}
{Durch zu schnelle Tastung der Treiberstufe.}
\end{QQuestion}

}
\only<2>{
\begin{QQuestion}{AD609}{Wodurch wird \glqq Chirp\grqq{} bei Morsetelegrafie hervorgerufen?}{Durch Amplitudenänderungen des Oszillators, weil die Tastung in der falschen Stufe erfolgt.}
{\textbf{\textcolor{DARCgreen}{Durch Betriebsspannungsänderungen des Oszillators bei der Tastung.}}}
{Durch zu steile Flanken des Tastsignals.}
{Durch zu schnelle Tastung der Treiberstufe.}
\end{QQuestion}

}
\end{frame}%ENDCONTENT
