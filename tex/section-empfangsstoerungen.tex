
\section{Störungen beim Empfang}
\label{section:empfangsstoerungen}
\begin{frame}%STARTCONTENT

\frametitle{Suche im eigenen Haushalt}
Häufige Ursachen

\begin{itemize}
  \item Wechselrichter von Solaranlagen
  \item Schaltnetzteile
  \item LED-Leuchten
  \item Powerline Communication
  \end{itemize}

\end{frame}

\begin{frame}
\only<1>{
\begin{QQuestion}{VE307}{Der Empfang Ihrer Amateurfunkstation ist auf allen Bändern gestört. Welche Maßnahme sollten Sie als erstes ergreifen?}{Störquellen im eigenen Haushalt suchen, z. B. Steckernetzteile, LED-Lampen, Computer und Bildschirme.}
{Die Funkstörungsannahme der Bundesnetzagentur telefonisch oder per E-Mail informieren.}
{Das Intruder Monitoring eines Amateurfunkverbandes informieren.}
{Den Empfangsbetrieb sofort einstellen und z. B. auf Sendebetrieb umstellen.}
\end{QQuestion}

}
\only<2>{
\begin{QQuestion}{VE307}{Der Empfang Ihrer Amateurfunkstation ist auf allen Bändern gestört. Welche Maßnahme sollten Sie als erstes ergreifen?}{\textbf{\textcolor{DARCgreen}{Störquellen im eigenen Haushalt suchen, z. B. Steckernetzteile, LED-Lampen, Computer und Bildschirme.}}}
{Die Funkstörungsannahme der Bundesnetzagentur telefonisch oder per E-Mail informieren.}
{Das Intruder Monitoring eines Amateurfunkverbandes informieren.}
{Den Empfangsbetrieb sofort einstellen und z. B. auf Sendebetrieb umstellen.}
\end{QQuestion}

}
\end{frame}

\begin{frame}
\frametitle{Hinnehmbare Störungen}
\begin{itemize}
  \item Versuchen, die Grenzwerte von Geräten in der Nachbarschaft festzustellen
  \item Werden die Grenzwerte (EMVG und FuAG) eingehalten, muss die Störung hingenommen werden
  \item Evtl. hat der Nachbar Kooperationsbereitschaft zur Behebung
  \end{itemize}

\end{frame}

\begin{frame}
\only<1>{
\begin{QQuestion}{VE308}{Muss ein Funkamateur eine Störung seines Empfangs durch andere Geräte hinnehmen?}{Er muss die Störungen grundsätzlich hinnehmen, wenn die störenden Geräte den Anforderungen des Gesetzes über die elektromagnetische Verträglichkeit von Betriebsmitteln (EMVG) oder des Funkanlagengesetzes (FuAG) genügen.}
{Er muss Störungen nicht hinnehmen.}
{Er muss die Störungen in jedem Fall hinnehmen.}
{Er muss die Störungen grundsätzlich hinnehmen, wenn das störende Gerät von erheblicher Bedeutung für den Betreiber ist (z.~B. von einer Alarmanlage).}
\end{QQuestion}

}
\only<2>{
\begin{QQuestion}{VE308}{Muss ein Funkamateur eine Störung seines Empfangs durch andere Geräte hinnehmen?}{\textbf{\textcolor{DARCgreen}{Er muss die Störungen grundsätzlich hinnehmen, wenn die störenden Geräte den Anforderungen des Gesetzes über die elektromagnetische Verträglichkeit von Betriebsmitteln (EMVG) oder des Funkanlagengesetzes (FuAG) genügen.}}}
{Er muss Störungen nicht hinnehmen.}
{Er muss die Störungen in jedem Fall hinnehmen.}
{Er muss die Störungen grundsätzlich hinnehmen, wenn das störende Gerät von erheblicher Bedeutung für den Betreiber ist (z.~B. von einer Alarmanlage).}
\end{QQuestion}

}
\end{frame}

\begin{frame}
\frametitle{BNetzA einbeziehen}
\begin{itemize}
  \item Über Funkstörannahme der Bundesnetzagentur
  \item Protokoll über Störungen erstellen
  \item Zeitpunkt, Art und vermutete Quelle
  \end{itemize}
\end{frame}

\begin{frame}
\only<1>{
\begin{QQuestion}{VE309}{Der Empfang Ihrer Amateurfunkstation ist wiederkehrend gestört. Die Ursache liegt nicht in Ihrem Haushalt. Sie wollen die Funkstörungsannahme der Bundesnetzagentur informieren. Wie sollten Sie die Bearbeitung durch die Behörde unterstützen?}{Ich sammele die Kontaktdaten aller Nachbarn und melde diese per E-Mail.}
{Ich sende bei jedem einzelnen Auftreten der Störung eine E-Mail.}
{Ich dränge auf ein schnelles Ausrücken des Prüf- und Messdienstes und frage regelmäßig telefonisch nach dem Stand.}
{Ich fertige ein Protokoll mit Zeitpunkt und Art der Störungen an und benenne die vermutete Quelle.}
\end{QQuestion}

}
\only<2>{
\begin{QQuestion}{VE309}{Der Empfang Ihrer Amateurfunkstation ist wiederkehrend gestört. Die Ursache liegt nicht in Ihrem Haushalt. Sie wollen die Funkstörungsannahme der Bundesnetzagentur informieren. Wie sollten Sie die Bearbeitung durch die Behörde unterstützen?}{Ich sammele die Kontaktdaten aller Nachbarn und melde diese per E-Mail.}
{Ich sende bei jedem einzelnen Auftreten der Störung eine E-Mail.}
{Ich dränge auf ein schnelles Ausrücken des Prüf- und Messdienstes und frage regelmäßig telefonisch nach dem Stand.}
{\textbf{\textcolor{DARCgreen}{Ich fertige ein Protokoll mit Zeitpunkt und Art der Störungen an und benenne die vermutete Quelle.}}}
\end{QQuestion}

}
\end{frame}%ENDCONTENT
