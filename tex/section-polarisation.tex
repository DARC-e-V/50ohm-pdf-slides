
\section{Polarisation}
\label{section:polarisation}
\begin{frame}%STARTCONTENT

\begin{columns}
    \begin{column}{0.48\textwidth}
    \begin{itemize}
  \item Polarisation kann \emph{vertikal} oder \emph{horizontal} sein
  \item Lässt sich bei den meisten Antennen leicht erkennen
  \item Auf VHF und höher sollten alle die gleiche Polarisation verwenden
  \end{itemize}

    \end{column}
   \begin{column}{0.48\textwidth}
       \begin{itemize}
  \item \emph{Zirkular} polarisiert
  \item Drehende Funkwellen mit besonderer Antennenbauform
  \item Unterscheidung in \enquote{linkszirkular} und \enquote{rechtszirkular} polarisiert
  \end{itemize}

   \end{column}
\end{columns}

\end{frame}

\begin{frame}
\only<1>{
\begin{QQuestion}{NB304}{Welche Polarisationen unterscheidet man üblicherweise bei der Funkwellenausbreitung im Amateurfunk und wieso sollte man diese beachten?}{Man unterscheidet parallele, koaxiale und drahtlose Polarisation. Die Polarisation der Antennenkabel muss auf die Antennen abgestimmt sein, um Verluste zu minimieren.}
{Man unterscheidet transversale, longitudinale und orthogonale Polarisation. Die Polarisation des Funkgeräts muss an das Stromnetz angepasst sein, um Kurzschlüsse zu vermeiden.}
{Man unterscheidet kohärente, inkohärente und korrelierte Polarisation. Die Polarisation der Funkwellen sollte regelmäßig geändert werden, um die Störfestigkeit zu erhöhen.}
{Man unterscheidet horizontale, vertikale sowie links- und rechtszirkulare Polarisation. Die Polarisation von Sende- und Empfangsantenne sollten angeglichen sein, um eine verlustarme Übertragung zu gewährleisten.}
\end{QQuestion}

}
\only<2>{
\begin{QQuestion}{NB304}{Welche Polarisationen unterscheidet man üblicherweise bei der Funkwellenausbreitung im Amateurfunk und wieso sollte man diese beachten?}{Man unterscheidet parallele, koaxiale und drahtlose Polarisation. Die Polarisation der Antennenkabel muss auf die Antennen abgestimmt sein, um Verluste zu minimieren.}
{Man unterscheidet transversale, longitudinale und orthogonale Polarisation. Die Polarisation des Funkgeräts muss an das Stromnetz angepasst sein, um Kurzschlüsse zu vermeiden.}
{Man unterscheidet kohärente, inkohärente und korrelierte Polarisation. Die Polarisation der Funkwellen sollte regelmäßig geändert werden, um die Störfestigkeit zu erhöhen.}
{\textbf{\textcolor{DARCgreen}{Man unterscheidet horizontale, vertikale sowie links- und rechtszirkulare Polarisation. Die Polarisation von Sende- und Empfangsantenne sollten angeglichen sein, um eine verlustarme Übertragung zu gewährleisten.}}}
\end{QQuestion}

}
\end{frame}%ENDCONTENT
