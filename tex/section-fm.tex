
\section{Frequenzmodulation (FM)}
\label{section:fm}
\begin{frame}%STARTCONTENT

\begin{figure}
    \DARCimage{0.85\linewidth}{357include}
    \caption{\scriptsize Frequenzmodulation}
    \label{n_fm_frequenzmodulation}
\end{figure}

\begin{itemize}
  \item Modulationssignal wird durch Änderung der Frequenz auf den Träger aufmoduliert
  \item Amplitude des Trägers wird nicht verändert und bleibt idealerweise konstant
  \end{itemize}
\end{frame}

\begin{frame}
\only<1>{
\begin{QQuestion}{NE301}{Welche Aussage zur Frequenzmodulation ist richtig? Durch das Informationssignal~...}{wird die Frequenz des Trägers beeinflusst. Die Amplitude des Trägers bleibt dabei konstant.}
{wird die Amplitude des Trägers beeinflusst. Die Frequenz des Trägers bleibt dabei konstant. }
{werden gleichzeitig Frequenz und Amplitude des Trägers beeinflusst.}
{wird zuerst die Frequenz und dann die Amplitude des Trägers beeinflusst.}
\end{QQuestion}

}
\only<2>{
\begin{QQuestion}{NE301}{Welche Aussage zur Frequenzmodulation ist richtig? Durch das Informationssignal~...}{\textbf{\textcolor{DARCgreen}{wird die Frequenz des Trägers beeinflusst. Die Amplitude des Trägers bleibt dabei konstant.}}}
{wird die Amplitude des Trägers beeinflusst. Die Frequenz des Trägers bleibt dabei konstant. }
{werden gleichzeitig Frequenz und Amplitude des Trägers beeinflusst.}
{wird zuerst die Frequenz und dann die Amplitude des Trägers beeinflusst.}
\end{QQuestion}

}
\end{frame}

\begin{frame}
\only<1>{
\begin{QQuestion}{NE302}{Welche Antwort beschreibt die Modulationsart \glqq FM\grqq{}?}{Die Richtung eines Trägersignals wird anhand eines zu übertragenden Signals verändert.}
{Die Amplitude eines Trägersignals wird anhand eines zu übertragenden Signals verändert.}
{Die Polarisation eines Trägersignals wird anhand eines zu übertragenden Signals verändert.}
{Die Frequenz eines Trägersignals wird anhand eines zu übertragenden Signals verändert.}
\end{QQuestion}

}
\only<2>{
\begin{QQuestion}{NE302}{Welche Antwort beschreibt die Modulationsart \glqq FM\grqq{}?}{Die Richtung eines Trägersignals wird anhand eines zu übertragenden Signals verändert.}
{Die Amplitude eines Trägersignals wird anhand eines zu übertragenden Signals verändert.}
{Die Polarisation eines Trägersignals wird anhand eines zu übertragenden Signals verändert.}
{\textbf{\textcolor{DARCgreen}{Die Frequenz eines Trägersignals wird anhand eines zu übertragenden Signals verändert.}}}
\end{QQuestion}

}
\end{frame}

\begin{frame}
\only<1>{
\begin{QQuestion}{NE303}{Welche Auswirkung hat Frequenzmodulation (FM) auf die Amplitude des Sendesignals?}{Je schneller die Schwingung des Modulationssignals ist, umso größer wird die Amplitude des Sendesignals.}
{Idealerweise entspricht die Amplitude des Sendesignals der Amplitude des Modulationssignals.}
{Idealerweise hat das Modulationssignal keine Auswirkung auf die Amplitude des Sendesignals.}
{Je größer die Amplitude des Modulationssignals ist, umso größer wird die Amplitude des Sendesignals.}
\end{QQuestion}

}
\only<2>{
\begin{QQuestion}{NE303}{Welche Auswirkung hat Frequenzmodulation (FM) auf die Amplitude des Sendesignals?}{Je schneller die Schwingung des Modulationssignals ist, umso größer wird die Amplitude des Sendesignals.}
{Idealerweise entspricht die Amplitude des Sendesignals der Amplitude des Modulationssignals.}
{\textbf{\textcolor{DARCgreen}{Idealerweise hat das Modulationssignal keine Auswirkung auf die Amplitude des Sendesignals.}}}
{Je größer die Amplitude des Modulationssignals ist, umso größer wird die Amplitude des Sendesignals.}
\end{QQuestion}

}
\end{frame}

\begin{frame}
\frametitle{Frequenzhub}
\begin{itemize}
  \item Je lauter in das Mikrofon gesprochen wird, umso größer die Änderung der Trägerfrequenz
  \item Dadurch steigt auch die belegte Bandbreite der Aussendung
  \item Maximalwert der Änderung der Trägerfrequenz wird als \emph{Frequenzhub} oder kurz \emph{Hub} bezeichnet
  \item In der Praxis kommt Schmalband-FM (englisch Narrow- FM, kurz NFM) mit \qty{12}{\kilo\hertz} Bandbreite zum Einsatz
  \end{itemize}

\end{frame}

\begin{frame}
\only<1>{
\begin{QQuestion}{BC216}{Warum sollten Sie bei FM-Telefonie auf \qty{145,525}{\MHz} darauf achten, ihr Funkgerät auf Schmalband-FM (Narrow FM) einzustellen? Der IARU-Bandplan empfiehlt~...}{ein Kanalraster von \qty{5}{\kHz} einzuhalten.}
{in diesem Frequenzbereich nicht mehr als \qty{25}{\kHz} Bandbreite zu belegen.}
{einen Kanalabstand von \qty{50}{\kHz} einzuhalten.}
{in diesem Frequenzbereich nicht mehr als \qty{12}{\kHz} Bandbreite zu belegen.}
\end{QQuestion}

}
\only<2>{
\begin{QQuestion}{BC216}{Warum sollten Sie bei FM-Telefonie auf \qty{145,525}{\MHz} darauf achten, ihr Funkgerät auf Schmalband-FM (Narrow FM) einzustellen? Der IARU-Bandplan empfiehlt~...}{ein Kanalraster von \qty{5}{\kHz} einzuhalten.}
{in diesem Frequenzbereich nicht mehr als \qty{25}{\kHz} Bandbreite zu belegen.}
{einen Kanalabstand von \qty{50}{\kHz} einzuhalten.}
{\textbf{\textcolor{DARCgreen}{in diesem Frequenzbereich nicht mehr als \qty{12}{\kHz} Bandbreite zu belegen.}}}
\end{QQuestion}

}
\end{frame}

\begin{frame}
\only<1>{
\begin{QQuestion}{NE306}{Was kann man tun, wenn der Hub bei einem Handfunkgerät oder Mobil-Transceiver zu groß ist?}{Weniger Leistung verwenden}
{Lauter ins Mikrofon sprechen}
{Leiser ins Mikrofon sprechen}
{Mehr Leistung verwenden}
\end{QQuestion}

}
\only<2>{
\begin{QQuestion}{NE306}{Was kann man tun, wenn der Hub bei einem Handfunkgerät oder Mobil-Transceiver zu groß ist?}{Weniger Leistung verwenden}
{Lauter ins Mikrofon sprechen}
{\textbf{\textcolor{DARCgreen}{Leiser ins Mikrofon sprechen}}}
{Mehr Leistung verwenden}
\end{QQuestion}

}

\end{frame}

\begin{frame}
\only<1>{
\begin{QQuestion}{NE304}{Sie senden mit \qty{2}{\W} in FM auf dem \qty{70}{\cm}-Band. Wie groß ist die angezeigte Sendeleistung, wenn Sie zuerst laut, danach leise und dann nicht mehr in das Mikrofon sprechen?}{zuerst \qty{1}{\W}, dann \qty{0,5}{\W} und zum Schluss \qty{0}{\W}}
{zuerst \qty{2}{\W}, dann \qty{1}{\W} und zum Schluss \qty{0}{\W}}
{immer \qty{2}{\W}}
{immer \qty{1}{\W}}
\end{QQuestion}

}
\only<2>{
\begin{QQuestion}{NE304}{Sie senden mit \qty{2}{\W} in FM auf dem \qty{70}{\cm}-Band. Wie groß ist die angezeigte Sendeleistung, wenn Sie zuerst laut, danach leise und dann nicht mehr in das Mikrofon sprechen?}{zuerst \qty{1}{\W}, dann \qty{0,5}{\W} und zum Schluss \qty{0}{\W}}
{zuerst \qty{2}{\W}, dann \qty{1}{\W} und zum Schluss \qty{0}{\W}}
{\textbf{\textcolor{DARCgreen}{immer \qty{2}{\W}}}}
{immer \qty{1}{\W}}
\end{QQuestion}

}

\end{frame}%ENDCONTENT
