
\section{Standortbescheinigung}
\label{section:standortbescheinigung}
\begin{frame}%STARTCONTENT

\begin{columns}
    \begin{column}{0.48\textwidth}
    \begin{itemize}
  \item Eine Standortbescheinigung kann auf \emph{Antrag kostenpflichtig} durch die BNetzA ausgestellt werden
  \item Funkamateur muss alle notwendigen Unterlagen und Informationen für die Berechnung bereitstellen
  \end{itemize}

    \end{column}
   \begin{column}{0.48\textwidth}
       
    \pause
    \begin{itemize}
  \item Lageplan
  \item Bauzeichnung mit Montageort der Antennen
  \item Informationen zum Abstrahlverhalten von allen Antennen
  \end{itemize}



   \end{column}
\end{columns}

\end{frame}

\begin{frame}
\only<1>{
\begin{QQuestion}{VC121}{Kann der Funkamateur laut Amateurfunkgesetz (AFuG) eine Standortbescheinigung erhalten?}{Die Standortbescheinigung kann mit der IT-Anwendung \glqq Watt-Wächter\grqq{} erstellt werden.}
{Der Funkamateur kann auch auf Antrag keine Standortbescheinigung der BNetzA erhalten.}
{Die BNetzA stellt auf Antrag eine Standortbescheinigung aus.}
{Die BNetzA stellt mit der Zuteilung des Rufzeichens eine Standortbescheinigung aus.}
\end{QQuestion}

}
\only<2>{
\begin{QQuestion}{VC121}{Kann der Funkamateur laut Amateurfunkgesetz (AFuG) eine Standortbescheinigung erhalten?}{Die Standortbescheinigung kann mit der IT-Anwendung \glqq Watt-Wächter\grqq{} erstellt werden.}
{Der Funkamateur kann auch auf Antrag keine Standortbescheinigung der BNetzA erhalten.}
{\textbf{\textcolor{DARCgreen}{Die BNetzA stellt auf Antrag eine Standortbescheinigung aus.}}}
{Die BNetzA stellt mit der Zuteilung des Rufzeichens eine Standortbescheinigung aus.}
\end{QQuestion}

}
\end{frame}

\begin{frame}
\frametitle{Verpflichtende Standortbescheinigung}
Verpflichtend ist eine Standortbescheinigung, wenn sich am Standort der vorgesehenen ortsfesten Amateurfunkstelle bereits ortsfeste Funkanlagen befinden, die selbst eine Standortbescheinigung benötigen.

\end{frame}

\begin{frame}
\only<1>{
\begin{QQuestion}{VE519}{Kann die Bundesnetzagentur für den Betrieb einer ortsfesten Amateurfunkstelle eine Standortbescheinigung fordern?}{Nein, für Amateurfunkanlagen gilt das Anzeigeverfahren}
{Nur wenn sich am Standort der vorgesehenen ortsfesten Amateurfunkstelle bereits ortsfeste Funkanlagen befinden, die selbst eine Standortbescheinigung benötigen.}
{Nur wenn die Amateurfunkstelle gewerblich genutzt wird}
{Ja, wenn die effektive Strahlungsleistung der Amateurfunkstelle \qty{750}{\W} überschreitet}
\end{QQuestion}

}
\only<2>{
\begin{QQuestion}{VE519}{Kann die Bundesnetzagentur für den Betrieb einer ortsfesten Amateurfunkstelle eine Standortbescheinigung fordern?}{Nein, für Amateurfunkanlagen gilt das Anzeigeverfahren}
{\textbf{\textcolor{DARCgreen}{Nur wenn sich am Standort der vorgesehenen ortsfesten Amateurfunkstelle bereits ortsfeste Funkanlagen befinden, die selbst eine Standortbescheinigung benötigen.}}}
{Nur wenn die Amateurfunkstelle gewerblich genutzt wird}
{Ja, wenn die effektive Strahlungsleistung der Amateurfunkstelle \qty{750}{\W} überschreitet}
\end{QQuestion}

}
\end{frame}

\begin{frame}
\only<1>{
\begin{QQuestion}{VE518}{Sie wollen eine Amateurfunkstelle an einem Standort errichten, an dem sich bereits andere ortsfeste Funkanlagen befinden. Welche Besonderheit müssen Sie in Bezug auf den Schutz von Personen in elektromagnetischen Feldern beachten?}{Sofern die Senderausgangsleistung der Amateurfunkstelle 10 W überschreitet, darf sie an diesem Standort nicht betrieben werden.}
{Sofern die Gesamtleistung aller Funkanlagen am Standort 10 W EIRP erreicht oder überschreitet, ist eine Standortbescheinigung erforderlich.}
{Es ist unzulässig, eine Amateurfunkstelle an einem Standort zu betreiben, an dem sich auch Funkanlagen anderer Funkdienste befinden.}
{Es ist ein mechanischer Sendeumschalter erforderlich, der verhindert, dass die Amateurfunkanlage gleichzeitig mit einer der anderen Funkanlagen sendet.}
\end{QQuestion}

}
\only<2>{
\begin{QQuestion}{VE518}{Sie wollen eine Amateurfunkstelle an einem Standort errichten, an dem sich bereits andere ortsfeste Funkanlagen befinden. Welche Besonderheit müssen Sie in Bezug auf den Schutz von Personen in elektromagnetischen Feldern beachten?}{Sofern die Senderausgangsleistung der Amateurfunkstelle 10 W überschreitet, darf sie an diesem Standort nicht betrieben werden.}
{\textbf{\textcolor{DARCgreen}{Sofern die Gesamtleistung aller Funkanlagen am Standort 10 W EIRP erreicht oder überschreitet, ist eine Standortbescheinigung erforderlich.}}}
{Es ist unzulässig, eine Amateurfunkstelle an einem Standort zu betreiben, an dem sich auch Funkanlagen anderer Funkdienste befinden.}
{Es ist ein mechanischer Sendeumschalter erforderlich, der verhindert, dass die Amateurfunkanlage gleichzeitig mit einer der anderen Funkanlagen sendet.}
\end{QQuestion}

}
\end{frame}%ENDCONTENT
