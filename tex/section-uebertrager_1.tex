
\section{Übertrager I}
\label{section:uebertrager_1}
\begin{frame}%STARTCONTENT

\begin{columns}
    \begin{column}{0.48\textwidth}
    \begin{itemize}
  \item Zwei Spulen auf gemeinsamen Kern magnetisch gekoppelt
  \item Energie wird darüber übertragen
  \item Ändern von Spannungen und Strömen ist möglich
  \item \emph{Übertrager} oder \emph{Transformator} kurz Trafo
  \end{itemize}

    \end{column}
   \begin{column}{0.48\textwidth}
       
\begin{figure}
    \DARCimage{0.85\linewidth}{197include}
    \caption{\scriptsize Schemazeichnung eines Übertragers}
    \label{e_uebertrager}
\end{figure}


   \end{column}
\end{columns}

\end{frame}

\begin{frame}
\frametitle{Übersetzungverhältnis}
\begin{columns}
    \begin{column}{0.48\textwidth}
    \begin{itemize}
  \item Spannungen an den Anschlüssen des Übertragers verhalten sich wie zur Anzahl der Wicklungen
  \end{itemize}
$ü = \dfrac{N_P}{N_S} = \dfrac{U_P}{U_S}$


    \end{column}
   \begin{column}{0.48\textwidth}
       \begin{itemize}
  \item $N_P$: Wicklungen auf der Primärseite
  \item $N_S$: Wicklungen auf der Sekundärseite
  \item $U_P$: Spannung an der Primärseite
  \item $U_S$: Spannung an der Sekundärseite
  \end{itemize}

   \end{column}
\end{columns}

\end{frame}

\begin{frame}
\only<1>{
\begin{PQuestion}{EC401}{Wie hoch ist die Spannung zwischen den Punkten a und b in dieser Schaltung für ein Transformationsverhältnis von 15:1?}{Etwa \qty{1}{\V}}
{Etwa \qty{15}{\V}}
{Etwa \qty{22}{\V}}
{Etwa \qty{11}{\V}}
{\DARCimage{0.5\linewidth}{197include}}\end{PQuestion}

}
\only<2>{
\begin{PQuestion}{EC401}{Wie hoch ist die Spannung zwischen den Punkten a und b in dieser Schaltung für ein Transformationsverhältnis von 15:1?}{Etwa \qty{1}{\V}}
{\textbf{\textcolor{DARCgreen}{Etwa \qty{15}{\V}}}}
{Etwa \qty{22}{\V}}
{Etwa \qty{11}{\V}}
{\DARCimage{0.5\linewidth}{197include}}\end{PQuestion}

}
\end{frame}

\begin{frame}
\only<1>{
\begin{QQuestion}{EC402}{Die Primärspule eines Übertragers hat die fünffache Anzahl von Windungen der Sekundärspule. Wie hoch ist die erwartete Sekundärspannung, wenn die Primärspule an eine \qty{230}{\V}~Spannungsversorgung angeschlossen wird?}{\qty{46}{\V}}
{\qty{9,2}{\V}}
{\qty{23}{\V}}
{\qty{1150}{\V}}
\end{QQuestion}

}
\only<2>{
\begin{QQuestion}{EC402}{Die Primärspule eines Übertragers hat die fünffache Anzahl von Windungen der Sekundärspule. Wie hoch ist die erwartete Sekundärspannung, wenn die Primärspule an eine \qty{230}{\V}~Spannungsversorgung angeschlossen wird?}{\textbf{\textcolor{DARCgreen}{\qty{46}{\V}}}}
{\qty{9,2}{\V}}
{\qty{23}{\V}}
{\qty{1150}{\V}}
\end{QQuestion}

}
\end{frame}

\begin{frame}
\only<1>{
\begin{QQuestion}{EC403}{An der Primärwicklung eines Transformators mit 600 Windungen liegt eine Spannung von \qty{230}{\V} an. Die Sekundärspannung beträgt \qty{11,5}{\V}. Wie groß ist die Sekundärwindungszahl?}{20 Windungen}
{30 Windungen}
{52 Windungen}
{180 Windungen}
\end{QQuestion}

}
\only<2>{
\begin{QQuestion}{EC403}{An der Primärwicklung eines Transformators mit 600 Windungen liegt eine Spannung von \qty{230}{\V} an. Die Sekundärspannung beträgt \qty{11,5}{\V}. Wie groß ist die Sekundärwindungszahl?}{20 Windungen}
{\textbf{\textcolor{DARCgreen}{30 Windungen}}}
{52 Windungen}
{180 Windungen}
\end{QQuestion}

}
\end{frame}

\begin{frame}
\only<1>{
\begin{QQuestion}{EC404}{An der Primärwicklung eines Transformators mit 150~Windungen liegt eine Spannung von \qty{45}{\V} an. Die Sekundärspannung beträgt \qty{180}{\V}. Wie groß ist die Sekundärwindungszahl?}{850~Windungen}
{600~Windungen}
{38~Windungen}
{30~Windungen}
\end{QQuestion}

}
\only<2>{
\begin{QQuestion}{EC404}{An der Primärwicklung eines Transformators mit 150~Windungen liegt eine Spannung von \qty{45}{\V} an. Die Sekundärspannung beträgt \qty{180}{\V}. Wie groß ist die Sekundärwindungszahl?}{850~Windungen}
{\textbf{\textcolor{DARCgreen}{600~Windungen}}}
{38~Windungen}
{30~Windungen}
\end{QQuestion}

}
\end{frame}%ENDCONTENT
