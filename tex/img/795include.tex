% Author: Stephan Kregel, DG1HXJ
% Year: 2024      
% Schaltung: Einweggleichrichter, Halbwelle

\begin{circuitikz}[background rectangle/.style={fill=white}, show background rectangle, european]
    \draw (0,0) to node[transformer core, american, name=t1]{} ++(0,0);
    \draw (t1.A2) -- node[ocirc]{} ++(-0.1,0) coordinate(t11);
    \draw (t1.A1) -- node[ocirc]{} ++(-0.1,0) coordinate(t12);
    \draw (t1.B1) -- ++(1,0) to [D, l=$D$] ++(2,0) coordinate(d1) -- ++(2,0) coordinate(rl1);
    \draw (t1.B2) -- ++(3,0) coordinate(d2) -- ++(2,0) coordinate(rl2);
    \draw (d1) to [C,l_=$C_L$, *-*] (d2);
    \draw (rl1) to [R, l_=$R_L$, v^=$U_L$, voltage=straight] (rl2);
    \draw (t1.B1) -- ++(0,0) to[open, v^=$U_E$, voltage=straight] (t1.B2);
    \draw (t1.A1) to[open, name=h1] (t1.A2);
    \def\x{0.15}
    \draw[] (h1.center) sin ++(\x,-\x)
                        cos ++(\x, \x);                      
    \draw[] (h1.center) sin ++(-\x, \x)
                        cos ++(-\x,-\x);                
\end{circuitikz}