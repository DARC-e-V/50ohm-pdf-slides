% Author: Prof. Dr. Matthias Jung, DL9MJ
% Year: 2024
\begin{circuitikz}
    \ctikzset{
        resistors/scale=\getDarcImageFactor,
        capacitors/scale=\getDarcImageFactor,
        inductors/scale=\getDarcImageFactor,
        blocks/scale=\getDarcImageFactor,
        RF/scale=\getDarcImageFactor,
        switches/scale=\getDarcImageFactor,
        misc/scale=\getDarcImageFactor
    }
    \draw(0,0) node[antenna]{}
        to [amp, box, >, l_=HF, name={amp1}, fill=DARCblue] ++ (2,0)
        node[inputarrow](){}
        node[mixer, box, anchor=west] (mix1) {};
    \draw(mix1.east)
        to [amp, box, >, l_=ZF, name={amp2}] ++ (2,0)
        node[inputarrow](){}
        node[mixer, box, anchor=west] (mix2) {};
    \draw(mix2.east)
        to [amp, box, >, l_=NF, name={amp3}] ++ (2,0)
        node[inputarrow](){}
        node[loudspeakershape, anchor=south, rotate=-90](L){};
    \draw(mix1.south) node[inputarrow, rotate=90] {} 
        to [short] ++(0,-1)
        node [vallpassshape, anchor=north](x){};
    \draw(mix2.south) node[inputarrow, rotate=90] {} 
        to [short] ++(0,-1)
        node [allpassshape, anchor=north](z){};
    \draw(x.east) node[anchor=west](){VFO};
    \draw(z.east) node[anchor=west](){BFO};
\end{circuitikz}