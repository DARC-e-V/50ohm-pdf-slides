% Author: Dr. Matthias Jung, DL9MJ
% Year: 2022
\begin{circuitikz}
    \draw (0.0,1.2) to [short, i={~}, name={foo}] ++(0.666666,0)
                    to [stroke diode] ++(0.666666,0)
                    to [short, i={~}, name ={bar}] ++(0.666666,0);
    \draw (0.0,0.0) to [short] ++(2,0);
    \draw (foo.center) node[anchor=south east](){Anode};
    \draw (bar.center) node[anchor=south west](){Kathode};
    %
    \draw[red, thick] (1.280,0) to [short] ++(-1.0,-0.626) coordinate(node);
    \draw[red, thick] (1.280,0) to [short] ++(-1.0, 0.626);
    \draw[red, thick] (0.720,0) to [short] ++(0.00, 0.35);
    \draw[red, thick] (0.720,0) to [short] ++(0.00,-0.35);
    %
    \draw[red] (node) node[rotate=-90, anchor=west]{node};
    %
    \draw[blue, thick] (1.280,0) to [short] ++(0.00, 0.35);
    \draw[blue, thick] (1.280,0) to [short] ++(0.00,-0.35);
    \draw[blue, thick] (1.280,0) to [short] ++(0.35, 0.35);
    \draw[blue, thick] (1.280,0) to [short] ++(0.35,-0.35) coordinate(athode);
    %
    \draw[blue] (athode) node[anchor=west]{athode};
\end{circuitikz}