% Author: Dr. Matthias Jung, DL9MJ
% Year: 2022
\begin{circuitikz}
    \draw (0,2) node[antenna]{}
                to [short, -*] ++(1.5,0) coordinate(n1)
                to [short] ++(1,0);
    \draw (0,0) node[ground]{}
                to [short,-*] ++(1.5,0) coordinate(n2)
                to [short] ++(1,0);

    \draw[draw=black] (2.5,-0.5) rectangle ++(1.5,3);
    \draw (3.25,1.0) node () {RX};

    \draw(n1)
        |- ++(-0.15,-1.0)
        node[tlineshape, rotate=180, anchor=west](foo){};
    \draw(foo.top left)
        to [short, *-] ++(0,0)
        -| (n2)
        to [short, -*] ++(0,0);

    \draw(foo.east)
        to [short] ++(-0.25,0)
        |- (foo.top right)
        to [short, -*] ++(0,0);

    \draw(-2,1.5) node {Elektrische Länge:};
    \draw(-2,1.0) node {1/4 der Wellenlänge};
    \draw(-2,0.5) node {der Störfrequenz};

    \draw [-{Triangle}] (-0.5,1.5) -- ++(0.7,-0.4);
\end{circuitikz}