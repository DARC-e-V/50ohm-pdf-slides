% Author: Dr. Matthias Jung, DL9MJ
% Year: 2020
\begin{circuitikz}
    \ctikzset{
        resistors/scale=\getDarcImageFactor,
        capacitors/scale=\getDarcImageFactor,
        inductors/scale=\getDarcImageFactor,
        blocks/scale=\getDarcImageFactor,
        RF/scale=\getDarcImageFactor,
        switches/scale=\getDarcImageFactor,
        misc/scale=\getDarcImageFactor
    }
    \draw(0,0) node[antenna]{}
        to [amp, box, >, l_=HF, name={amp1}] ++ (2,0)
        node[inputarrow](){}
        node[mixer, box, anchor=west] (mix1) {};
    \draw(mix1.east)
        to [amp, box, >, l_=1.ZF, name={amp2}] ++ (2,0)
        node[inputarrow](){}
        node[mixer, box, anchor=west] (mix2) {};
    \draw(mix2.east)
        to [amp, box, >, l_=2.ZF, name={amp3}] ++ (2,0)
        node[inputarrow](){}
        node[mixer, box, anchor=west](mix3){};
    \draw(mix3.east)
        to [amp, box, >, l_=NF, name={amp4}] ++ (2,0)
        node[inputarrow](){}
        node[loudspeakershape, anchor=south, rotate=-90](L){};

    \draw(mix1.south) node[inputarrow, rotate=90] {} 
        to [short] ++(0,-1)
        node [vallpassshape, anchor=north](x){};

    \draw(mix2.south) node[inputarrow, rotate=90] {} 
        to [short] ++(0,-1)
        node [qgeneratorshape, t={G}, anchor=north](y){};

    \draw(mix3.south) node[inputarrow, rotate=90] {} 
        to [short] ++(0,-1)
        node [allpassshape, anchor=north](z){};

    \draw(mix3.north) to [short] ++(0,1) coordinate(e1);
    \draw(e1) -| (amp3.north) node[inputarrow,rotate=-90](){};
    \draw(e1-|amp3.north) node[circ](){} -| (amp2.north) node[inputarrow,rotate=-90](){};
    \draw(e1-|amp2.north) node[circ](){} -| (amp1.north) node[inputarrow,rotate=-90](){};

    \draw(x.east) node[anchor=west](){X};
    \draw(y.east) node[anchor=west](){Y};
    \draw(z.east) node[anchor=west](e){Z};

    %Help Coordinates
    %\path(mix3.north) ++(+0.00,+0.25) coordinate(n);
    %\path(z.south)    ++(+0.00,-0.25) coordinate(s);
    %\path(z.west)     ++(-0.25,-0.00) coordinate(w);

    %\draw[dashed](n)
    %    -| (w) 
    %    |- (s)
    %    -| (e.south) (e.north)
    %    |- (n);
    %\draw(z.south) ++(0,-0.25) node[anchor=north](){Produktdetektor};
\end{circuitikz}