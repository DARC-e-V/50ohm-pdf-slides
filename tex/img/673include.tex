% Author: Dr. Matthias Jung, DL9MJ
% Year: 2022
\begin{circuitikz}
    \draw[dashed] (0,0) -- (-1.0,0);
    \draw[>=triangle 60, ->] (0,0) 
        -- ++(1,0) coordinate(c1)
        -- ++(3,0) coordinate(c2)
        -- ++(1,0) coordinate(c3)
        -- ++(1,0) coordinate(c4)
        -- ++(1,0) coordinate(c5)
        -- ++(0,0) coordinate(cx)
        -- ++(1,0) 
        node[anchor=west](){$f$};
        
    \draw(c1) -- ++(0,-0.25)
        node[anchor=north](n1){\qty{28,410}{\mega\hertz}};
    \draw(c2) -- ++(0,-0.25)
        node[anchor=north](n2){\qty{28,420}{\mega\hertz}};
    \draw(cx) -- ++(0,-0.25)
        node[anchor=north](n3){\qty{28,430}{\mega\hertz}};
        
    \draw(c3) -- ++(0,-0.15);
    \draw(c3) -- ++(0,+0.15);
    \draw(c4) -- ++(0,-0.15);
    \draw(c4) -- ++(0,+0.15);
    \draw(c5) -- ++(0,-0.15);
    \draw(c5) -- ++(0,+0.15);
        
    \draw(c1) -- ++(0,+0.25)
        node[anchor=south, align=center](n3){begehrte\\Station};
    \draw(c2) -- ++(0,+0.25) to [open] ++(0,0.1) coordinate(d1);
    \draw(cx) -- ++(0,+0.25) to [open] ++(0,0.1) coordinate(d2);
        
    \draw [decorate, decoration = {brace}] (d1) -- (d2)
    node[pos=0.5, above=0.15, black, align=center]{Antwort-\\bereich};
\end{circuitikz}