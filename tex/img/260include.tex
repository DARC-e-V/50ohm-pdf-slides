% Author: Dr. Matthias Jung, DL9MJ
% Year: 2021
\begin{circuitikz}
    \ctikzset{
        capacitors/scale=\getDarcImageFactor,
        inductors/scale=\getDarcImageFactor,
        RF/scale=\getDarcImageFactor,
    }

    \draw (1.5,2)
        node[tlineshape, anchor=east](bar){}
        to [american inductor, inductors/coils=3 , inductors/width=0.60, name={lo}, mirror] ++ (1.5,0) coordinate(x5)
        to [tline, name={tl2}, bipoles/tline/width=1.0] ++(2,0) coordinate(c1)
        to [short] ++(0.5,0)
        to [american inductor, inductors/coils=2 , inductors/width=0.45] ++(0,-1) coordinate(x0);
    \draw [dashed] (bar.west) -- ++(-0.5,0);
    \draw (c1) to [C, capacitors/scale=0.5, *-*, name={h4}] (c1|-x0) coordinate(x2);
    \draw (bar.bottom right)
        to [short, *-] (bar.bottom right|-x0) coordinate(x1)
        to [short] (x1-|bar.east) coordinate(x6)
        to [american inductor, inductors/coils=3 , inductors/width=0.60, name={lu}] (x6-|x5)
        -| (tl2.bottom left)
        to [short, -*] ++(0,0);
    \draw(tl2.bottom right)
        to [short, -*] ++(0,0)
        |- (x2)
        to [short, -*] (x0)
        to [short, name={h1}] ++(0.75,0) coordinate(z)
        to [american inductor, inductors/coils=14, inductors/width=3.25, name={l2}] ++(0,5) coordinate(feed);
    \draw[very thick]  (feed) -- ++ (5.5,0) coordinate(end);
    \draw[dashed] (l2.core west-|h1.center) coordinate(h2)
         to [short] (h2|-l2.core east);
    \draw[dashed] (lo.core west|-h4.center) coordinate(h3)
         to [short] (h3-|lo.core east);

    \DARCmeasure{feed}{end}{0,1.1}{0}{$\approx\frac{\lambda}{2}$}
    \DARCmeasure[black][top]{tl2.top left}{tl2.top right}{0,0.5}{0}{$\qty{0,05}\cdot\lambda$}

    \draw (lu.south) node[anchor=north]{\small{MWS}};
    \draw (bar.north) node[anchor=south]{\small{$\qty{50}{\ohm}$}};
    \draw (l2.south) node[anchor=west]{\small{ü~=~${1}:{7}$}};
\end{circuitikz}