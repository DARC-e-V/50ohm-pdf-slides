% Author: Dr. Matthias Jung, DL9MJ
% Year: 2020
\begin{tikzpicture} [
    pencildraw/.style={
      black,
      decorate,
      decoration={random steps,segment length=5pt,amplitude=2pt}
    }
]
\def\s{200}

% Innen unten:
\begin{scope}[shift={(0,-0.85)}]
    \draw[isometric view](0,0) plot [domain=360*2.694:360*2.787,samples=\s] ({\x/360*cos(-\x)},{\x/360*sin(-\x)});
    \draw[isometric view](0,0) plot [domain=360*1.723:360*1.817,samples=\s] ({\x/360*cos(-\x)},{\x/360*sin(-\x)});
    \draw[isometric view](0,0) plot [domain=360*1.968:360*2.068,samples=\s] ({\x/360*cos(-\x)},{\x/360*sin(-\x)});
    \draw[isometric view](0,0) plot [domain=360*0.905:360*1.025,samples=\s] ({\x/360*cos(-\x)},{\x/360*sin(-\x)});
\end{scope}

% Außen oben:
\draw[isometric view](0,0) % Anfang und Ende
    coordinate(sA)
    plot [domain=0:360*2.89,samples=\s] ({\x/360*cos(-\x)},{\x/360*sin(-\x)})
    coordinate(eA);

% Außen unten:
\begin{scope}[shift={(0,-4)}]
    \draw[isometric view](0,0)
        plot [domain=360*2.14:360*2.63,samples=\s] ({\x/360*cos(-\x)},{\x/360*sin(-\x)});
\end{scope}


% Koordinaten Außen:
\path[isometric view](0,0) plot [domain=0:360*2.14,samples=\s] ({\x/360*cos(-\x)},{\x/360*sin(-\x)}) coordinate(a1);
\path[isometric view](0,0) plot [domain=0:360*2.63,samples=\s] ({\x/360*cos(-\x)},{\x/360*sin(-\x)}) coordinate(a2);
\path[isometric view](0,0) plot [domain=0:360*1.63,samples=\s] ({\x/360*cos(-\x)},{\x/360*sin(-\x)}) coordinate(a3);
\path[isometric view](0,0) plot [domain=0:360*1.14,samples=\s] ({\x/360*cos(-\x)},{\x/360*sin(-\x)}) coordinate(a4);
\path[isometric view](0,0) plot [domain=0:360*0.63,samples=\s] ({\x/360*cos(-\x)},{\x/360*sin(-\x)}) coordinate(a5);
\path[isometric view](0,0) plot [domain=0:360*0.00,samples=\s] ({\x/360*cos(-\x)},{\x/360*sin(-\x)}) coordinate(a6);
\path[isometric view](0,0) plot [domain=0:360*2.00,samples=\s] ({\x/360*cos(-\x)},{\x/360*sin(-\x)}) coordinate(a7);
\path[isometric view](0,0) plot [domain=0:360*0.23,samples=\s] ({\x/360*cos(-\x)},{\x/360*sin(-\x)}) coordinate(a8);

% Innen:
\draw[isometric view](0,0)
    coordinate(sI)
    plot [domain=0:360*2.89,samples=\s] ({0.95*\x/360*cos(-\x)},{0.95*\x/360*sin(-\x)})
    coordinate(eI);

% Koordinaten Innen:
\path[isometric view](0,0) plot [domain=0:360*2.495,samples=\s] ({0.95*\x/360*cos(-\x)},{0.95*\x/360*sin(-\x)}) coordinate(i3);
\path[isometric view](0,0) plot [domain=0:360*2.288,samples=\s] ({0.95*\x/360*cos(-\x)},{0.95*\x/360*sin(-\x)}) coordinate(i4);
\path[isometric view](0,0) plot [domain=0:360*1.450,samples=\s] ({0.95*\x/360*cos(-\x)},{0.95*\x/360*sin(-\x)}) coordinate(i5);
\path[isometric view](0,0) plot [domain=0:360*0.378,samples=\s] ({0.95*\x/360*cos(-\x)},{0.95*\x/360*sin(-\x)}) coordinate(i6);
\path[isometric view](0,0) plot [domain=0:360*1.353,samples=\s] ({0.95*\x/360*cos(-\x)},{0.95*\x/360*sin(-\x)}) coordinate(i8);

% Außenwände:
\draw(a1) -- ++(0,-4) coordinate(a11);
\draw(a2) -- ++(0,-4) coordinate(a22);
\draw(a3) -- (i3);
\draw(a4) -- (i4);
\draw(a5) -- (i5);
\draw(a6) -- (i6);
\draw(a8) -- (i8);
\draw(eI) -- ++(4,0) coordinate(eII);
\pgfmathsetseed{43} % Um die Form gleich zu halten
\draw[pencildraw](eII) -- ++(0,-4) coordinate(eIII); 
\draw(eIII) -| (a11);
\draw(eA) -- ++(5,0) |- (eIII);
\draw(a7) -- ++(2,0) coordinate(x1) -- ++(0,-2.5) coordinate(x2) -| (a11);
\draw(x1) to [open] ++(0.9,0) -- ++(0.5,0) -- ++(0,-2.5) -- ++(-0.5,0) to [open] (x2);

% Beschriftung:
\draw(2.85,-2) -- ++(1,-1) coordinate(m1) -- ++(1,1);
\draw(m1) node[anchor=north](){Dielektrikum};
\draw(2.6,0.5) -- ++(1,1.5) coordinate(m2) -- ++(1,-1.5);
\draw(m2) node[anchor=south](){Metallbeläge};
\draw(1.5,-3.5) -- ++(1.5,-1) node[anchor=west](){Wickel};
\end{tikzpicture}