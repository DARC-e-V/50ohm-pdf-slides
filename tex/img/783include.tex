% Author: Dr. Matthias Jung, DL9MJ
% Year: 2022
% Relative Applied
\begin{circuitikz}[
        arrowset/.pic={
            \draw[>=Triangle,->] (0.15,0) -- ++(0,0.35);
            \draw[>=Triangle,<-] (0.30,0) -- ++(0,0.35);
        }
    ]
    \ctikzset{
        resistors/scale=\getDarcImageFactor,
        capacitors/scale=\getDarcImageFactor,
        inductors/scale=\getDarcImageFactor,
        transistors/scale=\getDarcImageFactor,
        RF/scale=\getDarcImageFactor,
        chips/scale=\getDarcImageFactor,
        misc/scale=\getDarcImageFactor,
        multipoles/dipchip/width=1.0,
    }
    \draw (0,0) node[dipchip, num pins=14, hide numbers, no topmark, external pins width=0](ea){};
    \draw(ea.bpin 14) 
        to [C, -*] ++(3.5,0) coordinate(c1)
        to [TL] ++(1.5,0) node[nigfete, anchor=gate, solderdot](T1){};
    \draw(T1.east)
        node[anchor=west](){};

    \draw(ea.bpin 8) 
        to [C, -*] ++(3.5,0) coordinate(c2)
        to [TL] ++(1.5,0) node[nigfete, anchor=gate,solderdot, yscale=-1](T2){};
    \draw(T2.east)
        node[anchor=west](){};

    \draw(T1.G)
        to [open] ++(2.5,0) % hack :-)
        node[dipchip, num pins=14, hide numbers, no topmark, external pins width=0, anchor=bpin 1](aa){};

    % Kondensatoren:
    \draw(T1.D)
        to [short] ++(0,1) coordinate(g1)
        to [C, *-] ++(1,0)
        |- (aa.bpin 1);
    \draw(T2.D) 
        to [short] ++(0,-1) coordinate(g2)
        to [C, *-] ++(1,0)
        |- (aa.bpin 7);

    \draw(aa.bpin 11) 
        to [short] ++(0.5,0) node[bnc, xscale=-1](hfout){};

    \draw(hfout.north)
        node[rotate=90, anchor=west] {$\mathrm{HF}_\mathrm{OUT}$};


    % Verbindung zwischen T1 und T2:
    \draw(T1.S)
        to [short, name=s1] (T2.S); % hack :-)
    \draw(s1.center)
        to [short, *-] ++ (-0.5,0) node[rground]{};

    % Drehko:
    \draw(c1)
        to [vC, tunable end arrow={Bar}, invert] (c2);

    % Spulen:
    \draw(c1)
        to [american inductor, name=l1] ++(0, 5.0)
        node[genericpotentiometershape, anchor=wiper, rotate=180, wiper end arrow={Triangle[]}](pr1){};
    \draw[dashed] (l1.core west) -- (l1.core east);
    \draw(pr1.left)
        to [short] ++(0.5,0) node[rground]{};

    \draw(c2)
        to [american inductor, name=l4, mirror] ++(0,-4.0)
        node[genericpotentiometershape, anchor=wiper, wiper end arrow={Triangle[]}](pr2){};
    \draw[dashed] (l4.core west) -- (l4.core east);
    \draw(pr2.right)
        to [short] ++(0.5,0) node[rground]{};

    \draw(g1)
        to [american inductor, name=l3, -o] ++(0, 3.5)
        node[above]{+\qty{50}{\volt}};
    \draw[dashed] (l3.core west) -- (l3.core east);

    \draw(g2)
        to [american inductor, name=l6, mirror, -o] ++(0,-2.5) node[below]{+\qty{50}{\volt}};
    \draw[dashed] (l6.core west) -- (l6.core east);

    % Potentionmeter mit UBias:
    \draw(pr1.right) 
        to [short, -*] ++(-0.50,0) coordinate(c3)
        to [short, -*] ++(-1.25,0) coordinate(c4)
        to [short, -*] ++(-1.25,0) coordinate(c5)
        to [R, -o] ++(-2,0) coordinate(g4)
        ++(0,0.25)
        node[anchor=south west]{$+\qty{5}{\volt}$};

    \draw(ea.bpin 4)
        to [short] ++(-0.5,0) node[bnc](bnc1){};
    \draw(bnc1.north) 
        node [rotate=90, anchor=west]{$\mathrm{HF}_\mathrm{IN}$}; 

    \draw(c3) |- (pr2.left);
    \draw(c4)
        to [C] ++(0,-3.5) node [rground]{};
    \draw(c5)
        to [R] ++(0,-1.75)
        to [thR, name=R] ++(0,-1.75) node[rground]{};
    \draw (R.south west) node[anchor=north east](){$\vartheta$};
    \draw (R.south east) ++(-0.5,0) pic {arrowset};

    % Beschriftungen:
    \draw(ea)  node[rotate=90, align=center](hh1){\scriptsize Eingangs-\\\scriptsize anpassung};
    \draw(aa)  node[rotate=90, align=center](hh3){\scriptsize Ausgangs-\\\scriptsize anpassung};
    \draw(c3)  node[anchor=south east]{$U_\mathrm{BIAS}$};
    \draw(pr1.270) node[above]{$R_1$};
    \draw(pr2.270) node[below]{$R_2$};
\end{circuitikz}