% Author: Dr. Matthias Jung, DL9MJ
% Year: 2020
\begin{circuitikz}
    \ctikzset{
        blocks/scale=\getDarcImageFactor,
        RF/scale=\getDarcImageFactor
    }
    \draw(0,0)
        node[twoportshape, t={PC}](pc){};

    \draw(pc.east)
        node[inputarrow, rotate=180](){}
        to [short] ++(1,0)
        node[twoportshape, t={SDR}, anchor=west](sdr){}
        node[inputarrow](){};

    \draw(sdr.right up) 
        node[inputarrow, rotate=180](){}
        to [short] ++(0.35,0)
        to [short] ++(0,1.25)
        to [short] ++(0.9,0)
        node[biastshape, box, anchor=west](bias){};

    \draw(bias.east)
        node[inputarrow, rotate=180](){}
        to [short] ++(1,0)
        node[twoportshape, t={LNB}, box, anchor=west](lnb){};

    \draw(lnb.east)
        to [short] ++(1,0)
        to [short] ++(0,1)
        node[bareantenna, xscale=-1](rx){};
    \draw(rx.top)
        node[anchor=south](){\qty{10}{\giga\hertz}};

    \draw(sdr.right down) 
        to [short] ++(0.35,0)
        to [short] ++(0,-1.25)
        to [short, name={h1}] ++(0.9,0)
        node[inputarrow](){}
        node[ampshape, box, anchor=west](preamp){};

    \draw(preamp.east)
        to [short, name={h2}] ++(1,0)
        node[inputarrow](){}
        node[ampshape, box, anchor=west](amp){};

    \draw(amp.east)
        to [short, name={h3}] ++(1,0)
        to [short] ++(0,1)
        node[bareantenna](tx){};
    \draw(tx.top)
        node[anchor=south](){\qty{2,4}{\giga\hertz}};

    \draw(sdr.north)
        node[inputarrow, rotate=-90](){}
        to [short] ++(0,2) coordinate(t1)
        node [qgeneratorshape, anchor=south, t={G}](gen1){};

    \draw(bias.north)
        node[inputarrow, rotate=-90](){} coordinate(t2)
        to [short] (t2|-t1)
        node [twoportshape, anchor=south, t={DC}](dc){};

    \draw(lnb.north)
        node[inputarrow, rotate=-90](){} coordinate(t3)
        to [short] (t3|-t1)
        node [qgeneratorshape, anchor=south, t={G}](gen2){};

    \draw[dashed](sdr.south)
        to [short, name={h0}] ++(0,-3) coordinate(b1);
    \draw[dashed](b1)
        -| (preamp.south)
        node[inputarrow, rotate=90](){};
    \draw[dashed](b1-|preamp.south)
        -| (amp.south)
        node[inputarrow, rotate=90](){};

    % Labels:
    \draw(gen1.north) node[anchor=south](){TXCO};
    \draw(gen2.north) node[anchor=south](){TXCO};
    \draw(h0.center)  node[anchor=east] (){PTT};
    \draw(dc.north)   node[anchor=south](){\qty{12}{\volt}};
    \draw(h1.center)  node[anchor=west, rotate=+90](){\footnotesize\qty{-5}{dBm}};
    \draw(h2.center)  node[anchor=west, rotate=+90](){\footnotesize\qty{16}{dBm}};
    \draw(h3.center)  node[anchor=west, rotate=+90](){\footnotesize\qty{43}{dBm}};
    \draw(h1.center)  node[anchor=east, rotate=+90](){\footnotesize\qty{0,3}{\milli\watt}};
    \draw(h2.center)  node[anchor=east, rotate=+90](){\footnotesize\qty{40}{\milli\watt}};
    \draw(h3.center)  node[anchor=east, rotate=+90](){\footnotesize\qty{20}{\watt}};
\end{circuitikz}