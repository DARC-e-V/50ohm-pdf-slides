% Author: Stephan Kregel, DG1HXJ
% Year: 2022
\begin{circuitikz}
    \tikzset{tdax/.style={muxdemux,muxdemux def={NL=1, Lh=6, w=6, NR=1, Rh=0, NB=1, NT=1}, scale=\getDarcImageFactor}}
    \ctikzset{
        resistors/scale=\getDarcImageFactor,
        capacitors/scale=\getDarcImageFactor,
        inductors/scale=\getDarcImageFactor,
        RF/scale=\getDarcImageFactor,
        switches/scale=\getDarcImageFactor,
        misc/scale=\getDarcImageFactor
    }
    \draw (0,0) node[tdax](A){$\text{MMIC}$};
    \draw (-4,0) coordinate(rfin)
        to [C=$C_{1}$, o-] (A.lpin 1);
    \draw (A.rpin 1)
        -- ++(1,0) coordinate(rbias-a) 
        to [C=$C_{3}$, -o] ++(3,0)
        coordinate(rfout);
    \draw (rbias-a)
        to [R=$R_\text{BIAS}$, *-*] ++(0,3) coordinate(rbias-b);
    \draw (rbias-b)
        to [C, l_=$C_{2}$, -*] (rbias-b-|A.tpin 1) coordinate(c0)
        to [short] ++(-1,0)
        to node[rground]{} ++(0,0);
    \draw (A.tpin 1) -- (c0);
    \draw (rbias-b)
        to [american inductor, l_=$R_1$, name={l1}, *-o] ++(3,0)
        coordinate (vcc);
    \draw [dashed] (l1.core west) -- (l1.core east);
    \draw (rfin)  node[rotate={90}, anchor=north east]{$\text{RF}_\text{IN}$};
    \draw (rfout) node[rotate={90}, anchor=south east]{$\text{RF}_\text{OUT}$};
    \draw (vcc)   node[rotate={90}, anchor=south east]{$U_\text{CC}$}++(0,0);
    \draw (A.lpin 1) node[anchor=north]{1};
    \draw (A.bpin 1) node[anchor=west ]{2};
    \draw (A.tpin 1) node[anchor=west ]{4};
    \draw (A.rpin 1) node[anchor=north]{3};
    \draw (rbias-a)
        to [open,-*, v={\rotatebox[origin=c]{90}{$U_\text{D}$~=~\qty{4}{\V}}}] ++(0,-3) coordinate(u0)
        node[rground]{};
    \draw (A.bpin 1)
        to [short] (A.bpin 1|-u0)
        node[rground]{};
\end{circuitikz}