% Author: Prof. Dr. Matthias Jung, DL9MJ
% Year: 2024
\begin{tikzpicture}
    \pgfplotsset{every tick label/.append style={rotate=-90}}
    \begin{axis}[%
        axis lines=middle,
        yticklabel=\empty,
        xticklabel=\empty,
        xmin=-0.5,
        xmax=1,
        ymin=-0.05,
        ymax=0.1,
        ytick={0},
        xtick={0},
        axis line style={-Triangle},
        xticklabel style={/pgf/number format/.cd,fixed,precision=1,use comma,fixed,fixed zerofill},
        xlabel style={at={(ticklabel* cs:1)},anchor=north east},
        ylabel style={at={(ticklabel* cs:1)},anchor=north east},
        ylabel={$I_\mathrm{D}$},
        xlabel={$U_\mathrm{D}$},
        scale only axis,
        height=5cm,
        width=\linewidth*\getDarcImageFactor,
    ]
    % Nach Shockley Gleichung Tieze Schenk (3.2)
    \draw (axis cs: 0.69,0.1) to [short] (axis cs: 0.65,0) node[rotate=0, anchor=north](){$U_\text{th}$};
    \addplot[DARCred, no marks, domain=-0.5:0.75, samples=1000, ultra thick] {10e-12*(exp(x/0.030)-1)};
    \draw[-Triangle] (axis cs: -0.25, 0.01) -- (axis cs: -0.25, 0.001);
    \draw[-Triangle] (axis cs: -0.25,-0.01) node[rotate=0, anchor=north](){$-I_\text{S}$} -- (axis cs: -0.25,-0.001);
    \end{axis}
\end{tikzpicture}%