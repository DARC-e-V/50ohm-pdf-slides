% Author: Prof. Dr. Matthias Jung, DL9MJ
% Year: 2023
\begin{tikzpicture}[american]
	\ctikzset{
	    capacitors/scale=\getDarcImageFactor,
	    inductors/scale=\getDarcImageFactor,
    	misc/scale=\getDarcImageFactor
	}
    \draw(0,0)
        node[transformer core, rotate=90, transform shape](T){};

    % Beschriftung T:
    \draw (T.inner dot A1)  node[circ]{};
    \draw (T.inner dot B1)  node[circ]{};
    \draw (T-L2.south east) node[anchor=south]{1};
    \draw (T-L2.south west) node[anchor=south]{2};
    \draw (T-L1.south west) node[anchor=north]{3};
    \draw (T-L1.south east) node[anchor=north]{4};
    \draw (T.east)          node[]{T};

    % Unten:
    \draw(T.A1)
        to [short] ++(-1,0) coordinate(u1)
        to [short,-o] ++(-1.25,0) coordinate(u0)
        node[anchor=east](){N};
    \draw(T.A2)
        to [short] ++(1.5,0) coordinate(u2)
        to [short, -o] ++(1,0) coordinate(u3)
        node[anchor=west](){N'};

    % Oben:
    \draw(T.B1)
        to [short] ++(-1,0) coordinate(o1)
        to [short,-o] ++(-1.25,0) coordinate(o0)
        node[anchor=east](){L1};
    \draw(T.B2)
        to [short] ++(1.5,0) coordinate(o2)
        to [short, -o] ++(1,0) coordinate(o3)
        node[anchor=west](){L1'};

    % Zwsichen:
    \draw(o0)
        to [open, name={h0}] (u0);
    \draw(o2)
        to [open, name={h2}] (u2);
    \draw(h0.center)
        node[anchor=east](){PE}
        to [short, o-] ++(0.5,0)
        node[rground](){};

    % Rest:
    \draw(o1)
        to [C, l={$C_1$}, *-*] (u1);
    \draw(o2)
        to [C, l={$C_2$}, *-*] (h2.center)
        to [C, l={$C_3$}, *-*] (u2);
    \draw(h2.center)
        to [short] ++(-1,0)
        node[rground](){};
\end{tikzpicture}