% Author: Dr. Matthias Jung, DL9MJ
% Year: 2022
\begin{circuitikz}
    \draw(0,0) coordinate(a1) node[ocirc]{} to [short,-*] ++(3,0) coordinate(n1)
               to [american inductor, inductors/coils=8, inductors/width=1.75, name=L1] ++(0,-3) coordinate(tap)
               to [american inductor, inductors/coils=8, inductors/width=1.75, name=L2] ++(0,-3)
               to [short, -o] ++(3,0) coordinate(b);

    \draw(n1) to [short, -o] ++(3,0) coordinate(a2);

    \draw(tap) 
        to [short,*-*] ++(-1.5,0) coordinate(n2)
        to [short, -o] ++(-1.5,0) coordinate(m);

    \draw(n2) node[ground]{};
    \draw(a1) node[left]{a};
    \draw(m)  node[left]{m};
    \draw(a2) node[right]{a};
    \draw(b)  node[right]{b};
    \draw(L1.midtap) node[left]{8};
    \draw(L2.midtap) node[left]{8};

    \draw(a2) to [open, name=o] (b);
    \draw(o.center) node[]{$Z_\mathrm{ant}=\qty{200}{\ohm}$};

    \draw[dashed] (L1.core west) -- (L2.core east);
\end{circuitikz}