
\section{Standortwahl}
\label{section:standortwahl}
\begin{frame}%STARTCONTENT
\begin{itemize}
  \item Wechselwirkungen mit anderen elektrischen Installationen und Geräten vermeiden
  \item Insbesondere in der eigenen Wohnung und der der Nachbarn
  \item Wechselwirkungen können auch den eigenen Funkempfang stören
  \item Und im Sendebetrieb die Funktionsweise des Geräts stören
  \item $\rightarrow$ Antenne möglichst im Außenbereich anbringen
  \end{itemize}
\end{frame}

\begin{frame}
\only<1>{
\begin{QQuestion}{EG223}{Eine im Außenbereich installierte Sendeantenne hat den Vorteil, dass~...}{sie in geringerem Ausmaß Ausstrahlungen unterworfen ist.}
{die Kopplung mit den elektrischen Leitungen im Haus reduziert wird.}
{sie eine geringere Anzahl von Harmonischen abstrahlt.}
{das Sendesignal einen niedrigeren Pegel aufweist.}
\end{QQuestion}

}
\only<2>{
\begin{QQuestion}{EG223}{Eine im Außenbereich installierte Sendeantenne hat den Vorteil, dass~...}{sie in geringerem Ausmaß Ausstrahlungen unterworfen ist.}
{\textbf{\textcolor{DARCgreen}{die Kopplung mit den elektrischen Leitungen im Haus reduziert wird.}}}
{sie eine geringere Anzahl von Harmonischen abstrahlt.}
{das Sendesignal einen niedrigeren Pegel aufweist.}
\end{QQuestion}

}
\end{frame}

\begin{frame}
\frametitle{Installation Kurzwellenantenne}
\begin{itemize}
  \item Am besten rechtwinklig vom Haus wegführen
  \item Hauptstrahlrichtung zeigt nicht auf das Gebäude und das der Nachbarn
  \item Weniger Störungen durch Einkopplung in die Leitungen im Haus
  \end{itemize}
\end{frame}

\begin{frame}
\only<1>{
\begin{QQuestion}{EJ110}{Ein Funkamateur wohnt in einem Reihenhaus. An welcher Stelle sollte eine Drahtantenne für den Sendebetrieb auf dem \qty{80}{\m}-Band angebracht werden, um störende Beeinflussungen möglichst zu vermeiden?}{Am gemeinsamen Schornstein neben der Fernsehantenne}
{Drahtführung rechtwinklig zur Häuserzeile}
{Entlang der Häuserzeile auf der Höhe der Dachrinne}
{Möglichst innerhalb des Dachbereichs}
\end{QQuestion}

}
\only<2>{
\begin{QQuestion}{EJ110}{Ein Funkamateur wohnt in einem Reihenhaus. An welcher Stelle sollte eine Drahtantenne für den Sendebetrieb auf dem \qty{80}{\m}-Band angebracht werden, um störende Beeinflussungen möglichst zu vermeiden?}{Am gemeinsamen Schornstein neben der Fernsehantenne}
{\textbf{\textcolor{DARCgreen}{Drahtführung rechtwinklig zur Häuserzeile}}}
{Entlang der Häuserzeile auf der Höhe der Dachrinne}
{Möglichst innerhalb des Dachbereichs}
\end{QQuestion}

}
\end{frame}

\begin{frame}
\frametitle{Installation Richtantenne}
\begin{itemize}
  \item Am besten so hoch und so weit weg wie möglich
  \item Feldstärke in Hauptstrahlrichtung nimmt mit der Entfernung ab
  \end{itemize}
\end{frame}

\begin{frame}
\only<1>{
\begin{QQuestion}{EG112}{Welcher Standort ist für eine HF-Richtantenne am besten geeignet, um mögliche Beeinflussungen bei den Geräten des Nachbarn zu vermeiden?}{An der Seitenwand zum Nachbarn}
{So hoch und weit weg wie möglich}
{Auf dem Dach, wobei die Dachfläche des Nachbarn mit abgedeckt werden sollte}
{So niedrig und nah am Haus wie möglich}
\end{QQuestion}

}
\only<2>{
\begin{QQuestion}{EG112}{Welcher Standort ist für eine HF-Richtantenne am besten geeignet, um mögliche Beeinflussungen bei den Geräten des Nachbarn zu vermeiden?}{An der Seitenwand zum Nachbarn}
{\textbf{\textcolor{DARCgreen}{So hoch und weit weg wie möglich}}}
{Auf dem Dach, wobei die Dachfläche des Nachbarn mit abgedeckt werden sollte}
{So niedrig und nah am Haus wie möglich}
\end{QQuestion}

}
\end{frame}%ENDCONTENT
