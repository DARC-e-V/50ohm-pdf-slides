
\section{Temperaturkompensation von Oszillatoren}
\label{section:oszillator_tcxo_ocxo}
\begin{frame}%STARTCONTENT

\only<1>{
\begin{QQuestion}{AF215}{Wie sollte ein bereits temperaturkompensierter VFO innerhalb eines Gerätes verbaut werden, um eine möglichst optimale Frequenzstabilität zu gewährleisten?}{Er sollte möglichst gut thermisch isoliert zu anderen Wärmequellen im Gerät sein.}
{Er sollte auf einem eigenen Kühlkörper montiert sein.}
{Er sollte auf dem gleichen Kühlkörper wie der Leistungsverstärker angebracht werden.}
{Er sollte durch einen kleinen Ventilator separat gekühlt werden. }
\end{QQuestion}

}
\only<2>{
\begin{QQuestion}{AF215}{Wie sollte ein bereits temperaturkompensierter VFO innerhalb eines Gerätes verbaut werden, um eine möglichst optimale Frequenzstabilität zu gewährleisten?}{\textbf{\textcolor{DARCgreen}{Er sollte möglichst gut thermisch isoliert zu anderen Wärmequellen im Gerät sein.}}}
{Er sollte auf einem eigenen Kühlkörper montiert sein.}
{Er sollte auf dem gleichen Kühlkörper wie der Leistungsverstärker angebracht werden.}
{Er sollte durch einen kleinen Ventilator separat gekühlt werden. }
\end{QQuestion}

}
\end{frame}

\begin{frame}
\only<1>{
\begin{QQuestion}{AD602}{Unter einem TCXO versteht man einen~...}{temperaturkompensierten Quarzoszillator.}
{kapazitiv abgestimmten Quarzoszillator.}
{temperaturkompensierten LC-Oszillator.}
{Oszillator, der auf konstanter Temperatur gehalten wird.}
\end{QQuestion}

}
\only<2>{
\begin{QQuestion}{AD602}{Unter einem TCXO versteht man einen~...}{\textbf{\textcolor{DARCgreen}{temperaturkompensierten Quarzoszillator.}}}
{kapazitiv abgestimmten Quarzoszillator.}
{temperaturkompensierten LC-Oszillator.}
{Oszillator, der auf konstanter Temperatur gehalten wird.}
\end{QQuestion}

}
\end{frame}

\begin{frame}
\only<1>{
\begin{QQuestion}{AD603}{Wie nennt man einen temperaturkompensierten Quarzoszillator?}{VCO}
{OCXO}
{TCXO}
{VFO}
\end{QQuestion}

}
\only<2>{
\begin{QQuestion}{AD603}{Wie nennt man einen temperaturkompensierten Quarzoszillator?}{VCO}
{OCXO}
{\textbf{\textcolor{DARCgreen}{TCXO}}}
{VFO}
\end{QQuestion}

}
\end{frame}

\begin{frame}
\only<1>{
\begin{QQuestion}{AD605}{Welcher der angegebenen Oszillatoren hat die größte Frequenzstabilität?}{VCO}
{TCXO}
{OCXO}
{XO}
\end{QQuestion}

}
\only<2>{
\begin{QQuestion}{AD605}{Welcher der angegebenen Oszillatoren hat die größte Frequenzstabilität?}{VCO}
{TCXO}
{\textbf{\textcolor{DARCgreen}{OCXO}}}
{XO}
\end{QQuestion}

}
\end{frame}

\begin{frame}
\only<1>{
\begin{QQuestion}{AD604}{Welcher Oszillator ist für einen SSB-SDR-Sender im \qty{3}{\cm} Band geeignet?}{TCXO}
{VCO}
{LC-Oszillator}
{RC-Oszillator}
\end{QQuestion}

}
\only<2>{
\begin{QQuestion}{AD604}{Welcher Oszillator ist für einen SSB-SDR-Sender im \qty{3}{\cm} Band geeignet?}{\textbf{\textcolor{DARCgreen}{TCXO}}}
{VCO}
{LC-Oszillator}
{RC-Oszillator}
\end{QQuestion}

}
\end{frame}%ENDCONTENT
