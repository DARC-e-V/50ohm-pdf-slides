
\section{Diode II}
\label{section:diode_2}
\begin{frame}%STARTCONTENT

\only<1>{
\begin{QQuestion}{AC401}{Ein in Durchlassrichtung betriebener PN-Übergang ermöglicht~...}{den Elektronenfluss von P nach N.}
{die Halbierung des Stromflusses.}
{keinen Stromfluss.}
{den Elektronenfluss von N nach P.}
\end{QQuestion}

}
\only<2>{
\begin{QQuestion}{AC401}{Ein in Durchlassrichtung betriebener PN-Übergang ermöglicht~...}{den Elektronenfluss von P nach N.}
{die Halbierung des Stromflusses.}
{keinen Stromfluss.}
{\textbf{\textcolor{DARCgreen}{den Elektronenfluss von N nach P.}}}
\end{QQuestion}

}
\end{frame}

\begin{frame}
\only<1>{
\begin{QQuestion}{AC403}{Wie verhält sich die Durchlassspannung einer Diode in Abhängigkeit von der Temperatur?}{Die Spannung ist unabhängig von der Temperatur.}
{Die Spannung sinkt bei steigender Temperatur.}
{Die Spannung oszilliert mit steigender Temperatur.}
{Die Spannung steigt bei steigender Temperatur.}
\end{QQuestion}

}
\only<2>{
\begin{QQuestion}{AC403}{Wie verhält sich die Durchlassspannung einer Diode in Abhängigkeit von der Temperatur?}{Die Spannung ist unabhängig von der Temperatur.}
{\textbf{\textcolor{DARCgreen}{Die Spannung sinkt bei steigender Temperatur.}}}
{Die Spannung oszilliert mit steigender Temperatur.}
{Die Spannung steigt bei steigender Temperatur.}
\end{QQuestion}

}
\end{frame}

\begin{frame}
\only<1>{
\begin{QQuestion}{AC404}{Wie verhält sich die Kapazität einer Kapazitätsdiode (Varicap)?}{Sie nimmt mit abnehmender Sperrspannung zu.}
{Sie nimmt mit abnehmendem Durchlassstrom zu.}
{Sie nimmt mit zunehmender Sperrspannung zu.}
{Sie nimmt mit zunehmendem Durchlassstrom zu.}
\end{QQuestion}

}
\only<2>{
\begin{QQuestion}{AC404}{Wie verhält sich die Kapazität einer Kapazitätsdiode (Varicap)?}{\textbf{\textcolor{DARCgreen}{Sie nimmt mit abnehmender Sperrspannung zu.}}}
{Sie nimmt mit abnehmendem Durchlassstrom zu.}
{Sie nimmt mit zunehmender Sperrspannung zu.}
{Sie nimmt mit zunehmendem Durchlassstrom zu.}
\end{QQuestion}

}
\end{frame}

\begin{frame}
\only<1>{
\begin{PQuestion}{AC405}{Das folgende Signal wird als $U_1$ an den Eingang der Schaltung mit Siliziumdioden gelegt. Wie sieht das zugehörige Ausgangssignal $U_2$ aus?}{\DARCimage{1.0\linewidth}{17include}}
{\DARCimage{1.0\linewidth}{15include}}
{\DARCimage{1.0\linewidth}{16include}}
{\DARCimage{1.0\linewidth}{14include}}
{\DARCimage{1.0\linewidth}{13include}}\end{PQuestion}

}
\only<2>{
\begin{PQuestion}{AC405}{Das folgende Signal wird als $U_1$ an den Eingang der Schaltung mit Siliziumdioden gelegt. Wie sieht das zugehörige Ausgangssignal $U_2$ aus?}{\DARCimage{1.0\linewidth}{17include}}
{\DARCimage{1.0\linewidth}{15include}}
{\DARCimage{1.0\linewidth}{16include}}
{\textbf{\textcolor{DARCgreen}{\DARCimage{1.0\linewidth}{14include}}}}
{\DARCimage{1.0\linewidth}{13include}}\end{PQuestion}

}
\end{frame}

\begin{frame}
\only<1>{
\begin{PQuestion}{AC406}{Das folgende Signal wird als $U_1$ an den Eingang der Schaltung mit Germaniumdioden gelegt. Wie sieht das zugehörige Ausgangssignal $U_2$ aus?}{\DARCimage{1.0\linewidth}{17include}}
{\DARCimage{1.0\linewidth}{15include}}
{\DARCimage{1.0\linewidth}{16include}}
{\DARCimage{1.0\linewidth}{14include}}
{\DARCimage{1.0\linewidth}{13include}}\end{PQuestion}

}
\only<2>{
\begin{PQuestion}{AC406}{Das folgende Signal wird als $U_1$ an den Eingang der Schaltung mit Germaniumdioden gelegt. Wie sieht das zugehörige Ausgangssignal $U_2$ aus?}{\textbf{\textcolor{DARCgreen}{\DARCimage{1.0\linewidth}{17include}}}}
{\DARCimage{1.0\linewidth}{15include}}
{\DARCimage{1.0\linewidth}{16include}}
{\DARCimage{1.0\linewidth}{14include}}
{\DARCimage{1.0\linewidth}{13include}}\end{PQuestion}

}
\end{frame}

\begin{frame}
\only<1>{
\begin{QQuestion}{AC407}{Welches Bauteil kann durch Lichteinfall elektrischen Strom erzeugen?}{Blindwiderstand}
{Fotowiderstand}
{Kapazitätsdiode}
{Fotodiode}
\end{QQuestion}

}
\only<2>{
\begin{QQuestion}{AC407}{Welches Bauteil kann durch Lichteinfall elektrischen Strom erzeugen?}{Blindwiderstand}
{Fotowiderstand}
{Kapazitätsdiode}
{\textbf{\textcolor{DARCgreen}{Fotodiode}}}
\end{QQuestion}

}
\end{frame}

\begin{frame}
\only<1>{
\begin{QQuestion}{AC408}{Die Hauptfunktion eines Optokopplers ist~...}{die galvanische Entkopplung zweier Stromkreise durch Licht.}
{die Erzeugung von hochfrequentem Wechselstrom durch Licht.}
{die Signalanzeige durch Licht.}
{die Erzeugung von Gleichstrom durch Licht.}
\end{QQuestion}

}
\only<2>{
\begin{QQuestion}{AC408}{Die Hauptfunktion eines Optokopplers ist~...}{\textbf{\textcolor{DARCgreen}{die galvanische Entkopplung zweier Stromkreise durch Licht.}}}
{die Erzeugung von hochfrequentem Wechselstrom durch Licht.}
{die Signalanzeige durch Licht.}
{die Erzeugung von Gleichstrom durch Licht.}
\end{QQuestion}

}
\end{frame}%ENDCONTENT
