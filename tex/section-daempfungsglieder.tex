
\section{Dämpfungsglieder}
\label{section:daempfungsglieder}
\begin{frame}%STARTCONTENT

\only<1>{
\begin{PQuestion}{AD801}{Was zeigt diese Schaltung?}{Dämpfungsglied}
{Verstärker}
{Hochpass}
{Tiefpass}
{\DARCimage{1.0\linewidth}{554include}}\end{PQuestion}

}
\only<2>{
\begin{PQuestion}{AD801}{Was zeigt diese Schaltung?}{\textbf{\textcolor{DARCgreen}{Dämpfungsglied}}}
{Verstärker}
{Hochpass}
{Tiefpass}
{\DARCimage{1.0\linewidth}{554include}}\end{PQuestion}

}
\end{frame}

\begin{frame}
\only<1>{
\begin{PQuestion}{AD802}{Was zeigt diese Schaltung?}{Tiefpass}
{Verstärker}
{Hochpass}
{Dämpfungsglied}
{\DARCimage{1.0\linewidth}{555include}}\end{PQuestion}

}
\only<2>{
\begin{PQuestion}{AD802}{Was zeigt diese Schaltung?}{Tiefpass}
{Verstärker}
{Hochpass}
{\textbf{\textcolor{DARCgreen}{Dämpfungsglied}}}
{\DARCimage{1.0\linewidth}{555include}}\end{PQuestion}

}
\end{frame}

\begin{frame}
\only<1>{
\begin{PQuestion}{AD803}{Dargestellt ist ein \qty{20}{\decibel} Dämpfungsglied. Wie groß ist das Leistungsverhältnis zwischen der Eingangsleistung $P_{\symup{IN}}$ und der Leistung am Lastwiderstand $P_{\symup{RL}}$?}{50}
{10}
{20}
{100}
{\DARCimage{1.0\linewidth}{341include}}\end{PQuestion}

}
\only<2>{
\begin{PQuestion}{AD803}{Dargestellt ist ein \qty{20}{\decibel} Dämpfungsglied. Wie groß ist das Leistungsverhältnis zwischen der Eingangsleistung $P_{\symup{IN}}$ und der Leistung am Lastwiderstand $P_{\symup{RL}}$?}{50}
{10}
{20}
{\textbf{\textcolor{DARCgreen}{100}}}
{\DARCimage{1.0\linewidth}{341include}}\end{PQuestion}

}
\end{frame}

\begin{frame}
\frametitle{Lösungweg}
\begin{itemize}
  \item 20dB entsprechen einer Leistungdämpfung mit dem Faktor 100
  \end{itemize}
\end{frame}

\begin{frame}
\only<1>{
\begin{PQuestion}{AD804}{Dargestellt ist ein \qty{6}{\decibel} Dämpfungsglied. Wie groß ist das Leistungsverhältnis zwischen der Eingangsleistung $P_{\symup{IN}}$ und der Leistung am Lastwiderstand $P_{\symup{RL}}$?}{2}
{4}
{3}
{6}
{\DARCimage{1.0\linewidth}{424include}}\end{PQuestion}

}
\only<2>{
\begin{PQuestion}{AD804}{Dargestellt ist ein \qty{6}{\decibel} Dämpfungsglied. Wie groß ist das Leistungsverhältnis zwischen der Eingangsleistung $P_{\symup{IN}}$ und der Leistung am Lastwiderstand $P_{\symup{RL}}$?}{2}
{\textbf{\textcolor{DARCgreen}{4}}}
{3}
{6}
{\DARCimage{1.0\linewidth}{424include}}\end{PQuestion}

}
\end{frame}

\begin{frame}
\frametitle{Lösungsweg}
\begin{itemize}
  \item 6dB entsprechen einer Leistungsdämpfung mit dem Faktor 4
  \end{itemize}
\end{frame}

\begin{frame}
\only<1>{
\begin{PQuestion}{AD805}{Dargestellt ist ein symmetrisches \qty{50}{\ohm} Dämpfungsglied. Welche Impedanz ist zwischen $a$ und $b$ messbar, wenn $R_{\symup{L}}$~=~\qty{50}{\ohm} beträgt?}{$R_1$ + $R_2$ + \qty{50}{\ohm}}
{\qty{100}{\ohm}}
{$R_1$ + \qty{50}{\ohm}}
{\qty{50}{\ohm}}
{\DARCimage{1.0\linewidth}{423include}}\end{PQuestion}

}
\only<2>{
\begin{PQuestion}{AD805}{Dargestellt ist ein symmetrisches \qty{50}{\ohm} Dämpfungsglied. Welche Impedanz ist zwischen $a$ und $b$ messbar, wenn $R_{\symup{L}}$~=~\qty{50}{\ohm} beträgt?}{$R_1$ + $R_2$ + \qty{50}{\ohm}}
{\qty{100}{\ohm}}
{$R_1$ + \qty{50}{\ohm}}
{\textbf{\textcolor{DARCgreen}{\qty{50}{\ohm}}}}
{\DARCimage{1.0\linewidth}{423include}}\end{PQuestion}

}
\end{frame}

\begin{frame}
\frametitle{Lösungsweg}
\begin{itemize}
  \item Die Impedanz für die Gesamtschaltung ändert sich nicht – also 50Ω
  \end{itemize}
\end{frame}

\begin{frame}
\only<1>{
\begin{PQuestion}{AD806}{In einem \qty{50}{\ohm} System wird in ein symmetrisches \qty{20}{\decibel} Dämpfungsglied die Leistung von \qty{100}{\W} eingespeist. Der Widerstand $R_{\symup{L}}$~=~\qty{50}{\ohm} ist an das Dämpfungsglied angepasst. Welche Leistung wird insgesamt im Dämpfungsglied in Wärme umgesetzt?}{\qty{99}{\W}}
{\qty{50}{\W}}
{\qty{2}{\W}}
{\qty{1}{\W}}
{\DARCimage{1.0\linewidth}{342include}}\end{PQuestion}

}
\only<2>{
\begin{PQuestion}{AD806}{In einem \qty{50}{\ohm} System wird in ein symmetrisches \qty{20}{\decibel} Dämpfungsglied die Leistung von \qty{100}{\W} eingespeist. Der Widerstand $R_{\symup{L}}$~=~\qty{50}{\ohm} ist an das Dämpfungsglied angepasst. Welche Leistung wird insgesamt im Dämpfungsglied in Wärme umgesetzt?}{\textbf{\textcolor{DARCgreen}{\qty{99}{\W}}}}
{\qty{50}{\W}}
{\qty{2}{\W}}
{\qty{1}{\W}}
{\DARCimage{1.0\linewidth}{342include}}\end{PQuestion}

}
\end{frame}

\begin{frame}
\frametitle{Lösungsweg}
\begin{itemize}
  \item gegeben: $P_1 = 100W$
  \item gegeben: $a = 20dB$
  \item gesucht: $\Delta P = P_2 -- P_1$
  \end{itemize}
    \pause
    $$\begin{align}\nonumber a &= 10 \cdot \log_{10}{(\frac{P_1}{P_2})}dB\\ \nonumber \Rightarrow \frac{a}{10} &= \log_{10}{(\frac{P_1}{P_2})}dB\\ \nonumber \Rightarrow 10^{\frac{a}{10}} &= \frac{P_1}{P_2}\\ \nonumber \Rightarrow P_2 &= \frac{P_1}{10^{\frac{a}{10}}}\end{align}$$



\end{frame}

\begin{frame}
    \pause
    $P_2 = \frac{P_1}{10^{\frac{a}{10}}} = \frac{100W}{10^{\frac{20}{10}}} = 1W$
    \pause
    $\Delta P = P_2 -- P_1 = 100W -- 1W = 99W$



\end{frame}%ENDCONTENT
