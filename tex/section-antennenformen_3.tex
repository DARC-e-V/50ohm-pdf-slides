
\section{Antennenformen III}
\label{section:antennenformen_3}
\begin{frame}%STARTCONTENT

\only<1>{
\begin{PQuestion}{AG419}{Was ist beim Aufbau des dargestellten Drahtantennensystems zu beachten? Die Drahtlänge des Strahlers sollte~...}{gleich 5/8~$\lambda$ der benutzten Frequenz sein oder einem Vielfachen davon entsprechen.}
{gleich 1/2~$\lambda$ der benutzten Frequenz sein oder einem Vielfachen davon entsprechen.}
{genau 1/4~$\lambda$ der benutzten Frequenzen sein.}
{genau 3/8~$\lambda$ der benutzten Frequenzen sein.}
{\DARCimage{1.0\linewidth}{310include}}\end{PQuestion}

}
\only<2>{
\begin{PQuestion}{AG419}{Was ist beim Aufbau des dargestellten Drahtantennensystems zu beachten? Die Drahtlänge des Strahlers sollte~...}{gleich 5/8~$\lambda$ der benutzten Frequenz sein oder einem Vielfachen davon entsprechen.}
{\textbf{\textcolor{DARCgreen}{gleich 1/2~$\lambda$ der benutzten Frequenz sein oder einem Vielfachen davon entsprechen.}}}
{genau 1/4~$\lambda$ der benutzten Frequenzen sein.}
{genau 3/8~$\lambda$ der benutzten Frequenzen sein.}
{\DARCimage{1.0\linewidth}{310include}}\end{PQuestion}

}
\end{frame}

\begin{frame}
\only<1>{
\begin{PQuestion}{AG123}{Wie wird die dargestellte Antenne bezeichnet (MWS~=~Mantelwellensperre)?}{endgespeiste, magnetische Multibandantenne}
{Windomantenne}
{W3DZZ}
{endgespeiste Multibandantenne}
{\DARCimage{1.0\linewidth}{315include}}\end{PQuestion}

}
\only<2>{
\begin{PQuestion}{AG123}{Wie wird die dargestellte Antenne bezeichnet (MWS~=~Mantelwellensperre)?}{endgespeiste, magnetische Multibandantenne}
{Windomantenne}
{W3DZZ}
{\textbf{\textcolor{DARCgreen}{endgespeiste Multibandantenne}}}
{\DARCimage{1.0\linewidth}{315include}}\end{PQuestion}

}
\end{frame}

\begin{frame}
\only<1>{
\begin{PQuestion}{AG124}{Wie wird die in der nachfolgenden Skizze dargestellte Antenne bezeichnet (MWS~=~Mantelwellensperre)? Es handelt sich um eine~...}{elektrisch verkürzte Windomantenne}
{mit magnetischem Balun aufgebaute Multibandantenne}
{endgespeiste Multibandantenne mit einem Trap}
{endgespeiste, resonante Multibandantenne}
{\DARCimage{1.0\linewidth}{260include}}\end{PQuestion}

}
\only<2>{
\begin{PQuestion}{AG124}{Wie wird die in der nachfolgenden Skizze dargestellte Antenne bezeichnet (MWS~=~Mantelwellensperre)? Es handelt sich um eine~...}{elektrisch verkürzte Windomantenne}
{mit magnetischem Balun aufgebaute Multibandantenne}
{endgespeiste Multibandantenne mit einem Trap}
{\textbf{\textcolor{DARCgreen}{endgespeiste, resonante Multibandantenne}}}
{\DARCimage{1.0\linewidth}{260include}}\end{PQuestion}

}
\end{frame}

\begin{frame}
\only<1>{
\begin{PQuestion}{AG120}{Wie wird die folgende Antenne in der Amateurfunkliteratur bezeichnet?  }{Fuchs-Antenne}
{Windom-Antenne}
{Zeppelin-Antenne}
{Marconi-Antenne}
{\DARCimage{1.0\linewidth}{314include}}\end{PQuestion}

}
\only<2>{
\begin{PQuestion}{AG120}{Wie wird die folgende Antenne in der Amateurfunkliteratur bezeichnet?  }{Fuchs-Antenne}
{Windom-Antenne}
{\textbf{\textcolor{DARCgreen}{Zeppelin-Antenne}}}
{Marconi-Antenne}
{\DARCimage{1.0\linewidth}{314include}}\end{PQuestion}

}
\end{frame}

\begin{frame}
\only<1>{
\begin{PQuestion}{AG117}{Wie wird die folgende Antenne in der Amateurfunkliteratur üblicherweise bezeichnet?}{Dreieck-Antenne}
{Delta-Loop (Ganzwellenschleife)}
{Koaxial-Stub-Antenne}
{koaxial gespeiste Dreilinien-Antenne}
{\DARCimage{1.0\linewidth}{311include}}\end{PQuestion}

}
\only<2>{
\begin{PQuestion}{AG117}{Wie wird die folgende Antenne in der Amateurfunkliteratur üblicherweise bezeichnet?}{Dreieck-Antenne}
{\textbf{\textcolor{DARCgreen}{Delta-Loop (Ganzwellenschleife)}}}
{Koaxial-Stub-Antenne}
{koaxial gespeiste Dreilinien-Antenne}
{\DARCimage{1.0\linewidth}{311include}}\end{PQuestion}

}
\end{frame}

\begin{frame}
\only<1>{
\begin{QQuestion}{AG119}{Bei einer Quad-Antenne beträgt die elektrische Länge jeder Seite~...}{eine ganze Wellenlänge.}
{die Hälfte der Wellenlänge.}
{dreiviertel der Wellenlänge.}
{ein Viertel der Wellenlänge.}
\end{QQuestion}

}
\only<2>{
\begin{QQuestion}{AG119}{Bei einer Quad-Antenne beträgt die elektrische Länge jeder Seite~...}{eine ganze Wellenlänge.}
{die Hälfte der Wellenlänge.}
{dreiviertel der Wellenlänge.}
{\textbf{\textcolor{DARCgreen}{ein Viertel der Wellenlänge.}}}
\end{QQuestion}

}
\end{frame}

\begin{frame}
\only<1>{
\begin{PQuestion}{AG121}{Wie wird die folgende Antenne in der Amateurfunkliteratur bezeichnet?  }{Windom-Antenne}
{G5RV-Antenne}
{Fuchs-Antenne}
{Zeppelin-Antenne}
{\DARCimage{1.0\linewidth}{313include}}\end{PQuestion}

}
\only<2>{
\begin{PQuestion}{AG121}{Wie wird die folgende Antenne in der Amateurfunkliteratur bezeichnet?  }{Windom-Antenne}
{\textbf{\textcolor{DARCgreen}{G5RV-Antenne}}}
{Fuchs-Antenne}
{Zeppelin-Antenne}
{\DARCimage{1.0\linewidth}{313include}}\end{PQuestion}

}
\end{frame}

\begin{frame}
\only<1>{
\begin{PQuestion}{AG122}{Wie wird die folgende Antenne in der Amateurfunkliteratur bezeichnet?  }{Fuchs-Antenne}
{Windom-Antenne}
{Zeppelin-Antenne}
{Marconi-Antenne}
{\DARCimage{1.0\linewidth}{309include}}\end{PQuestion}

}
\only<2>{
\begin{PQuestion}{AG122}{Wie wird die folgende Antenne in der Amateurfunkliteratur bezeichnet?  }{Fuchs-Antenne}
{\textbf{\textcolor{DARCgreen}{Windom-Antenne}}}
{Zeppelin-Antenne}
{Marconi-Antenne}
{\DARCimage{1.0\linewidth}{309include}}\end{PQuestion}

}
\end{frame}

\begin{frame}
\only<1>{
\begin{QQuestion}{AG223}{Bei welcher Länge erreicht eine Vertikalantenne für den Kurzwellenbereich über einer Erdoberfläche mittlerer Leitfähigkeit eine möglichst flache Abstrahlung?}{5/8$~\lambda$}
{$\lambda$/4}
{$\lambda$/2}
{3/4$~\lambda$}
\end{QQuestion}

}
\only<2>{
\begin{QQuestion}{AG223}{Bei welcher Länge erreicht eine Vertikalantenne für den Kurzwellenbereich über einer Erdoberfläche mittlerer Leitfähigkeit eine möglichst flache Abstrahlung?}{\textbf{\textcolor{DARCgreen}{5/8$~\lambda$}}}
{$\lambda$/4}
{$\lambda$/2}
{3/4$~\lambda$}
\end{QQuestion}

}
\end{frame}%ENDCONTENT
