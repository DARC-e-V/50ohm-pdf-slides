
\section{Sinusschwingung}
\label{section:sinusschwingung}
\begin{frame}%STARTCONTENT

\begin{columns}
    \begin{column}{0.48\textwidth}
    
\begin{figure}
    \DARCimage{0.85\linewidth}{725include}
    \caption{\scriptsize Die Spannung des Stromnetzes im zeitlichen Verlauf. Da die Spannung nicht die ganze Zeit den Höchstwert von \qty{325}{\volt} aufweist, wirkt sie effektiv übrigens nur mit \qty{230}{\volt}.}
    \label{n_frequenz_sinusschwingung}
\end{figure}


    \end{column}
   \begin{column}{0.48\textwidth}
       \begin{itemize}
  \item Die Wechselspannung aus dem Stromnetz schwingt nicht direkt zurück
  \item Es gibt einen sanften Übergang über 0
  \item Wie bei einem Pendel
  \item Diese Art der Schwingung ist eine Sinusschwingung
  \end{itemize}

   \end{column}
\end{columns}

\end{frame}

\begin{frame}
\begin{figure}
    \DARCimage{0.85\linewidth}{505include}
    \caption{\scriptsize Rechteckförmige Schwingung}
    \label{sinusschwingung_rechteck}
\end{figure}

\end{frame}

\begin{frame}
\begin{figure}
    \DARCimage{0.85\linewidth}{504include}
    \caption{\scriptsize Dreieckförmige Schwingung}
    \label{sinusschwingung_dreieck}
\end{figure}

\end{frame}

\begin{frame}
\begin{figure}
    \DARCimage{0.85\linewidth}{506include}
    \caption{\scriptsize Sägezahnförmige Schwingung}
    \label{sinusschwingung_saegezahn}
\end{figure}

\end{frame}

\begin{frame}
\only<1>{
\begin{question2x2}{NB401}{Welches Bild zeigt eine sinusförmige Wechselspannung?}
{\DARCimage{0.75\linewidth}{505include}}
{\DARCimage{0.75\linewidth}{504include}}
{\DARCimage{0.75\linewidth}{503include}}
{\DARCimage{0.75\linewidth}{506include}}
\end{question2x2}

}
\only<2>{
\begin{question2x2}{NB401}{Welches Bild zeigt eine sinusförmige Wechselspannung?}
{\DARCimage{0.75\linewidth}{505include}}
{\DARCimage{0.75\linewidth}{504include}}
{\textbf{\textcolor{DARCgreen}{\DARCimage{0.75\linewidth}{503include}}}}
{\DARCimage{0.75\linewidth}{506include}}
\end{question2x2}

}
\end{frame}%ENDCONTENT
