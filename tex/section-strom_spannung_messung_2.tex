
\section{Strom- und Spannungsmessung II}
\label{section:strom_spannung_messung_2}
\begin{frame}%STARTCONTENT

\begin{columns}
    \begin{column}{0.48\textwidth}
    \begin{itemize}
  \item Der Strom wird im Stromkreis eingeschleift gemessen
  \item Die Spannung wird über den Widerstand gemessen
  \item Der Widerstand im Voltmeter soll hochohmig sein $\rightarrow$ Strom nimmt den Weg des geringsten Widerstandes
  \end{itemize}

    \end{column}
   \begin{column}{0.48\textwidth}
       
\begin{figure}
    \DARCimage{0.85\linewidth}{238include}
    \caption{\scriptsize Korrekte Anordnung zur Messung von Strom und Spannung an einem Widerstand}
    \label{e_strom_und_spannungsmessung}
\end{figure}


   \end{column}
\end{columns}

\end{frame}

\begin{frame}
\only<1>{
\begin{QQuestion}{EI101}{Wie werden elektrische Spannungsmessgeräte an Messobjekte angeschlossen und welche Anforderungen muss das Messgerät erfüllen, damit der Messfehler möglichst gering bleibt? Das Spannungsmessgerät ist~...}{in den Stromkreis einzuschleifen und sollte niederohmig sein.}
{parallel zum Messobjekt anzuschließen und sollte hochohmig sein.}
{parallel zum Messobjekt anzuschließen und sollte niederohmig sein.}
{in den Stromkreis einzuschleifen und sollte hochohmig sein.}
\end{QQuestion}

}
\only<2>{
\begin{QQuestion}{EI101}{Wie werden elektrische Spannungsmessgeräte an Messobjekte angeschlossen und welche Anforderungen muss das Messgerät erfüllen, damit der Messfehler möglichst gering bleibt? Das Spannungsmessgerät ist~...}{in den Stromkreis einzuschleifen und sollte niederohmig sein.}
{\textbf{\textcolor{DARCgreen}{parallel zum Messobjekt anzuschließen und sollte hochohmig sein.}}}
{parallel zum Messobjekt anzuschließen und sollte niederohmig sein.}
{in den Stromkreis einzuschleifen und sollte hochohmig sein.}
\end{QQuestion}

}
\end{frame}

\begin{frame}
\only<1>{
\begin{question2x2}{EI102}{Welche Schaltung mit idealen Messgeräten könnte dazu verwendet werden, den Wert eines Widerstandes anhand des ohmschen Gesetzes zu ermitteln?}{\DARCimage{0.75\linewidth}{240include}}
{\DARCimage{0.75\linewidth}{239include}}
{\DARCimage{0.75\linewidth}{241include}}
{\DARCimage{0.75\linewidth}{238include}}
\end{question2x2}

}
\only<2>{
\begin{question2x2}{EI102}{Welche Schaltung mit idealen Messgeräten könnte dazu verwendet werden, den Wert eines Widerstandes anhand des ohmschen Gesetzes zu ermitteln?}{\DARCimage{0.75\linewidth}{240include}}
{\DARCimage{0.75\linewidth}{239include}}
{\DARCimage{0.75\linewidth}{241include}}
{\textbf{\textcolor{DARCgreen}{\DARCimage{0.75\linewidth}{238include}}}}
\end{question2x2}

}
\end{frame}%ENDCONTENT
