
\section{Spannungsquellen}
\label{section:spannungsquelle}
\begin{frame}%STARTCONTENT

\frametitle{Netzgerät}
\begin{itemize}
  \item Die für uns Funkamateuere wichtigste Spannungsquelle ist, neben Batterien, das Netzgerät. Es wird über das Stromnetz mit 230 Volt Wechselspannung versorgt und erzeut eine Gleichspannung von 13,8 Volt. Damit lassen sich Funkgeräte und Zubehör versorgen.
  \item Wichtig ist, dass die Ausggangsspannung bei Last konstant bleibt.
  \end{itemize}
\end{frame}

\begin{frame}
\only<1>{
\begin{QQuestion}{ED301}{Welche Eigenschaften sollten Gleichspannungsquellen aufweisen?}{Gleichspannungsquellen sollten bei Belastung die Spannung erhöhen.}
{Gleichspannungsquellen sollten bei Belastung eine niedrige Spannungskonstanz haben.}
{Gleichspannungsquellen sollten bei Belastung eine hohe Spannungskonstanz haben.}
{Gleichspannungsquellen sollten bei Belastung einen Wechselspannungsanteil haben.  }
\end{QQuestion}

}
\only<2>{
\begin{QQuestion}{ED301}{Welche Eigenschaften sollten Gleichspannungsquellen aufweisen?}{Gleichspannungsquellen sollten bei Belastung die Spannung erhöhen.}
{Gleichspannungsquellen sollten bei Belastung eine niedrige Spannungskonstanz haben.}
{\textbf{\textcolor{DARCgreen}{Gleichspannungsquellen sollten bei Belastung eine hohe Spannungskonstanz haben.}}}
{Gleichspannungsquellen sollten bei Belastung einen Wechselspannungsanteil haben.  }
\end{QQuestion}

}
\end{frame}

\begin{frame}
\frametitle{Elektrische Sicherheit}
Bei Netzgeräten, besonders mit einem Metallgehäuse, ist ein normgerechter Anschluss an das Stromnetz wichtig. Der Schutzleiter (grün/gelb) hat dabei die Aufgabe im Fehlerfall die Spannung zur \enquote{Erde} abzuleiten und damit die Haussicherung auszulösen, damit keine gefährliche Spannung am Metallgehäuse anliegt. Bei einer 3-adrigen Leitung sind die Adernkennfarben wie folgt festgelegt:

\end{frame}

\begin{frame}
\frametitle{Aderfarben}
\begin{columns}
    \begin{column}{0.48\textwidth}
    \begin{itemize}
  \item Außenleiter (L) $\rightarrow$ braun
  \item Neutralleiter (N) $\rightarrow$ blau
  \item Schutzleiter (PE) $\rightarrow$ grün/gelb
  \end{itemize}

    \end{column}
   \begin{column}{0.48\textwidth}
       
\begin{figure}
    \DARCimage{0.85\linewidth}{791include}
    \caption{\scriptsize Aderfarben einer 3-adrigen Leitung}
    \label{e_NYM_Aderfarben}
\end{figure}


   \end{column}
\end{columns}

\end{frame}

\begin{frame}
\only<1>{
\begin{QQuestion}{EK205}{Wählen Sie die normgerechten Adernkennfarben von 3-adrigen, isolierten Energieleitungen und -kabeln in der Reihenfolge: Schutzleiter, Außenleiter, Neutralleiter!}{grüngelb, braun, blau}
{braun, grüngelb, blau}
{grau, schwarz, rot}
{grüngelb, blau, braun oder schwarz}
\end{QQuestion}

}
\only<2>{
\begin{QQuestion}{EK205}{Wählen Sie die normgerechten Adernkennfarben von 3-adrigen, isolierten Energieleitungen und -kabeln in der Reihenfolge: Schutzleiter, Außenleiter, Neutralleiter!}{\textbf{\textcolor{DARCgreen}{grüngelb, braun, blau}}}
{braun, grüngelb, blau}
{grau, schwarz, rot}
{grüngelb, blau, braun oder schwarz}
\end{QQuestion}

}
\end{frame}%ENDCONTENT
