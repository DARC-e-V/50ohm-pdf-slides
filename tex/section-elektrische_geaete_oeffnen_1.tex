
\section{Öffnen elektrischer Geräte I}
\label{section:elektrische_geaete_oeffnen_1}
\begin{frame}%STARTCONTENT
\begin{itemize}
  \item Als Funkamateure dürfen wir Geräte öffnen und verändern
  \item Zum Eigenschutz das \emph{Gerät vom Netz trennen!}
  \item Kondensatoren können längere Zeit Energie speichern, die gefährlich werden können
  \item Auch ohne Netzanschluss besteht weiterhin \emph{Lebensgefahr bei der Berührung von Kondensatoren!}
  \end{itemize}
\end{frame}

\begin{frame}
\only<1>{
\begin{QQuestion}{EK203}{Mit welchen Gefahren muss beim Öffnen eines vom Netz getrennten Funk- oder anderen elektrisch betriebenen Gerätes gerechnet werden?}{Elektrischer Schlag durch Ladungen im Netztransformator.}
{Elektrischer Schlag durch aufgeladene Kondensatoren im Netzteil.}
{Keine Gefahr, da das Gerät vorher von der Stromversorgung getrennt worden ist.}
{In der Ladedrossel eines Schaltnetzteiles können Spannungen gespeichert sein, die deutlich höher sind als die angelegte Versorgungsspannung.}
\end{QQuestion}

}
\only<2>{
\begin{QQuestion}{EK203}{Mit welchen Gefahren muss beim Öffnen eines vom Netz getrennten Funk- oder anderen elektrisch betriebenen Gerätes gerechnet werden?}{Elektrischer Schlag durch Ladungen im Netztransformator.}
{\textbf{\textcolor{DARCgreen}{Elektrischer Schlag durch aufgeladene Kondensatoren im Netzteil.}}}
{Keine Gefahr, da das Gerät vorher von der Stromversorgung getrennt worden ist.}
{In der Ladedrossel eines Schaltnetzteiles können Spannungen gespeichert sein, die deutlich höher sind als die angelegte Versorgungsspannung.}
\end{QQuestion}

}
\end{frame}%ENDCONTENT
