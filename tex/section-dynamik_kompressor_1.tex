
\section{Dynamikkompressor I}
\label{section:dynamik_kompressor_1}
\begin{frame}%STARTCONTENT

\begin{columns}
    \begin{column}{0.48\textwidth}
    \emph{Ohne Kompressor}

\begin{itemize}
  \item Sprache unterliegt starken Schwankungen in der Amplitude
  \item Das führt zu unterschiedlicher Modulation des Signals
  \item Teilweise kann das Signal beim Empfänger schlecht verstanden werden
  \end{itemize}

    \end{column}
   \begin{column}{0.48\textwidth}
       \emph{Mit Kompressor}

\begin{itemize}
  \item Ein \emph{Dynamikkompressor} hebt leise Signale gegenüber den lauten an
  \item Das Signal wird hinsichtlich seiner Amplitudenschwankungen komprimiert
  \item Führt zu einem besseren Verständnis beim Empfänger
  \end{itemize}

   \end{column}
\end{columns}

\end{frame}

\begin{frame}
\only<1>{
\begin{QQuestion}{EF306}{Wie heißt die Stufe in einem Sender, welche die Eigenschaft hat, leise Anteile eines Sprachsignale gegenüber den lauten etwas anzuheben?}{Notchfilter}
{Noise Blanker}
{Clarifier}
{Dynamic Compressor}
\end{QQuestion}

}
\only<2>{
\begin{QQuestion}{EF306}{Wie heißt die Stufe in einem Sender, welche die Eigenschaft hat, leise Anteile eines Sprachsignale gegenüber den lauten etwas anzuheben?}{Notchfilter}
{Noise Blanker}
{Clarifier}
{\textbf{\textcolor{DARCgreen}{Dynamic Compressor}}}
\end{QQuestion}

}
\end{frame}%ENDCONTENT
