
\section{Leistung beim Wechselstrom}
\label{section:wechselstrom_leistung}
\begin{frame}%STARTCONTENT
\begin{itemize}
  \item Berechnung mit Effektivwert
  \item $U_{\textrm{eff}} = \frac{\^{U}}{\sqrt{2}}$
  \item $I_{\textrm{eff}} = \frac{\^{I}}{\sqrt{2}}$
  \end{itemize}
\end{frame}

\begin{frame}
\only<1>{
\begin{QQuestion}{AB301}{Ein sinusförmiger Wechselstrom mit einer Amplitude $I_{\symup{max}}$ von 0,5 Ampere fließt durch einen Widerstand von \qty{20}{\ohm}. Wieviel Leistung wird in Wärme umgesetzt?}{\qty{3,5}{\W}}
{\qty{5,0}{\W}}
{\qty{10}{\W}}
{\qty{2,5}{\W}}
\end{QQuestion}

}
\only<2>{
\begin{QQuestion}{AB301}{Ein sinusförmiger Wechselstrom mit einer Amplitude $I_{\symup{max}}$ von 0,5 Ampere fließt durch einen Widerstand von \qty{20}{\ohm}. Wieviel Leistung wird in Wärme umgesetzt?}{\qty{3,5}{\W}}
{\qty{5,0}{\W}}
{\qty{10}{\W}}
{\textbf{\textcolor{DARCgreen}{\qty{2,5}{\W}}}}
\end{QQuestion}

}
\end{frame}

\begin{frame}
\frametitle{Lösungsweg}
\begin{itemize}
  \item gegeben: $I_{\textrm{max}} = 0,5A$
  \item gegeben: $R = 20\Omega$
  \item gesucht: $P$
  \end{itemize}
    \pause
    \begin{equation}\begin{split} \nonumber P &=  I^2 \cdot R = (\frac{I_{\textrm{max}}}{\sqrt{2}})^2 \cdot R\\ &= \frac{(0,5A)^2}{2} \cdot 20\Omega \\ &= \frac{1}{8}A^2 \cdot 20\Omega = 2,5W \end{split}\end{equation}



\end{frame}%ENDCONTENT
