
\section{Notfunk}
\label{section:notfunk}
\begin{frame}%STARTCONTENT

\frametitle{Dürfen Funkamateure helfen?}
\begin{itemize}
  \item Ja! Funkamateure dürfen in Not- und Katastrophenfällen durch Übermittlung von Nachrichten für und an Dritte bei der Bewältigung einer Notlage unterstützen.
  \end{itemize}
\end{frame}

\begin{frame}
\only<1>{
\begin{QQuestion}{BF103}{Sie erreichen eine Unfallstelle. Der Ersthelfer bittet Sie, über Ihre mobile Amateurfunkstelle Hilfe zu holen, da das Mobiltelefonnetz nicht verfügbar ist. Wie verhalten Sie sich?}{Ich rufe per Funk einen Funkamateur und fordere diesen auf, die Polizei oder Rettungsleitstelle zu informieren.}
{Ich lehne es ab zu helfen, da im Amateurfunk keine Informationen für Dritte übermittelt werden dürfen.}
{Ich rufe per Funk mindestens dreimal MAYDAY gefolgt von meinem Rufzeichen, dem Standort und der Art der Notlage.}
{Ich lehne es ab zu helfen, da im Amateurfunk keine Notzeichen verwendet werden dürfen.}
\end{QQuestion}

}
\only<2>{
\begin{QQuestion}{BF103}{Sie erreichen eine Unfallstelle. Der Ersthelfer bittet Sie, über Ihre mobile Amateurfunkstelle Hilfe zu holen, da das Mobiltelefonnetz nicht verfügbar ist. Wie verhalten Sie sich?}{\textbf{\textcolor{DARCgreen}{Ich rufe per Funk einen Funkamateur und fordere diesen auf, die Polizei oder Rettungsleitstelle zu informieren.}}}
{Ich lehne es ab zu helfen, da im Amateurfunk keine Informationen für Dritte übermittelt werden dürfen.}
{Ich rufe per Funk mindestens dreimal MAYDAY gefolgt von meinem Rufzeichen, dem Standort und der Art der Notlage.}
{Ich lehne es ab zu helfen, da im Amateurfunk keine Notzeichen verwendet werden dürfen.}
\end{QQuestion}

}
\end{frame}

\begin{frame}
\frametitle{Was  tun, wenn man eine Notmeldung erhält?}
\begin{itemize}
  \item Bei einer Notmeldung sollte man zunächst aufmerksam zuhören und alle wichtigen Informationen notieren.
  \end{itemize}
\end{frame}

\begin{frame}
\only<1>{
\begin{QQuestion}{BF104}{Sie hören eine Notmeldung. Was tun Sie als Erstes?}{Ich wechsle die Frequenz oder schalte ab.}
{Ich sende dreimal MAYDAY, mein Rufzeichen und warte auf Antwort.}
{Ich stimme meinen Sender auf der Frequenz ab.}
{Ich höre aufmerksam zu und notiere alle wichtigen Informationen.}
\end{QQuestion}

}
\only<2>{
\begin{QQuestion}{BF104}{Sie hören eine Notmeldung. Was tun Sie als Erstes?}{Ich wechsle die Frequenz oder schalte ab.}
{Ich sende dreimal MAYDAY, mein Rufzeichen und warte auf Antwort.}
{Ich stimme meinen Sender auf der Frequenz ab.}
{\textbf{\textcolor{DARCgreen}{Ich höre aufmerksam zu und notiere alle wichtigen Informationen.}}}
\end{QQuestion}

}
\end{frame}

\begin{frame}
\frametitle{Was tun, wenn eine Rettungsorganisation sich der Sache annimmt?}
\begin{itemize}
  \item Wird die Notmeldung von einer Rettungsorganisation beantwortet, hält man sich zurück um den Funkbetrieb nicht zu stören.
  \end{itemize}
\end{frame}

\begin{frame}
\only<1>{
\begin{QQuestion}{BF105}{Sie haben eine Notmeldung aufgenommen, die nach kurzer Zeit von einer Rettungsorganisation beantwortet wird. Wie verhalten Sie sich?}{Ich bitte möglichst viele Funkamateure um Hilfe.}
{Ich biete zwischen zwei Durchgängen meine Hilfe an.}
{Ich stimme meinen Sender auf der Frequenz ab.}
{Ich störe auf keinen Fall den Funkbetrieb.}
\end{QQuestion}

}
\only<2>{
\begin{QQuestion}{BF105}{Sie haben eine Notmeldung aufgenommen, die nach kurzer Zeit von einer Rettungsorganisation beantwortet wird. Wie verhalten Sie sich?}{Ich bitte möglichst viele Funkamateure um Hilfe.}
{Ich biete zwischen zwei Durchgängen meine Hilfe an.}
{Ich stimme meinen Sender auf der Frequenz ab.}
{\textbf{\textcolor{DARCgreen}{Ich störe auf keinen Fall den Funkbetrieb.}}}
\end{QQuestion}

}
\end{frame}

\begin{frame}
\frametitle{Was tun, wenn zunächst niemand antwortet?}
\begin{itemize}
  \item Reagiert keine andere Funkstelle, beantwortet man den Ruf und informiert die Polizei oder Rettungsleitstelle.
  \end{itemize}
\end{frame}

\begin{frame}
\only<1>{
\begin{QQuestion}{BF106}{Sie haben eine Notmeldung aufgenommen. Keine andere Funkstelle reagiert und Sie könnten helfen. Wie verhalten Sie sich?}{Ich beantworte den Ruf und informiere die Polizei oder Rettungsleitstelle.}
{Ich schalte mein Funkgerät ab, um keine Probleme zu bekommen.}
{Ich wiederhole die Notmeldung umgehend auf derselben Frequenz.}
{Ich warte etwa eine Stunde, ob sich eine Rettungsorganisation meldet.}
\end{QQuestion}

}
\only<2>{
\begin{QQuestion}{BF106}{Sie haben eine Notmeldung aufgenommen. Keine andere Funkstelle reagiert und Sie könnten helfen. Wie verhalten Sie sich?}{\textbf{\textcolor{DARCgreen}{Ich beantworte den Ruf und informiere die Polizei oder Rettungsleitstelle.}}}
{Ich schalte mein Funkgerät ab, um keine Probleme zu bekommen.}
{Ich wiederhole die Notmeldung umgehend auf derselben Frequenz.}
{Ich warte etwa eine Stunde, ob sich eine Rettungsorganisation meldet.}
\end{QQuestion}

}
\end{frame}

\begin{frame}
\frametitle{Was tun, wenn man die zuständigen Stellen informiert hat?}
\begin{itemize}
  \item Idealerweise bleibt man erreichbar und gibt Informationen weiter, bis Hilfe eingetroffen ist.
  \end{itemize}
\end{frame}

\begin{frame}
\only<1>{
\begin{QQuestion}{BF107}{Sie haben eine Notmeldung beantwortet und die Polizei oder Rettungsleitstelle informiert. Welches Verhalten ist im Anschluss vorbildlich?}{Ich rufe regelmäßig die Polizei oder Rettungsleitstelle an und erkundige mich nach dem Stand.}
{Ich bleibe erreichbar und gebe Informationen weiter, bis Hilfe eingetroffen ist.}
{Ich schalte mein Funkgerät ab, da ich meiner Pflicht nachgekommen bin.}
{Ich informiere die Medien, damit über den Rettungseinsatz live berichtet werden kann.}
\end{QQuestion}

}
\only<2>{
\begin{QQuestion}{BF107}{Sie haben eine Notmeldung beantwortet und die Polizei oder Rettungsleitstelle informiert. Welches Verhalten ist im Anschluss vorbildlich?}{Ich rufe regelmäßig die Polizei oder Rettungsleitstelle an und erkundige mich nach dem Stand.}
{\textbf{\textcolor{DARCgreen}{Ich bleibe erreichbar und gebe Informationen weiter, bis Hilfe eingetroffen ist.}}}
{Ich schalte mein Funkgerät ab, da ich meiner Pflicht nachgekommen bin.}
{Ich informiere die Medien, damit über den Rettungseinsatz live berichtet werden kann.}
\end{QQuestion}

}
\end{frame}

\begin{frame}
\frametitle{Was ist bei internationaler Beteiligung zu beachten?}
\begin{itemize}
  \item Bei internationaler Beteiligung nutzt man UTC anstelle von lokaler Zeit.
  \end{itemize}
\end{frame}

\begin{frame}
\only<1>{
\begin{QQuestion}{BF108}{Sie haben am 16. August (Ortsdatum) um 20:00 Uhr mitteleuropäischer Sommerzeit (MESZ) von 9J2NG eine Notfunkmeldung aufgenommen und an eine Hilfeleistungsorganisation per Telefon weitergemeldet. Die Amateurfunkstelle 9J2NG hat Sie gebeten, um 23:00 Uhr UTC erneut mit ihr in Verbindung zu treten. Welcher Zeitpunkt ist dies in Deutschland?}{22:00 MESZ am 16. August (Ortsdatum)}
{21:00 MESZ am 16. August (Ortsdatum)}
{01:00 MESZ am 17. August (Ortsdatum)}
{00:00 MESZ am 18. August (Ortsdatum)}
\end{QQuestion}

}
\only<2>{
\begin{QQuestion}{BF108}{Sie haben am 16. August (Ortsdatum) um 20:00 Uhr mitteleuropäischer Sommerzeit (MESZ) von 9J2NG eine Notfunkmeldung aufgenommen und an eine Hilfeleistungsorganisation per Telefon weitergemeldet. Die Amateurfunkstelle 9J2NG hat Sie gebeten, um 23:00 Uhr UTC erneut mit ihr in Verbindung zu treten. Welcher Zeitpunkt ist dies in Deutschland?}{22:00 MESZ am 16. August (Ortsdatum)}
{21:00 MESZ am 16. August (Ortsdatum)}
{\textbf{\textcolor{DARCgreen}{01:00 MESZ am 17. August (Ortsdatum)}}}
{00:00 MESZ am 18. August (Ortsdatum)}
\end{QQuestion}

}
\end{frame}

\begin{frame}
\frametitle{Notzeichen}
\begin{itemize}
  \item Die Notzeichen außerhalb des Amateurfunks sind SOS und Mayday, diese dürfen im Amateurfunk nicht verwendet werden.
  \end{itemize}

\end{frame}

\begin{frame}
\only<1>{
\begin{QQuestion}{VD105}{Dürfen im Amateurfunkverkehr internationale Not-, Dringlichkeits- und Sicherheitszeichen (z. B. MAYDAY, PAN PAN, SÉCURITÉ) ausgesendet werden?}{Der Gebrauch dieser Zeichen ist auf den Kurzwellenbändern erlaubt.}
{Amateurfunkstellen in Küstennähe ist es erlaubt, diese Zeichen auszusenden.}
{Bei einem Notfall dürfen die Zeichen ausgesendet werden.}
{Der Gebrauch dieser Zeichen ist ausdrücklich untersagt.}
\end{QQuestion}

}
\only<2>{
\begin{QQuestion}{VD105}{Dürfen im Amateurfunkverkehr internationale Not-, Dringlichkeits- und Sicherheitszeichen (z. B. MAYDAY, PAN PAN, SÉCURITÉ) ausgesendet werden?}{Der Gebrauch dieser Zeichen ist auf den Kurzwellenbändern erlaubt.}
{Amateurfunkstellen in Küstennähe ist es erlaubt, diese Zeichen auszusenden.}
{Bei einem Notfall dürfen die Zeichen ausgesendet werden.}
{\textbf{\textcolor{DARCgreen}{Der Gebrauch dieser Zeichen ist ausdrücklich untersagt.}}}
\end{QQuestion}

}
\end{frame}

\begin{frame}
\only<1>{
\begin{QQuestion}{BF101}{Wie heißen die internationalen Notzeichen außerhalb des Amateurfunks?}{Prudence und TTT}
{Securité und Distresse}
{Distresse und DDD}
{Mayday und SOS}
\end{QQuestion}

}
\only<2>{
\begin{QQuestion}{BF101}{Wie heißen die internationalen Notzeichen außerhalb des Amateurfunks?}{Prudence und TTT}
{Securité und Distresse}
{Distresse und DDD}
{\textbf{\textcolor{DARCgreen}{Mayday und SOS}}}
\end{QQuestion}

}
\end{frame}

\begin{frame}
\only<1>{
\begin{QQuestion}{BF102}{Dürfen Sie im Notfall SOS oder Mayday innerhalb des Amateurfunks gebrauchen?}{SOS nicht, aber Mayday im Notfall.}
{Mayday nicht, aber SOS im Notfall.}
{Ja}
{Nein}
\end{QQuestion}

}
\only<2>{
\begin{QQuestion}{BF102}{Dürfen Sie im Notfall SOS oder Mayday innerhalb des Amateurfunks gebrauchen?}{SOS nicht, aber Mayday im Notfall.}
{Mayday nicht, aber SOS im Notfall.}
{Ja}
{\textbf{\textcolor{DARCgreen}{Nein}}}
\end{QQuestion}

}
\end{frame}

\begin{frame}
\frametitle{ Notfunkfrequenzen}
Die IARU hat für die ITU-Region 1 die folgenden Notfunkfrequenzen in den Bandplänen festgelegt, die für den Notfunkbetrieb frei zu halten sind:

\begin{itemize}
  \item 3.\qty{760}{\kilo\hertz}
  \item 7.\qty{110}{\kilo\hertz}
  \item 14.\qty{300}{\kilo\hertz}
  \item 18.\qty{160}{\kilo\hertz}
  \item 21.\qty{360}{\kilo\hertz}
  \end{itemize}
\end{frame}%ENDCONTENT
