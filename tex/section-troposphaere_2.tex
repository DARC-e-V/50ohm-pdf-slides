
\section{Troposphäre II}
\label{section:troposphaere_2}
\begin{frame}%STARTCONTENT

\frametitle{Begriffe}
Im folgenden Kapitel werden mehrere Begriffe verwendet, die vorab erklärt werden

\begin{itemize}
  \item \emph{Beugung}: Wellen werden an einem Hindernis abgelenkt
  \item \emph{Streuung}: Ablenkung der Wellen durch Interaktion von Teilchen
  \item \emph{Reflexion}: Gleichgerichtete Streuung
  \item \emph{Brechung} oder \emph{Refraktion}: Ablenkung der Wellen durch Änderung der Ausbreitungsgeschwindigkeit durch ein anderes Medium mit anderer Dichte
  \end{itemize}
\end{frame}

\begin{frame}
\begin{columns}
    \begin{column}{0.48\textwidth}
    \begin{itemize}
  \item Bereits bekannt: Die für den Amateurfunk relevanten Schichten in der Atmosphäre
  \item In der Troposphäre finden Erscheinungen des Wetters statt
  \end{itemize}

    \end{column}
   \begin{column}{0.48\textwidth}
       
\begin{figure}
    \DARCimage{0.85\linewidth}{731include}
    \caption{\scriptsize Für den Amateurfunk relevante Schichten in der Atmosphäre}
    \label{e_atmosphaeren_schichten}
\end{figure}


   \end{column}
\end{columns}

\end{frame}

\begin{frame}
\frametitle{DX in VHF/UHF}
\begin{columns}
    \begin{column}{0.48\textwidth}
    \begin{itemize}
  \item Überhorizontverbindungen bei VHF/UHF entstehen durch Beugung, Reflexion und Streuung in der Troposphäre
  \item Bereiche mit unterschiedlicher Temperatur und Dichte
  \end{itemize}

    \end{column}
   \begin{column}{0.48\textwidth}
       
\begin{figure}
    \DARCimage{0.85\linewidth}{734include}
    \caption{\scriptsize Troposphärische Ausbreitung an verschiedenen Luftschichten}
    \label{e_tropo}
\end{figure}


   \end{column}
\end{columns}

\end{frame}

\begin{frame}
\only<1>{
\begin{QQuestion}{EH301}{Was ist die \glqq Troposphäre\grqq{}? Die Troposphäre ist der Teil der Atmosphäre,~...}{in welchem Aurora-Erscheinungen auftreten können.}
{der sich über den Tropen befindet.}
{in dem es zur Bildung sporadischer E-Regionen kommen kann.}
{in der die Erscheinungen des Wetters stattfinden.}
\end{QQuestion}

}
\only<2>{
\begin{QQuestion}{EH301}{Was ist die \glqq Troposphäre\grqq{}? Die Troposphäre ist der Teil der Atmosphäre,~...}{in welchem Aurora-Erscheinungen auftreten können.}
{der sich über den Tropen befindet.}
{in dem es zur Bildung sporadischer E-Regionen kommen kann.}
{\textbf{\textcolor{DARCgreen}{in der die Erscheinungen des Wetters stattfinden.}}}
\end{QQuestion}

}
\end{frame}

\begin{frame}
\only<1>{
\begin{QQuestion}{EH302}{Überhorizontverbindungen im VHF/UHF-Bereich kommen u.~a. zustande durch~...}{Polarisationsdrehungen in der Troposphäre bei hoch liegender Bewölkung.}
{Beugung, Reflexion und Streuung der Wellen in der Troposphäre durch das Auftreten sporadischer D-Regionen.}
{Beugung, Reflexion und Streuung der Wellen an troposphärischen Bereichen unterschiedlicher Temperatur und Dichte.}
{Polarisationsdrehungen in der Troposphäre an Gewitterfronten.}
\end{QQuestion}

}
\only<2>{
\begin{QQuestion}{EH302}{Überhorizontverbindungen im VHF/UHF-Bereich kommen u.~a. zustande durch~...}{Polarisationsdrehungen in der Troposphäre bei hoch liegender Bewölkung.}
{Beugung, Reflexion und Streuung der Wellen in der Troposphäre durch das Auftreten sporadischer D-Regionen.}
{\textbf{\textcolor{DARCgreen}{Beugung, Reflexion und Streuung der Wellen an troposphärischen Bereichen unterschiedlicher Temperatur und Dichte.}}}
{Polarisationsdrehungen in der Troposphäre an Gewitterfronten.}
\end{QQuestion}

}
\end{frame}

\begin{frame}
\only<1>{
\begin{QQuestion}{EH303}{Für VHF-Weitverkehrsverbindungen wird hauptsächlich die~...}{ionosphärische Ausbreitung genutzt.}
{troposphärische Ausbreitung genutzt.}
{Bodenwellenausbreitung genutzt.}
{Oberflächenwellenausbreitung genutzt.}
\end{QQuestion}

}
\only<2>{
\begin{QQuestion}{EH303}{Für VHF-Weitverkehrsverbindungen wird hauptsächlich die~...}{ionosphärische Ausbreitung genutzt.}
{\textbf{\textcolor{DARCgreen}{troposphärische Ausbreitung genutzt.}}}
{Bodenwellenausbreitung genutzt.}
{Oberflächenwellenausbreitung genutzt.}
\end{QQuestion}

}
\end{frame}%ENDCONTENT
