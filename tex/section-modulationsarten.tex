
\section{Modulationsarten}
\label{section:modulationsarten}
\begin{frame}%STARTCONTENT

\begin{columns}
    \begin{column}{0.48\textwidth}
    
\begin{figure}
    \DARCimage{0.85\linewidth}{790include}
    \caption{\scriptsize Schwingung}
    \label{n_schwingung}
\end{figure}


    \end{column}
   \begin{column}{0.48\textwidth}
       Eigenschaften einer elektrischen Schwingung:

\begin{itemize}
  \item Amplitude
  \item Frequenz
  \end{itemize}

   \end{column}
\end{columns}

\end{frame}

\begin{frame}
\frametitle{Amplitudenmodulation (AM)}

\begin{figure}
    \DARCimage{0.85\linewidth}{358include}
    \caption{\scriptsize Bei der Amplitudenmodulation (AM) wird die Amplitude einer elektrischen Schwingung verändert.}
    \label{n_modulationsarten_am}
\end{figure}

\end{frame}

\begin{frame}
\frametitle{Frequenzmodulation (FM)}

\begin{figure}
    \DARCimage{0.85\linewidth}{357include}
    \caption{\scriptsize Bei der Frequenzmodulation (FM) wird die Schwingungsdauer und somit die Frequenz verändert.}
    \label{n_modulationsarten_am}
\end{figure}

\end{frame}%ENDCONTENT
