
\section{Brückenschaltung}
\label{section:brueckenschaltung}
\begin{frame}%STARTCONTENT

\only<1>{
\begin{PQuestion}{AD111}{In welchem Verhältnis müssen die Widerstände $R_1$ bis $R_4$ zueinander stehen, damit das Messinstrument im Brückenzweig keine Spannung anzeigt?}{$\dfrac{R_1}{R_4} = \dfrac{R_2}{R_3}$}
{$\dfrac{R_1}{R_2} = \dfrac{R_4}{R_3}$}
{$\dfrac{R_2}{R_1} = \dfrac{R_3}{R_4}$}
{$\dfrac{R_1}{R_2} = \dfrac{R_3}{R_4}$}
{\DARCimage{1.0\linewidth}{343include}}\end{PQuestion}

}
\only<2>{
\begin{PQuestion}{AD111}{In welchem Verhältnis müssen die Widerstände $R_1$ bis $R_4$ zueinander stehen, damit das Messinstrument im Brückenzweig keine Spannung anzeigt?}{$\dfrac{R_1}{R_4} = \dfrac{R_2}{R_3}$}
{$\dfrac{R_1}{R_2} = \dfrac{R_4}{R_3}$}
{$\dfrac{R_2}{R_1} = \dfrac{R_3}{R_4}$}
{\textbf{\textcolor{DARCgreen}{$\dfrac{R_1}{R_2} = \dfrac{R_3}{R_4}$}}}
{\DARCimage{1.0\linewidth}{343include}}\end{PQuestion}

}
\end{frame}

\begin{frame}
\only<1>{
\begin{PQuestion}{AD112}{Die Spannung an der Brückenschaltung beträgt \qty{10}{\V}. Alle Widerstände haben einen Wert von \qty{50}{\ohm}. Wie groß ist die Spannung zwischen A und B im Brückenzweig (gemessen von A nach B)?}{\qty{2,5}{\V}}
{\qty{-5}{\V}}
{\qty{5}{\V}}
{\qty{0}{\V}}
{\DARCimage{1.0\linewidth}{343include}}\end{PQuestion}

}
\only<2>{
\begin{PQuestion}{AD112}{Die Spannung an der Brückenschaltung beträgt \qty{10}{\V}. Alle Widerstände haben einen Wert von \qty{50}{\ohm}. Wie groß ist die Spannung zwischen A und B im Brückenzweig (gemessen von A nach B)?}{\qty{2,5}{\V}}
{\qty{-5}{\V}}
{\qty{5}{\V}}
{\textbf{\textcolor{DARCgreen}{\qty{0}{\V}}}}
{\DARCimage{1.0\linewidth}{343include}}\end{PQuestion}

}
\end{frame}

\begin{frame}
\only<1>{
\begin{PQuestion}{AD113}{Die Spannung an der Brückenschaltung beträgt \qty{11}{\V}. Die Widerstände haben folgende Werte: $R_1$ = \qty{1}{\kohm}; $R_2$ = \qty{10}{\kohm}; $R_3$ = \qty{10}{\kohm}; $R_4$ = \qty{1}{\kohm}. Wie groß ist die Spannung zwischen A und B im Brückenzweig (gemessen von A nach B)?}{$U_{AB} = \qty{-9}{\V}$}
{$U_{AB} = \qty{9}{\V}$ }
{$U_{AB} = \qty{10}{\V}$}
{$U_{AB} = \qty{-10}{\V}$}
{\DARCimage{1.0\linewidth}{343include}}\end{PQuestion}

}
\only<2>{
\begin{PQuestion}{AD113}{Die Spannung an der Brückenschaltung beträgt \qty{11}{\V}. Die Widerstände haben folgende Werte: $R_1$ = \qty{1}{\kohm}; $R_2$ = \qty{10}{\kohm}; $R_3$ = \qty{10}{\kohm}; $R_4$ = \qty{1}{\kohm}. Wie groß ist die Spannung zwischen A und B im Brückenzweig (gemessen von A nach B)?}{$U_{AB} = \qty{-9}{\V}$}
{\textbf{\textcolor{DARCgreen}{$U_{AB} = \qty{9}{\V}$ }}}
{$U_{AB} = \qty{10}{\V}$}
{$U_{AB} = \qty{-10}{\V}$}
{\DARCimage{1.0\linewidth}{343include}}\end{PQuestion}

}
\end{frame}

\begin{frame}
\frametitle{Lösungsweg}
\begin{itemize}
  \item gegeben: $R_1 = R_4 = 1kΩ$
  \item gegeben: $R_2 = R_3 = 10kΩ$
  \item gegeben: $U = 11V$
  \item gesucht: $U_{AB}$
  \end{itemize}
    \pause
    $\frac{U_A}{U} = \frac{R_1}{R_1 + R_2} \Rightarrow U_A = \frac{R_1}{R_1 + R_2} \cdot U = \frac{1kΩ}{1kΩ + 10kΩ} \cdot 11V = 1V$
    \pause
    $\frac{U_B}{U} = \frac{R_3}{R_3 + R_4} \Rightarrow U_B = \frac{R_3}{R_3 + R_4} \cdot U = \frac{10kΩ}{10kΩ + 1kΩ} \cdot 11V = 10V$
    \pause
    $U_{AB} = |U_A -- U_B| = |1V -- 10V| = 9V$



\end{frame}%ENDCONTENT
