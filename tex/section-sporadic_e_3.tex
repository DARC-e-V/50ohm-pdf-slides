
\section{Sporadic-E III}
\label{section:sporadic_e_3}
\begin{frame}%STARTCONTENT

\frametitle{Foliensatz in Arbeit}
2024-04-28: Die Inhalte werden noch aufbereitet.

Derzeit sind in diesem Abschnitt nur die Fragen sortiert enthalten.

Für das Selbststudium verweisen wir aktuell auf den Abschnitt Wellenausbreitung im DARC Online Lehrgang (\textcolor{DARCblue}{\faLink~\href{https://www.darc.de/der-club/referate/ajw/lehrgang-te/e09/}{www.darc.de/der-club/referate/ajw/lehrgang-te/e09/}}) für die Prüfung bis Juni 2024. Bis auf die Fragen hat sich an der Thematik nichts geändert. Das Thema war bisher Stoff der Klasse~E und wurde mit der neuen Prüfungsordnung auf alle drei Klassen aufgeteilt.

\end{frame}

\begin{frame}
\only<1>{
\begin{QQuestion}{AH301}{Bei \glqq Sporadic~E\grqq{}-Ausbreitung werden Wellen im VHF-Bereich gebrochen an~...}{besonders stark ionisierten Bereichen der E-Region.}
{Inversionen am unteren Rand der E-Region.}
{geomagnetischen Störungen am unteren Rand der E-Region.}
{Ionisationsspuren von Meteoriten in der E-Region.}
\end{QQuestion}

}
\only<2>{
\begin{QQuestion}{AH301}{Bei \glqq Sporadic~E\grqq{}-Ausbreitung werden Wellen im VHF-Bereich gebrochen an~...}{\textbf{\textcolor{DARCgreen}{besonders stark ionisierten Bereichen der E-Region.}}}
{Inversionen am unteren Rand der E-Region.}
{geomagnetischen Störungen am unteren Rand der E-Region.}
{Ionisationsspuren von Meteoriten in der E-Region.}
\end{QQuestion}

}
\end{frame}

\begin{frame}
\only<1>{
\begin{QQuestion}{AH214}{Wie groß ist in etwa die maximale Entfernung, die ein KW-Signal bei Refraktion (Brechung) in der E-Region auf der Erdoberfläche mit einem Sprung (Hop) überbrücken kann? Sie beträgt etwa~...}{\qty{9000}{\km}}
{\qty{1100}{\km}}
{\qty{4500}{\km}}
{\qty{2200}{\km}}
\end{QQuestion}

}
\only<2>{
\begin{QQuestion}{AH214}{Wie groß ist in etwa die maximale Entfernung, die ein KW-Signal bei Refraktion (Brechung) in der E-Region auf der Erdoberfläche mit einem Sprung (Hop) überbrücken kann? Sie beträgt etwa~...}{\qty{9000}{\km}}
{\qty{1100}{\km}}
{\qty{4500}{\km}}
{\textbf{\textcolor{DARCgreen}{\qty{2200}{\km}}}}
\end{QQuestion}

}

\end{frame}

\begin{frame}
\only<1>{
\begin{QQuestion}{AH220}{Wie wirkt sich \glqq Sporadic~E\grqq{} auf die höheren Kurzwellenbänder aus?}{Bei Überseeverbindungen tritt Flatterfading auf.}
{Die Signale werden stark verbrummt empfangen.}
{Die \glqq tote Zone\grqq{} wird reduziert oder verschwindet ganz.}
{Die ionosphärische Ausbreitung fällt komplett aus.}
\end{QQuestion}

}
\only<2>{
\begin{QQuestion}{AH220}{Wie wirkt sich \glqq Sporadic~E\grqq{} auf die höheren Kurzwellenbänder aus?}{Bei Überseeverbindungen tritt Flatterfading auf.}
{Die Signale werden stark verbrummt empfangen.}
{\textbf{\textcolor{DARCgreen}{Die \glqq tote Zone\grqq{} wird reduziert oder verschwindet ganz.}}}
{Die ionosphärische Ausbreitung fällt komplett aus.}
\end{QQuestion}

}
\end{frame}%ENDCONTENT
