
\section{Messungen am Sender}
\label{section:sender_messungen}
\begin{frame}%STARTCONTENT

\only<1>{
\begin{PQuestion}{AI608}{Was stellt die folgende Schaltung dar? }{Absorptionsfrequenzmesser}
{Messkopf zur HF-Leistungsmessung}
{Antennenimpedanzmesser}
{HF-Dipmeter}
{\DARCimage{1.0\linewidth}{576include}}\end{PQuestion}

}
\only<2>{
\begin{PQuestion}{AI608}{Was stellt die folgende Schaltung dar? }{Absorptionsfrequenzmesser}
{\textbf{\textcolor{DARCgreen}{Messkopf zur HF-Leistungsmessung}}}
{Antennenimpedanzmesser}
{HF-Dipmeter}
{\DARCimage{1.0\linewidth}{576include}}\end{PQuestion}

}
\end{frame}

\begin{frame}
\only<1>{
\begin{PQuestion}{AI605}{Was stellt die folgende Schaltung dar? }{Absorptionsfrequenzmesser}
{HF-Tastkopf}
{Antennenimpedanzmesser}
{HF-Dipmeter}
{\DARCimage{1.0\linewidth}{770include}}\end{PQuestion}

}
\only<2>{
\begin{PQuestion}{AI605}{Was stellt die folgende Schaltung dar? }{Absorptionsfrequenzmesser}
{\textbf{\textcolor{DARCgreen}{HF-Tastkopf}}}
{Antennenimpedanzmesser}
{HF-Dipmeter}
{\DARCimage{1.0\linewidth}{770include}}\end{PQuestion}

}
\end{frame}

\begin{frame}
\only<1>{
\begin{PQuestion}{AI604}{Wozu dient diese Schaltung? Sie dient~...}{als Messkopf zum Abgleich von HF-Schaltungen.}
{als hochohmiger Messkopf für einen vektoriellen Netzwerkanalyzer.}
{zur Messung der Resonanzfrequenz mit einem Frequenzzähler.}
{als Gleichspannungstastkopf zur genauen Einstellung der Versorgungsspannung.}
{\DARCimage{1.0\linewidth}{770include}}\end{PQuestion}

}
\only<2>{
\begin{PQuestion}{AI604}{Wozu dient diese Schaltung? Sie dient~...}{\textbf{\textcolor{DARCgreen}{als Messkopf zum Abgleich von HF-Schaltungen.}}}
{als hochohmiger Messkopf für einen vektoriellen Netzwerkanalyzer.}
{zur Messung der Resonanzfrequenz mit einem Frequenzzähler.}
{als Gleichspannungstastkopf zur genauen Einstellung der Versorgungsspannung.}
{\DARCimage{1.0\linewidth}{770include}}\end{PQuestion}

}
\end{frame}

\begin{frame}
\only<1>{
\begin{PQuestion}{AI609}{Sie wollen mit der folgenden Messschaltung die Ausgangsleistung eines \qty{2}{\m}-Senders überprüfen, der voraussichtlich ca. \qty{15}{\W} HF-Leistung liefert. Was sollte für die Messung vor die dargestellte Messschaltung geschaltet werden?}{Dämpfungsglied \qty{20}{\decibel}, \qty{20}{\W}}
{\qty{25}{\m} langes Koaxialkabel vom Typ RG213 (MIL)}
{Stehwellenmessgerät}
{Adapter BNC-Buchse auf N-Stecker}
{\DARCimage{1.0\linewidth}{576include}}\end{PQuestion}

}
\only<2>{
\begin{PQuestion}{AI609}{Sie wollen mit der folgenden Messschaltung die Ausgangsleistung eines \qty{2}{\m}-Senders überprüfen, der voraussichtlich ca. \qty{15}{\W} HF-Leistung liefert. Was sollte für die Messung vor die dargestellte Messschaltung geschaltet werden?}{\textbf{\textcolor{DARCgreen}{Dämpfungsglied \qty{20}{\decibel}, \qty{20}{\W}}}}
{\qty{25}{\m} langes Koaxialkabel vom Typ RG213 (MIL)}
{Stehwellenmessgerät}
{Adapter BNC-Buchse auf N-Stecker}
{\DARCimage{1.0\linewidth}{576include}}\end{PQuestion}

}
\end{frame}

\begin{frame}
\only<1>{
\begin{PQuestion}{AI612}{Was muss für die genaue Messung der HF-Ausgangsleistung eines Senders mit einer solchen Schaltung berücksichtigt werden?}{Bei den Umrechnungen darf nur mit dem Effektivwert gerechnet werden.}
{$R_1$ muss genau \qty{50}{\ohm} betragen.}
{Korrekturwerte für die Schaltung, die aus einer Kalibrierung stammen.}
{Die Schaltung muss vor jeder Messung mit einem Spektrumanalysator überprüft werden.}
{\DARCimage{1.0\linewidth}{577include}}\end{PQuestion}

}
\only<2>{
\begin{PQuestion}{AI612}{Was muss für die genaue Messung der HF-Ausgangsleistung eines Senders mit einer solchen Schaltung berücksichtigt werden?}{Bei den Umrechnungen darf nur mit dem Effektivwert gerechnet werden.}
{$R_1$ muss genau \qty{50}{\ohm} betragen.}
{\textbf{\textcolor{DARCgreen}{Korrekturwerte für die Schaltung, die aus einer Kalibrierung stammen.}}}
{Die Schaltung muss vor jeder Messung mit einem Spektrumanalysator überprüft werden.}
{\DARCimage{1.0\linewidth}{577include}}\end{PQuestion}

}
\end{frame}

\begin{frame}
\only<1>{
\begin{PQuestion}{AI610}{Dem Eingang der folgenden Messschaltung wird eine HF-Leistung von \qty{1}{\W} zugeführt. D ist eine Schottkydiode mit $U_F$ = \qty{0,23}{\V}. Welche Spannung $U_A$ ist am Ausgang A zu erwarten, wenn die Messung mit einem hochohmigen Spannungsmessgerät erfolgt?}{\qty{9,8}{\V}}
{\qty{3,3}{\V}}
{\qty{7,1}{\V}}
{\qty{4,8}{\V}}
{\DARCimage{1.0\linewidth}{576include}}\end{PQuestion}

}
\only<2>{
\begin{PQuestion}{AI610}{Dem Eingang der folgenden Messschaltung wird eine HF-Leistung von \qty{1}{\W} zugeführt. D ist eine Schottkydiode mit $U_F$ = \qty{0,23}{\V}. Welche Spannung $U_A$ ist am Ausgang A zu erwarten, wenn die Messung mit einem hochohmigen Spannungsmessgerät erfolgt?}{\qty{9,8}{\V}}
{\qty{3,3}{\V}}
{\qty{7,1}{\V}}
{\textbf{\textcolor{DARCgreen}{\qty{4,8}{\V}}}}
{\DARCimage{1.0\linewidth}{576include}}\end{PQuestion}

}
\end{frame}

\begin{frame}
\frametitle{Lösungsweg}
\begin{itemize}
  \item gegeben: $P_E = 1W$
  \item gegeben: $U_F = 0,23V$
  \item gegeben: $R_V = 110Ω$, $R_T = 330Ω$
  \item gesucht: $U_A$
  \end{itemize}
    \pause
    $R = (\frac{1}{R_T + R_T} + \frac{1}{R_V} + \frac{1}{R_V})^{-1} = (\frac{1}{330Ω + 330Ω} + \frac{1}{110Ω} + \frac{1}{110Ω})^{-1} = 50,77Ω$
    \pause
    $P_E = \frac{U_{E,eff}^2}{R} \Rightarrow U_{E,eff} = \sqrt{P_E \cdot R} = \sqrt{1W \cdot 50,77Ω} = 7,125V$

$U_S = U_{E,eff} \cdot \sqrt{2} = 7,071V \cdot 1,414 = 10,07V$
    \pause
    $U_A = \frac{U_S}{2} -- U_F = \frac{10,07V}{2} -- 0,23V = 5,035V -- 0,23V = 4,805V \approx 4,8V$



\end{frame}

\begin{frame}
\only<1>{
\begin{PQuestion}{AI611}{Bei der folgenden Schaltung besteht $R_1$ aus einer Zusammenschaltung von Widerständen, die einen Gesamtwiderstand von \qty{54,1}{\ohm} hat und etwa \qty{200}{\W} aufnehmen kann. Die Diode ist eine Siliziumdiode mit $U_{\symup{F}}$ = \qty{0,7}{\V}. Am Ausgang wird mit einem digitalen Spannungsmessgerät eine Gleichspannung von \qty{14,9}{\V} gemessen. Wie groß ist etwa die HF-Leistung am Eingang der Schaltung?}{\qty{9,7}{\W}}
{\qty{37,8}{\W}}
{\qty{4,9}{\W}}
{\qty{19,4}{\W}}
{\DARCimage{1.0\linewidth}{577include}}\end{PQuestion}

}
\only<2>{
\begin{PQuestion}{AI611}{Bei der folgenden Schaltung besteht $R_1$ aus einer Zusammenschaltung von Widerständen, die einen Gesamtwiderstand von \qty{54,1}{\ohm} hat und etwa \qty{200}{\W} aufnehmen kann. Die Diode ist eine Siliziumdiode mit $U_{\symup{F}}$ = \qty{0,7}{\V}. Am Ausgang wird mit einem digitalen Spannungsmessgerät eine Gleichspannung von \qty{14,9}{\V} gemessen. Wie groß ist etwa die HF-Leistung am Eingang der Schaltung?}{\textbf{\textcolor{DARCgreen}{\qty{9,7}{\W}}}}
{\qty{37,8}{\W}}
{\qty{4,9}{\W}}
{\qty{19,4}{\W}}
{\DARCimage{1.0\linewidth}{577include}}\end{PQuestion}

}
\end{frame}

\begin{frame}
\frametitle{Lösungsweg}
\begin{itemize}
  \item gegeben: $U_A = 14,9V DC$
  \item gegeben: $U_F = 0,7V$
  \item gegeben: $R_1 = 54,1Ω$, $R_T = 330Ω$
  \item gesucht: $P_E$
  \end{itemize}
    \pause
    $R = (\frac{1}{R_T + R_T} + \frac{1}{R_1})^{-1} = (\frac{1}{330Ω + 330Ω} + \frac{1}{54,1Ω})^{-1} = 50Ω$
    \pause
    $U_S = (U_A + U_F) \cdot 2 = (14,9V + 0,7V) \cdot 2 = 31,2V$

$U_{E,eff} = \frac{U_S}{\sqrt{2}} = \frac{31,2V}{1,414} = 22,06V$
    \pause
    $P_E = \frac{U_{E,eff}^2}{R} = \frac{(22,06V)^2}{50Ω} \approx 9,7W$



\end{frame}

\begin{frame}
\only<1>{
\begin{PQuestion}{AI607}{Mit der folgenden Schaltung soll die Ausgangsleistung eines \qty{2}{\m}-FM-Handfunkgerätes gemessen werden. Die Dioden sind Schottkydioden mit $U_{\symup{F}}$~=~\qty{0,23}{\V}. Am Ausgang wird mit einem digitalen Spannungsmessgerät eine Gleichspannung von \qty{15,3}{\V} gemessen. Wie groß ist etwa die HF-Leistung am Eingang der Schaltung?}{Zirka \qty{1,2}{\W}}
{Zirka \qty{4,7}{\W}}
{Zirka \qty{600}{\mW}}
{Zirka \qty{2,4}{\W}}
{\DARCimage{1.0\linewidth}{771include}}\end{PQuestion}

}
\only<2>{
\begin{PQuestion}{AI607}{Mit der folgenden Schaltung soll die Ausgangsleistung eines \qty{2}{\m}-FM-Handfunkgerätes gemessen werden. Die Dioden sind Schottkydioden mit $U_{\symup{F}}$~=~\qty{0,23}{\V}. Am Ausgang wird mit einem digitalen Spannungsmessgerät eine Gleichspannung von \qty{15,3}{\V} gemessen. Wie groß ist etwa die HF-Leistung am Eingang der Schaltung?}{Zirka \qty{1,2}{\W}}
{Zirka \qty{4,7}{\W}}
{\textbf{\textcolor{DARCgreen}{Zirka \qty{600}{\mW}}}}
{Zirka \qty{2,4}{\W}}
{\DARCimage{1.0\linewidth}{771include}}\end{PQuestion}

}
\end{frame}

\begin{frame}
\frametitle{Lösungsweg}
\begin{itemize}
  \item gegeben: $U_A = 15,3V DC$
  \item gegeben: $U_F = 0,23V$
  \item gegeben: $R_{V1} = 56Ω$, $R_{V2} = 470Ω$
  \item gesucht: $P_E$
  \end{itemize}
    \pause
    $R = (\frac{1}{R_{V1}} + \frac{1}{R_{V2}})^{-1} = (\frac{1}{R_{56Ω}} + \frac{1}{R_{470Ω}})^{-1} = 50,04Ω$
    \pause
    $U_S = \frac{U_A}{2} + U_F = \frac{15,3V}{2} + 0,23V = 7,88V$

$U_{E,eff} = \frac{U_S}{\sqrt{2}} = \frac{7,88V}{1,414} = 5,57V$
    \pause
    $P_E = \frac{U_{E,eff}^2}{R} = \frac{{5,57V}^2}{50,04Ω} \approx 600mW$



\end{frame}

\begin{frame}
\only<1>{
\begin{PQuestion}{AI606}{Die Leistung eines \qty{2}{\metre}-Senders soll mit einer künstlichen \qty{50}{\ohm}-Antenne bestimmt werden, die über eine Anzapfung bei \qty{5}{\ohm} vom erdnahen Ende verfügt. Zur Messung an diesem Punkt wird die folgende Schaltung eingesetzt. Die Dioden sind Schottkydioden mit $U_{\symup{F}}~=$~\qty{0,23}{\V}. Am Ausgang der Schaltung wird dabei mit einem digitalen Spannungsmessgerät eine Gleichspannung von \qty{15,3}{\V} gemessen. Wie groß ist etwa die HF-Leistung des Senders?}{Zirka \qty{60}{\W}}
{Zirka \qty{480}{\W}}
{Zirka \qty{340}{\W}}
{Zirka \qty{240}{\W}}
{\DARCimage{1.0\linewidth}{770include}}\end{PQuestion}

}
\only<2>{
\begin{PQuestion}{AI606}{Die Leistung eines \qty{2}{\metre}-Senders soll mit einer künstlichen \qty{50}{\ohm}-Antenne bestimmt werden, die über eine Anzapfung bei \qty{5}{\ohm} vom erdnahen Ende verfügt. Zur Messung an diesem Punkt wird die folgende Schaltung eingesetzt. Die Dioden sind Schottkydioden mit $U_{\symup{F}}~=$~\qty{0,23}{\V}. Am Ausgang der Schaltung wird dabei mit einem digitalen Spannungsmessgerät eine Gleichspannung von \qty{15,3}{\V} gemessen. Wie groß ist etwa die HF-Leistung des Senders?}{\textbf{\textcolor{DARCgreen}{Zirka \qty{60}{\W}}}}
{Zirka \qty{480}{\W}}
{Zirka \qty{340}{\W}}
{Zirka \qty{240}{\W}}
{\DARCimage{1.0\linewidth}{770include}}\end{PQuestion}

}
\end{frame}

\begin{frame}
\frametitle{Lösungsweg}
\begin{itemize}
  \item gegeben: $U_A = 15,3V DC$
  \item gegeben: $U_F = 0,23V$
  \item gegeben: $R = 50Ω$ aus dem Messsystem
  \item gegeben: $R_A = 5Ω$ (10:1 Spannungsteiler)
  \item gesucht: $P_E$
  \end{itemize}
    \pause
    $U_S = \frac{U_A}{2} + U_F = \frac{15,3V}{2} + 0,23V = 7,88V$

$U_{E,eff} = \frac{U_S}{\sqrt{2}} = \frac{7,88V}{1,414} = 5,57V$
    \pause
    $P_E = \frac{(U_{E,eff} \cdot 10)^2}{R} = \frac{(5,57V \cdot 10)^2}{50Ω} \approx 60W$



\end{frame}

\begin{frame}
\only<1>{
\begin{PQuestion}{AI613}{Was stellt die folgende Schaltung dar? }{Antennenimpedanzmesser}
{Einfacher Peilsender}
{Feldstärkeanzeiger}
{Resonanzmessgerät}
{\DARCimage{1.0\linewidth}{496include}}\end{PQuestion}

}
\only<2>{
\begin{PQuestion}{AI613}{Was stellt die folgende Schaltung dar? }{Antennenimpedanzmesser}
{Einfacher Peilsender}
{\textbf{\textcolor{DARCgreen}{Feldstärkeanzeiger}}}
{Resonanzmessgerät}
{\DARCimage{1.0\linewidth}{496include}}\end{PQuestion}

}
\end{frame}%ENDCONTENT
