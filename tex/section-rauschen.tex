
\section{Rauschen}
\label{section:rauschen}
\begin{frame}%STARTCONTENT

\only<1>{
\begin{QQuestion}{AB408}{Für Messzwecke speisen Sie in den Antenneneingang Ihres Empfängers ein gleichmäßig über alle Frequenzen verteiltes Rauschsignal aus einem Messender ein (weißes Rauschen). Welche Aussage über die Leistung, die man beim Empfang dieses Signals misst, stimmt?}{Sie ist umgekehrt proportional zur Bandbreite des Empfängers.}
{Sie ist proportional zur Bandbreite des Empfängers.}
{Sie ist proportional zum Signal-Rausch-Abstand des Empfängers}
{Sie ist umgekehrt proportional zum Eingangswiderstand des Empfängers.}
\end{QQuestion}

}
\only<2>{
\begin{QQuestion}{AB408}{Für Messzwecke speisen Sie in den Antenneneingang Ihres Empfängers ein gleichmäßig über alle Frequenzen verteiltes Rauschsignal aus einem Messender ein (weißes Rauschen). Welche Aussage über die Leistung, die man beim Empfang dieses Signals misst, stimmt?}{Sie ist umgekehrt proportional zur Bandbreite des Empfängers.}
{\textbf{\textcolor{DARCgreen}{Sie ist proportional zur Bandbreite des Empfängers.}}}
{Sie ist proportional zum Signal-Rausch-Abstand des Empfängers}
{Sie ist umgekehrt proportional zum Eingangswiderstand des Empfängers.}
\end{QQuestion}

}
\end{frame}

\begin{frame}
\only<1>{
\begin{QQuestion}{AB409}{Wie verhält sich der Pegel des thermischen Rauschens am Empfängerausgang, wenn von einem Quarzfilter mit einer Bandbreite von \qty{2,5}{\kHz} auf ein Quarzfilter mit einer Bandbreite von \qty{0,5}{\kHz} mit gleicher Durchlassdämpfung und Flankensteilheit umgeschaltet wird? Der Rauschleistungspegel~...}{verringert sich um etwa \qty{14}{\decibel}.}
{erhöht sich um etwa \qty{7}{\decibel}.}
{verringert sich um etwa \qty{7}{\decibel}.}
{erhöht sich um etwa \qty{14}{\decibel}.}
\end{QQuestion}

}
\only<2>{
\begin{QQuestion}{AB409}{Wie verhält sich der Pegel des thermischen Rauschens am Empfängerausgang, wenn von einem Quarzfilter mit einer Bandbreite von \qty{2,5}{\kHz} auf ein Quarzfilter mit einer Bandbreite von \qty{0,5}{\kHz} mit gleicher Durchlassdämpfung und Flankensteilheit umgeschaltet wird? Der Rauschleistungspegel~...}{verringert sich um etwa \qty{14}{\decibel}.}
{erhöht sich um etwa \qty{7}{\decibel}.}
{\textbf{\textcolor{DARCgreen}{verringert sich um etwa \qty{7}{\decibel}.}}}
{erhöht sich um etwa \qty{14}{\decibel}.}
\end{QQuestion}

}
\end{frame}

\begin{frame}
\frametitle{Lösungsweg}
\begin{itemize}
  \item gegeben: $B_1 = 2,5kHz$
  \item gegeben: $B_2 = 0,5kHz$
  \item gesucht: $\Delta P_R$
  \end{itemize}
    \pause
    $\Delta P_R = 10 \cdot \log_{10}{(\frac{B_1}{B_2})}dB = 10 \cdot \log_{10}{(\frac{2,5kHz}{0,5kHz})}dB \approx 7dB$



\end{frame}%ENDCONTENT
