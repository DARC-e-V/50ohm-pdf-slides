
\section{Halbwertsbreite}
\label{section:halbwertsbreite}
\begin{frame}%STARTCONTENT

\only<1>{
\begin{QQuestion}{AG219}{Die Halbwertsbreite einer Antenne ist der Winkelbereich, innerhalb dessen~...}{die Feldstärke auf nicht weniger als den 0,707-fachen Wert der maximalen Feldstärke absinkt.}
{die Feldstärke auf nicht weniger als die Hälfte der maximalen Feldstärke absinkt.}
{die Strahlungsdichte auf nicht weniger als den $\dfrac{1}{\sqrt{2}}$-fachen Wert der maximalen Strahlungsdichte absinkt.}
{die abgestrahlte Leistung auf nicht weniger als den $\dfrac{1}{\sqrt{2}}$-fachen Wert des Leistungsmaximums absinkt.}
\end{QQuestion}

}
\only<2>{
\begin{QQuestion}{AG219}{Die Halbwertsbreite einer Antenne ist der Winkelbereich, innerhalb dessen~...}{\textbf{\textcolor{DARCgreen}{die Feldstärke auf nicht weniger als den 0,707-fachen Wert der maximalen Feldstärke absinkt.}}}
{die Feldstärke auf nicht weniger als die Hälfte der maximalen Feldstärke absinkt.}
{die Strahlungsdichte auf nicht weniger als den $\dfrac{1}{\sqrt{2}}$-fachen Wert der maximalen Strahlungsdichte absinkt.}
{die abgestrahlte Leistung auf nicht weniger als den $\dfrac{1}{\sqrt{2}}$-fachen Wert des Leistungsmaximums absinkt.}
\end{QQuestion}

}
\end{frame}

\begin{frame}
\only<1>{
\begin{PQuestion}{AG220}{In dem folgenden Richtdiagramm sind auf der Skala der relativen Feldstärke $\frac{E}{{E}_{\symup{max}}}$ die Punkte~a bis~d markiert. Durch welchen der Punkte a bis d ziehen Sie den Kreisbogen, um die Halbwertsbreite der Antenne an den Schnittpunkten des Kreises mit der Richtkeule ablesen zu können?}{Durch den Punkt~a.}
{Durch den Punkt~b.}
{Durch den Punkt~d.}
{Durch den Punkt~c.}
{\DARCimage{1.0\linewidth}{266include}}\end{PQuestion}

}
\only<2>{
\begin{PQuestion}{AG220}{In dem folgenden Richtdiagramm sind auf der Skala der relativen Feldstärke $\frac{E}{{E}_{\symup{max}}}$ die Punkte~a bis~d markiert. Durch welchen der Punkte a bis d ziehen Sie den Kreisbogen, um die Halbwertsbreite der Antenne an den Schnittpunkten des Kreises mit der Richtkeule ablesen zu können?}{Durch den Punkt~a.}
{Durch den Punkt~b.}
{Durch den Punkt~d.}
{\textbf{\textcolor{DARCgreen}{Durch den Punkt~c.}}}
{\DARCimage{1.0\linewidth}{266include}}\end{PQuestion}

}
\end{frame}

\begin{frame}
\only<1>{
\begin{PQuestion}{AG221}{Die folgende Skizze zeigt das Horizontaldiagramm der relativen Feldstärke einer Yagi-Uda-Antenne. Wie groß ist im vorliegenden Fall die Halbwertsbreite (Öffnungswinkel)?}{Etwa \qty{27}{\degree}}
{Etwa \qty{34}{\degree}}
{Etwa \qty{69}{\degree}}
{Etwa \qty{55}{\degree}}
{\DARCimage{1.0\linewidth}{266include}}\end{PQuestion}

}
\only<2>{
\begin{PQuestion}{AG221}{Die folgende Skizze zeigt das Horizontaldiagramm der relativen Feldstärke einer Yagi-Uda-Antenne. Wie groß ist im vorliegenden Fall die Halbwertsbreite (Öffnungswinkel)?}{Etwa \qty{27}{\degree}}
{Etwa \qty{34}{\degree}}
{Etwa \qty{69}{\degree}}
{\textbf{\textcolor{DARCgreen}{Etwa \qty{55}{\degree}}}}
{\DARCimage{1.0\linewidth}{266include}}\end{PQuestion}

}
\end{frame}%ENDCONTENT
