
\section{Ionosphäre}
\label{section:ionosphaere}
\begin{frame}%STARTCONTENT

\begin{columns}
    \begin{column}{0.48\textwidth}
     
\begin{figure}
    \DARCimage{0.85\linewidth}{732include}
    \caption{\scriptsize Brechung an der Ionosphäre}
    \label{n_ionosphaere}
\end{figure}


    \end{column}
   \begin{column}{0.48\textwidth}
       \begin{itemize}
  \item Im oberen Teil der Erdatmosphäre
  \item Hat großen Einfluss im Kurzwellenbereich
  \item Sonnenstrahlung erzeugt elektrisch geladene Teilchen
  \item Funkwellen werden daran gebrochen (refraktiert) $\rightarrow$ Raumwelle
  \item Weltweite Funkverbindungen möglich
  \end{itemize}

   \end{column}
\end{columns}

\end{frame}

\begin{frame}
\only<1>{
\begin{QQuestion}{NH101}{Wie nennt sich der Bereich in der Atmosphäre, in dem die Kurzwellenausbreitung durch Brechung (Refraktion) ermöglicht wird?}{Hemisphäre}
{Magnetosphäre}
{Ionosphäre}
{Hydrosphäre}
\end{QQuestion}

}
\only<2>{
\begin{QQuestion}{NH101}{Wie nennt sich der Bereich in der Atmosphäre, in dem die Kurzwellenausbreitung durch Brechung (Refraktion) ermöglicht wird?}{Hemisphäre}
{Magnetosphäre}
{\textbf{\textcolor{DARCgreen}{Ionosphäre}}}
{Hydrosphäre}
\end{QQuestion}

}
\end{frame}

\begin{frame}
\only<1>{
\begin{QQuestion}{NH102}{Warum ist die Ionosphäre ausschlaggebend für die Kurzwellenausbreitung? In der Ionosphäre werden elektromagnetische Wellen durch~...}{Wärme verstärkt und reflektiert.}
{elektrisch geladene Teilchen gebrochen (refraktiert).}
{Kälte gebrochen und reflektiert.}
{Temperaturübergänge gebrochen (refraktiert).}
\end{QQuestion}

}
\only<2>{
\begin{QQuestion}{NH102}{Warum ist die Ionosphäre ausschlaggebend für die Kurzwellenausbreitung? In der Ionosphäre werden elektromagnetische Wellen durch~...}{Wärme verstärkt und reflektiert.}
{\textbf{\textcolor{DARCgreen}{elektrisch geladene Teilchen gebrochen (refraktiert).}}}
{Kälte gebrochen und reflektiert.}
{Temperaturübergänge gebrochen (refraktiert).}
\end{QQuestion}

}
\end{frame}

\begin{frame}
\begin{columns}
    \begin{column}{0.48\textwidth}
    
\begin{figure}
    \DARCimage{0.85\linewidth}{729include}
    \caption{\scriptsize Die Anzahl der Sonnenflecken, die über den elfjährige Sonnenzyklus schwankt}
    \label{n_ionosphaere_sonnenflecken}
\end{figure}


    \end{column}
   \begin{column}{0.48\textwidth}
       \begin{itemize}
  \item Ausbreitungsbedingungen wechseln täglich und nach Jahreszeit
  \item Sonnenflecken haben einen großen Einfluss
  \item Alle 11 Jahre treten starke Sonnenflecken auf
  \item Mehr elektromagnetische Strahlung und Materie in der Ionosphäre
  \end{itemize}

   \end{column}
\end{columns}

\end{frame}

\begin{frame}
\only<1>{
\begin{QQuestion}{NH201}{Was ist ein wesentlicher Faktor für die Ausbreitung von Kurzwellen über die Ionosphäre?}{Die Bandbreite der Antenne}
{Die Filterfunktion des Empfängers}
{Der elfjährige Sonnenzyklus}
{Die präzise Antennenausrichtung zum Äquator}
\end{QQuestion}

}
\only<2>{
\begin{QQuestion}{NH201}{Was ist ein wesentlicher Faktor für die Ausbreitung von Kurzwellen über die Ionosphäre?}{Die Bandbreite der Antenne}
{Die Filterfunktion des Empfängers}
{\textbf{\textcolor{DARCgreen}{Der elfjährige Sonnenzyklus}}}
{Die präzise Antennenausrichtung zum Äquator}
\end{QQuestion}

}
\end{frame}

\begin{frame}
\begin{columns}
    \begin{column}{0.48\textwidth}
    
\begin{figure}
    \DARCimage{0.85\linewidth}{741include}
    \caption{\scriptsize Die Tote Zone, die für die Bodenwelle zu nah und für die Raumwelle zu weit weg ist.}
    \label{n_ionosphaere_tote_zone}
\end{figure}


    \end{column}
   \begin{column}{0.48\textwidth}
       \begin{itemize}
  \item Auf Kurzwelle kann es zu einer \emph{Toten Zone} kommen
  \item Für die Bodenwelle zu weit -- für die Raumwelle zu nah
  \item Nur eine Seite einer Funkverbindung hörbar
  \end{itemize}

   \end{column}
\end{columns}

\end{frame}

\begin{frame}
\only<1>{
\begin{QQuestion}{BE106}{Eine Frequenz auf einem höheren Kurzwellenband erscheint zunächst frei, stellt sich aber anschließend als besetzt heraus. Was ist die häufigste Ursache dafür?}{Die auf dieser Frequenz sendende Station wurde durch den Mögel-Dellinger-Effekt kurzfristig unterbrochen.}
{Eine Station auf dieser Frequenz verwendet das andere Seitenband.}
{Für die auf dieser Frequenz sendenden Stationen sind die Ausbreitungsbedingungen zu schlecht.}
{Eine auf dieser Frequenz sendende Station liegt innerhalb der toten Zone und konnte daher von mir nicht gehört werden.}
\end{QQuestion}

}
\only<2>{
\begin{QQuestion}{BE106}{Eine Frequenz auf einem höheren Kurzwellenband erscheint zunächst frei, stellt sich aber anschließend als besetzt heraus. Was ist die häufigste Ursache dafür?}{Die auf dieser Frequenz sendende Station wurde durch den Mögel-Dellinger-Effekt kurzfristig unterbrochen.}
{Eine Station auf dieser Frequenz verwendet das andere Seitenband.}
{Für die auf dieser Frequenz sendenden Stationen sind die Ausbreitungsbedingungen zu schlecht.}
{\textbf{\textcolor{DARCgreen}{Eine auf dieser Frequenz sendende Station liegt innerhalb der toten Zone und konnte daher von mir nicht gehört werden.}}}
\end{QQuestion}

}
\end{frame}%ENDCONTENT
