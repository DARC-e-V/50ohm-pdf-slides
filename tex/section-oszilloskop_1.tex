
\section{Oszilloskop I}
\label{section:oszilloskop_1}
\begin{frame}%STARTCONTENT

\frametitle{Periode}
\begin{columns}
    \begin{column}{0.48\textwidth}
    \begin{itemize}
  \item Dauer einer vollständigen Schwingung
  \item Wird zur Ermittlung der Frequenz benötigt, z.B. Oszilloskop
  \end{itemize}

    \end{column}
   \begin{column}{0.48\textwidth}
       
\begin{figure}
    \DARCimage{0.85\linewidth}{790include}
    \caption{\scriptsize Periode und Amplitude in einer Sinusschwingung}
    \label{e_periode_amplitude}
\end{figure}


   \end{column}
\end{columns}

\end{frame}

\begin{frame}\begin{itemize}
  \item Periode steht im umgekehrten Verhältnis zur Frequenz
  \item Formelzeichen T, Einheit Sekunde (s)
  \end{itemize}
    \pause
    $T = \dfrac{1}{f} \Rightarrow f = \dfrac{1}{T}$



\end{frame}

\begin{frame}\end{frame}

\begin{frame}
\only<1>{
\begin{QQuestion}{EB408}{Die Periodendauer von \qty{50}{\us} entspricht einer Frequenz von~...}{\qty{2}{\MHz}.}
{\qty{20}{\kHz}.}
{\qty{200}{\kHz}.}
{\qty{20}{\MHz}.}
\end{QQuestion}

}
\only<2>{
\begin{QQuestion}{EB408}{Die Periodendauer von \qty{50}{\us} entspricht einer Frequenz von~...}{\qty{2}{\MHz}.}
{\textbf{\textcolor{DARCgreen}{\qty{20}{\kHz}.}}}
{\qty{200}{\kHz}.}
{\qty{20}{\MHz}.}
\end{QQuestion}

}
\end{frame}

\begin{frame}
\frametitle{Periodendauer ablesen}
\begin{columns}
    \begin{column}{0.48\textwidth}
    \begin{itemize}
  \item Kästchen einer ganzen Periode im Nulldurchgang zählen
  \item Mit der Zeiteinheit multiplizieren
  \item Bei 8 Kästchen und 2 ms pro Kästchen $\rightarrow$ 8 $\cdot$ 2 ms = 16 ms
  \end{itemize}

    \end{column}
   \begin{column}{0.48\textwidth}
       
\begin{figure}
    \DARCimage{0.85\linewidth}{36include}
    \caption{\scriptsize Eine Sinuswelle auf dem Bildschirm eine Oszilloskops}
    \label{e_sinuswelle_oszilloskop}
\end{figure}


   \end{column}
\end{columns}

\end{frame}

\begin{frame}
\only<1>{
\begin{PQuestion}{EI301}{Die Zeitbasis eines Oszilloskop ist so eingestellt, dass ein Skalenteil \qty{0,5}{\ms} entspricht. Welche Periodendauer hat die angelegte Spannung?}{\qty{4}{\ms}}
{\qty{2}{\ms}}
{\qty{1,5}{\ms}}
{\qty{3}{\ms}}
{\DARCimage{1.0\linewidth}{36include}}\end{PQuestion}

}
\only<2>{
\begin{PQuestion}{EI301}{Die Zeitbasis eines Oszilloskop ist so eingestellt, dass ein Skalenteil \qty{0,5}{\ms} entspricht. Welche Periodendauer hat die angelegte Spannung?}{\textbf{\textcolor{DARCgreen}{\qty{4}{\ms}}}}
{\qty{2}{\ms}}
{\qty{1,5}{\ms}}
{\qty{3}{\ms}}
{\DARCimage{1.0\linewidth}{36include}}\end{PQuestion}

}
\end{frame}

\begin{frame}
\frametitle{Frequenz ermitteln}
$f = \dfrac{1}{T}$

Erst Periodendauer ermitteln, dann Frequenz ausrechnen

\end{frame}

\begin{frame}
\only<1>{
\begin{PQuestion}{EB410}{Welche Frequenz hat die in diesem Oszillogramm dargestellte Spannung?}{\qty{50}{\Hz}}
{\qty{100}{\Hz}}
{\qty{500}{\Hz}}
{\qty{20}{\Hz}}
{\DARCimage{1.0\linewidth}{56include}}\end{PQuestion}

}
\only<2>{
\begin{PQuestion}{EB410}{Welche Frequenz hat die in diesem Oszillogramm dargestellte Spannung?}{\textbf{\textcolor{DARCgreen}{\qty{50}{\Hz}}}}
{\qty{100}{\Hz}}
{\qty{500}{\Hz}}
{\qty{20}{\Hz}}
{\DARCimage{1.0\linewidth}{56include}}\end{PQuestion}

}

\end{frame}

\begin{frame}
\frametitle{Lösungsweg}
Eine Periode ist 4 Kästchen lang

$T = 4\cdot 5ms = 20ms$

$f = \dfrac{1}{T} = \dfrac{1}{20\cdot10^{-3}s} = $

$0,05\frac{1}{10^{-3}s} = 0,05\cdot10^3Hz = 0,05kHz = 50Hz$

\end{frame}

\begin{frame}
\only<1>{
\begin{PQuestion}{EI302}{Die Zeitbasis eines Oszilloskops ist so eingestellt, dass ein Skalenteil \qty{0,5}{\ms} entspricht. Welche Frequenz hat die angelegte Spannung? }{\qty{500}{\Hz}}
{\qty{250}{\Hz}}
{\qty{667}{\Hz}}
{\qty{333}{\Hz}}
{\DARCimage{1.0\linewidth}{36include}}\end{PQuestion}

}
\only<2>{
\begin{PQuestion}{EI302}{Die Zeitbasis eines Oszilloskops ist so eingestellt, dass ein Skalenteil \qty{0,5}{\ms} entspricht. Welche Frequenz hat die angelegte Spannung? }{\qty{500}{\Hz}}
{\textbf{\textcolor{DARCgreen}{\qty{250}{\Hz}}}}
{\qty{667}{\Hz}}
{\qty{333}{\Hz}}
{\DARCimage{1.0\linewidth}{36include}}\end{PQuestion}

}
\end{frame}

\begin{frame}
\only<1>{
\begin{PQuestion}{EB409}{Welche Frequenz hat die in diesem Oszillogramm dargestellte Spannung in etwa?}{\qty{833}{\kHz}}
{\qty{83,3}{\kHz}}
{\qty{8,33}{\MHz}}
{\qty{83,3}{\MHz}}
{\DARCimage{1.0\linewidth}{53include}}\end{PQuestion}

}
\only<2>{
\begin{PQuestion}{EB409}{Welche Frequenz hat die in diesem Oszillogramm dargestellte Spannung in etwa?}{\qty{833}{\kHz}}
{\textbf{\textcolor{DARCgreen}{\qty{83,3}{\kHz}}}}
{\qty{8,33}{\MHz}}
{\qty{83,3}{\MHz}}
{\DARCimage{1.0\linewidth}{53include}}\end{PQuestion}

}

\end{frame}

\begin{frame}
\frametitle{Lösungsweg}
Eine Periode ist 4 Kästchen lang

$T = 4\cdot 3\mu s = 12\mu s$

$f = \dfrac{1}{T} = \dfrac{1}{12\cdot10^{-6}s} = $

$0,0833\frac{1}{10^{-6}s} = 0,0833\cdot10^6Hz = 0,0833MHz = 83,3kHz$

\end{frame}

\begin{frame}
\only<1>{
\begin{PQuestion}{EB411}{Welche Frequenz hat das in diesem Schirmbild dargestellte Signal?}{\qty{8,33}{\MHz}}
{\qty{83,3}{\MHz}}
{\qty{8,33}{\kHz}}
{\qty{833}{\kHz}}
{\DARCimage{1.0\linewidth}{55include}}\end{PQuestion}

}
\only<2>{
\begin{PQuestion}{EB411}{Welche Frequenz hat das in diesem Schirmbild dargestellte Signal?}{\textbf{\textcolor{DARCgreen}{\qty{8,33}{\MHz}}}}
{\qty{83,3}{\MHz}}
{\qty{8,33}{\kHz}}
{\qty{833}{\kHz}}
{\DARCimage{1.0\linewidth}{55include}}\end{PQuestion}

}
\end{frame}

\begin{frame}
\frametitle{Impuls}
\begin{columns}
    \begin{column}{0.48\textwidth}
    \begin{itemize}
  \item Ein Signal springt von einem Wert auf einen höheren und zu einem späteren Zeitpunkt zurück
  \item Dauer des Impulses wird von Mitte der ansteigenden Flanke bis zur Mitte der abfallenden Flanke gemessen
  \end{itemize}

    \end{column}
   \begin{column}{0.48\textwidth}
       
\begin{figure}
    \DARCimage{0.85\linewidth}{57include}
    \caption{\scriptsize Impuls in einem Oszilloskop}
    \label{e_impuls}
\end{figure}

 
   \end{column}
\end{columns}

\end{frame}

\begin{frame}
\only<1>{
\begin{PQuestion}{EI303}{Die Impulsdauer beträgt hier~...}{\qty{230}{\micro\s}.}
{\qty{260}{\micro\s}.}
{\qty{200}{\micro\s}.}
{\qty{150}{\micro\s}.}
{\DARCimage{1.0\linewidth}{57include}}\end{PQuestion}

}
\only<2>{
\begin{PQuestion}{EI303}{Die Impulsdauer beträgt hier~...}{\qty{230}{\micro\s}.}
{\qty{260}{\micro\s}.}
{\textbf{\textcolor{DARCgreen}{\qty{200}{\micro\s}.}}}
{\qty{150}{\micro\s}.}
{\DARCimage{1.0\linewidth}{57include}}\end{PQuestion}

}
\end{frame}

\begin{frame}
\frametitle{NF-Verzerrungen}
\begin{itemize}
  \item Verzerrungen sind Abweichungen von der Sinusform
  \item Diese können mit einem Oszilloskop sichtbar gemacht werden
  \end{itemize}
\end{frame}

\begin{frame}
\only<1>{
\begin{QQuestion}{EI304}{Welches dieser Geräte wird für die Anzeige von NF-Verzerrungen verwendet?}{Ein Oszilloskop}
{Ein Transistorvoltmeter}
{Ein Vielfachmessgerät}
{Ein Frequenzzähler}
\end{QQuestion}

}
\only<2>{
\begin{QQuestion}{EI304}{Welches dieser Geräte wird für die Anzeige von NF-Verzerrungen verwendet?}{\textbf{\textcolor{DARCgreen}{Ein Oszilloskop}}}
{Ein Transistorvoltmeter}
{Ein Vielfachmessgerät}
{Ein Frequenzzähler}
\end{QQuestion}

}
\end{frame}%ENDCONTENT
