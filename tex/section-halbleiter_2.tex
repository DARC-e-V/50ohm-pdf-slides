
\section{Halbleiter II}
\label{section:halbleiter_2}
\begin{frame}%STARTCONTENT

\only<1>{
\begin{QQuestion}{AC402}{Wie verhalten sich die Elektronen in einem in Durchlassrichtung betriebenen PN-Übergang?}{Sie wandern von N nach P.}
{Sie wandern von P nach N.}
{Sie bleiben im N-Bereich.}
{Sie zerfallen beim Übergang.}
\end{QQuestion}

}
\only<2>{
\begin{QQuestion}{AC402}{Wie verhalten sich die Elektronen in einem in Durchlassrichtung betriebenen PN-Übergang?}{\textbf{\textcolor{DARCgreen}{Sie wandern von N nach P.}}}
{Sie wandern von P nach N.}
{Sie bleiben im N-Bereich.}
{Sie zerfallen beim Übergang.}
\end{QQuestion}

}
\end{frame}

\begin{frame}
\only<1>{
\begin{QQuestion}{AB104}{Was versteht man unter Halbleitermaterialien?}{Einige Stoffe (z. B. Silizium) sind in reinem Zustand bei Raumtemperatur gute Elektrolyten. Durch geringfügige Zusätze von geeigneten anderen Stoffen (z. B. Bismut, Tellur) kann man daraus entweder N-leitendes- oder P-leitendes Material für Anoden bzw. Kathoden von Batterien herstellen.}
{Einige Stoffe (z. B. Silizium) sind in reinem Zustand bei Raumtemperatur gute Leiter. Durch geringfügige Zusätze von geeigneten anderen Stoffen (z. B. Bor, Phosphor) oder bei hohen Temperaturen nimmt jedoch ihre Leitfähigkeit ab.}
{Einige Stoffe (z. B. Silizium) sind in reinem Zustand bei Raumtemperatur gute Leiter. Durch geringfügige Zusätze von geeigneten anderen Stoffen (z. B. Bismut, Tellur) fällt ihr Widerstand auf den halben Wert.}
{Einige Stoffe (z. B. Silizium) sind in reinem Zustand bei Raumtemperatur gute Isolatoren. Durch geringfügige Zusätze von geeigneten anderen Stoffen (z. B. Bor, Phosphor) oder bei hohen Temperaturen werden sie jedoch zu Leitern.}
\end{QQuestion}

}
\only<2>{
\begin{QQuestion}{AB104}{Was versteht man unter Halbleitermaterialien?}{Einige Stoffe (z. B. Silizium) sind in reinem Zustand bei Raumtemperatur gute Elektrolyten. Durch geringfügige Zusätze von geeigneten anderen Stoffen (z. B. Bismut, Tellur) kann man daraus entweder N-leitendes- oder P-leitendes Material für Anoden bzw. Kathoden von Batterien herstellen.}
{Einige Stoffe (z. B. Silizium) sind in reinem Zustand bei Raumtemperatur gute Leiter. Durch geringfügige Zusätze von geeigneten anderen Stoffen (z. B. Bor, Phosphor) oder bei hohen Temperaturen nimmt jedoch ihre Leitfähigkeit ab.}
{Einige Stoffe (z. B. Silizium) sind in reinem Zustand bei Raumtemperatur gute Leiter. Durch geringfügige Zusätze von geeigneten anderen Stoffen (z. B. Bismut, Tellur) fällt ihr Widerstand auf den halben Wert.}
{\textbf{\textcolor{DARCgreen}{Einige Stoffe (z. B. Silizium) sind in reinem Zustand bei Raumtemperatur gute Isolatoren. Durch geringfügige Zusätze von geeigneten anderen Stoffen (z. B. Bor, Phosphor) oder bei hohen Temperaturen werden sie jedoch zu Leitern.}}}
\end{QQuestion}

}
\end{frame}

\begin{frame}
\only<1>{
\begin{QQuestion}{AB105}{Was versteht man unter Dotierung?}{Das Einbringen von magnetischen Nord- oder Südpolen in einen Halbleitergrundstoff, um die Induktivität zu erhöhen.}
{Das Entfernen von Atomen aus dem Halbleitergrundstoff, um die elektrische Leitfähigkeit zu senken.}
{Das Einbringen von chemisch anderswertigen Fremdatomen in einen Halbleitergrundstoff, um freie Ladungsträger zur Verfügung zu stellen.}
{Das Entfernen von Verunreinigungen aus einem Halbleitergrundstoff, um Elektronen zu generieren.}
\end{QQuestion}

}
\only<2>{
\begin{QQuestion}{AB105}{Was versteht man unter Dotierung?}{Das Einbringen von magnetischen Nord- oder Südpolen in einen Halbleitergrundstoff, um die Induktivität zu erhöhen.}
{Das Entfernen von Atomen aus dem Halbleitergrundstoff, um die elektrische Leitfähigkeit zu senken.}
{\textbf{\textcolor{DARCgreen}{Das Einbringen von chemisch anderswertigen Fremdatomen in einen Halbleitergrundstoff, um freie Ladungsträger zur Verfügung zu stellen.}}}
{Das Entfernen von Verunreinigungen aus einem Halbleitergrundstoff, um Elektronen zu generieren.}
\end{QQuestion}

}
\end{frame}

\begin{frame}
\only<1>{
\begin{QQuestion}{AB106}{N-leitendes Halbleitermaterial ist gekennzeichnet durch~...}{einen Überschuss an beweglichen Elektronenlöchern.}
{ein Fehlen von Dotierungsatomen.}
{ein Fehlen von Atomen im Gitter des Halbleiterkristalls.}
{einen Überschuss an beweglichen Elektronen.}
\end{QQuestion}

}
\only<2>{
\begin{QQuestion}{AB106}{N-leitendes Halbleitermaterial ist gekennzeichnet durch~...}{einen Überschuss an beweglichen Elektronenlöchern.}
{ein Fehlen von Dotierungsatomen.}
{ein Fehlen von Atomen im Gitter des Halbleiterkristalls.}
{\textbf{\textcolor{DARCgreen}{einen Überschuss an beweglichen Elektronen.}}}
\end{QQuestion}

}
\end{frame}

\begin{frame}
\only<1>{
\begin{QQuestion}{AB107}{P-leitendes Halbleitermaterial ist gekennzeichnet durch~...}{einen Überschuss an beweglichen Elektronen.}
{ein Fehlen von Dotierungsatomen.}
{ein Fehlen von Atomen im Gitter des Halbleiterkristalls.}
{einen Überschuss an beweglichen Elektronenlöchern.}
\end{QQuestion}

}
\only<2>{
\begin{QQuestion}{AB107}{P-leitendes Halbleitermaterial ist gekennzeichnet durch~...}{einen Überschuss an beweglichen Elektronen.}
{ein Fehlen von Dotierungsatomen.}
{ein Fehlen von Atomen im Gitter des Halbleiterkristalls.}
{\textbf{\textcolor{DARCgreen}{einen Überschuss an beweglichen Elektronenlöchern.}}}
\end{QQuestion}

}
\end{frame}

\begin{frame}
\only<1>{
\begin{PQuestion}{AB108}{Das folgende Bild zeigt den prinzipiellen Aufbau einer Halbleiterdiode. Wie entsteht die Sperrschicht?}{An der Grenzschicht wandern Atome aus dem P-Teil in den N-Teil. Dadurch wird auf der P-Seite der Atommangel abgebaut, auf der N-Seite der Atommangel vergrößert. Es bildet sich auf beiden Seiten der Grenzfläche eine leitende Schicht.}
{An der Grenzschicht wandern Elektronen aus dem P-Teil in den N-Teil. Dadurch wird auf der P-Seite der Elektronenüberschuss teilweise abgebaut, auf der N-Seite der Elektronenmangel teilweise neutralisiert. Es bildet sich auf beiden Seiten der Grenzfläche eine isolierende Schicht.}
{An der Grenzschicht wandern Elektronen aus dem N-Teil in den P-Teil. Dadurch wird auf der N-Seite der Elektronenüberschuss teilweise abgebaut, auf der P-Seite der Elektronenmangel teilweise neutralisiert. Es bildet sich auf beiden Seiten der Grenzfläche eine isolierende Schicht.}
{An der Grenzschicht wandern Atome aus dem N-Teil in den P-Teil. Dadurch wird auf der N-Seite der Atommangel abgebaut, auf der P-Seite der Atommangel vergrößert. Es bildet sich auf beiden Seiten der Grenzfläche eine leitende Schicht.}
{\DARCimage{1.0\linewidth}{48include}}\end{PQuestion}

}
\only<2>{
\begin{PQuestion}{AB108}{Das folgende Bild zeigt den prinzipiellen Aufbau einer Halbleiterdiode. Wie entsteht die Sperrschicht?}{An der Grenzschicht wandern Atome aus dem P-Teil in den N-Teil. Dadurch wird auf der P-Seite der Atommangel abgebaut, auf der N-Seite der Atommangel vergrößert. Es bildet sich auf beiden Seiten der Grenzfläche eine leitende Schicht.}
{An der Grenzschicht wandern Elektronen aus dem P-Teil in den N-Teil. Dadurch wird auf der P-Seite der Elektronenüberschuss teilweise abgebaut, auf der N-Seite der Elektronenmangel teilweise neutralisiert. Es bildet sich auf beiden Seiten der Grenzfläche eine isolierende Schicht.}
{\textbf{\textcolor{DARCgreen}{An der Grenzschicht wandern Elektronen aus dem N-Teil in den P-Teil. Dadurch wird auf der N-Seite der Elektronenüberschuss teilweise abgebaut, auf der P-Seite der Elektronenmangel teilweise neutralisiert. Es bildet sich auf beiden Seiten der Grenzfläche eine isolierende Schicht.}}}
{An der Grenzschicht wandern Atome aus dem N-Teil in den P-Teil. Dadurch wird auf der N-Seite der Atommangel abgebaut, auf der P-Seite der Atommangel vergrößert. Es bildet sich auf beiden Seiten der Grenzfläche eine leitende Schicht.}
{\DARCimage{1.0\linewidth}{48include}}\end{PQuestion}

}
\end{frame}

\begin{frame}
\only<1>{
\begin{PQuestion}{AB109}{Wie verhält sich die Verarmungszone in der hier dargestellten Halbleiterdiode?}{Sie verändert sich nicht.}
{Sie verengt sich.}
{Sie erweitert sich.}
{Sie verschwindet. }
{\DARCimage{1.0\linewidth}{486include}}\end{PQuestion}

}
\only<2>{
\begin{PQuestion}{AB109}{Wie verhält sich die Verarmungszone in der hier dargestellten Halbleiterdiode?}{Sie verändert sich nicht.}
{Sie verengt sich.}
{\textbf{\textcolor{DARCgreen}{Sie erweitert sich.}}}
{Sie verschwindet. }
{\DARCimage{1.0\linewidth}{486include}}\end{PQuestion}

}
\end{frame}%ENDCONTENT
