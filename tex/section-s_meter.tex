
\section{S-Meter}
\label{section:s_meter}
\begin{frame}%STARTCONTENT
\begin{itemize}
  \item Bis S9: Eine S-Stufe entspricht 6dB
  \item 6dB: $2\cdot U$ oder $4\cdot P$
  \end{itemize}
\end{frame}

\begin{frame}
\only<1>{
\begin{QQuestion}{AA113}{Wie groß ist der Unterschied zwischen den S-Stufen S4 und S7 in dB?}{\qty{18}{\decibel}}
{\qty{9}{\decibel}}
{\qty{15}{\decibel}}
{\qty{3}{\decibel}}
\end{QQuestion}

}
\only<2>{
\begin{QQuestion}{AA113}{Wie groß ist der Unterschied zwischen den S-Stufen S4 und S7 in dB?}{\textbf{\textcolor{DARCgreen}{\qty{18}{\decibel}}}}
{\qty{9}{\decibel}}
{\qty{15}{\decibel}}
{\qty{3}{\decibel}}
\end{QQuestion}

}
\end{frame}

\begin{frame}
\frametitle{Lösungsweg}
\begin{itemize}
  \item von S3 bis S7 sind 3-Stufen
  \item $3\cdot 6dB = 18dB$
  \end{itemize}
\end{frame}

\begin{frame}
\only<1>{
\begin{QQuestion}{AF104}{Ein Funkamateur kommt laut S-Meter mit S7 an. Dann schaltet dieser seine Endstufe ein und bittet um einen erneuten Rapport. Das S-Meter zeigt nun S9+\qty{8}{\decibel} an. Um welchen Faktor hat der Funkamateur seine Leistung erhöht?}{100-fach}
{20-fach}
{10-fach}
{120-fach}
\end{QQuestion}

}
\only<2>{
\begin{QQuestion}{AF104}{Ein Funkamateur kommt laut S-Meter mit S7 an. Dann schaltet dieser seine Endstufe ein und bittet um einen erneuten Rapport. Das S-Meter zeigt nun S9+\qty{8}{\decibel} an. Um welchen Faktor hat der Funkamateur seine Leistung erhöht?}{\textbf{\textcolor{DARCgreen}{100-fach}}}
{20-fach}
{10-fach}
{120-fach}
\end{QQuestion}

}
\end{frame}

\begin{frame}
\frametitle{Lösungsweg}
\begin{itemize}
  \item von S7 auf S9+8dB sind 6dB+6dB+8dB = 20dB
  \item 20dB entsprechen der 100-fachen Leistung
  \end{itemize}
\end{frame}

\begin{frame}
\only<1>{
\begin{QQuestion}{AF101}{Um wie viele S-Stufen müsste die S-Meter-Anzeige Ihres Empfängers steigen, wenn Ihr Partner die Sendeleistung von \qty{25}{\W} auf \qty{100}{\W} erhöht?}{Um zwei S-Stufen}
{Um eine S-Stufe}
{Um vier S-Stufen}
{Um acht S-Stufen}
\end{QQuestion}

}
\only<2>{
\begin{QQuestion}{AF101}{Um wie viele S-Stufen müsste die S-Meter-Anzeige Ihres Empfängers steigen, wenn Ihr Partner die Sendeleistung von \qty{25}{\W} auf \qty{100}{\W} erhöht?}{Um zwei S-Stufen}
{\textbf{\textcolor{DARCgreen}{Um eine S-Stufe}}}
{Um vier S-Stufen}
{Um acht S-Stufen}
\end{QQuestion}

}
\end{frame}

\begin{frame}
\frametitle{Lösungsweg}
\begin{itemize}
  \item von 25W auf 100W sind $\frac{100W}{25W} = 4$-fache Leistung
  \item 4-fache Leistung entsprich einer S-Stufe
  \end{itemize}
\end{frame}

\begin{frame}
\only<1>{
\begin{QQuestion}{AF102}{Um wie viel S-Stufen müsste die S-Meter-Anzeige Ihres Empfängers steigen, wenn Ihr Funkpartner die Sendeleistung von \qty{100}{\W} auf \qty{400}{\W} erhöht?}{Um eine S-Stufe}
{Um zwei S-Stufen}
{Um vier S-Stufen}
{Um acht S-Stufen}
\end{QQuestion}

}
\only<2>{
\begin{QQuestion}{AF102}{Um wie viel S-Stufen müsste die S-Meter-Anzeige Ihres Empfängers steigen, wenn Ihr Funkpartner die Sendeleistung von \qty{100}{\W} auf \qty{400}{\W} erhöht?}{\textbf{\textcolor{DARCgreen}{Um eine S-Stufe}}}
{Um zwei S-Stufen}
{Um vier S-Stufen}
{Um acht S-Stufen}
\end{QQuestion}

}
\end{frame}

\begin{frame}
\frametitle{Lösungsweg}
\begin{itemize}
  \item von 100W auf 400W sind $\frac{400W}{100W} = 4$-fache Leistung
  \item 4-fache Leistung entspricht einer S-Stufe
  \end{itemize}
\end{frame}

\begin{frame}
\only<1>{
\begin{QQuestion}{AF103}{Ein Funkamateur erhöht seine Sendeleistung von 10 auf \qty{100}{\W}. Vor der Leistungserhöhung zeigte Ihr S-Meter genau S8. Auf welchen Wert müsste die Anzeige Ihres S-Meters nach der Leistungserhöhung ansteigen?}{S9}
{S9+\qty{7}{\decibel}}
{S9+\qty{4}{\decibel}}
{S9+\qty{9}{\decibel}}
\end{QQuestion}

}
\only<2>{
\begin{QQuestion}{AF103}{Ein Funkamateur erhöht seine Sendeleistung von 10 auf \qty{100}{\W}. Vor der Leistungserhöhung zeigte Ihr S-Meter genau S8. Auf welchen Wert müsste die Anzeige Ihres S-Meters nach der Leistungserhöhung ansteigen?}{S9}
{S9+\qty{7}{\decibel}}
{\textbf{\textcolor{DARCgreen}{S9+\qty{4}{\decibel}}}}
{S9+\qty{9}{\decibel}}
\end{QQuestion}

}
\end{frame}

\begin{frame}
\frametitle{Lösungsweg}
\begin{itemize}
  \item von 10W auf 100W sind $\frac{100W}{10W} = 10$-fache Leistung
  \item 10-fache Leistung entspricht 10dB
  \item von S8 auf S9 sind 6dB
  \item die restlichen 4dB kommen als +4dB oben drauf
  \end{itemize}
\end{frame}

\begin{frame}
\only<1>{
\begin{QQuestion}{AA114}{Wie stark ist die Empfängereingangsspannung abgesunken, wenn die S-Meter-Anzeige durch Änderung der Ausbreitungsbedingungen von S9+\qty{20}{\decibel} auf S8 zurückgeht? Die Empfängereingangsspannung sinkt um~...}{\qty{6}{\decibel}.}
{\qty{23}{\decibel}.}
{\qty{26}{\decibel}.}
{\qty{20}{\decibel}.}
\end{QQuestion}

}
\only<2>{
\begin{QQuestion}{AA114}{Wie stark ist die Empfängereingangsspannung abgesunken, wenn die S-Meter-Anzeige durch Änderung der Ausbreitungsbedingungen von S9+\qty{20}{\decibel} auf S8 zurückgeht? Die Empfängereingangsspannung sinkt um~...}{\qty{6}{\decibel}.}
{\qty{23}{\decibel}.}
{\textbf{\textcolor{DARCgreen}{\qty{26}{\decibel}.}}}
{\qty{20}{\decibel}.}
\end{QQuestion}

}
\end{frame}

\begin{frame}
\frametitle{Lösungsweg}
\begin{itemize}
  \item von S9+20dB auf S8 sind 26dB
  \end{itemize}
\end{frame}

\begin{frame}
\only<1>{
\begin{QQuestion}{AF105}{Durch \glqq Fading\grqq{} sinkt die S-Meter-Anzeige von S9 auf S8. Auf welchen Wert sinkt dabei die Empfänger-Eingangsspannung ab, wenn bei S9 am Empfängereingang \qty{50}{\micro\V} anliegen? Die Empfänger-Eingangsspannung sinkt auf}{\qty{25}{\micro\V}}
{\qty{37}{\micro\V}}
{\qty{40}{\micro\V}}
{\qty{30}{\micro\V}}
\end{QQuestion}

}
\only<2>{
\begin{QQuestion}{AF105}{Durch \glqq Fading\grqq{} sinkt die S-Meter-Anzeige von S9 auf S8. Auf welchen Wert sinkt dabei die Empfänger-Eingangsspannung ab, wenn bei S9 am Empfängereingang \qty{50}{\micro\V} anliegen? Die Empfänger-Eingangsspannung sinkt auf}{\textbf{\textcolor{DARCgreen}{\qty{25}{\micro\V}}}}
{\qty{37}{\micro\V}}
{\qty{40}{\micro\V}}
{\qty{30}{\micro\V}}
\end{QQuestion}

}
\end{frame}

\begin{frame}
\frametitle{Lösungsweg}
\end{frame}%ENDCONTENT
