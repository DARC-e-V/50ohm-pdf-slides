
\section{Transistor I}
\label{section:transistor_1}
\begin{frame}%STARTCONTENT

\frametitle{Von der Diode zum Transistor}
\begin{columns}
    \begin{column}{0.48\textwidth}
    Die Funktion kann man sich so vorstellen:

\begin{itemize}
  \item Mittels eines Steuerkanals wird der Durchfluss eines Wehrs geregelt
  \item Fließt kein Wasser im Steuerkanal ist das Wehr geschlossen
  \end{itemize}

    \end{column}
   \begin{column}{0.48\textwidth}
       
\begin{figure}
    \DARCimage{0.85\linewidth}{835include}
    \caption{\scriptsize Steuerkanal schließt Wehr komplett}
    \label{e_transistor_wehr_geschlossen}
\end{figure}


   \end{column}
\end{columns}

\end{frame}

\begin{frame}
\frametitle{Von der Diode zum Transistor}
\begin{columns}
    \begin{column}{0.48\textwidth}
    Die Funktion kann man sich so vorstellen:

\begin{itemize}
  \item Fließt etwas Wasser im Steuerkanal, öffnet das Wehr zur Hälfte
  \end{itemize}

    \end{column}
   \begin{column}{0.48\textwidth}
       
\begin{figure}
    \DARCimage{0.85\linewidth}{837include}
    \caption{\scriptsize Steuerkanal öffnet Wehr halb}
    \label{e_transistor_wehr_halb_offen}
\end{figure}


   \end{column}
\end{columns}

\end{frame}

\begin{frame}
\frametitle{Von der Diode zum Transistor}
\begin{columns}
    \begin{column}{0.48\textwidth}
    Die Funktion kann man sich so vorstellen:

\begin{itemize}
  \item Fließt mehr Wasser im Steuerkanal, ist das Wehr ganz geöffnet
  \end{itemize}

    \end{column}
   \begin{column}{0.48\textwidth}
       
\begin{figure}
    \DARCimage{0.85\linewidth}{836include}
    \caption{\scriptsize Steuerkanal öffnet Wehr komplett}
    \label{e_transistor_wehr_geoeffnet}
\end{figure}


   \end{column}
\end{columns}

\end{frame}

\begin{frame}
\only<1>{
\begin{QQuestion}{EC602}{Ein Transistor ist~...}{ein Nichtleiterbauelement.}
{ein Laserbauelement.}
{ein Halbleiterbauelement.}
{ein Kaltleiterbauelement.}
\end{QQuestion}

}
\only<2>{
\begin{QQuestion}{EC602}{Ein Transistor ist~...}{ein Nichtleiterbauelement.}
{ein Laserbauelement.}
{\textbf{\textcolor{DARCgreen}{ein Halbleiterbauelement.}}}
{ein Kaltleiterbauelement.}
\end{QQuestion}

}
\end{frame}

\begin{frame}
\only<1>{
\begin{QQuestion}{EC608}{Wie lauten die Bezeichnungen der Anschlüsse eines bipolaren Transistors?}{Emitter, Basis, Kollektor}
{Emitter, Drain, Source}
{Gate, Source, Kollektor}
{Drain, Gate, Source}
\end{QQuestion}

}
\only<2>{
\begin{QQuestion}{EC608}{Wie lauten die Bezeichnungen der Anschlüsse eines bipolaren Transistors?}{\textbf{\textcolor{DARCgreen}{Emitter, Basis, Kollektor}}}
{Emitter, Drain, Source}
{Gate, Source, Kollektor}
{Drain, Gate, Source}
\end{QQuestion}

}
\end{frame}

\begin{frame}
\frametitle{Bipolarer Transistor und Schaltbild}
\begin{columns}
    \begin{column}{0.48\textwidth}
    Merksatz für PNP $\rightarrow$ Pfeil Nach Platte


    \end{column}
   \begin{column}{0.48\textwidth}
       
\begin{figure}
    \DARCimage{0.85\linewidth}{374include}
    \caption{\scriptsize Schaltbild NPN-Transistor}
    \label{e_schaltbild_npn_transistor}
\end{figure}


\begin{figure}
    \DARCimage{0.85\linewidth}{375include}
    \caption{\scriptsize Schaltbild PNP-Transistor}
    \label{e_schaltbild_pnp_transistor}
\end{figure}


   \end{column}
\end{columns}

\end{frame}

\begin{frame}
\only<1>{
\begin{PQuestion}{EC607}{Bei diesem Bauelement handelt es sich um einen }{PNP-Transistor.}
{NPN-Transistor.}
{P-Kanal-FET.}
{N-Kanal-FET.}
{\DARCimage{0.25\linewidth}{375include}}\end{PQuestion}

}
\only<2>{
\begin{PQuestion}{EC607}{Bei diesem Bauelement handelt es sich um einen }{\textbf{\textcolor{DARCgreen}{PNP-Transistor.}}}
{NPN-Transistor.}
{P-Kanal-FET.}
{N-Kanal-FET.}
{\DARCimage{0.25\linewidth}{375include}}\end{PQuestion}

}
\end{frame}

\begin{frame}
\only<1>{
\begin{PQuestion}{EC606}{Bei diesem Bauelement handelt es sich um einen }{PNP-Transistor.}
{NPN-Transistor.}
{N-Kanal-FET.}
{P-Kanal-FET.}
{\DARCimage{0.25\linewidth}{374include}}\end{PQuestion}

}
\only<2>{
\begin{PQuestion}{EC606}{Bei diesem Bauelement handelt es sich um einen }{PNP-Transistor.}
{\textbf{\textcolor{DARCgreen}{NPN-Transistor.}}}
{N-Kanal-FET.}
{P-Kanal-FET.}
{\DARCimage{0.25\linewidth}{374include}}\end{PQuestion}

}
\end{frame}

\begin{frame}
\only<1>{
\begin{question2x2}{EC605}{Welches Schaltzeichen stellt einen bipolaren Transistor dar?}{\DARCimage{1.0\linewidth}{433include}}
{\DARCimage{1.0\linewidth}{381include}}
{\DARCimage{1.0\linewidth}{374include}}
{\DARCimage{1.0\linewidth}{432include}}
\end{question2x2}

}
\only<2>{
\begin{question2x2}{EC605}{Welches Schaltzeichen stellt einen bipolaren Transistor dar?}{\DARCimage{1.0\linewidth}{433include}}
{\DARCimage{1.0\linewidth}{381include}}
{\textbf{\textcolor{DARCgreen}{\DARCimage{1.0\linewidth}{374include}}}}
{\DARCimage{1.0\linewidth}{432include}}
\end{question2x2}

}
\end{frame}

\begin{frame}
\only<1>{
\begin{PQuestion}{EC609}{Wie bezeichnet man die Anschlüsse des abgebildeten Transistors?}{1~=~Kollektor, 2~=~Basis, 3~=~Emitter}
{1~=~Emitter, 2~=~Basis, 3~=~Kollektor}
{1~=~Kollektor, 2~=~Emitter, 3~=~Basis}
{1~=~Basis, 2~=~Emitter, 3~=~Kollektor}
{\DARCimage{0.25\linewidth}{502include}}\end{PQuestion}

}
\only<2>{
\begin{PQuestion}{EC609}{Wie bezeichnet man die Anschlüsse des abgebildeten Transistors?}{\textbf{\textcolor{DARCgreen}{1~=~Kollektor, 2~=~Basis, 3~=~Emitter}}}
{1~=~Emitter, 2~=~Basis, 3~=~Kollektor}
{1~=~Kollektor, 2~=~Emitter, 3~=~Basis}
{1~=~Basis, 2~=~Emitter, 3~=~Kollektor}
{\DARCimage{0.25\linewidth}{502include}}\end{PQuestion}

}
\end{frame}

\begin{frame}
\frametitle{Schalter oder Verstärker?}
\begin{itemize}
  \item Die Ansteuerung kann so eingestellt werden, dass der Transistor sperrt oder voll durchsteuert, dann spricht man von einem Schalttransistor.
  \item Die Ansteuerung kann so eingestellt werden, dass der Transistor stufenlos gesteuert wird, dann spricht man von einem Verstärker.
  \end{itemize}
\end{frame}

\begin{frame}
\only<1>{
\begin{QQuestion}{EC601}{Welches Bauteil kann als Schalter, Verstärker oder Widerstand eingesetzt werden?}{Diode}
{Transformator}
{Kondensator}
{Transistor}
\end{QQuestion}

}
\only<2>{
\begin{QQuestion}{EC601}{Welches Bauteil kann als Schalter, Verstärker oder Widerstand eingesetzt werden?}{Diode}
{Transformator}
{Kondensator}
{\textbf{\textcolor{DARCgreen}{Transistor}}}
\end{QQuestion}

}
\end{frame}

\begin{frame}
\only<1>{
\begin{QQuestion}{EC603}{Was versteht man unter Stromverstärkung beim Transistor?}{Mit einem geringen Kollektorstrom wird ein großer Emitterstrom gesteuert.}
{Mit einem geringen Emitterstrom wird ein großer Kollektorstrom gesteuert.}
{Mit einem geringen Emitterstrom wird ein großer Basisstrom gesteuert.}
{Mit einem geringen Basisstrom wird ein großer Kollektorstrom gesteuert.}
\end{QQuestion}

}
\only<2>{
\begin{QQuestion}{EC603}{Was versteht man unter Stromverstärkung beim Transistor?}{Mit einem geringen Kollektorstrom wird ein großer Emitterstrom gesteuert.}
{Mit einem geringen Emitterstrom wird ein großer Kollektorstrom gesteuert.}
{Mit einem geringen Emitterstrom wird ein großer Basisstrom gesteuert.}
{\textbf{\textcolor{DARCgreen}{Mit einem geringen Basisstrom wird ein großer Kollektorstrom gesteuert.}}}
\end{QQuestion}

}
\end{frame}

\begin{frame}
\frametitle{Ansteuerspannung und deren Polarität}
Je Art des bipolaren Transistor hat man verschiedene Polaritäten.

\begin{itemize}
  \item Bei einem NPN-Transistor benötigt man zum Durchschalten eine positive Steuerspannung.
  \item Bei einem PNP-Transistor benötigt man zum Durchschalten eine negative Steuerspannung.
  \end{itemize}
Die Steuerspannung liegt wie bei einer Siliziumdiode bei etwa \qty{0,6}{\volt}.

\end{frame}

\begin{frame}
\only<1>{
\begin{PQuestion}{EC610}{Wie groß muss die Spannung $U_{BE}$ in etwa sein, sodass sich der Transistor im leitenden Betriebszustand befindet?}{\qty{-0,6}{\V}}
{\qty{0,6}{\V}}
{\qty{0,6}{\V} oder \qty{-0,6}{\V}}
{\qty{0}{\V}}
{\DARCimage{0.5\linewidth}{559include}}\end{PQuestion}

}
\only<2>{
\begin{PQuestion}{EC610}{Wie groß muss die Spannung $U_{BE}$ in etwa sein, sodass sich der Transistor im leitenden Betriebszustand befindet?}{\qty{-0,6}{\V}}
{\textbf{\textcolor{DARCgreen}{\qty{0,6}{\V}}}}
{\qty{0,6}{\V} oder \qty{-0,6}{\V}}
{\qty{0}{\V}}
{\DARCimage{0.5\linewidth}{559include}}\end{PQuestion}

}
\end{frame}

\begin{frame}Da neben dem Kollektorstrom auch der Basisstrom durch den Transistor fließt, fließt durch den Emitteranschluss der größte Strom.

\end{frame}

\begin{frame}
\only<1>{
\begin{QQuestion}{EC611}{Durch welchen Transistoranschluss fliesst im leitenden Zustand der größte Strom?}{Gehäuse}
{Kollektor}
{Basis}
{Emitter}
\end{QQuestion}

}
\only<2>{
\begin{QQuestion}{EC611}{Durch welchen Transistoranschluss fliesst im leitenden Zustand der größte Strom?}{Gehäuse}
{Kollektor}
{Basis}
{\textbf{\textcolor{DARCgreen}{Emitter}}}
\end{QQuestion}

}
\end{frame}

\begin{frame}
\frametitle{ Wann schaltet der NPN Transistor durch?}
Ist die Basis-Emitter-Spannung ausreichend und liegt sie im positiven Potential vor?

Hier muss man auf die Vorzeichen achten und bei negativen Vorzeichen umdenken, Beispiele:

\begin{itemize}
  \item Basis +\qty{2}{\volt} und Emitter +\qty{1,4}{\volt}<br/> $\rightarrow$ Die Basis-Emitter-Spannung ist positiv und beträgt +\qty{0,6}{\volt}
  \item Basis -\qty{5,6}{\volt} und Emitter -\qty{6,2}{\volt}<br/> $\rightarrow$ Die Basis-Emitter-Spannung ist positiv und beträgt +\qty{0,6}{\volt}
  \end{itemize}
\end{frame}

\begin{frame}Entweder erkennet man das intuitiv oder man rechnet es (unter Beachtung der Vorzeichen) aus.

$U_{ BE } = U_{ B } -- U_{ E }$

\end{frame}

\begin{frame}
\only<1>{
\begin{question2x2}{EC612}{In einer Schaltung wurden die Spannungen der Transistoranschlüsse gegenüber Massepotential gemessen. Bei welchem der folgenden Transistoren fließt Kollektorstrom?}{\DARCimage{1.0\linewidth}{278include}}
{\DARCimage{1.0\linewidth}{279include}}
{\DARCimage{1.0\linewidth}{280include}}
{\DARCimage{1.0\linewidth}{281include}}
\end{question2x2}

}
\only<2>{
\begin{question2x2}{EC612}{In einer Schaltung wurden die Spannungen der Transistoranschlüsse gegenüber Massepotential gemessen. Bei welchem der folgenden Transistoren fließt Kollektorstrom?}{\textbf{\textcolor{DARCgreen}{\DARCimage{1.0\linewidth}{278include}}}}
{\DARCimage{1.0\linewidth}{279include}}
{\DARCimage{1.0\linewidth}{280include}}
{\DARCimage{1.0\linewidth}{281include}}
\end{question2x2}

}
\end{frame}

\begin{frame}
\only<1>{
\begin{question2x2}{EC613}{In einer Schaltung wurden die Spannungen der Transistoranschlüsse gegenüber Massepotential gemessen. Bei welchem der folgenden Transistoren fließt Kollektorstrom?}{\DARCimage{1.0\linewidth}{284include}}
{\DARCimage{1.0\linewidth}{283include}}
{\DARCimage{1.0\linewidth}{282include}}
{\DARCimage{1.0\linewidth}{285include}}
\end{question2x2}

}
\only<2>{
\begin{question2x2}{EC613}{In einer Schaltung wurden die Spannungen der Transistoranschlüsse gegenüber Massepotential gemessen. Bei welchem der folgenden Transistoren fließt Kollektorstrom?}{\DARCimage{1.0\linewidth}{284include}}
{\DARCimage{1.0\linewidth}{283include}}
{\textbf{\textcolor{DARCgreen}{\DARCimage{1.0\linewidth}{282include}}}}
{\DARCimage{1.0\linewidth}{285include}}
\end{question2x2}

}
\end{frame}

\begin{frame}
\frametitle{Wann schaltet der PNP Transistor durch?}
Ist die Basis-Emitter-Spannung ausreichend und liegt sie im negativen Potential vor?

Hier muss man auf die Vorzeichen achten und bei negativen Vorzeichen umdenken, Beispiele:

\begin{itemize}
  \item Basis +\qty{5,6}{\volt} und Emitter +\qty{6,2}{\volt}<br/> $\rightarrow$ Die Basis-Emitter-Spannung ist ist negativ und beträgt -\qty{0,6}{\volt}
  \item Basis -\qty{2}{\volt} und Emitter -\qty{1,4}{\volt}<br/> $\rightarrow$ Die Basis-Emitter-Spannung ist negativ und beträgt -\qty{0,6}{\volt}
  \end{itemize}
\end{frame}

\begin{frame}Entweder erkennet man das intuitiv oder man rechnet es (unter Beachtung der Vorzeichen) aus.

$U_{ BE } = U_{ B } -- U_{ E }$

\end{frame}

\begin{frame}
\only<1>{
\begin{question2x2}{EC614}{In einer Schaltung wurden die Spannungen der Transistoranschlüsse gegenüber Massepotential gemessen. Bei welchem der folgenden Transistoren fließt Kollektorstrom?}{\DARCimage{1.0\linewidth}{289include}}
{\DARCimage{1.0\linewidth}{287include}}
{\DARCimage{1.0\linewidth}{286include}}
{\DARCimage{1.0\linewidth}{288include}}
\end{question2x2}

}
\only<2>{
\begin{question2x2}{EC614}{In einer Schaltung wurden die Spannungen der Transistoranschlüsse gegenüber Massepotential gemessen. Bei welchem der folgenden Transistoren fließt Kollektorstrom?}{\DARCimage{1.0\linewidth}{289include}}
{\DARCimage{1.0\linewidth}{287include}}
{\textbf{\textcolor{DARCgreen}{\DARCimage{1.0\linewidth}{286include}}}}
{\DARCimage{1.0\linewidth}{288include}}
\end{question2x2}

}
\end{frame}

\begin{frame}
\only<1>{
\begin{question2x2}{EC615}{In einer Schaltung wurden die Spannungen der Transistoranschlüsse gegenüber Massepotential gemessen. Bei welchem der folgenden Transistoren fließt Kollektorstrom?}{\DARCimage{1.0\linewidth}{293include}}
{\DARCimage{1.0\linewidth}{291include}}
{\DARCimage{1.0\linewidth}{292include}}
{\DARCimage{1.0\linewidth}{290include}}
\end{question2x2}

}
\only<2>{
\begin{question2x2}{EC615}{In einer Schaltung wurden die Spannungen der Transistoranschlüsse gegenüber Massepotential gemessen. Bei welchem der folgenden Transistoren fließt Kollektorstrom?}{\DARCimage{1.0\linewidth}{293include}}
{\DARCimage{1.0\linewidth}{291include}}
{\DARCimage{1.0\linewidth}{292include}}
{\textbf{\textcolor{DARCgreen}{\DARCimage{1.0\linewidth}{290include}}}}
\end{question2x2}

}
\end{frame}

\begin{frame}
\frametitle{Typen von Transistoren}
Die bisher behandelten Transistoren nennt man \emph{Bipolare Transistoren}. Sie sind die Art der Transistoren, die in den 50er Jahren eine technische Revolution einläuteten und die Elektronenröhre ablösten. Im Gegensatz zu den stromgesteuerten Bipolartransistoren sind \emph{Feldeffekttransistoren (FET)} spannungsgesteuert, es fließt also kein Steuerstrom in ihn hinein. Mit diesen werden wir uns im Klasse~A Kurs intensiver auseinandersetzen.

\end{frame}

\begin{frame}
\only<1>{
\begin{QQuestion}{EC604}{Welche Transistortypen sind bipolare Transistoren?}{Isolierschicht-FETs}
{Dual-Gate-MOS-FETs}
{NPN- und PNP-Transistoren}
{Sperrschicht-FETs}
\end{QQuestion}

}
\only<2>{
\begin{QQuestion}{EC604}{Welche Transistortypen sind bipolare Transistoren?}{Isolierschicht-FETs}
{Dual-Gate-MOS-FETs}
{\textbf{\textcolor{DARCgreen}{NPN- und PNP-Transistoren}}}
{Sperrschicht-FETs}
\end{QQuestion}

}
\end{frame}%ENDCONTENT
