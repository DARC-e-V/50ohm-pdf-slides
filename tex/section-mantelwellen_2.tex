
\section{Mantelwellen II}
\label{section:mantelwellen_2}
\begin{frame}%STARTCONTENT

\only<1>{
\begin{QQuestion}{AG425}{Wann liegen Mantelwellen auf einem Koaxialkabel vor? Wenn~...}{vor- und rücklaufende Leistung nicht identisch sind.}
{der Schirm geerdet ist.}
{Stehwellen vorhanden sind.}
{Gleichtaktanteile vorhanden sind.}
\end{QQuestion}

}
\only<2>{
\begin{QQuestion}{AG425}{Wann liegen Mantelwellen auf einem Koaxialkabel vor? Wenn~...}{vor- und rücklaufende Leistung nicht identisch sind.}
{der Schirm geerdet ist.}
{Stehwellen vorhanden sind.}
{\textbf{\textcolor{DARCgreen}{Gleichtaktanteile vorhanden sind.}}}
\end{QQuestion}

}
\end{frame}

\begin{frame}
\only<1>{
\begin{QQuestion}{AG426}{Wie wirkt eine stromkompensierte Drossel (z.~B. Koaxialkabel um einen Ferritkern gewickelt) Mantelwellen entgegen? Sie wirkt~...}{hochohmig für Oberschwingungen und niederohmig für Grundschwingungen.}
{hochohmig für Gleichtaktanteile und niederohmig für Gegentaktanteile.}
{hochohmig für alle Ströme im Außenleiter und niederohmig für alle Ströme im Innenleiter.}
{hochohmig für Wechselströme des Innenleiters und niederohmig für Gleichströme des Außenleiters.}
\end{QQuestion}

}
\only<2>{
\begin{QQuestion}{AG426}{Wie wirkt eine stromkompensierte Drossel (z.~B. Koaxialkabel um einen Ferritkern gewickelt) Mantelwellen entgegen? Sie wirkt~...}{hochohmig für Oberschwingungen und niederohmig für Grundschwingungen.}
{\textbf{\textcolor{DARCgreen}{hochohmig für Gleichtaktanteile und niederohmig für Gegentaktanteile.}}}
{hochohmig für alle Ströme im Außenleiter und niederohmig für alle Ströme im Innenleiter.}
{hochohmig für Wechselströme des Innenleiters und niederohmig für Gleichströme des Außenleiters.}
\end{QQuestion}

}
\end{frame}

\begin{frame}
\only<1>{
\begin{QQuestion}{AJ115}{Zur Verhinderung von Rundfunk-Empfangsstörungen (z.~B. UKW, DAB, DVB-T), die durch Mantelwellen hervorgerufen werden, ist anstelle einer Mantelwellendrossel alternativ~...}{der Einbau einer seriellen Drosselspule in den Innenleiter der Empfangsantennenleitung möglich.}
{der Einbau eines Tiefpassfilters nach dem Senderausgang möglich.}
{der Einbau eines Bandpassfilters nach dem Senderausgang möglich.}
{der Einbau eines HF-Trenntrafos in die Empfangsantennenleitung möglich.}
\end{QQuestion}

}
\only<2>{
\begin{QQuestion}{AJ115}{Zur Verhinderung von Rundfunk-Empfangsstörungen (z.~B. UKW, DAB, DVB-T), die durch Mantelwellen hervorgerufen werden, ist anstelle einer Mantelwellendrossel alternativ~...}{der Einbau einer seriellen Drosselspule in den Innenleiter der Empfangsantennenleitung möglich.}
{der Einbau eines Tiefpassfilters nach dem Senderausgang möglich.}
{der Einbau eines Bandpassfilters nach dem Senderausgang möglich.}
{\textbf{\textcolor{DARCgreen}{der Einbau eines HF-Trenntrafos in die Empfangsantennenleitung möglich.}}}
\end{QQuestion}

}
\end{frame}

\begin{frame}
\only<1>{
\begin{QQuestion}{AG427}{Wodurch können Mantelwellen auf Koaxialkabeln verursacht werden?}{Durch symmetrische Antennen, schlechte Erdung asymmetrischer Antennen oder Einkopplung in den Koax-Schirm}
{Durch Asymmetrie der Spannungsversorgung oder durch Dielektrika der Speiseleitung, die einen hohen Widerstand aufweisen}
{Durch Stehwellen in Koaxialkabeln mit geflochtenem Mantel, deren Länge ein Vielfaches von $\lambda$/2 betragen}
{Durch Oberwellen auf Speiseleitungen, deren Länge ein Vielfaches von $\lambda$/4 oder 5/8 $\lambda$ betragen}
\end{QQuestion}

}
\only<2>{
\begin{QQuestion}{AG427}{Wodurch können Mantelwellen auf Koaxialkabeln verursacht werden?}{\textbf{\textcolor{DARCgreen}{Durch symmetrische Antennen, schlechte Erdung asymmetrischer Antennen oder Einkopplung in den Koax-Schirm}}}
{Durch Asymmetrie der Spannungsversorgung oder durch Dielektrika der Speiseleitung, die einen hohen Widerstand aufweisen}
{Durch Stehwellen in Koaxialkabeln mit geflochtenem Mantel, deren Länge ein Vielfaches von $\lambda$/2 betragen}
{Durch Oberwellen auf Speiseleitungen, deren Länge ein Vielfaches von $\lambda$/4 oder 5/8 $\lambda$ betragen}
\end{QQuestion}

}
\end{frame}

\begin{frame}
\only<1>{
\begin{PQuestion}{AG421}{Für welche Antennenimpedanz ist der folgende Balun-Transformator aus zweimal acht Windungen ausgelegt?}{\qty{400}{\ohm}}
{\qty{50}{\ohm}}
{\qty{100}{\ohm}}
{\qty{200}{\ohm}}
{\DARCimage{1.0\linewidth}{447include}}\end{PQuestion}

}
\only<2>{
\begin{PQuestion}{AG421}{Für welche Antennenimpedanz ist der folgende Balun-Transformator aus zweimal acht Windungen ausgelegt?}{\qty{400}{\ohm}}
{\qty{50}{\ohm}}
{\qty{100}{\ohm}}
{\textbf{\textcolor{DARCgreen}{\qty{200}{\ohm}}}}
{\DARCimage{1.0\linewidth}{447include}}\end{PQuestion}

}
\end{frame}

\begin{frame}
\only<1>{
\begin{PQuestion}{AG422}{Dargestellt ist ein HF-Übertrager (Balun). An den Anschlüssen a und b wird ein Faltdipol mit \qty{200}{\ohm} Impedanz angeschlossen. Welche Impedanz misst man zwischen den Anschlüssen a und m?}{\qty{100}{\ohm}}
{\qty{0}{\ohm}}
{\qty{50}{\ohm}}
{\qty{200}{\ohm}}
{\DARCimage{1.0\linewidth}{448include}}\end{PQuestion}

}
\only<2>{
\begin{PQuestion}{AG422}{Dargestellt ist ein HF-Übertrager (Balun). An den Anschlüssen a und b wird ein Faltdipol mit \qty{200}{\ohm} Impedanz angeschlossen. Welche Impedanz misst man zwischen den Anschlüssen a und m?}{\qty{100}{\ohm}}
{\qty{0}{\ohm}}
{\textbf{\textcolor{DARCgreen}{\qty{50}{\ohm}}}}
{\qty{200}{\ohm}}
{\DARCimage{1.0\linewidth}{448include}}\end{PQuestion}

}
\end{frame}

\begin{frame}
\only<1>{
\begin{PQuestion}{AG428}{Die Darstellung zeigt die bei Ankopplung eines Koaxialkabels an eine Antenne auftretenden Ströme. Wie kann man den als $I_3$ bezeichneten, unerwünschten Mantelstrom reduzieren?}{Einfügen eines Oberwellenfilters oder bei unsymmetrischen Störeinflüssen auch eines Spannungs-Baluns}
{Einfügen einer Gleichtaktdrossel oder bei symmetrischen Antennen auch eines Spannungs-Baluns}
{Auftrennen des Koax-Schirms vom Arm 2 der dargestellten Antenne (direkt an oder kurz vor der Antenne)}
{Herstellung einer direkten Verbindung zwischen dem Arm 1 der Antenne mit einer guten HF-Erde}
{\DARCimage{1.0\linewidth}{633include}}\end{PQuestion}

}
\only<2>{
\begin{PQuestion}{AG428}{Die Darstellung zeigt die bei Ankopplung eines Koaxialkabels an eine Antenne auftretenden Ströme. Wie kann man den als $I_3$ bezeichneten, unerwünschten Mantelstrom reduzieren?}{Einfügen eines Oberwellenfilters oder bei unsymmetrischen Störeinflüssen auch eines Spannungs-Baluns}
{\textbf{\textcolor{DARCgreen}{Einfügen einer Gleichtaktdrossel oder bei symmetrischen Antennen auch eines Spannungs-Baluns}}}
{Auftrennen des Koax-Schirms vom Arm 2 der dargestellten Antenne (direkt an oder kurz vor der Antenne)}
{Herstellung einer direkten Verbindung zwischen dem Arm 1 der Antenne mit einer guten HF-Erde}
{\DARCimage{1.0\linewidth}{633include}}\end{PQuestion}

}
\end{frame}

\begin{frame}
\only<1>{
\begin{QQuestion}{AG429}{Wodurch können Mantelwellen im Falle einer koax-gespeisten symmetrischen Antenne auftreten, obwohl ein Spannungs-Balun verwendet wird?}{Fehlanpassung durch Impedanztransformation des Baluns (z.\,B. 4:1-Spartransformator) sowie Stehwellen in der Zuleitung}
{Ungleichmäßige Belastung der Antenne durch Störeinflüsse der Umgebung (z.\,B. Bäume oder Gebäude) sowie Einkopplung in den Koax-Schirm}
{Dämpfung der Abstrahlung durch als Oberwellenfilter wirkenden Balun (z.\,B. 1:1-Transformator) sowie Einkopplung in den Koax-Schirm}
{Erhitzung des Ringkerns durch unzureichende Abschirmung (z.\,B. Kunststoffgehäuse) des Baluns sowie Stehwellen in der Zuleitung}
\end{QQuestion}

}
\only<2>{
\begin{QQuestion}{AG429}{Wodurch können Mantelwellen im Falle einer koax-gespeisten symmetrischen Antenne auftreten, obwohl ein Spannungs-Balun verwendet wird?}{Fehlanpassung durch Impedanztransformation des Baluns (z.\,B. 4:1-Spartransformator) sowie Stehwellen in der Zuleitung}
{\textbf{\textcolor{DARCgreen}{Ungleichmäßige Belastung der Antenne durch Störeinflüsse der Umgebung (z.\,B. Bäume oder Gebäude) sowie Einkopplung in den Koax-Schirm}}}
{Dämpfung der Abstrahlung durch als Oberwellenfilter wirkenden Balun (z.\,B. 1:1-Transformator) sowie Einkopplung in den Koax-Schirm}
{Erhitzung des Ringkerns durch unzureichende Abschirmung (z.\,B. Kunststoffgehäuse) des Baluns sowie Stehwellen in der Zuleitung}
\end{QQuestion}

}
\end{frame}%ENDCONTENT
