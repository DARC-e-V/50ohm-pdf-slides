
\section{Relaisfunkstellen}
\label{section:relaisfunkstellen}
\begin{frame}%STARTCONTENT

\begin{columns}
    \begin{column}{0.48\textwidth}
    
\begin{figure}
    \DARCimage{0.85\linewidth}{648include}
    \caption{\scriptsize Schematische Darstellung einer Relaisfunkstelle mit Nutzern}
    \label{n_relaisfunkstellen_aufbau}
\end{figure}


    \end{column}
   \begin{column}{0.48\textwidth}
       \begin{itemize}
  \item Ermöglicht eine größere Reichweite als bei direkter Verbindung
  \item Meist an exponierten Standorten, z.B. Berggipfeln, Hochhäusern, (Kirch-)Türmen
  \item Oder in Satelliten
  \end{itemize}

   \end{column}
\end{columns}

\end{frame}

\begin{frame}
\frametitle{Definition Relaisfunkstelle}
eine fernbediente Amateurfunkstelle (auch in Satelliten), die empfangene Amateurfunkaussendungen, Teile davon oder sonstige eingespeiste oder eingespeicherte Signale fern ausgelöst aussendet und dabei zur Erhöhung der Erreichbarkeit von Amateurfunkstellen dient

\end{frame}

\begin{frame}
\begin{columns}
    \begin{column}{0.48\textwidth}
    \begin{itemize}
  \item Auch kurz genannt: Relais oder Repeater
  \item Senden regelmäßig ihr Rufzeichen aus
  \item Rufzeichen beginnt in der Regel mit DB0, DM0 oder DO0
  \end{itemize}

    \end{column}
    \pause
    
   \begin{column}{0.48\textwidth}
       \begin{itemize}
  \item Relaisfunkstellen werden nicht mit persönlichen Rufzeichen betrieben.
  \item Relaisfunkstellen sind üblicherweise nicht ständig besetzt.
  \item Relaisfunkstellen müssen nicht zwingend an geografisch exponierten Standorten betrieben werden.
  \end{itemize}

   \end{column}
\end{columns}



\end{frame}

\begin{frame}
\only<1>{
\begin{QQuestion}{NF118}{Was wird unter einem Digipeater verstanden?}{Ein Lineartransponder, der empfangene Datenpakete auf ein anderes Frequenzband umsetzt. Hierbei bleiben die verwendete Modulationsart sowie der Inhalt des Pakets erhalten.}
{Eine Funkstation, die empfangene Datenpakete oder Teile davon automatisch erneut aussendet, ggf. auch zeitversetzt oder wiederholt. Hierbei können einzelne Datenfelder geändert werden.}
{Ein integrierter Schaltkreis, der digitale Signale für die Modulation im Funkgerät vorbereitet. Hierbei wird das Rufzeichen der Station regelmäßig in den Datenstrom eingefügt.}
{Eine Relaisstation, die Sprachübertragungen auf einer anderen Frequenz erneut aussendet. Hierbei wird die Lautstärke adaptiv mittels digitaler Signalverarbeitung angepasst.}
\end{QQuestion}

}
\only<2>{
\begin{QQuestion}{NF118}{Was wird unter einem Digipeater verstanden?}{Ein Lineartransponder, der empfangene Datenpakete auf ein anderes Frequenzband umsetzt. Hierbei bleiben die verwendete Modulationsart sowie der Inhalt des Pakets erhalten.}
{\textbf{\textcolor{DARCgreen}{Eine Funkstation, die empfangene Datenpakete oder Teile davon automatisch erneut aussendet, ggf. auch zeitversetzt oder wiederholt. Hierbei können einzelne Datenfelder geändert werden.}}}
{Ein integrierter Schaltkreis, der digitale Signale für die Modulation im Funkgerät vorbereitet. Hierbei wird das Rufzeichen der Station regelmäßig in den Datenstrom eingefügt.}
{Eine Relaisstation, die Sprachübertragungen auf einer anderen Frequenz erneut aussendet. Hierbei wird die Lautstärke adaptiv mittels digitaler Signalverarbeitung angepasst.}
\end{QQuestion}

}
\end{frame}

\begin{frame}
\frametitle{Funktionsweise}
\begin{columns}
    \begin{column}{0.48\textwidth}
    \begin{itemize}
  \item Empfängt auf der Eingangsfrequenz das Signal einer Amateurfunkstation
  \item Stahlt es zeitgleich auf der Ausgabefrequenz aus
  \item Damit der Sender nicht stört, sind die Frequenzen meistens unterschiedlich
  \end{itemize}

    \end{column}
   \begin{column}{0.48\textwidth}
       
    \pause
    Den Abstand nennt man \emph{Frequenzablage} oder kurz \emph{Ablage}

\begin{table}
\begin{DARCtabular}{rr}
     Band  & Ablage   \\
     \qty{10}{\metre}  & \qty{100}{\kilo\hertz}   \\
     \qty{2}{\metre}  & \qty{600}{\kilo\hertz}   \\
     \qty{70}{\centi\metre}  & 7.\qty{6}{\mega\hertz}   \\
     \qty{23}{\centi\metre}  & \qty{28}{\mega\hertz}   \\
\end{DARCtabular}
\caption{Frequenzablage}
\label{n_relaisfunkstellen_ablage}
\end{table}



   \end{column}
\end{columns}

\end{frame}

\begin{frame}Beispiel eines 70cm-Relais:

\begin{itemize}
  \item Ausgabefrequenz: 438.\qty{875}{\mega\hertz}
  \item Ablage: -7.\qty{600}{\mega\hertz}
  \item Eingabefrequenz: 431.\qty{275}{\mega\hertz}
  \end{itemize}
\end{frame}

\begin{frame}
\only<1>{
\begin{QQuestion}{BE401}{Was ist damit gemeint, wenn man sagt, die Relaisfunkstelle hat eine Eingabe- und eine Ausgabefrequenz?}{Die Relaisfunkstelle empfängt auf der Eingabefrequenz und sendet auf der Ausgabefrequenz.}
{Die Relaisfunkstelle stellt bei starker Belegung der Eingabefrequenz eine zusätzliche Ausgabefrequenz zur Verfügung.}
{Die Relaisfunkstelle benutzt eine Eingabefrequenz zur Umsetzung des empfangenen Signals und die Ausgabefrequenz zur Fernsteuerung.}
{Die Relaisfunkstelle muss auf der Ausgabefrequenz mit einem Tonruf geöffnet werden, bevor sie auf der Eingabefrequenz in Betrieb gehen kann.}
\end{QQuestion}

}
\only<2>{
\begin{QQuestion}{BE401}{Was ist damit gemeint, wenn man sagt, die Relaisfunkstelle hat eine Eingabe- und eine Ausgabefrequenz?}{\textbf{\textcolor{DARCgreen}{Die Relaisfunkstelle empfängt auf der Eingabefrequenz und sendet auf der Ausgabefrequenz.}}}
{Die Relaisfunkstelle stellt bei starker Belegung der Eingabefrequenz eine zusätzliche Ausgabefrequenz zur Verfügung.}
{Die Relaisfunkstelle benutzt eine Eingabefrequenz zur Umsetzung des empfangenen Signals und die Ausgabefrequenz zur Fernsteuerung.}
{Die Relaisfunkstelle muss auf der Ausgabefrequenz mit einem Tonruf geöffnet werden, bevor sie auf der Eingabefrequenz in Betrieb gehen kann.}
\end{QQuestion}

}
\end{frame}

\begin{frame}
\only<1>{
\begin{QQuestion}{BE402}{Bei deutschen \qty{2}{\m}-Relaisfunkstellen liegt die Eingabefrequenz üblicherweise~...}{\qty{7,6}{\MHz} höher die Ausgabefrequenz.}
{\qty{600}{\kHz} höher als die Ausgabefrequenz.}
{\qty{7,6}{\MHz} niedriger als die Ausgabefrequenz.}
{\qty{600}{\kHz} niedriger als die Ausgabefrequenz.}
\end{QQuestion}

}
\only<2>{
\begin{QQuestion}{BE402}{Bei deutschen \qty{2}{\m}-Relaisfunkstellen liegt die Eingabefrequenz üblicherweise~...}{\qty{7,6}{\MHz} höher die Ausgabefrequenz.}
{\qty{600}{\kHz} höher als die Ausgabefrequenz.}
{\qty{7,6}{\MHz} niedriger als die Ausgabefrequenz.}
{\textbf{\textcolor{DARCgreen}{\qty{600}{\kHz} niedriger als die Ausgabefrequenz.}}}
\end{QQuestion}

}
\end{frame}

\begin{frame}
\only<1>{
\begin{QQuestion}{BE403}{Bei deutschen \qty{70}{\cm}-Relaisfunkstellen liegt die Eingabefrequenz üblicherweise~...}{\qty{600}{\kHz} niedriger als die Ausgabefrequenz.}
{\qty{600}{\kHz} höher als die Ausgabefrequenz.}
{\qty{7,6}{\MHz} niedriger als die Ausgabefrequenz.}
{\qty{7,6}{\MHz} höher als die Ausgabefrequenz.}
\end{QQuestion}

}
\only<2>{
\begin{QQuestion}{BE403}{Bei deutschen \qty{70}{\cm}-Relaisfunkstellen liegt die Eingabefrequenz üblicherweise~...}{\qty{600}{\kHz} niedriger als die Ausgabefrequenz.}
{\qty{600}{\kHz} höher als die Ausgabefrequenz.}
{\textbf{\textcolor{DARCgreen}{\qty{7,6}{\MHz} niedriger als die Ausgabefrequenz.}}}
{\qty{7,6}{\MHz} höher als die Ausgabefrequenz.}
\end{QQuestion}

}
\end{frame}

\begin{frame}
\frametitle{Crossband-Betrieb}
\begin{itemize}
  \item Sendet und empfängt gleichzeitig auf zwei verschiedenen Bändern, z.B. 2m und 70cm
  \item Umsetzung der Sendeart auch möglich, z.B. SSB auf FM
  \end{itemize}
\end{frame}

\begin{frame}
\frametitle{Digipeater}
\begin{itemize}
  \item Vermittelt Daten statt Sprache
  \item Empfängt und sendet Datenpakete
  \item Aussendung kann nur in Teilen oder zeitversetzt geschehen
  \item Datenpakete können wiederholt werden
  \item Einzelne Datenfelder können geändert werden
  \end{itemize}

\end{frame}

\begin{frame}
\frametitle{Besondere Einstellungen}
\begin{itemize}
  \item Ggf. sind weitere Einstellungen für die Verbindung zum Relais notwendig
  \item Diese Informationen sind in Repeaterverzeichnissen, auf Webseiten oder beim Relaisverantwortlichen erhältlich
  \item Neben FM-Repeatern gibt es welche für digitale Sprache wie DMR oder D-Star
  \end{itemize}

\end{frame}

\begin{frame}
\only<1>{
\begin{QQuestion}{NE309}{Welche Modulationsart wird üblicherweise bei analogen VHF/UHF-Relaisfunkstellen für Sprache verwendet?}{FM}
{AM}
{SSB}
{DMR}
\end{QQuestion}

}
\only<2>{
\begin{QQuestion}{NE309}{Welche Modulationsart wird üblicherweise bei analogen VHF/UHF-Relaisfunkstellen für Sprache verwendet?}{\textbf{\textcolor{DARCgreen}{FM}}}
{AM}
{SSB}
{DMR}
\end{QQuestion}

}
\end{frame}

\begin{frame}
\only<1>{
\begin{QQuestion}{NE308}{Welche Übertragungsverfahren werden bei VHF/UHF-Relaisfunkstellen für Sprache benutzt?}{SSB-Sprechfunk, DMR, RTTY}
{CW-Morsetelegrafie, FT8, D-STAR}
{FM-Sprechfunk, DMR, D-STAR}
{AM-Sprechfunk, C4FM, FT8 }
\end{QQuestion}

}
\only<2>{
\begin{QQuestion}{NE308}{Welche Übertragungsverfahren werden bei VHF/UHF-Relaisfunkstellen für Sprache benutzt?}{SSB-Sprechfunk, DMR, RTTY}
{CW-Morsetelegrafie, FT8, D-STAR}
{\textbf{\textcolor{DARCgreen}{FM-Sprechfunk, DMR, D-STAR}}}
{AM-Sprechfunk, C4FM, FT8 }
\end{QQuestion}

}
\end{frame}

\begin{frame}
\frametitle{Kanalbandbreite}
\begin{itemize}
  \item Der benötigte Platz im Frequenzspektrum
  \item Wide-FM: \qty{25}{\kilo\hertz}
  \item Narrow-FM: \qty{12,5}{\kilo\hertz}
  \item Repeater mögen Narrow-FM, da sonst Signale verzerrt sind und benachbarte Frequenzen gestört werden
  \end{itemize}
\end{frame}

\begin{frame}
\only<1>{
\begin{QQuestion}{BE407}{Warum sollten Sie bei Nutzung eines FM-Repeaters darauf achten, Schmalband-FM (Narrow-FM) an Ihrem Handfunkgerät einzustellen? Da ansonsten~...}{zu starke Oberwellen entstehen können und Funkdienste auf anderen Bändern durch Spiegelfrequenzen gestört werden könnten.}
{eine übermäßige Abnutzung des Vorverstärkers des Repeaters durch Überlastung eintreten könnte und der Repeater dann ausfallen würde.}
{Repeater-Eingaben auf benachbarten Frequenzen gestört werden können und der verwendete Repeater das Signal verzerrt ausgeben könnte.}
{die Batterien der Notstromversorgung des Repeaters übermäßig belastet werden könnten und dann im Notfall nicht mehr nutzbar wären.}
\end{QQuestion}

}
\only<2>{
\begin{QQuestion}{BE407}{Warum sollten Sie bei Nutzung eines FM-Repeaters darauf achten, Schmalband-FM (Narrow-FM) an Ihrem Handfunkgerät einzustellen? Da ansonsten~...}{zu starke Oberwellen entstehen können und Funkdienste auf anderen Bändern durch Spiegelfrequenzen gestört werden könnten.}
{eine übermäßige Abnutzung des Vorverstärkers des Repeaters durch Überlastung eintreten könnte und der Repeater dann ausfallen würde.}
{\textbf{\textcolor{DARCgreen}{Repeater-Eingaben auf benachbarten Frequenzen gestört werden können und der verwendete Repeater das Signal verzerrt ausgeben könnte.}}}
{die Batterien der Notstromversorgung des Repeaters übermäßig belastet werden könnten und dann im Notfall nicht mehr nutzbar wären.}
\end{QQuestion}

}
\end{frame}

\begin{frame}
\frametitle{Störungsfreier Betrieb}
\begin{itemize}
  \item Grundsätzlich können alle Funkamateure mit ihrem zugeteilten Rufzeichen fernbediente Amateurfunkstellen nutzen
  \item Betreiber kann zur Sicherstellung des störungsfreien Betriebs Funkamateure ausschließen
  \item Die BNetzA ist hiervon zu unterrichten
  \end{itemize}
\end{frame}

\begin{frame}
\only<1>{
\begin{QQuestion}{VD504}{Wann kann ein verantwortlicher Funkamateur einen bestimmten Funkamateur vom Betrieb über die von ihm betreute Relaisfunkstelle ausschließen?}{Wenn ein Funkamateur die Relaisfunkstelle zu häufig benutzt}
{Wenn dies dazu dient, den störungsfreien Betrieb der Relaisfunkstelle sicherzustellen}
{Wenn die Relaisnutzungsgebühr nicht entrichtet wurde}
{Wenn ein Funkamateur das Mindestalter noch nicht erreicht hat}
\end{QQuestion}

}
\only<2>{
\begin{QQuestion}{VD504}{Wann kann ein verantwortlicher Funkamateur einen bestimmten Funkamateur vom Betrieb über die von ihm betreute Relaisfunkstelle ausschließen?}{Wenn ein Funkamateur die Relaisfunkstelle zu häufig benutzt}
{\textbf{\textcolor{DARCgreen}{Wenn dies dazu dient, den störungsfreien Betrieb der Relaisfunkstelle sicherzustellen}}}
{Wenn die Relaisnutzungsgebühr nicht entrichtet wurde}
{Wenn ein Funkamateur das Mindestalter noch nicht erreicht hat}
\end{QQuestion}

}
\end{frame}

\begin{frame}
\frametitle{Funkbetrieb auf Repeatern}
\begin{itemize}
  \item Kurze Durchgänge
  \item Mobile und portable Stationen sind oft nur kurzzeitig in Empfangsreichweite
  \item Pause zwischen den Durchgängen zum Reinmelden anderer Stationen
  \end{itemize}
\end{frame}

\begin{frame}
\only<1>{
\begin{QQuestion}{BE406}{Warum sollten bei Relaisfunkbetrieb die Durchgänge möglichst kurz gehalten werden?}{Die Sprachspeicher einer Relaisfunkstelle haben eine zeitlich begrenzte Kapazität.}
{Um zeitweilig Simplex-Verkehr zu ermöglichen.}
{Nach der Amateurfunkverordnung darf ein Durchgang höchstens 60 Sekunden betragen.}
{Damit es besonders Mobil- und Portabelstationen leichter möglich ist, die Relaisfunkstelle zu nutzen.}
\end{QQuestion}

}
\only<2>{
\begin{QQuestion}{BE406}{Warum sollten bei Relaisfunkbetrieb die Durchgänge möglichst kurz gehalten werden?}{Die Sprachspeicher einer Relaisfunkstelle haben eine zeitlich begrenzte Kapazität.}
{Um zeitweilig Simplex-Verkehr zu ermöglichen.}
{Nach der Amateurfunkverordnung darf ein Durchgang höchstens 60 Sekunden betragen.}
{\textbf{\textcolor{DARCgreen}{Damit es besonders Mobil- und Portabelstationen leichter möglich ist, die Relaisfunkstelle zu nutzen.}}}
\end{QQuestion}

}
\end{frame}

\begin{frame}
\only<1>{
\begin{QQuestion}{BE404}{Wodurch sollte es Stationen erleichtert werden, sich in eine laufende Funkrunde oder ein Gespräch auf einem Repeater hereinzumelden?}{Durch Freihalten der Eingabefrequenz}
{Durch Verwendung eines Auftasttons}
{Durch eine kurze Pause vor jedem Durchgang}
{Durch Freihalten der Ausgabefrequenz}
\end{QQuestion}

}
\only<2>{
\begin{QQuestion}{BE404}{Wodurch sollte es Stationen erleichtert werden, sich in eine laufende Funkrunde oder ein Gespräch auf einem Repeater hereinzumelden?}{Durch Freihalten der Eingabefrequenz}
{Durch Verwendung eines Auftasttons}
{\textbf{\textcolor{DARCgreen}{Durch eine kurze Pause vor jedem Durchgang}}}
{Durch Freihalten der Ausgabefrequenz}
\end{QQuestion}

}
\end{frame}

\begin{frame}
\frametitle{Doppeln}
\begin{itemize}
  \item Bei gleichzeitiger Spracheingabe wird die Aussendung bis zur Unlesbarkeit gestört
  \item \enquote{Doppeln} durch ordentliche Übergabe vermeiden
  \item Aussendung erst dann beginnen, wenn die vorige Station beendet hat
  \end{itemize}
\end{frame}

\begin{frame}
\only<1>{
\begin{QQuestion}{NE310}{Wie sind zwei FM-Stationen auf der Relaisausgabe zu hören, wenn sie gleich stark und gleichzeitig auf der Relaiseingabe empfangen werden?}{Sie stören sich gegenseitig bis zur Unlesbarkeit.}
{Sie stören sich nicht, jede Station ist mit halber Lautstärke zu hören.}
{Sie sind auf der Ausgabe abwechselnd  zu empfangen.}
{Es ist nur die Station zu hören, die zuerst mit der Sendung begonnen hat.}
\end{QQuestion}

}
\only<2>{
\begin{QQuestion}{NE310}{Wie sind zwei FM-Stationen auf der Relaisausgabe zu hören, wenn sie gleich stark und gleichzeitig auf der Relaiseingabe empfangen werden?}{\textbf{\textcolor{DARCgreen}{Sie stören sich gegenseitig bis zur Unlesbarkeit.}}}
{Sie stören sich nicht, jede Station ist mit halber Lautstärke zu hören.}
{Sie sind auf der Ausgabe abwechselnd  zu empfangen.}
{Es ist nur die Station zu hören, die zuerst mit der Sendung begonnen hat.}
\end{QQuestion}

}
\end{frame}

\begin{frame}
\only<1>{
\begin{QQuestion}{BE405}{Wodurch sollte gleichzeitiges Sprechen (Doppeln) bei Nutzung eines Repeaters und in Funkrunden vermieden werden?}{Durch Nutzung eines Sendezeitbegrenzers}
{Durch ordentliche Übergabe nach jedem Durchgang}
{Durch Senden mit möglichst großer Sendeleistung}
{Durch leichte Verstimmung der Sendefrequenz}
\end{QQuestion}

}
\only<2>{
\begin{QQuestion}{BE405}{Wodurch sollte gleichzeitiges Sprechen (Doppeln) bei Nutzung eines Repeaters und in Funkrunden vermieden werden?}{Durch Nutzung eines Sendezeitbegrenzers}
{\textbf{\textcolor{DARCgreen}{Durch ordentliche Übergabe nach jedem Durchgang}}}
{Durch Senden mit möglichst großer Sendeleistung}
{Durch leichte Verstimmung der Sendefrequenz}
\end{QQuestion}

}
\end{frame}

\begin{frame}
\frametitle{Sendeleistung}
\begin{itemize}
  \item Nach Anlage 1 der AFuV
  \item Für automatische Station oberhalb von \qty{30}{\mega\hertz} mit \qty{50}{\watt} ERP
  \end{itemize}
\end{frame}

\begin{frame}
\only<1>{
\begin{QQuestion}{VD503}{Wie hoch ist die maximal zulässige Strahlungsleistung einer Relaisfunkstelle oberhalb \qty{30}{\MHz}?}{\qty{100}{\W} PEP}
{\qty{750}{\W} PEP für Klasse A, \qty{75}{\W} PEP für Klasse E und \qty{10}{\W} PEP für Klasse N}
{\qty{50}{\W} ERP}
{\qty{150}{\W} ERP}
\end{QQuestion}

}
\only<2>{
\begin{QQuestion}{VD503}{Wie hoch ist die maximal zulässige Strahlungsleistung einer Relaisfunkstelle oberhalb \qty{30}{\MHz}?}{\qty{100}{\W} PEP}
{\qty{750}{\W} PEP für Klasse A, \qty{75}{\W} PEP für Klasse E und \qty{10}{\W} PEP für Klasse N}
{\textbf{\textcolor{DARCgreen}{\qty{50}{\W} ERP}}}
{\qty{150}{\W} ERP}
\end{QQuestion}

}
\end{frame}

\begin{frame}
\frametitle{Rapport}
\begin{itemize}
  \item Empfangene Signalstärke (S) ist die des Relais
  \item Es wird darauf verzichtet
  \item Nur die Lesbarkeit (R) wird im Rapport beurteilt
  \end{itemize}
\end{frame}

\begin{frame}
\only<1>{
\begin{QQuestion}{BE408}{Wie wird eine Funkverbindung beurteilt, wenn über eine Relaisfunkstelle gearbeitet wird?}{Es wird nur die Lesbarkeit \glqq R\grqq{} beurteilt, weil sich die Signalstärke \glqq S\grqq{} auf die Relaisfunkstelle bezieht.}
{Es werden die Lesbarkeit \glqq R\grqq{} und die Signalstärke  \glqq S\grqq{} beurteilt, weil das zu einem vollständigen Rapport dazugehört.}
{Es wird nur die Signalstärke  \glqq S\grqq{} beurteilt, weil die Lesbarkeit \glqq R\grqq{} bei einem Relais immer gleich gut ist.}
{Es werden nur verbale Aussagen gemacht, da die exakte Einschätzung bei Betrieb über eine Relaisfunkstelle nicht möglich ist.}
\end{QQuestion}

}
\only<2>{
\begin{QQuestion}{BE408}{Wie wird eine Funkverbindung beurteilt, wenn über eine Relaisfunkstelle gearbeitet wird?}{\textbf{\textcolor{DARCgreen}{Es wird nur die Lesbarkeit \glqq R\grqq{} beurteilt, weil sich die Signalstärke \glqq S\grqq{} auf die Relaisfunkstelle bezieht.}}}
{Es werden die Lesbarkeit \glqq R\grqq{} und die Signalstärke  \glqq S\grqq{} beurteilt, weil das zu einem vollständigen Rapport dazugehört.}
{Es wird nur die Signalstärke  \glqq S\grqq{} beurteilt, weil die Lesbarkeit \glqq R\grqq{} bei einem Relais immer gleich gut ist.}
{Es werden nur verbale Aussagen gemacht, da die exakte Einschätzung bei Betrieb über eine Relaisfunkstelle nicht möglich ist.}
\end{QQuestion}

}
\end{frame}%ENDCONTENT
