
\section{Umwegleitung}
\label{section:umwegleitung}
\begin{frame}%STARTCONTENT

\only<1>{
\begin{QQuestion}{AG420}{Ein Dipol soll mit einem Koaxkabel gleicher Impedanz gespeist werden. Dabei erreicht man einen Symmetriereffekt zum Beispiel durch~...}{Parallelschalten eines am freien Ende offenen $\lambda$/4 langen Leitungsstücks (Stub) am Speisepunkt der Antenne.}
{die Einfügung von Sperrkreisen (Traps) in den Dipol.}
{Symmetrierglieder wie Umwegleitung oder Balun.}
{Parallelschalten eines am freien Ende kurzgeschlossenen $\lambda$/2 langen Leitungsstücks (Stub) am Speisepunkt der Antenne.}
\end{QQuestion}

}
\only<2>{
\begin{QQuestion}{AG420}{Ein Dipol soll mit einem Koaxkabel gleicher Impedanz gespeist werden. Dabei erreicht man einen Symmetriereffekt zum Beispiel durch~...}{Parallelschalten eines am freien Ende offenen $\lambda$/4 langen Leitungsstücks (Stub) am Speisepunkt der Antenne.}
{die Einfügung von Sperrkreisen (Traps) in den Dipol.}
{\textbf{\textcolor{DARCgreen}{Symmetrierglieder wie Umwegleitung oder Balun.}}}
{Parallelschalten eines am freien Ende kurzgeschlossenen $\lambda$/2 langen Leitungsstücks (Stub) am Speisepunkt der Antenne.}
\end{QQuestion}

}
\end{frame}

\begin{frame}
\only<1>{
\begin{PQuestion}{AG423}{Was zeigt diese Darstellung?}{Sie zeigt einen symmetrischen \qty{60}{\ohm}-Schleifendipol mit einem koaxialen Leitungskreis, der als Sperrfilter zur Unterdrückung von unerwünschten Aussendungen eingesetzt ist.}
{Sie zeigt einen symmetrischen \qty{60}{\ohm}-Schleifendipol mit Koaxialkabel-Balun. Durch die Anordnung wird die symmetrische Antenne an ein unsymmetrisches \qty{60}{\ohm}-Antennenkabel angepasst.}
{Sie zeigt einen $\lambda$/2-Dipol mit symmetrierender $\lambda$/2-Umwegleitung. Durch die Anordnung wird der Fußpunktwiderstand der symmetrischen Antenne von \qty{120}{\ohm} an ein unsymmetrisches \qty{60}{\ohm}-Antennenkabel angepasst.}
{Sie zeigt einen $\lambda$/2-Faltdipol mit $\lambda$/2-Umwegleitung. Durch die Anordnung wird der Fußpunktwiderstand der symmetrischen Antenne von \qty{240}{\ohm} an ein unsymmetrisches \qty{60}{\ohm}-Antennenkabel angepasst.}
{\DARCimage{1.0\linewidth}{562include}}\end{PQuestion}

}
\only<2>{
\begin{PQuestion}{AG423}{Was zeigt diese Darstellung?}{Sie zeigt einen symmetrischen \qty{60}{\ohm}-Schleifendipol mit einem koaxialen Leitungskreis, der als Sperrfilter zur Unterdrückung von unerwünschten Aussendungen eingesetzt ist.}
{Sie zeigt einen symmetrischen \qty{60}{\ohm}-Schleifendipol mit Koaxialkabel-Balun. Durch die Anordnung wird die symmetrische Antenne an ein unsymmetrisches \qty{60}{\ohm}-Antennenkabel angepasst.}
{Sie zeigt einen $\lambda$/2-Dipol mit symmetrierender $\lambda$/2-Umwegleitung. Durch die Anordnung wird der Fußpunktwiderstand der symmetrischen Antenne von \qty{120}{\ohm} an ein unsymmetrisches \qty{60}{\ohm}-Antennenkabel angepasst.}
{\textbf{\textcolor{DARCgreen}{Sie zeigt einen $\lambda$/2-Faltdipol mit $\lambda$/2-Umwegleitung. Durch die Anordnung wird der Fußpunktwiderstand der symmetrischen Antenne von \qty{240}{\ohm} an ein unsymmetrisches \qty{60}{\ohm}-Antennenkabel angepasst.}}}
{\DARCimage{1.0\linewidth}{562include}}\end{PQuestion}

}
\end{frame}

\begin{frame}
\only<1>{
\begin{PQuestion}{AG424}{Zur Anpassung von Antennen werden häufig Umwegleitungen verwendet. Wie arbeitet die folgende Schaltung?}{Der $\lambda$/2-Faltdipol hat eine Impedanz von \qty{240}{\ohm}. Durch die $\lambda$/2-Umwegleitung erfolgt eine Widerstandstransformation von 4:1 mit Phasendrehung um \qty{360}{\degree}, womit an der Seite der Antennenleitung eine Ausgangsimpedanz von \qty{60}{\ohm} erreicht wird.}
{Der $\lambda$/2-Faltdipol hat an jedem seiner Anschlüsse eine Impedanz von \qty{120}{\ohm} gegen Erde. Durch die $\lambda$/2-Umwegleitung erfolgt eine 1:1-Widerstandstransformation mit Phasendrehung um \qty{180}{\degree}. An der Seite der Antennenleitung erfolgt eine phasenrichtige Parallelschaltung von 2~mal \qty{120}{\ohm} gegen Erde, womit eine Ausgangsimpedanz von \qty{60}{\ohm} erreicht wird.
}
{Der $\lambda$/2-Dipol hat eine Impedanz von \qty{60}{\ohm}. Durch die $\lambda$/2-Umwegleitung erfolgt eine Widerstandstransformation von 1:2 mit Phasendrehung um \qty{180}{\degree}. An der Seite der Antennenleitung erfolgt eine phasenrichtige Parallelschaltung von 2~mal \qty{120}{\ohm} gegen Erde, womit eine Ausgangsimpedanz von \qty{60}{\ohm} erreicht wird.}
{Der $\lambda$/2-Dipol hat eine Impedanz von \qty{240}{\ohm}. Durch die $\lambda$/2-Umwegleitung erfolgt eine Widerstandstransformation von 4:1 mit Phasendrehung um \qty{360}{\degree}, womit an der Seite der Antennenleitung eine Ausgangsimpedanz von \qty{60}{\ohm} erreicht wird.}
{\DARCimage{1.0\linewidth}{562include}}\end{PQuestion}

}
\only<2>{
\begin{PQuestion}{AG424}{Zur Anpassung von Antennen werden häufig Umwegleitungen verwendet. Wie arbeitet die folgende Schaltung?}{Der $\lambda$/2-Faltdipol hat eine Impedanz von \qty{240}{\ohm}. Durch die $\lambda$/2-Umwegleitung erfolgt eine Widerstandstransformation von 4:1 mit Phasendrehung um \qty{360}{\degree}, womit an der Seite der Antennenleitung eine Ausgangsimpedanz von \qty{60}{\ohm} erreicht wird.}
{\textbf{\textcolor{DARCgreen}{Der $\lambda$/2-Faltdipol hat an jedem seiner Anschlüsse eine Impedanz von \qty{120}{\ohm} gegen Erde. Durch die $\lambda$/2-Umwegleitung erfolgt eine 1:1-Widerstandstransformation mit Phasendrehung um \qty{180}{\degree}. An der Seite der Antennenleitung erfolgt eine phasenrichtige Parallelschaltung von 2~mal \qty{120}{\ohm} gegen Erde, womit eine Ausgangsimpedanz von \qty{60}{\ohm} erreicht wird.
}}}
{Der $\lambda$/2-Dipol hat eine Impedanz von \qty{60}{\ohm}. Durch die $\lambda$/2-Umwegleitung erfolgt eine Widerstandstransformation von 1:2 mit Phasendrehung um \qty{180}{\degree}. An der Seite der Antennenleitung erfolgt eine phasenrichtige Parallelschaltung von 2~mal \qty{120}{\ohm} gegen Erde, womit eine Ausgangsimpedanz von \qty{60}{\ohm} erreicht wird.}
{Der $\lambda$/2-Dipol hat eine Impedanz von \qty{240}{\ohm}. Durch die $\lambda$/2-Umwegleitung erfolgt eine Widerstandstransformation von 4:1 mit Phasendrehung um \qty{360}{\degree}, womit an der Seite der Antennenleitung eine Ausgangsimpedanz von \qty{60}{\ohm} erreicht wird.}
{\DARCimage{1.0\linewidth}{562include}}\end{PQuestion}

}
\end{frame}%ENDCONTENT
