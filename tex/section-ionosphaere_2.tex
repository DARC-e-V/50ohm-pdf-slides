
\section{Ionosphäre II}
\label{section:ionosphaere_2}
\begin{frame}%STARTCONTENT

\begin{columns}
    \begin{column}{0.48\textwidth}
    \begin{itemize}
  \item Ionosphäre enthält große Menge von Ionen und freier Elektronen
  \item In ca. \qtyrange{50}{450}{\kilo\metre} Höhe
  \item Refraktion von Kurzwellen, wodurch weltweite Kommunikation ermöglicht wird
  \end{itemize}

    \end{column}
   \begin{column}{0.48\textwidth}
       
\begin{figure}
    \DARCimage{0.85\linewidth}{731include}
    \caption{\scriptsize Für den Amateurfunk relevante Schichten in der Atmosphäre}
    \label{e_atmosphaeren_schichten}
\end{figure}


   \end{column}
\end{columns}

\end{frame}

\begin{frame}
\only<1>{
\begin{QQuestion}{EH101}{Wie kommt die Fernausbreitung einer Funkwelle auf den Kurzwellenbändern zustande? Sie kommt zustande durch die Refraktion (Brechung) an~...}{den Wolken in der niedrigen Atmosphäre.}
{Hoch- und Tiefdruckgebieten der hohen Atmosphäre.}
{elektrisch aufgeladenen Luftschichten in der Ionosphäre.}
{den parasitären Elementen einer Richtantenne.}
\end{QQuestion}

}
\only<2>{
\begin{QQuestion}{EH101}{Wie kommt die Fernausbreitung einer Funkwelle auf den Kurzwellenbändern zustande? Sie kommt zustande durch die Refraktion (Brechung) an~...}{den Wolken in der niedrigen Atmosphäre.}
{Hoch- und Tiefdruckgebieten der hohen Atmosphäre.}
{\textbf{\textcolor{DARCgreen}{elektrisch aufgeladenen Luftschichten in der Ionosphäre.}}}
{den parasitären Elementen einer Richtantenne.}
\end{QQuestion}

}
\end{frame}

\begin{frame}
\frametitle{Ausbreitung von Funkwellen}
\begin{columns}
    \begin{column}{0.48\textwidth}
    \begin{itemize}
  \item In den kommenden Abschnitten werden Einflüsse der Funkwellenausbreitung an der Ionosphäre besprochen
  \item Und mit welchen Maßnahmen wir die Ausbreitung unserer Funkwellen optimieren können
  \item Zuerst eine exemplarische Betrachtung der Wellenausbreitung
  \end{itemize}

    \end{column}
   \begin{column}{0.48\textwidth}
       
\begin{figure}
    \DARCimage{0.85\linewidth}{865include}
    \caption{\scriptsize Refraktion an Schichten der Ionosphäre}
    \label{e_wellenausbreitung_refraktion}
\end{figure}


   \end{column}
\end{columns}

\end{frame}

\begin{frame}\end{frame}

\begin{frame}
\frametitle{Schichten der Ionosphäre}
\begin{itemize}
  \item Es gibt in verschiedenen Höhen verschiedene \enquote{Schichten} bzw. Regionen mit unterschiedlich starker Ionisierung
  \item Diese tragen die Namen
  \end{itemize}
\begin{enumerate}
  \item[1] D-Schicht
  \item[2] E-Schicht
  \item[3] F<sub>1</sub>-Schicht
  \item[4] F<sub>2</sub>-Schicht
  \end{enumerate}
\end{frame}

\begin{frame}
\frametitle{D-Region}
\begin{itemize}
  \item In ca. 50–\qty{90}{\kilo\metre} Höhe
  \item Existiert \emph{nur am Tag}
  \item Nach Sonnenuntergang sehr schnell verschwunden
  \item Starke \emph{Dämpfung} von Funkwellen unter \qty{10}{\mega\hertz}
  \item Keine Raumwelle für Amateurfunkbänder wie \qty{160}{\metre} oder 80m
  \end{itemize}

\end{frame}

\begin{frame}
\only<1>{
\begin{QQuestion}{EH210}{Warum sind Signale im 160- und \qty{80}{\m}-Band tagsüber nur schwach und nicht für den weltweiten Funkverkehr geeignet? Sie sind ungeeignet wegen der Tagesdämpfung in der~...}{A-Region.}
{F1-Region.}
{F2-Region.}
{D-Region.}
\end{QQuestion}

}
\only<2>{
\begin{QQuestion}{EH210}{Warum sind Signale im 160- und \qty{80}{\m}-Band tagsüber nur schwach und nicht für den weltweiten Funkverkehr geeignet? Sie sind ungeeignet wegen der Tagesdämpfung in der~...}{A-Region.}
{F1-Region.}
{F2-Region.}
{\textbf{\textcolor{DARCgreen}{D-Region.}}}
\end{QQuestion}

}
\end{frame}

\begin{frame}
\only<1>{
\begin{QQuestion}{EH105}{Welchen Einfluss hat die D-Region auf die Fernausbreitung?}{Die D-Region absorbiert tagsüber die Wellen im \qty{10}{\m}-Band.}
{Die D-Region reflektiert tagsüber die Wellen im 80- und \qty{160}{\m}-Band.}
{Die D-Region führt tagsüber zu starker Dämpfung im 80- und \qty{160}{\m}-Band.}
{Die D-Region verhindert nachts die Fernausbreitung im Lang-, Mittel- und unteren Kurzwellenbereich.}
\end{QQuestion}

}
\only<2>{
\begin{QQuestion}{EH105}{Welchen Einfluss hat die D-Region auf die Fernausbreitung?}{Die D-Region absorbiert tagsüber die Wellen im \qty{10}{\m}-Band.}
{Die D-Region reflektiert tagsüber die Wellen im 80- und \qty{160}{\m}-Band.}
{\textbf{\textcolor{DARCgreen}{Die D-Region führt tagsüber zu starker Dämpfung im 80- und \qty{160}{\m}-Band.}}}
{Die D-Region verhindert nachts die Fernausbreitung im Lang-, Mittel- und unteren Kurzwellenbereich.}
\end{QQuestion}

}
\end{frame}

\begin{frame}
\frametitle{E-Region}
\begin{itemize}
  \item In ca. 90–\qty{130}{\kilo\metre} Höhe
  \item Entsteht \emph{tagsüber} mit Maximum zur Mittagszeit
  \item Verschwindet etwa 1 Stunde nach Sonnenuntergang
  \item Starke Ionisation $\rightarrow$ Sporadic-E
  \item Namensgebene: \emph{E}(lektrische)\emph{-Schicht}
  \end{itemize}

\end{frame}

\begin{frame}
\only<1>{
\begin{QQuestion}{EH106}{Welche ionosphärische Region sorgt während der Sommermonate für gelegentliche gute Ausbreitung vom oberen Kurzwellenbereich bis in den UKW-Bereich?}{Die E-Region}
{Die D-Region}
{Die F1-Region}
{Die F2-Region}
\end{QQuestion}

}
\only<2>{
\begin{QQuestion}{EH106}{Welche ionosphärische Region sorgt während der Sommermonate für gelegentliche gute Ausbreitung vom oberen Kurzwellenbereich bis in den UKW-Bereich?}{\textbf{\textcolor{DARCgreen}{Die E-Region}}}
{Die D-Region}
{Die F1-Region}
{Die F2-Region}
\end{QQuestion}

}
\end{frame}

\begin{frame}
\frametitle{F-Regionen}
\begin{itemize}
  \item In ca. 200–\qty{400}{\kilo\metre} Höhe
  \item Am stärksten ionisierte Schicht
  \item F<sub>1</sub>-Schicht existiert \emph{nur am Tag}
  \item F<sub>2</sub>-Schicht bleibt \emph{nachts} bestehen
  \end{itemize}
\end{frame}

\begin{frame}
\only<1>{
\begin{QQuestion}{EH104}{Welche ionosphärische Region ermöglicht DX-Verbindungen im \qty{80}{\m}-Band in der Nacht?}{Die E-Region}
{Die F2-Region}
{Die D-Region}
{Die F1-Region}
\end{QQuestion}

}
\only<2>{
\begin{QQuestion}{EH104}{Welche ionosphärische Region ermöglicht DX-Verbindungen im \qty{80}{\m}-Band in der Nacht?}{Die E-Region}
{\textbf{\textcolor{DARCgreen}{Die F2-Region}}}
{Die D-Region}
{Die F1-Region}
\end{QQuestion}

}
\end{frame}

\begin{frame}
\only<1>{
\begin{QQuestion}{EH103}{Welche ionosphärische Region ermöglicht im wesentlichen Weitverkehrsverbindungen im Kurzwellenbereich?}{D-Region}
{F2-Region}
{E-Region}
{F1-Region}
\end{QQuestion}

}
\only<2>{
\begin{QQuestion}{EH103}{Welche ionosphärische Region ermöglicht im wesentlichen Weitverkehrsverbindungen im Kurzwellenbereich?}{D-Region}
{\textbf{\textcolor{DARCgreen}{F2-Region}}}
{E-Region}
{F1-Region}
\end{QQuestion}

}
\end{frame}

\begin{frame}
\only<1>{
\begin{QQuestion}{EH102}{In welcher Höhe befinden sich für die Kurzwellen-Fernausbreitung (DX) wichtige ionosphärische Regionen? Sie befinden sich in ungefähr~...}{\qtyrange{130}{450}{\km} Höhe.}
{\qtyrange{50}{90}{\km} Höhe.}
{\qtyrange{90}{130}{\km} Höhe.}
{\qtyrange{130}{200}{\km} Höhe.}
\end{QQuestion}

}
\only<2>{
\begin{QQuestion}{EH102}{In welcher Höhe befinden sich für die Kurzwellen-Fernausbreitung (DX) wichtige ionosphärische Regionen? Sie befinden sich in ungefähr~...}{\textbf{\textcolor{DARCgreen}{\qtyrange{130}{450}{\km} Höhe.}}}
{\qtyrange{50}{90}{\km} Höhe.}
{\qtyrange{90}{130}{\km} Höhe.}
{\qtyrange{130}{200}{\km} Höhe.}
\end{QQuestion}

}

\end{frame}

\begin{frame}
\frametitle{Sonnenzyklus}
\begin{columns}
    \begin{column}{0.48\textwidth}
    \begin{itemize}
  \item Im Schnitt alle 11 Jahre durch Umkehrung des Magnetfelds
  \item Führt zu starker Ionisation der F<sub>2</sub>-Region
  \end{itemize}

    \end{column}
   \begin{column}{0.48\textwidth}
       
\begin{figure}
    \DARCimage{0.85\linewidth}{729include}
    \caption{\scriptsize Zählung der monatlichen Sonnenflecken seit 1749}
    \label{e_sonnenzyklus}
\end{figure}


   \end{column}
\end{columns}

\end{frame}

\begin{frame}
\frametitle{Ursachen des Sonnenzyklus}
\begin{itemize}
  \item Polregionen der Sonne rotieren langsamer als Äquator
  \item Führt zu inneren Spannungen des Magnetfelds
  \item Magnetfelder zwischen Sonnenflecken fallen abrupt zusammen $\rightarrow$ Plasma wird freigesetzt und hat Einfluss auf die Ionosphäre der Erde
  \end{itemize}
\end{frame}

\begin{frame}
\only<1>{
\begin{QQuestion}{EH107}{Die Sonnenaktivität ist einem regelmäßigen Zyklus unterworfen. Welchen Zeitraum hat dieser Zyklus ungefähr?}{12 Monate}
{6 Monate}
{11 Jahre}
{7 Jahre}
\end{QQuestion}

}
\only<2>{
\begin{QQuestion}{EH107}{Die Sonnenaktivität ist einem regelmäßigen Zyklus unterworfen. Welchen Zeitraum hat dieser Zyklus ungefähr?}{12 Monate}
{6 Monate}
{\textbf{\textcolor{DARCgreen}{11 Jahre}}}
{7 Jahre}
\end{QQuestion}

}
\end{frame}

\begin{frame}
\only<1>{
\begin{QQuestion}{EH205}{Welche Aussage ist für das Sonnenfleckenmaximum richtig?}{Die Sonnenaktivität ist sehr hoch und führt zu stärkerer Ionisation in der F-Region.}
{Die Sonnenaktivität ist sehr hoch und führt zu schwächerer Ionisation in der F-Region.}
{Die Sonnenaktivität verringert sich stark und führt zu stärkerer Ionisation in der F-Region.}
{Die Sonnenaktivität ist in der Nacht sehr hoch, am Tag sehr schwach und führt deshalb zu keiner Ionisation in der D-Region.}
\end{QQuestion}

}
\only<2>{
\begin{QQuestion}{EH205}{Welche Aussage ist für das Sonnenfleckenmaximum richtig?}{\textbf{\textcolor{DARCgreen}{Die Sonnenaktivität ist sehr hoch und führt zu stärkerer Ionisation in der F-Region.}}}
{Die Sonnenaktivität ist sehr hoch und führt zu schwächerer Ionisation in der F-Region.}
{Die Sonnenaktivität verringert sich stark und führt zu stärkerer Ionisation in der F-Region.}
{Die Sonnenaktivität ist in der Nacht sehr hoch, am Tag sehr schwach und führt deshalb zu keiner Ionisation in der D-Region.}
\end{QQuestion}

}
\end{frame}

\begin{frame}
\only<1>{
\begin{QQuestion}{EH219}{Welches Frequenzband kann im Sonnenfleckenmaximum tagsüber auch mit kleiner Leistung für weltweite Funkverbindungen verwendet werden?}{\qty{10}{\m}-Band}
{\qty{2}{\m}-Band}
{\qty{80}{\m}-Band}
{\qty{160}{\m}-Band}
\end{QQuestion}

}
\only<2>{
\begin{QQuestion}{EH219}{Welches Frequenzband kann im Sonnenfleckenmaximum tagsüber auch mit kleiner Leistung für weltweite Funkverbindungen verwendet werden?}{\textbf{\textcolor{DARCgreen}{\qty{10}{\m}-Band}}}
{\qty{2}{\m}-Band}
{\qty{80}{\m}-Band}
{\qty{160}{\m}-Band}
\end{QQuestion}

}
\end{frame}%ENDCONTENT
