
\section{Akkus}
\label{section:akku}
\begin{frame}%STARTCONTENT

\only<1>{
\begin{PQuestion}{AB209}{Folgende Schaltung eines Akkus besteht aus Zellen von je \qty{2}{\V}. Jede Zelle kann \qty{10}{\A\hour} Ladung liefern. Welche Daten hat der Akku?}{\qty{2}{\V}/\qty{10}{\A\hour}}
{\qty{12}{\V}/\qty{60}{\A\hour}}
{\qty{12}{\V}/\qty{10}{\A\hour}}
{\qty{2}{\V}/\qty{60}{\A\hour}}
{\DARCimage{1.0\linewidth}{431include}}\end{PQuestion}

}
\only<2>{
\begin{PQuestion}{AB209}{Folgende Schaltung eines Akkus besteht aus Zellen von je \qty{2}{\V}. Jede Zelle kann \qty{10}{\A\hour} Ladung liefern. Welche Daten hat der Akku?}{\qty{2}{\V}/\qty{10}{\A\hour}}
{\qty{12}{\V}/\qty{60}{\A\hour}}
{\textbf{\textcolor{DARCgreen}{\qty{12}{\V}/\qty{10}{\A\hour}}}}
{\qty{2}{\V}/\qty{60}{\A\hour}}
{\DARCimage{1.0\linewidth}{431include}}\end{PQuestion}

}
\end{frame}

\begin{frame}
\frametitle{Lösungsweg}
\begin{itemize}
  \item gegeben: $U = 2V$
  \item gegeben: $Q = 10Ah$
  \item gegeben: $N = 6$
  \item gesucht: $U_{ges}, Q_{ges}$
  \end{itemize}
    \pause
    $U_{ges} = N \cdot U = 6 \cdot 2V = 12V$
    \pause
    $Q_{ges} = Q =10Ah$



\end{frame}

\begin{frame}
\only<1>{
\begin{QQuestion}{AB210}{Auf dem Akku-Pack eines Handfunksprechgerätes stehen folgende Angaben: \qty{7,4}{\V} - \qty{2200}{\milli\A\hour} - \qty{16,28}{\W\hour}. Welcher Begriff ist für die Angabe \qty{2200}{\milli\A\hour} zutreffend.}{Nennkapazität}
{Nennleistung}
{maximaler Ladestrom pro Stunde}
{maximaler Entladestrom pro Stunde}
\end{QQuestion}

}
\only<2>{
\begin{QQuestion}{AB210}{Auf dem Akku-Pack eines Handfunksprechgerätes stehen folgende Angaben: \qty{7,4}{\V} - \qty{2200}{\milli\A\hour} - \qty{16,28}{\W\hour}. Welcher Begriff ist für die Angabe \qty{2200}{\milli\A\hour} zutreffend.}{\textbf{\textcolor{DARCgreen}{Nennkapazität}}}
{Nennleistung}
{maximaler Ladestrom pro Stunde}
{maximaler Entladestrom pro Stunde}
\end{QQuestion}

}
\end{frame}

\begin{frame}
\only<1>{
\begin{QQuestion}{AB211}{Wie lange könnte man idealerweise mit einem voll geladenen Akku mit \qty{60}{\A\hour} einen Amateurfunkempfänger betreiben, bis dieser auf \qty{10}{\percent} seiner Kapazität entladen ist und einen Strom von \qty{0,8}{\A} aufnimmt?}{74~Stunden und 60~Minuten}
{43~Stunden und 12~Minuten}
{67~Stunden und 30~Minuten}
{48~Stunden und 0~Minuten}
\end{QQuestion}

}
\only<2>{
\begin{QQuestion}{AB211}{Wie lange könnte man idealerweise mit einem voll geladenen Akku mit \qty{60}{\A\hour} einen Amateurfunkempfänger betreiben, bis dieser auf \qty{10}{\percent} seiner Kapazität entladen ist und einen Strom von \qty{0,8}{\A} aufnimmt?}{74~Stunden und 60~Minuten}
{43~Stunden und 12~Minuten}
{\textbf{\textcolor{DARCgreen}{67~Stunden und 30~Minuten}}}
{48~Stunden und 0~Minuten}
\end{QQuestion}

}
\end{frame}

\begin{frame}
\frametitle{Lösungsweg}
\begin{itemize}
  \item gegeben: $Q_{max} = 60Ah$
  \item gegeben: $Q_{10\%} = 0,1 \cdot Q_{max} = 6Ah$
  \item gegeben: $I = 0,8A$
  \item gesucht: $t$
  \end{itemize}
    \pause
    $Q = I \cdot t \Rightarrow t = \frac{Q}{I} = \frac{Q_{max} -- Q_{10\%}}{I} = \frac{54Ah}{0,8A} = 67,5h$



\end{frame}

\begin{frame}
\only<1>{
\begin{QQuestion}{AB501}{Ein \qty{12}{\V} Akku hat eine Kapazität von \qty{5}{\A\hour}. Welcher speicherbaren Energie entspricht das?}{\qty{5,0}{\W\hour}}
{\qty{12,0}{\W\hour}}
{\qty{2,4}{\W\hour}}
{\qty{60,0}{\W\hour}}
\end{QQuestion}

}
\only<2>{
\begin{QQuestion}{AB501}{Ein \qty{12}{\V} Akku hat eine Kapazität von \qty{5}{\A\hour}. Welcher speicherbaren Energie entspricht das?}{\qty{5,0}{\W\hour}}
{\qty{12,0}{\W\hour}}
{\qty{2,4}{\W\hour}}
{\textbf{\textcolor{DARCgreen}{\qty{60,0}{\W\hour}}}}
\end{QQuestion}

}
\end{frame}

\begin{frame}
\frametitle{Lösungsweg}
\begin{itemize}
  \item gegeben: $U = 12V$
  \item gegeben: $Q = 5Ah$
  \item gesucht: $W$
  \end{itemize}
    \pause
    $W = P \cdot t = U \cdot I \cdot t = U \cdot Q = 12V \cdot 5Ah = 60,0Wh$



\end{frame}%ENDCONTENT
