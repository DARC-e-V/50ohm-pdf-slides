
\section{Yagi-Uda Antenne II}
\label{section:yagi_uda_2}
\begin{frame}%STARTCONTENT

\frametitle{Funktionsprinzip}
\begin{columns}
    \begin{column}{0.48\textwidth}
    \begin{itemize}
  \item Einspeisung an \emph{Strahler} ausgeführt als Dipol oder Faltdipol
  \item Welle trifft auf längeren \emph{Reflektor} und kürzeren \emph{Direktor}
  \item Es kann auch mehrere Direktoren geben
  \end{itemize}

    \end{column}
   \begin{column}{0.48\textwidth}
       
\begin{figure}
    \DARCimage{0.85\linewidth}{531include}
    \caption{\scriptsize Die Elemente einer Yagi-Uda-Antenne}
    \label{e_yagi_uda_aufbau}
\end{figure}


   \end{column}
\end{columns}

\end{frame}

\begin{frame}
\only<1>{
\begin{PQuestion}{EG111}{Das folgende Bild enthält eine einfache Richtantenne. Die Bezeichnungen der Elemente in numerischer Reihenfolge lauten~...}{1 Strahler, 2 Direktor und 3 Reflektor.}
{1 Reflektor, 2 Strahler und 3 Direktor.}
{1 Direktor, 2 Strahler und 3 Reflektor.}
{1 Direktor, 2 Reflektor und 3 Strahler.}
{\DARCimage{1.0\linewidth}{531include}}\end{PQuestion}

}
\only<2>{
\begin{PQuestion}{EG111}{Das folgende Bild enthält eine einfache Richtantenne. Die Bezeichnungen der Elemente in numerischer Reihenfolge lauten~...}{1 Strahler, 2 Direktor und 3 Reflektor.}
{\textbf{\textcolor{DARCgreen}{1 Reflektor, 2 Strahler und 3 Direktor.}}}
{1 Direktor, 2 Strahler und 3 Reflektor.}
{1 Direktor, 2 Reflektor und 3 Strahler.}
{\DARCimage{1.0\linewidth}{531include}}\end{PQuestion}

}
\end{frame}

\begin{frame}
\frametitle{Parasitäre Elemente}
\begin{itemize}
  \item Reflektor und Direktor schwingen ohne elektrische Verbindung zum Strahler zu haben
  \item Haben auch keine Antenneneinspeisung
  \item Nehmen dennoch Energie auf und geben sie wieder ab
  \end{itemize}
\end{frame}

\begin{frame}
\only<1>{
\begin{QQuestion}{EG212}{An welchem Element einer Yagi-Uda-Antenne erfolgt die Energieeinspeisung? Sie erfolgt am~...}{Direktor}
{Strahler}
{Reflektor}
{Strahler und am Reflektor gleichzeitig}
\end{QQuestion}

}
\only<2>{
\begin{QQuestion}{EG212}{An welchem Element einer Yagi-Uda-Antenne erfolgt die Energieeinspeisung? Sie erfolgt am~...}{Direktor}
{\textbf{\textcolor{DARCgreen}{Strahler}}}
{Reflektor}
{Strahler und am Reflektor gleichzeitig}
\end{QQuestion}

}
\end{frame}

\begin{frame}
\frametitle{Richtwirkung}
\begin{columns}
    \begin{column}{0.48\textwidth}
    \begin{itemize}
  \item Zwischen Strahler und Elementen gibt es eine räumliche und zeitliche Phasenverschiebung
  \item Durch die Überlagerung der Abstrahlung entsteht eine Richtwirkung
  \item \emph{Destruktive Interferenz}: Wellen löschen sich aus
  \item \emph{Konstruktive Interferenz}: Wellen verstärken sich
  \end{itemize}

    \end{column}
   \begin{column}{0.48\textwidth}
       
   \end{column}
\end{columns}

\end{frame}

\begin{frame}
\frametitle{Strahlungsdiagramm}
\begin{columns}
    \begin{column}{0.48\textwidth}
    \begin{itemize}
  \item Große Hauptkeule in Richtung der Direktoren
  \item Kleine Nebenkeulen und insbesondere Rückkeule
  \end{itemize}

    \end{column}
   \begin{column}{0.48\textwidth}
       
\begin{figure}
    \DARCimage{0.85\linewidth}{265include}
    \caption{\scriptsize Strahlungsdiagramm einer Yagi-Uda-Antenne}
    \label{e_yagi_uda_strahlungsdiagramm}
\end{figure}


   \end{column}
\end{columns}

\end{frame}

\begin{frame}
\only<1>{
\begin{PQuestion}{EG218}{Für welche Antenne ist dieses Strahlungsdiagramm typisch?}{Yagi-Uda}
{Groundplane}
{Dipol}
{Kugelstrahler}
{\DARCimage{1.0\linewidth}{265include}}\end{PQuestion}

}
\only<2>{
\begin{PQuestion}{EG218}{Für welche Antenne ist dieses Strahlungsdiagramm typisch?}{\textbf{\textcolor{DARCgreen}{Yagi-Uda}}}
{Groundplane}
{Dipol}
{Kugelstrahler}
{\DARCimage{1.0\linewidth}{265include}}\end{PQuestion}

}
\end{frame}%ENDCONTENT
