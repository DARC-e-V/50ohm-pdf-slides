
\section{Binäres Zahlensystem}
\label{section:binaer}
\begin{frame}%STARTCONTENT

\begin{columns}
    \begin{column}{0.48\textwidth}
    \emph{Dezimalsystem}

\begin{itemize}
  \item Menschen sind es gewohnt, die zehn Ziffern von 0 bis 9 zu benutzen
  \item Man spricht von einem Zehner- oder Dezimalsystem
  \end{itemize}

    \end{column}
   \begin{column}{0.48\textwidth}
       \emph{Binärsystem}

\begin{itemize}
  \item Für Computer ist es hingegen einfacher mit nur 2 Ziffern zu arbeiten: Der 0 und der 1
  \item Dies entspricht zwei Zuständen: Beispielsweise ausgeschaltet und eingeschaltet oder auch 0 V und 5 V
  \end{itemize}

   \end{column}
\end{columns}

\end{frame}

\begin{frame}
\only<1>{
\begin{QQuestion}{EA201}{Was ist der Vorteil des binären Zahlensystems gegenüber dem dezimalen Zahlensystem in elektronischen Schaltungen?}{Die Genauigkeit des binären Systems (mit zwei Ziffern) ist um den Faktor 5 höher als die des Dezimalsystems (mit 10 Ziffern).}
{Die binären Ziffern 0 und 1 können als zwei elektrische Zustände dargestellt und dadurch einfach mittels Schaltelementen (z.~B. Transistoren) verarbeitet werden.}
{Der Zwischenbereich zwischen 0 und 1 kann von analogen Verstärkerschaltungen mit hoher Genauigkeit abgebildet werden.}
{Je Ziffer kann mehr als ein Bit an Information übertragen werden (1 binäre Ziffer erlaubt die Übertragung von 8 Dezimalziffern).}
\end{QQuestion}

}
\only<2>{
\begin{QQuestion}{EA201}{Was ist der Vorteil des binären Zahlensystems gegenüber dem dezimalen Zahlensystem in elektronischen Schaltungen?}{Die Genauigkeit des binären Systems (mit zwei Ziffern) ist um den Faktor 5 höher als die des Dezimalsystems (mit 10 Ziffern).}
{\textbf{\textcolor{DARCgreen}{Die binären Ziffern 0 und 1 können als zwei elektrische Zustände dargestellt und dadurch einfach mittels Schaltelementen (z.~B. Transistoren) verarbeitet werden.}}}
{Der Zwischenbereich zwischen 0 und 1 kann von analogen Verstärkerschaltungen mit hoher Genauigkeit abgebildet werden.}
{Je Ziffer kann mehr als ein Bit an Information übertragen werden (1 binäre Ziffer erlaubt die Übertragung von 8 Dezimalziffern).}
\end{QQuestion}

}
\end{frame}

\begin{frame}\begin{itemize}
  \item Mit einem Bit sind zwei Werte möglich (0 und 1)
  \item Mit zwei Bits schon vier (00, 01, 10 und 11) und mit jedem weiteren Bit jeweils doppelt so viele
  \item Mathematisch ausgedrückt: Mit n Bits lassen sich 2<sup>n</sup> verschiedene Zahlen darstellen
  \item Neben Binärzahl wird auch Dualzahl gesagt
  \end{itemize}
\end{frame}

\begin{frame}
\only<1>{
\begin{QQuestion}{EA202}{Wie viele unterschiedliche Zustände können mit einer Dualzahl dargestellt werden, die aus einer Folge von \qty{3}{\bit} besteht?}{8}
{4}
{6}
{16}
\end{QQuestion}

}
\only<2>{
\begin{QQuestion}{EA202}{Wie viele unterschiedliche Zustände können mit einer Dualzahl dargestellt werden, die aus einer Folge von \qty{3}{\bit} besteht?}{\textbf{\textcolor{DARCgreen}{8}}}
{4}
{6}
{16}
\end{QQuestion}

}
\end{frame}

\begin{frame}
\only<1>{
\begin{QQuestion}{EA203}{Wie viele unterschiedliche Zustände können mit einer Dualzahl dargestellt werden, die aus einer Folge von \qty{4}{\bit} besteht?}{16}
{4}
{6}
{8}
\end{QQuestion}

}
\only<2>{
\begin{QQuestion}{EA203}{Wie viele unterschiedliche Zustände können mit einer Dualzahl dargestellt werden, die aus einer Folge von \qty{4}{\bit} besteht?}{\textbf{\textcolor{DARCgreen}{16}}}
{4}
{6}
{8}
\end{QQuestion}

}
\end{frame}

\begin{frame}
\only<1>{
\begin{QQuestion}{EA204}{Wie viele unterschiedliche Werte können mit einer fünfstelligen Dualzahl dargestellt werden?}{5}
{32}
{64}
{128}
\end{QQuestion}

}
\only<2>{
\begin{QQuestion}{EA204}{Wie viele unterschiedliche Werte können mit einer fünfstelligen Dualzahl dargestellt werden?}{5}
{\textbf{\textcolor{DARCgreen}{32}}}
{64}
{128}
\end{QQuestion}

}
\end{frame}

\begin{frame}
\frametitle{Umwandlung}
Binärzahlen in Dezimale Zahlen am Beispiel von 10001110

\begin{table}
\begin{DARCtabular}{cccccccc}
      &  &  &  &  &  &  &   \\
     2<sup>7</sup>  & 2<sup>6</sup>  & 2<sup>5</sup>  & 2<sup>4</sup>  & 2<sup>3</sup>  & 2<sup>2</sup>  & 2<sup>1</sup>  & 2<sup>0</sup>   \\
     128  & 64  & 32  & 16  & 8  & 4  & 2  & 1   \\
     1  & 0  & 0  & 0  & 1  & 1  & 1  & 0   \\
\end{DARCtabular}
\caption{Stellenwerte der achtstelligen Dualzahl 10001110}
\label{binar_stellenwert_dual}
\end{table}
    \pause
    128 + 8 + 4 + 2 = 142

\end{frame}

\begin{frame}
\only<1>{
\begin{QQuestion}{EA205}{Berechnen Sie den dezimalen Wert der Dualzahl 01001110. Die Dezimalzahl lautet:}{156}
{78}
{142}
{248}
\end{QQuestion}

}
\only<2>{
\begin{QQuestion}{EA205}{Berechnen Sie den dezimalen Wert der Dualzahl 01001110. Die Dezimalzahl lautet:}{156}
{\textbf{\textcolor{DARCgreen}{78}}}
{142}
{248}
\end{QQuestion}

}
\end{frame}

\begin{frame}
\only<1>{
\begin{QQuestion}{EA206}{Berechnen Sie den dezimalen Wert der Dualzahl 10001110. Die Dezimalzahl lautet:}{78}
{142}
{156}
{248}
\end{QQuestion}

}
\only<2>{
\begin{QQuestion}{EA206}{Berechnen Sie den dezimalen Wert der Dualzahl 10001110. Die Dezimalzahl lautet:}{78}
{\textbf{\textcolor{DARCgreen}{142}}}
{156}
{248}
\end{QQuestion}

}
\end{frame}

\begin{frame}
\only<1>{
\begin{QQuestion}{EA207}{Berechnen Sie den dezimalen Wert der Dualzahl 10011100. Die Dezimalzahl lautet:}{78}
{142}
{156}
{248}
\end{QQuestion}

}
\only<2>{
\begin{QQuestion}{EA207}{Berechnen Sie den dezimalen Wert der Dualzahl 10011100. Die Dezimalzahl lautet:}{78}
{142}
{\textbf{\textcolor{DARCgreen}{156}}}
{248}
\end{QQuestion}

}
\end{frame}

\begin{frame}
\only<1>{
\begin{QQuestion}{EA208}{Berechnen Sie den dezimalen Wert der Dualzahl 11111000. Die Dezimalzahl lautet:}{142}
{78}
{156}
{248}
\end{QQuestion}

}
\only<2>{
\begin{QQuestion}{EA208}{Berechnen Sie den dezimalen Wert der Dualzahl 11111000. Die Dezimalzahl lautet:}{142}
{78}
{156}
{\textbf{\textcolor{DARCgreen}{248}}}
\end{QQuestion}

}
\end{frame}%ENDCONTENT
