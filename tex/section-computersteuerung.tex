
\section{Computersteuerung}
\label{section:computersteuerung}
\begin{frame}%STARTCONTENT

\frametitle{Steuersignale}
\begin{columns}
    \begin{column}{0.48\textwidth}
    \begin{itemize}
  \item Übertragung von Audio- sowie Steuersignalen (CAT) zwischen Computer und Transceiver
  \item Z.B. Transceiver auf Sendung schalten und Signal vom Computer übertragen
  \end{itemize}

    \end{column}
   \begin{column}{0.48\textwidth}
       
\begin{figure}
    \DARCimage{0.85\linewidth}{630include}
    \caption{\scriptsize Beispiele für Verbindungen zwischen Computer und Funkgerät}
    \label{n_computersteuerung_verbindungen}
\end{figure}


   \end{column}
\end{columns}

\end{frame}

\begin{frame}
\frametitle{Datenanschluss}
\begin{itemize}
  \item Hinter dem Mikrofonanschluss im Funkgerät können Verstärker- und Filterstufen für Sprachübertragung liegen $\rightarrow$ ungeeignet für Datenübertragung
  \item Eigener Datenanschluss am Transceiver
  \item Lässt Signale vom Computer unverfälscht passieren
  \end{itemize}

\end{frame}

\begin{frame}
\only<1>{
\begin{QQuestion}{NF114}{Wie kann eine Verbindung zwischen Funkgerät und Computer für digitale Übertragungsverfahren (z.~B. FT8 oder WSPR) hergestellt werden?}{Der ALC-Anschluss des Funkgeräts wird mittels eines Hardware-Modems mit Audio- oder Datenanschlüssen des Computers verbunden.}
{Eine Audioverbindung (NF-Signal oder digital z.~B. per USB-Kabel) wird zwischen Computer und Funkgerät hergestellt oder es wird ein Hardware-Modem verwendet.}
{Es wird ein Software-Modem installiert und der ALC-Anschluss des Funkgeräts direkt mit dem Computer verbunden (ggf. auch mittels Adapter).}
{Der HF-Anschluss (z.~B. Antennenausgang) des Funkgeräts wird mittels eines Y-Kabels mit einer geeigneten Datenschnittstelle des Computers verbunden.}
\end{QQuestion}

}
\only<2>{
\begin{QQuestion}{NF114}{Wie kann eine Verbindung zwischen Funkgerät und Computer für digitale Übertragungsverfahren (z.~B. FT8 oder WSPR) hergestellt werden?}{Der ALC-Anschluss des Funkgeräts wird mittels eines Hardware-Modems mit Audio- oder Datenanschlüssen des Computers verbunden.}
{\textbf{\textcolor{DARCgreen}{Eine Audioverbindung (NF-Signal oder digital z.~B. per USB-Kabel) wird zwischen Computer und Funkgerät hergestellt oder es wird ein Hardware-Modem verwendet.}}}
{Es wird ein Software-Modem installiert und der ALC-Anschluss des Funkgeräts direkt mit dem Computer verbunden (ggf. auch mittels Adapter).}
{Der HF-Anschluss (z.~B. Antennenausgang) des Funkgeräts wird mittels eines Y-Kabels mit einer geeigneten Datenschnittstelle des Computers verbunden.}
\end{QQuestion}

}
\end{frame}

\begin{frame}
\only<1>{
\begin{QQuestion}{NF116}{Manche Transceiver verfügen über eine sogenannte CAT-Schnittstelle. Dieser Anschluss dient dazu,~...}{mittels eines seriellen Kommunikationsprotokolls den Transceiver z.~B. mit einem Computer zu steuern oder Werte abzufragen, z.~B. Frequenz, Sendeleistung oder PTT.}
{durch Umgehung von Verstärker- und Filterstufen ein NF-Signal (z.~B. für DV oder POCSAG) möglichst verzerrungsfrei abzugreifen oder einzuspeisen.}
{das empfangene HF-Signal möglichst ungefiltert an einen Computer zur Weiterverarbeitung mittels digitaler Signalverarbeitung auszuleiten.}
{ohne weitere Beschaltung einen Drehwinkelgeber (Encoder) oder ein Potentiometer zur präzisen Frequenzeinstellung anzuschließen.}
\end{QQuestion}

}
\only<2>{
\begin{QQuestion}{NF116}{Manche Transceiver verfügen über eine sogenannte CAT-Schnittstelle. Dieser Anschluss dient dazu,~...}{\textbf{\textcolor{DARCgreen}{mittels eines seriellen Kommunikationsprotokolls den Transceiver z.~B. mit einem Computer zu steuern oder Werte abzufragen, z.~B. Frequenz, Sendeleistung oder PTT.}}}
{durch Umgehung von Verstärker- und Filterstufen ein NF-Signal (z.~B. für DV oder POCSAG) möglichst verzerrungsfrei abzugreifen oder einzuspeisen.}
{das empfangene HF-Signal möglichst ungefiltert an einen Computer zur Weiterverarbeitung mittels digitaler Signalverarbeitung auszuleiten.}
{ohne weitere Beschaltung einen Drehwinkelgeber (Encoder) oder ein Potentiometer zur präzisen Frequenzeinstellung anzuschließen.}
\end{QQuestion}

}
\end{frame}

\begin{frame}
\only<1>{
\begin{QQuestion}{NF117}{Welcher unerwünschte Effekt kann eintreten, wenn ein Funkgerät mittels Computer gesteuert wird?}{Der Computer kann wie ein Elektrolytkondensator im Antennenkreis wirken und somit die Sendefrequenz verschieben.}
{Der Vorverstärker ist außer Funktion, wodurch Nachbarkanäle und Frequenzen in anderen Bändern gestört werden könnten.}
{Die automatische Pegelregelung (ALC) könnte ausgelöst werden und andere digitale Geräte stören.}
{Das Funkgerät könnte unerwartet auf Sendung schalten und somit unerwünschte Aussendungen verursachen oder Menschen in Gefahr bringen.}
\end{QQuestion}

}
\only<2>{
\begin{QQuestion}{NF117}{Welcher unerwünschte Effekt kann eintreten, wenn ein Funkgerät mittels Computer gesteuert wird?}{Der Computer kann wie ein Elektrolytkondensator im Antennenkreis wirken und somit die Sendefrequenz verschieben.}
{Der Vorverstärker ist außer Funktion, wodurch Nachbarkanäle und Frequenzen in anderen Bändern gestört werden könnten.}
{Die automatische Pegelregelung (ALC) könnte ausgelöst werden und andere digitale Geräte stören.}
{\textbf{\textcolor{DARCgreen}{Das Funkgerät könnte unerwartet auf Sendung schalten und somit unerwünschte Aussendungen verursachen oder Menschen in Gefahr bringen.}}}
\end{QQuestion}

}

\end{frame}

\begin{frame}
\only<1>{
\begin{QQuestion}{NF115}{Manche FM-Transceiver verfügen über einen analogen Datenanschluss (z.~B. mit DATA beschriftet oder als 9600-Port bezeichnet). Dieser dient im Wesentlichen dazu,~...}{durch Umgehung von Verstärker- und Filterstufen ein NF-Signal (z.~B. für DV oder POCSAG) möglichst verzerrungsfrei abzugreifen oder einzuspeisen.}
{mittels eines seriellen Kommunikationsprotokolls den Transceiver z.~B. mit einem Computer zu steuern und Werte abzufragen, z.~B. Frequenz, Sendeleistung oder PTT.}
{das empfangene HF-Signal möglichst ungefiltert an einen Computer auszuleiten und mittels digitaler Signalverarbeitung weiterzuverarbeiten.}
{ohne weitere Beschaltung einen Drehwinkelgeber (Encoder) oder ein Potentiometer zur präzisen Frequenzeinstellung anzuschließen.}
\end{QQuestion}

}
\only<2>{
\begin{QQuestion}{NF115}{Manche FM-Transceiver verfügen über einen analogen Datenanschluss (z.~B. mit DATA beschriftet oder als 9600-Port bezeichnet). Dieser dient im Wesentlichen dazu,~...}{\textbf{\textcolor{DARCgreen}{durch Umgehung von Verstärker- und Filterstufen ein NF-Signal (z.~B. für DV oder POCSAG) möglichst verzerrungsfrei abzugreifen oder einzuspeisen.}}}
{mittels eines seriellen Kommunikationsprotokolls den Transceiver z.~B. mit einem Computer zu steuern und Werte abzufragen, z.~B. Frequenz, Sendeleistung oder PTT.}
{das empfangene HF-Signal möglichst ungefiltert an einen Computer auszuleiten und mittels digitaler Signalverarbeitung weiterzuverarbeiten.}
{ohne weitere Beschaltung einen Drehwinkelgeber (Encoder) oder ein Potentiometer zur präzisen Frequenzeinstellung anzuschließen.}
\end{QQuestion}

}
\end{frame}%ENDCONTENT
