
\section{Schwingkreis I}
\label{section:schwingkreis_1}
\begin{frame}%STARTCONTENT
\begin{itemize}
  \item Kondensatoren und Spulen haben frequenzabhängige Widerstände
  \item Damit sind passive Filterschaltungen möglich, um nur bestimmte Frequenzen passieren zu lassen
  \end{itemize}
    \pause
    \emph{Zur Erinnerung}

\begin{itemize}
  \item Kondensator blockiert niedrige Frequenzen und  lässt hohe Frequenzen durch
  \item Spule blockiert hohe Frequenzen und lässt niedrige Frequenzen durch
  \end{itemize}


\end{frame}

\begin{frame}
\frametitle{Hochpass}
\begin{columns}
    \begin{column}{0.48\textwidth}
    \begin{itemize}
  \item Bei niedrigen Frequenzen hat der Kondensator einen sehr hohen Widerstand
  \item Schaltung wirkt wie ein frequenzabhängiger Spannungsteiler
  \item U<sub>A</sub> ist dadurch sehr klein
  \end{itemize}

    \end{column}
   \begin{column}{0.48\textwidth}
       
\begin{figure}
    \DARCimage{0.85\linewidth}{592include}
    \caption{\scriptsize Filtercharakteristik eines Hochpass}
    \label{e_hochpass}
\end{figure}


\begin{figure}
    \DARCimage{0.85\linewidth}{195include}
    \caption{\scriptsize Hochpass aus Kondensator und Widerstand}
    \label{e_hochpass_rc}
\end{figure}


   \end{column}
\end{columns}

\end{frame}

\begin{frame}
\only<1>{
\begin{PQuestion}{ED202}{Wie wird die dargestellte Filtercharakteristik bezeichnet?}{Bandsperre}
{Tiefpass}
{Bandpass}
{Hochpass}
{\DARCimage{0.75\linewidth}{592include}}\end{PQuestion}

}
\only<2>{
\begin{PQuestion}{ED202}{Wie wird die dargestellte Filtercharakteristik bezeichnet?}{Bandsperre}
{Tiefpass}
{Bandpass}
{\textbf{\textcolor{DARCgreen}{Hochpass}}}
{\DARCimage{0.75\linewidth}{592include}}\end{PQuestion}

}
\end{frame}

\begin{frame}
\only<1>{
\begin{PQuestion}{ED211}{Was stellt die folgende Schaltung dar? }{Hochpass}
{Sperrkreis}
{Bandpass}
{Tiefpass}
{\DARCimage{1.0\linewidth}{195include}}\end{PQuestion}

}
\only<2>{
\begin{PQuestion}{ED211}{Was stellt die folgende Schaltung dar? }{\textbf{\textcolor{DARCgreen}{Hochpass}}}
{Sperrkreis}
{Bandpass}
{Tiefpass}
{\DARCimage{1.0\linewidth}{195include}}\end{PQuestion}

}
\end{frame}

\begin{frame}
\only<1>{
\begin{PQuestion}{ED212}{Was stellt die folgende Schaltung dar? }{Hochpass}
{Sperrkreis}
{Bandpass}
{Tiefpass}
{\DARCimage{1.0\linewidth}{754include}}\end{PQuestion}

}
\only<2>{
\begin{PQuestion}{ED212}{Was stellt die folgende Schaltung dar? }{\textbf{\textcolor{DARCgreen}{Hochpass}}}
{Sperrkreis}
{Bandpass}
{Tiefpass}
{\DARCimage{1.0\linewidth}{754include}}\end{PQuestion}

}

\end{frame}

\begin{frame}
\only<1>{
\begin{question2x2}{ED213}{Welche Schaltung stellt ein Hochpassfilter dar?}{\DARCimage{1.0\linewidth}{182include}}
{\DARCimage{1.0\linewidth}{167include}}
{\DARCimage{1.0\linewidth}{172include}}
{\DARCimage{1.0\linewidth}{165include}}
\end{question2x2}

}
\only<2>{
\begin{question2x2}{ED213}{Welche Schaltung stellt ein Hochpassfilter dar?}{\DARCimage{1.0\linewidth}{182include}}
{\DARCimage{1.0\linewidth}{167include}}
{\DARCimage{1.0\linewidth}{172include}}
{\textbf{\textcolor{DARCgreen}{\DARCimage{1.0\linewidth}{165include}}}}
\end{question2x2}

}

\end{frame}

\begin{frame}
\frametitle{Tiefpass}
\begin{columns}
    \begin{column}{0.48\textwidth}
    \begin{itemize}
  \item Bei niedrigen Frequenzen hat der Kondensator einen sehr hohen Widerstand
  \item Schaltung wirkt wie ein frequenzabhängiger Spannungsteiler
  \item U<sub>A</sub> ist dadurch sehr groß
  \end{itemize}

    \end{column}
   \begin{column}{0.48\textwidth}
       
\begin{figure}
    \DARCimage{0.85\linewidth}{591include}
    \caption{\scriptsize Filtercharakteristik eines Tiefpass}
    \label{e_hochpass}
\end{figure}


\begin{figure}
    \DARCimage{0.85\linewidth}{175include}
    \caption{\scriptsize Tiefpass aus Kondensator und Widerstand}
    \label{e_tiefpass_rc}
\end{figure}


   \end{column}
\end{columns}

\end{frame}

\begin{frame}
\only<1>{
\begin{PQuestion}{ED201}{Wie wird die dargestellte Filtercharakteristik bezeichnet?}{Bandsperre}
{Hochpass}
{Bandpass}
{Tiefpass}
{\DARCimage{0.75\linewidth}{591include}}\end{PQuestion}

}
\only<2>{
\begin{PQuestion}{ED201}{Wie wird die dargestellte Filtercharakteristik bezeichnet?}{Bandsperre}
{Hochpass}
{Bandpass}
{\textbf{\textcolor{DARCgreen}{Tiefpass}}}
{\DARCimage{0.75\linewidth}{591include}}\end{PQuestion}

}
\end{frame}

\begin{frame}
\only<1>{
\begin{PQuestion}{ED208}{Was stellt die folgende Schaltung dar? }{Bandpass}
{Sperrkreis}
{Tiefpass}
{Hochpass}
{\DARCimage{1.0\linewidth}{175include}}\end{PQuestion}

}
\only<2>{
\begin{PQuestion}{ED208}{Was stellt die folgende Schaltung dar? }{Bandpass}
{Sperrkreis}
{\textbf{\textcolor{DARCgreen}{Tiefpass}}}
{Hochpass}
{\DARCimage{1.0\linewidth}{175include}}\end{PQuestion}

}
\end{frame}

\begin{frame}
\only<1>{
\begin{PQuestion}{ED209}{Was stellt die folgende Schaltung dar? }{Tiefpass}
{Sperrkreis}
{Bandpass}
{Hochpass}
{\DARCimage{1.0\linewidth}{756include}}\end{PQuestion}

}
\only<2>{
\begin{PQuestion}{ED209}{Was stellt die folgende Schaltung dar? }{\textbf{\textcolor{DARCgreen}{Tiefpass}}}
{Sperrkreis}
{Bandpass}
{Hochpass}
{\DARCimage{1.0\linewidth}{756include}}\end{PQuestion}

}

\end{frame}

\begin{frame}
\only<1>{
\begin{question2x2}{ED210}{Welche Schaltung könnte für die Tiefpassfilterung in einem Mikrofonverstärker eingesetzt werden?}{\DARCimage{1.0\linewidth}{170include}}
{\DARCimage{1.0\linewidth}{173include}}
{\DARCimage{1.0\linewidth}{161include}}
{\DARCimage{1.0\linewidth}{169include}}
\end{question2x2}

}
\only<2>{
\begin{question2x2}{ED210}{Welche Schaltung könnte für die Tiefpassfilterung in einem Mikrofonverstärker eingesetzt werden?}{\textbf{\textcolor{DARCgreen}{\DARCimage{1.0\linewidth}{170include}}}}
{\DARCimage{1.0\linewidth}{173include}}
{\DARCimage{1.0\linewidth}{161include}}
{\DARCimage{1.0\linewidth}{169include}}
\end{question2x2}

}

\end{frame}

\begin{frame}
\frametitle{Serienschwingkreis}
\begin{columns}
    \begin{column}{0.48\textwidth}
    \begin{itemize}
  \item Es gibt eine Resonanzfrequenz, bei der der Wechselstromwiderstand (Impedanz) sehr gering ist
  \item Bandpass, Saugkreis (eine Frequenz wird rausgesaugt)
  \end{itemize}

    \end{column}
   \begin{column}{0.48\textwidth}
       
\begin{figure}
    \DARCimage{0.85\linewidth}{189include}
    \caption{\scriptsize Impedanzverlauf eines Serienschwingkreis}
    \label{e_serienschwingkreis_z}
\end{figure}


\begin{figure}
    \DARCimage{0.85\linewidth}{757include}
    \caption{\scriptsize Serienschwingkreis aus Kondensator und Spule}
    \label{e_serienschwingkreis_cl}
\end{figure}


   \end{column}
\end{columns}

\end{frame}%ENDCONTENT
