
\section{Baken}
\label{section:baken}
\begin{frame}%STARTCONTENT

\begin{columns}
    \begin{column}{0.48\textwidth}
    \begin{itemize}
  \item Automatisch arbeitende Amateurfunk-Sendeanlage
  \item Ständig wiederkeherende Aussendungen
  \item Zu Feldstärkebeobachtungen oder Empfangsversuchen
  \item Kann auch in Satelliten sein
  \end{itemize}

    \end{column}
   \begin{column}{0.48\textwidth}
       \begin{itemize}
  \item Fest zugewiesene Frequenz
  \item Fester Standort
  \item Rufzeichen in regelmäßigen Abständen
  \item Meist in Morsetelegrafie
  \end{itemize}

   \end{column}
\end{columns}

\end{frame}

\begin{frame}
\only<1>{
\begin{QQuestion}{VD119}{Wie ist der Begriff \glqq Funkbake\grqq{} nach dem Wortlaut der Amateurfunkverordnung (AFuV) definiert?}{Eine \glqq Funkbake\grqq{} ist eine automatisch arbeitende Amateurfunk-Sendeanlage (auch in Satelliten), die selbsttätig ständig wiederkehrende Aussendungen zur Feldstärkebeobachtung oder zu Empfangsversuchen erzeugt.}
{Eine \glqq Funkbake\grqq{} ist eine automatisch arbeitende Amateurfunk-Sendeanlage (auch in Satelliten), die selbsttätig ständig wiederkehrende Aussendungen zur Positionsbestimmung auf hoher See erzeugt.}
{Eine \glqq Funkbake\grqq{} ist eine  Amateurfunk-Sendeanlage, die ständig wiederkehrende Aussendungen zur Positionsbestimmung in Not- und Katastrophenfällen erzeugt.}
{Eine \glqq Funkbake\grqq{} ist eine  Amateurfunk-Sendeanlage, die ständig wiederkehrende Signale zur Identifikation der Kurzwellen-Bandgrenzen aussendet.}
\end{QQuestion}

}
\only<2>{
\begin{QQuestion}{VD119}{Wie ist der Begriff \glqq Funkbake\grqq{} nach dem Wortlaut der Amateurfunkverordnung (AFuV) definiert?}{\textbf{\textcolor{DARCgreen}{Eine \glqq Funkbake\grqq{} ist eine automatisch arbeitende Amateurfunk-Sendeanlage (auch in Satelliten), die selbsttätig ständig wiederkehrende Aussendungen zur Feldstärkebeobachtung oder zu Empfangsversuchen erzeugt.}}}
{Eine \glqq Funkbake\grqq{} ist eine automatisch arbeitende Amateurfunk-Sendeanlage (auch in Satelliten), die selbsttätig ständig wiederkehrende Aussendungen zur Positionsbestimmung auf hoher See erzeugt.}
{Eine \glqq Funkbake\grqq{} ist eine  Amateurfunk-Sendeanlage, die ständig wiederkehrende Aussendungen zur Positionsbestimmung in Not- und Katastrophenfällen erzeugt.}
{Eine \glqq Funkbake\grqq{} ist eine  Amateurfunk-Sendeanlage, die ständig wiederkehrende Signale zur Identifikation der Kurzwellen-Bandgrenzen aussendet.}
\end{QQuestion}

}
\end{frame}

\begin{frame}
\frametitle{Nutzung von Baken}
\begin{itemize}
  \item Empfangbarkeit abhängig von wechselnden Ausbreitungsbedingungen
  \item Indikator für die Machbarkeit einer Funkverbindung
  \item Reflexion an Polarlichtern im VHF-Band durch \enquote{Aurora-Baken} testen
  \item Durch Peilung Antennenausrichtung überprüfen
  \end{itemize}

\end{frame}

\begin{frame}
\only<1>{
\begin{QQuestion}{BE409}{Was ist eine häufige Anwendung von Amateurfunkbaken? Sie~...}{helfen bei der Beobachtung der Ausbreitungsbedingungen.}
{reservieren Frequenzen für einen Funkamateur.}
{stellen Empfangsberichte in das Internet ein.}
{ionisieren die D-Region der Atmosphäre.}
\end{QQuestion}

}
\only<2>{
\begin{QQuestion}{BE409}{Was ist eine häufige Anwendung von Amateurfunkbaken? Sie~...}{\textbf{\textcolor{DARCgreen}{helfen bei der Beobachtung der Ausbreitungsbedingungen.}}}
{reservieren Frequenzen für einen Funkamateur.}
{stellen Empfangsberichte in das Internet ein.}
{ionisieren die D-Region der Atmosphäre.}
\end{QQuestion}

}
\end{frame}

\begin{frame}
\frametitle{Internationales Bakenprojekt (IBP)}
\begin{columns}
    \begin{column}{0.48\textwidth}
    \begin{itemize}
  \item Größere Anzahl Baken auf allen Kontinenten verteilt
  \item Senden in einem festgelegten zeitlichen Ablauf nacheinander aus
  \item Alle auf der gleichen Frequenz
  \end{itemize}

    \end{column}
   \begin{column}{0.48\textwidth}
       Spezielle Frequenzbereiche im IARU-Bandplan

\begin{table}
\begin{DARCtabular}{lX}
     Band  & Frequenzbereich   \\
     \qty{10}{\metre}  & \qtyrange{28190}{28225}{\kilo\hertz}   \\
     \qty{12}{\metre}  & \qtyrange{24929}{24931}{\kilo\hertz}   \\
     \qty{15}{\metre}  & \qtyrange{21149}{21151}{\kilo\hertz}   \\
     \qty{17}{\metre}  & \qtyrange{18109}{18111}{\kilo\hertz}   \\
     \qty{20}{\metre}  & \qtyrange{14099}{14101}{\kilo\hertz}   \\
\end{DARCtabular}
\caption{Frequenzbereiche für Baken gemäß IARU-Bandplan}
\label{n_baken_frequenzbereiche}
\end{table}
    \pause
    Keinen Funkbetrieb dort durchführen!




   \end{column}
\end{columns}

\end{frame}

\begin{frame}
\only<1>{
\begin{QQuestion}{BE410}{Weshalb sind die Frequenzen \qtyrange{14099}{14101}{\kHz}, \qtyrange{18109}{18111}{\kHz}, \qtyrange{21149}{21151}{\kHz}, \qtyrange{24929}{24931}{\kHz} und \qtyrange{28190}{28225}{\kHz} freizuhalten?}{Diese Frequenzen sind nach der IARU-Empfehlung besonders für DX-Verkehr vorgesehen und sollen möglichst für Funkverkehr bei \glqq DX-Expeditionen\grqq{} genutzt werden.}
{Diese Frequenzbereiche sind nach der IARU-Empfehlung für HAMNET vorgesehen und sollen für die Beobachtung dieser Sendungen freigehalten werden.}
{Diese Frequenzen sind nach der IARU-Empfehlung für das Internationale Bakenprojekt (IBP) vorgesehen und sind für die Beobachtung der Ausbreitungsbedingungen anhand von Bakensignalen freizuhalten.}
{Diese Frequenzbereiche sind nach Empfehlung der Radio Regulations (VO Funk) für besondere Amateurfunk-Zeitzeichen- und Normalfrequenzaussendungen vorgesehen und sollen möglichst freigehalten werden.}
\end{QQuestion}

}
\only<2>{
\begin{QQuestion}{BE410}{Weshalb sind die Frequenzen \qtyrange{14099}{14101}{\kHz}, \qtyrange{18109}{18111}{\kHz}, \qtyrange{21149}{21151}{\kHz}, \qtyrange{24929}{24931}{\kHz} und \qtyrange{28190}{28225}{\kHz} freizuhalten?}{Diese Frequenzen sind nach der IARU-Empfehlung besonders für DX-Verkehr vorgesehen und sollen möglichst für Funkverkehr bei \glqq DX-Expeditionen\grqq{} genutzt werden.}
{Diese Frequenzbereiche sind nach der IARU-Empfehlung für HAMNET vorgesehen und sollen für die Beobachtung dieser Sendungen freigehalten werden.}
{\textbf{\textcolor{DARCgreen}{Diese Frequenzen sind nach der IARU-Empfehlung für das Internationale Bakenprojekt (IBP) vorgesehen und sind für die Beobachtung der Ausbreitungsbedingungen anhand von Bakensignalen freizuhalten.}}}
{Diese Frequenzbereiche sind nach Empfehlung der Radio Regulations (VO Funk) für besondere Amateurfunk-Zeitzeichen- und Normalfrequenzaussendungen vorgesehen und sollen möglichst freigehalten werden.}
\end{QQuestion}

}
\end{frame}%ENDCONTENT
