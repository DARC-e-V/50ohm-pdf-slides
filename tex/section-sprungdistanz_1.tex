
\section{Sprungdistanz I}
\label{section:sprungdistanz_1}
\begin{frame}%STARTCONTENT

\begin{columns}
    \begin{column}{0.48\textwidth}
    \begin{itemize}
  \item Je flacher meine Antenne im Winkel zur Erdoberfläche abstrahlt, umso weiter ist die Sprungdistanz
  \item Je steiler meine Antenne nach oben strahlt, umso kürzer ist die Sprungdistanz
  \end{itemize}

    \end{column}
   \begin{column}{0.48\textwidth}
       
\begin{figure}
    \DARCimage{0.85\linewidth}{732include}
    \caption{\scriptsize Ausbreitung von Raum- und Bodenwelle}
    \label{e_raum_und_bodenwelle}
\end{figure}


   \end{column}
\end{columns}

\end{frame}

\begin{frame}
\only<1>{
\begin{QQuestion}{EH208}{Von welchem der genannten Parameter ist die Sprungdistanz abhängig, die ein KW-Signal auf der Erdoberfläche überbrücken kann? Sie ist abhängig~...}{von der Sendeleistung.}
{von der Polarisation der Antenne.}
{vom Abstrahlwinkel der Antenne.}
{vom Antennengewinn.}
\end{QQuestion}

}
\only<2>{
\begin{QQuestion}{EH208}{Von welchem der genannten Parameter ist die Sprungdistanz abhängig, die ein KW-Signal auf der Erdoberfläche überbrücken kann? Sie ist abhängig~...}{von der Sendeleistung.}
{von der Polarisation der Antenne.}
{\textbf{\textcolor{DARCgreen}{vom Abstrahlwinkel der Antenne.}}}
{vom Antennengewinn.}
\end{QQuestion}

}
\end{frame}%ENDCONTENT
