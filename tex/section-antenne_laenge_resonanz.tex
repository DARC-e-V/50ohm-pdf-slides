
\section{Antennenlänge und -resonanz}
\label{section:antenne_laenge_resonanz}
\begin{frame}%STARTCONTENT
\begin{itemize}
  \item Die Drähte einer Antennen können eine beliebige Länge oder Form haben
  \item Jedoch ist dann deren Wellenwiderstand anders
  \item Dieser Wellenwiderstand muss an die Speiseleitung angepasst werden, z.B. durch einen Balun
  \end{itemize}
\end{frame}

\begin{frame}
\only<1>{
\begin{QQuestion}{EG102}{Eine Drahtantenne für den Amateurfunk im KW-Bereich~...}{kann grundsätzlich eine beliebige Länge haben.}
{muss unbedingt $\lambda/2$ lang sein.}
{muss genau $\lambda/4$ lang sein.}
{muss eine Länge von $3/4~\lambda$  haben.}
\end{QQuestion}

}
\only<2>{
\begin{QQuestion}{EG102}{Eine Drahtantenne für den Amateurfunk im KW-Bereich~...}{\textbf{\textcolor{DARCgreen}{kann grundsätzlich eine beliebige Länge haben.}}}
{muss unbedingt $\lambda/2$ lang sein.}
{muss genau $\lambda/4$ lang sein.}
{muss eine Länge von $3/4~\lambda$  haben.}
\end{QQuestion}

}
\end{frame}

\begin{frame}
\only<1>{
\begin{QQuestion}{EG109}{Berechnen Sie die elektrische Länge eines 5/8 $\lambda$ langen Vertikalstrahlers für das \qty{10}{\m}-Band (\qty{28,5}{\MHz}).}{\qty{2,08}{\m}}
{\qty{3,29}{\m}}
{\qty{6,58}{\m}}
{\qty{5,26}{\m}}
\end{QQuestion}

}
\only<2>{
\begin{QQuestion}{EG109}{Berechnen Sie die elektrische Länge eines 5/8 $\lambda$ langen Vertikalstrahlers für das \qty{10}{\m}-Band (\qty{28,5}{\MHz}).}{\qty{2,08}{\m}}
{\qty{3,29}{\m}}
{\textbf{\textcolor{DARCgreen}{\qty{6,58}{\m}}}}
{\qty{5,26}{\m}}
\end{QQuestion}

}
\end{frame}

\begin{frame}
\frametitle{Lösungsweg}
Anstatt direkt die ungefähre Wellenlänge des 10m-Bands zu verwenden, wird hier erst die angegebene Frequenz in die exakte Wellenlänge umgerechnet.

\begin{equation} \begin{split} l &= \frac{5}{8}\lambda\\ &= \frac{5}{8} \cdot \dfrac{300}{28,5MHz}\\ &= \frac{5}{8} \cdot 10,53m\\ &= 6,58m\\ \end{split} \end{equation}

\end{frame}

\begin{frame}
\frametitle{Faltdipol}
\begin{columns}
    \begin{column}{0.48\textwidth}
    \begin{itemize}
  \item Ein Draht einer Wellenlänge wird an den Enden zur Länge eines Halbwellen-Dipols umgebogen
  \item Die Einspeisung ist immer noch in der Mitte
  \end{itemize}

    \end{column}
   \begin{column}{0.48\textwidth}
       
\begin{figure}
    \DARCimage{0.85\linewidth}{531include}
    \caption{\scriptsize Die Elemente einer Yagi-Uda-Antenne mit einem Faltdipol als Strahler bei der Nummer 2}
    \label{e_antenne_laenge_resonanz}
\end{figure}


   \end{column}
\end{columns}

\end{frame}

\begin{frame}
\only<1>{
\begin{QQuestion}{EG110}{Die Länge des Drahtes zur Herstellung eines Faltdipols entspricht~...}{einer Wellenlänge.}
{einer Halbwellenlänge.}
{zwei Wellenlängen.}
{vier Wellenlängen.}
\end{QQuestion}

}
\only<2>{
\begin{QQuestion}{EG110}{Die Länge des Drahtes zur Herstellung eines Faltdipols entspricht~...}{\textbf{\textcolor{DARCgreen}{einer Wellenlänge.}}}
{einer Halbwellenlänge.}
{zwei Wellenlängen.}
{vier Wellenlängen.}
\end{QQuestion}

}
\end{frame}%ENDCONTENT
