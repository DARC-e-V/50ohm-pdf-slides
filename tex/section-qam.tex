
\section{Quadraturamplitudenmodulation  (QAM)}
\label{section:qam}
\begin{frame}%STARTCONTENT
\begin{itemize}
  \item Es scheint zunächst nahe zu liegen, die Anzahl der Symbole möglichst groß zu wählen, damit pro Symbol möglichst viele Informationen übertragen werden können.
  \item Doch dann muss ein Empfänger z.B. zwischen vielen unterschiedlichen Amplituden unterscheiden können. Somit wird das Verfahren anfälliger für Störungen.
  \end{itemize}
\end{frame}

\begin{frame}\begin{itemize}
  \item Trick: Anstelle der Änderung nur eines Parameters (z.B. der Amplitude) werden pro Symbol zwei Parameter verändert, nämlich die Amplitude und die Phase.
  \item Ein Symbol entspricht dann einer Kombination einer bestimmten Amplitude mit einer bestimmten Phasenlage.
  \end{itemize}

\begin{figure}
    \DARCimage{0.85\linewidth}{702include}
    \caption{\scriptsize Signalverlauf eines 8QAM-Signals, je Symbol mit Amplitude (0,5 bzw. 1), Phasenlage und 3-stelliger Bitfolge}
    \label{8qam}
\end{figure}

\end{frame}

\begin{frame}
\only<1>{
\begin{QQuestion}{AE403}{Wie werden Informationen bei der Quadraturamplitudenmodulation (QAM) mittels eines Trägers übertragen? Durch~...}{richtungsabhängige Änderung der Frequenz}
{nichtlineare Änderung der Amplitude}
{separate Änderung des elektrischen und magnetischen Feldwellenanteils}
{Änderung der Amplitude und der Phase}
\end{QQuestion}

}
\only<2>{
\begin{QQuestion}{AE403}{Wie werden Informationen bei der Quadraturamplitudenmodulation (QAM) mittels eines Trägers übertragen? Durch~...}{richtungsabhängige Änderung der Frequenz}
{nichtlineare Änderung der Amplitude}
{separate Änderung des elektrischen und magnetischen Feldwellenanteils}
{\textbf{\textcolor{DARCgreen}{Änderung der Amplitude und der Phase}}}
\end{QQuestion}

}
\end{frame}%ENDCONTENT
