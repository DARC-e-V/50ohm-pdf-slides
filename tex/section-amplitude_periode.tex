
\section{Amplitude und Periode}
\label{section:amplitude_periode}
\begin{frame}%STARTCONTENT

\frametitle{Amplitude}
\begin{columns}
    \begin{column}{0.48\textwidth}
    
\begin{figure}
    \DARCimage{0.85\linewidth}{726include}
    \caption{\scriptsize  Amplitude einer Sinusschwingung}
    \label{amplitude_periode_amplitudee}
\end{figure}


    \end{column}
   \begin{column}{0.48\textwidth}
       Der maximale Abstand von der Nulllinie zum höchsten oder tiefsten Punkt heißt \emph{Amplitude}


   \end{column}
\end{columns}

\end{frame}

\begin{frame}
\only<1>{
\begin{PQuestion}{NB404}{Was ist im Oszillogramm mit 1 markiert?}{Periode}
{Frequenz}
{Amplitude}
{Wellenlänge}
{\DARCimage{1.0\linewidth}{627include}}\end{PQuestion}

}
\only<2>{
\begin{PQuestion}{NB404}{Was ist im Oszillogramm mit 1 markiert?}{Periode}
{Frequenz}
{\textbf{\textcolor{DARCgreen}{Amplitude}}}
{Wellenlänge}
{\DARCimage{1.0\linewidth}{627include}}\end{PQuestion}

}
\end{frame}

\begin{frame}
\frametitle{Halbwellen}
\begin{columns}
    \begin{column}{0.48\textwidth}
    
\begin{figure}
    \DARCimage{0.85\linewidth}{727include}
    \caption{\scriptsize Positive und negative Halbwellen einer Sinusschwingung}
    \label{amplitude_periode_halbwellen}
\end{figure}


    \end{column}
   \begin{column}{0.48\textwidth}
       Bei einer Sinusschwingung gibt es positive und negative \emph{Halbwellen}


   \end{column}
\end{columns}

\end{frame}

\begin{frame}
\frametitle{Periode}
\begin{columns}
    \begin{column}{0.48\textwidth}
    
\begin{figure}
    \DARCimage{0.85\linewidth}{728include}
    \caption{\scriptsize Perioden einer Sinusschwingung}
    \label{amplitude_periode_perioden}
\end{figure}


    \end{column}
   \begin{column}{0.48\textwidth}
       Die Zeit ($t$) vom Beginn einer positiven Halbwelle bis zum Ende der darauf folgenden negativen Halbwelle heißt \emph{Periode} oder \emph{Periodendauer}


   \end{column}
\end{columns}

\end{frame}

\begin{frame}
\frametitle{Interaktiv}

\end{frame}

\begin{frame}
\only<1>{
\begin{PQuestion}{NB405}{Was ist im Oszillogramm mit 2 markiert?}{Spannung}
{Amplitude}
{Strom}
{Periode}
{\DARCimage{1.0\linewidth}{627include}}\end{PQuestion}

}
\only<2>{
\begin{PQuestion}{NB405}{Was ist im Oszillogramm mit 2 markiert?}{Spannung}
{Amplitude}
{Strom}
{\textbf{\textcolor{DARCgreen}{Periode}}}
{\DARCimage{1.0\linewidth}{627include}}\end{PQuestion}

}
 \end{frame}

\begin{frame}
\only<1>{
\begin{QQuestion}{NA213}{Welche Aussage ist für eine Schwingung von \num{145000000} Perioden pro Sekunde richtig?}{Ihre Ausbreitungsgeschwindigkeit beträgt \qty{145}{\km}/s.}
{Ihre Periodendauer beträgt \qty{145}{\us}.}
{Ihre Amplitude beträgt \qty{145}{\pps}.}
{Ihre Frequenz beträgt \qty{145}{\MHz}.}
\end{QQuestion}

}
\only<2>{
\begin{QQuestion}{NA213}{Welche Aussage ist für eine Schwingung von \num{145000000} Perioden pro Sekunde richtig?}{Ihre Ausbreitungsgeschwindigkeit beträgt \qty{145}{\km}/s.}
{Ihre Periodendauer beträgt \qty{145}{\us}.}
{Ihre Amplitude beträgt \qty{145}{\pps}.}
{\textbf{\textcolor{DARCgreen}{Ihre Frequenz beträgt \qty{145}{\MHz}.}}}
\end{QQuestion}

}

\end{frame}%ENDCONTENT
