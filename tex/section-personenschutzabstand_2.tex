
\section{Personenschutzabstand II}
\label{section:personenschutzabstand_2}
\begin{frame}%STARTCONTENT

\frametitle{Personenschutzgrenzwerte}
\begin{itemize}
  \item Müssen ab einer EIRP von \qty{10}{\watt} nachgewiesen werden
  \item Trotz kleiner Leistung kann es einen hohen Antennengewinn geben
  \item Dann besteht eine Pflicht zur Nachweisführung
  \end{itemize}
\end{frame}

\begin{frame}
\only<1>{
\begin{QQuestion}{EK104}{Muss ein Funkamateur als Betreiber einer ortsfesten Amateurfunkstelle bei FM-Telefonie und einer Sendeleistung von \qty{6}{\W} an einer 15-Element-Yagi-Uda-Antenne mit 13 dBd Gewinn im \qty{2}{\m}-Band die Einhaltung der Personenschutzgrenzwerte nachweisen?}{Ja, er ist in diesem Fall verpflichtet die Einhaltung der Personenschutzgrenzwerte nachzuweisen.}
{Nein, der Schutz von Personen in elektromagnetischen Feldern ist durch den Funkamateur erst bei einer Strahlungsleistung von mehr als \qty{10}{\W} EIRP sicherzustellen.}
{Ja, für ortsfeste Amateurfunkstellen ist die Einhaltung der Personenschutzgrenzwerte in jedem Fall nachzuweisen.}
{Nein, bei FM-Telefonie und Sendezeiten unter 6 Minuten in der Stunde kann der Schutz von Personen in elektromagnetischen Feldern durch den Funkamateur vernachlässigt werden.}
\end{QQuestion}

}
\only<2>{
\begin{QQuestion}{EK104}{Muss ein Funkamateur als Betreiber einer ortsfesten Amateurfunkstelle bei FM-Telefonie und einer Sendeleistung von \qty{6}{\W} an einer 15-Element-Yagi-Uda-Antenne mit 13 dBd Gewinn im \qty{2}{\m}-Band die Einhaltung der Personenschutzgrenzwerte nachweisen?}{\textbf{\textcolor{DARCgreen}{Ja, er ist in diesem Fall verpflichtet die Einhaltung der Personenschutzgrenzwerte nachzuweisen.}}}
{Nein, der Schutz von Personen in elektromagnetischen Feldern ist durch den Funkamateur erst bei einer Strahlungsleistung von mehr als \qty{10}{\W} EIRP sicherzustellen.}
{Ja, für ortsfeste Amateurfunkstellen ist die Einhaltung der Personenschutzgrenzwerte in jedem Fall nachzuweisen.}
{Nein, bei FM-Telefonie und Sendezeiten unter 6 Minuten in der Stunde kann der Schutz von Personen in elektromagnetischen Feldern durch den Funkamateur vernachlässigt werden.}
\end{QQuestion}

}
\end{frame}

\begin{frame}
\frametitle{Sicherheitsabstand}
\begin{itemize}
  \item Bewertungsverfahren nach BEMFV (Verordnung über das Nachweisverfahren zur Begrenzung elektromagnetischer Felder)
  \item Fernfeldberechnung ist für das Fernfeld möglich
  \item Fernfeld bildet sich bei Dipolen in einem Abstand von etwa bei 4λ aus
  \item Bei Berechnung mit der Fernfeldnäherung gilt der Sicherheitsabstand von jedem Punkt der Antenne
  \end{itemize}
\end{frame}

\begin{frame}
\only<1>{
\begin{QQuestion}{EK107}{Sie errechnen einen Sicherheitsabstand für Ihre Antenne. Von welchem Punkt aus muss dieser Sicherheitsabstand eingehalten werden, wenn Sie bei der Berechnung die Fernfeldnäherung verwendet haben? Er muss eingehalten werden~...}{von jedem Punkt der Antenne.}
{vom Einspeisepunkt der Antenne.}
{von der Mitte der Antenne, d.~h. dort, wo sie am Mast befestigt ist.}
{vom untersten Punkt der Antenne.}
\end{QQuestion}

}
\only<2>{
\begin{QQuestion}{EK107}{Sie errechnen einen Sicherheitsabstand für Ihre Antenne. Von welchem Punkt aus muss dieser Sicherheitsabstand eingehalten werden, wenn Sie bei der Berechnung die Fernfeldnäherung verwendet haben? Er muss eingehalten werden~...}{\textbf{\textcolor{DARCgreen}{von jedem Punkt der Antenne.}}}
{vom Einspeisepunkt der Antenne.}
{von der Mitte der Antenne, d.~h. dort, wo sie am Mast befestigt ist.}
{vom untersten Punkt der Antenne.}
\end{QQuestion}

}
\end{frame}%ENDCONTENT
