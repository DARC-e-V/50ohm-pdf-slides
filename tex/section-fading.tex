
\section{Fading}
\label{section:fading}
\begin{frame}%STARTCONTENT

\begin{columns}
    \begin{column}{0.48\textwidth}
    \begin{itemize}
  \item Raumwelle trifft noch im Bereich der Bodenwelle wieder zum Empfänger
  \item Durch Wellenüberlagerung können sich Raum- und Bodenwelle gegenseitig abschwächen
  \item Signal verliert an Stärke $\rightarrow$ \emph{Fading}
  \end{itemize}

    \end{column}
   \begin{column}{0.48\textwidth}
       
   \end{column}
\end{columns}

\end{frame}

\begin{frame}
\only<1>{
\begin{QQuestion}{EH203}{Wie nennt man den Feldstärkeschwund durch Überlagerung von Boden- und Raumwelle?}{Fading}
{Backscatter}
{MUF}
{Mögel-Dellinger-Effekt}
\end{QQuestion}

}
\only<2>{
\begin{QQuestion}{EH203}{Wie nennt man den Feldstärkeschwund durch Überlagerung von Boden- und Raumwelle?}{\textbf{\textcolor{DARCgreen}{Fading}}}
{Backscatter}
{MUF}
{Mögel-Dellinger-Effekt}
\end{QQuestion}

}
\end{frame}

\begin{frame}
\only<1>{
\begin{QQuestion}{EH202}{Was kann durch das Zusammenwirken von Raum- und Bodenwelle verursacht werden?}{Frequenzverschiebung (Doppler-Effekt)}
{Feldstärkeschwankungen (Fading)}
{Rückstreuung (Backscatter)}
{Rauschen (Noise)}
\end{QQuestion}

}
\only<2>{
\begin{QQuestion}{EH202}{Was kann durch das Zusammenwirken von Raum- und Bodenwelle verursacht werden?}{Frequenzverschiebung (Doppler-Effekt)}
{\textbf{\textcolor{DARCgreen}{Feldstärkeschwankungen (Fading)}}}
{Rückstreuung (Backscatter)}
{Rauschen (Noise)}
\end{QQuestion}

}
\end{frame}%ENDCONTENT
