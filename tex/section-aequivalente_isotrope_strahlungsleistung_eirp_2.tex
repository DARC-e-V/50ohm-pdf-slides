
\section{Äquivalente isotrope Strahlungsleistung (EIRP) II}
\label{section:aequivalente_isotrope_strahlungsleistung_eirp_2}
\begin{frame}%STARTCONTENT
\begin{itemize}
  \item Bei der Berechnung nur die Energie berücksichtigen, die an der Antenne ankommt
  \item Verluste $a$ durch Kabel, Stecker oder andere Bauteile abziehen
  \item Erst dann mit dem Gewinnfaktor multiplizieren
  \item Es folgen diverse allgemeine Formeln für ERP und EIRP
  \end{itemize}
\end{frame}

\begin{frame}
\frametitle{ERP}
Aus Klasse~N bekannt:

$P_{\mathrm{ERP}} = (P_{\mathrm{Sender}} -- P_{\mathrm{Verluste}}) \cdot G_{\mathrm{Antenne}}$
    \pause
    Bei der Rechnung mit dB zu verwenden:

$P_{\mathrm{ERP}} = P_{\mathrm{Sender}} -- a + g_d$
    \pause
    Aus der Formelsammlung mit Umwandlung von dB in Leistungsfaktor:

$P_{\mathrm{ERP}} = P_{\mathrm{Sender}} \cdot 10^{\frac{g_d -- a}{10\mathrm{dB}}}$



\end{frame}

\begin{frame}
\frametitle{EIRP}
Umrechnung ERP zu EIRP:

$P_{\mathrm{EIRP}} = P_{\mathrm{ERP}} + 2,15 \mathrm{dB}$
    \pause
    Aus der Formelsammlung mit Umwandlung von dB in Leistungsfaktor:

$P_{\mathrm{EIRP}} = P_{\mathrm{Sender}} \cdot 10^{\frac{g_d -- a + 2,15\mathrm{dB}}{10\mathrm{dB}}}$
    \pause
    Wenn der Gewinn in dBi angegeben ist:

$P_{\mathrm{EIRP}} = P_{\mathrm{Sender}} \cdot 10^{\frac{g_i -- a}{10\mathrm{dB}}}$



\end{frame}

\begin{frame}
\only<1>{
\begin{QQuestion}{EG501}{Die äquivalente isotrope Strahlungsleistung (EIRP) ist~...}{die durchschnittliche Leistung bei der höchsten Spitze der Modulationshüllkurve, die der Antenne zugeführt wird, und ihrem Gewinn in einer Richtung, bezogen auf den isotropen Strahler.}
{das Produkt aus der Leistung, die unmittelbar der Antenne zugeführt wird, und ihrem Gewinn in einer Richtung, bezogen auf den Dipol.}
{das Produkt aus der Leistung, die unmittelbar der Antenne zugeführt wird, und ihrem Gewinn in einer Richtung, bezogen auf den isotropen Strahler.}
{die durchschnittliche Leistung bei der höchsten Spitze der Modulationshüllkurve, die der Antenne zugeführt wird, und ihrem Gewinn in einer Richtung, bezogen auf den Dipol.}
\end{QQuestion}

}
\only<2>{
\begin{QQuestion}{EG501}{Die äquivalente isotrope Strahlungsleistung (EIRP) ist~...}{die durchschnittliche Leistung bei der höchsten Spitze der Modulationshüllkurve, die der Antenne zugeführt wird, und ihrem Gewinn in einer Richtung, bezogen auf den isotropen Strahler.}
{das Produkt aus der Leistung, die unmittelbar der Antenne zugeführt wird, und ihrem Gewinn in einer Richtung, bezogen auf den Dipol.}
{\textbf{\textcolor{DARCgreen}{das Produkt aus der Leistung, die unmittelbar der Antenne zugeführt wird, und ihrem Gewinn in einer Richtung, bezogen auf den isotropen Strahler.}}}
{die durchschnittliche Leistung bei der höchsten Spitze der Modulationshüllkurve, die der Antenne zugeführt wird, und ihrem Gewinn in einer Richtung, bezogen auf den Dipol.}
\end{QQuestion}

}
\end{frame}

\begin{frame}
\only<1>{
\begin{QQuestion}{EG502}{Nach welcher der Antworten kann die EIRP berechnet werden?}{$P_{\symup{EIRP}} = (P_{\symup{Sender}} - P_{\symup{Verluste}}) \cdot G_{\symup{Antenne}}$, bezogen auf einen isotropen Strahler}
{$P_{\symup{EIRP}} = (P_{\symup{Sender}} \cdot P_{\symup{Verluste}}) \cdot G_{\symup{Antenne}}$, bezogen auf einen Halbwellendipol}
{$P_{\symup{EIRP}} = (P_{\symup{Sender}} - P_{\symup{Verluste}}) + G_{\symup{Antenne}}$, bezogen auf einen isotropen Strahler}
{$P_{\symup{EIRP}} = (P_{\symup{Sender}} - P_{\symup{Verluste}}) + G_{\symup{Antenne}}$, bezogen auf einen Halbwellendipol}
\end{QQuestion}

}
\only<2>{
\begin{QQuestion}{EG502}{Nach welcher der Antworten kann die EIRP berechnet werden?}{\textbf{\textcolor{DARCgreen}{$P_{\symup{EIRP}} = (P_{\symup{Sender}} - P_{\symup{Verluste}}) \cdot G_{\symup{Antenne}}$, bezogen auf einen isotropen Strahler}}}
{$P_{\symup{EIRP}} = (P_{\symup{Sender}} \cdot P_{\symup{Verluste}}) \cdot G_{\symup{Antenne}}$, bezogen auf einen Halbwellendipol}
{$P_{\symup{EIRP}} = (P_{\symup{Sender}} - P_{\symup{Verluste}}) + G_{\symup{Antenne}}$, bezogen auf einen isotropen Strahler}
{$P_{\symup{EIRP}} = (P_{\symup{Sender}} - P_{\symup{Verluste}}) + G_{\symup{Antenne}}$, bezogen auf einen Halbwellendipol}
\end{QQuestion}

}
\end{frame}

\begin{frame}
\frametitle{Ortsfeste Amateurfunkanlage}
Eine ortsfeste Amateurfunkanlage ist nach §~9 BEMFV bei der BNetzA anzuzeigen, wenn eine Strahlungsleistung von \qty{10}{\watt} EIRP überschritten wird.

\end{frame}

\begin{frame}
\only<1>{
\begin{QQuestion}{EG503}{Ein HF-Verstärker für \qty{5,7}{\GHz} speist eine Ausgangsleistung von \qty{250}{\mW} ohne Leitungsverluste direkt in einen Parabolspiegel mit einem Gewinn von \qty{26}{\dBi} ein. Wie hoch ist die äquivalente Strahlungsleistung (EIRP)?}{\qty{100}{\W}}
{\qty{61}{\W}}
{\qty{6,5}{\W}}
{\qty{3,4}{\W}}
\end{QQuestion}

}
\only<2>{
\begin{QQuestion}{EG503}{Ein HF-Verstärker für \qty{5,7}{\GHz} speist eine Ausgangsleistung von \qty{250}{\mW} ohne Leitungsverluste direkt in einen Parabolspiegel mit einem Gewinn von \qty{26}{\dBi} ein. Wie hoch ist die äquivalente Strahlungsleistung (EIRP)?}{\textbf{\textcolor{DARCgreen}{\qty{100}{\W}}}}
{\qty{61}{\W}}
{\qty{6,5}{\W}}
{\qty{3,4}{\W}}
\end{QQuestion}

}
\end{frame}

\begin{frame}
\frametitle{Lösungsweg}
\begin{itemize}
  \item gegeben: $P_{\mathrm{Sender}} = 250mW$
  \item gegeben: $g_i = 26\mathrm{dB}$
  \item gegeben: $a = 0$
  \item gesucht: $P_{\mathrm{EIRP}}$
  \end{itemize}
    \pause
    \begin{equation}\begin{split} \nonumber P_{\mathrm{EIRP}} &= P_{\mathrm{Sender}} \cdot 10^{\frac{g_i -- a}{10\mathrm{dB}}}\\ &= 250mW \cdot 10^{\frac{26\mathrm{dB}}{10\mathrm{dB}}}\\ &= 250mW \cdot 398\\ &\approx 100W \end{split}\end{equation}



\end{frame}

\begin{frame}
\only<1>{
\begin{QQuestion}{EG504}{Ein HF-Verstärker für \qty{10,4}{\GHz} speist eine Ausgangsleistung von \qty{5}{\W} direkt in einen Parabolspiegel mit einem Gewinn von \qty{36}{\dBi} ein. Wie hoch ist die äquivalente Strahlungsleistung (EIRP)?}{\qty{180}{\W}}
{\qty{12195}{\W}}
{\qty{20000}{\W}}
{\qty{110}{\W}}
\end{QQuestion}

}
\only<2>{
\begin{QQuestion}{EG504}{Ein HF-Verstärker für \qty{10,4}{\GHz} speist eine Ausgangsleistung von \qty{5}{\W} direkt in einen Parabolspiegel mit einem Gewinn von \qty{36}{\dBi} ein. Wie hoch ist die äquivalente Strahlungsleistung (EIRP)?}{\qty{180}{\W}}
{\qty{12195}{\W}}
{\textbf{\textcolor{DARCgreen}{\qty{20000}{\W}}}}
{\qty{110}{\W}}
\end{QQuestion}

}

\end{frame}

\begin{frame}
\only<1>{
\begin{QQuestion}{EG511}{Sie möchten für Ihre Sendeanlage keine Anzeige einer ortsfesten Amateurfunkanlage nach \S~9~BEMFV abgeben. Wie hoch darf die Sendeleistung für ihre Vertikalantenne mit \qty{5,15}{\dBi} Gewinn ohne Berücksichtigung der Kabelverluste maximal sein, damit die Strahlungsleistung von \qty{10}{\W} EIRP nicht überschritten wird?}{\qty{3}{\W}}
{\qty{10}{\W}}
{\qty{5}{\W}}
{\qty{2}{\W}}
\end{QQuestion}

}
\only<2>{
\begin{QQuestion}{EG511}{Sie möchten für Ihre Sendeanlage keine Anzeige einer ortsfesten Amateurfunkanlage nach \S~9~BEMFV abgeben. Wie hoch darf die Sendeleistung für ihre Vertikalantenne mit \qty{5,15}{\dBi} Gewinn ohne Berücksichtigung der Kabelverluste maximal sein, damit die Strahlungsleistung von \qty{10}{\W} EIRP nicht überschritten wird?}{\textbf{\textcolor{DARCgreen}{\qty{3}{\W}}}}
{\qty{10}{\W}}
{\qty{5}{\W}}
{\qty{2}{\W}}
\end{QQuestion}

}
\end{frame}

\begin{frame}
\frametitle{Lösungsweg}
\begin{itemize}
  \item gegeben: $P_{\mathrm{EIRP}} = 10W$
  \item gegeben: $g_i = 5,15\mathrm{dB}$
  \item gegeben: $a = 0$
  \item gesucht: $P_{\mathrm{Sender}}$
  \end{itemize}
    \pause
    \begin{equation}\begin{split} \nonumber P_{\mathrm{EIRP}} &= P_{\mathrm{Sender}} \cdot 10^{\frac{g_i -- a}{10\mathrm{dB}}}\\ \Rightarrow P_{\mathrm{Sender}} &= \dfrac{P_{\mathrm{EIRP}}}{10^{\frac{g_i -- a}{10\mathrm{dB}}}}\\ &= \dfrac{10W}{10^{\frac{5,15\mathrm{dB}}{10\mathrm{dB}}}}\\ &\approx \frac{10W}{3,27} \approx 3W \end{split}\end{equation}

\end{frame}

\begin{frame}
\only<1>{
\begin{QQuestion}{EG505}{An einen Sender mit \qty{100}{\W} Ausgangsleistung ist eine Antenne mit einem Gewinn von \qty{11}{\dBi} angeschlossen. Die Dämpfung des Kabels beträgt \qty{1}{\decibel}. Wie hoch ist die äquivalente Strahlungsleistung (EIRP)?}{\qty{164}{\W}}
{\qty{1000}{\W}}
{\qty{100}{\W}}
{\qty{1640}{\W}}
\end{QQuestion}

}
\only<2>{
\begin{QQuestion}{EG505}{An einen Sender mit \qty{100}{\W} Ausgangsleistung ist eine Antenne mit einem Gewinn von \qty{11}{\dBi} angeschlossen. Die Dämpfung des Kabels beträgt \qty{1}{\decibel}. Wie hoch ist die äquivalente Strahlungsleistung (EIRP)?}{\qty{164}{\W}}
{\textbf{\textcolor{DARCgreen}{\qty{1000}{\W}}}}
{\qty{100}{\W}}
{\qty{1640}{\W}}
\end{QQuestion}

}

\end{frame}

\begin{frame}
\only<1>{
\begin{QQuestion}{EG507}{An einen Sender mit \qty{100}{\W} Ausgangsleistung ist eine Dipol-Antenne angeschlossen. Die Dämpfung des Kabels beträgt \qty{10}{\decibel}. Wie hoch ist die äquivalente isotrope Strahlungsleistung (EIRP)?}{\qty{16,4}{\W}}
{\qty{90}{\W}}
{\qty{164}{\W}}
{\qty{10}{\W}}
\end{QQuestion}

}
\only<2>{
\begin{QQuestion}{EG507}{An einen Sender mit \qty{100}{\W} Ausgangsleistung ist eine Dipol-Antenne angeschlossen. Die Dämpfung des Kabels beträgt \qty{10}{\decibel}. Wie hoch ist die äquivalente isotrope Strahlungsleistung (EIRP)?}{\textbf{\textcolor{DARCgreen}{\qty{16,4}{\W}}}}
{\qty{90}{\W}}
{\qty{164}{\W}}
{\qty{10}{\W}}
\end{QQuestion}

}

\end{frame}

\begin{frame}
\only<1>{
\begin{QQuestion}{EG506}{Ein Sender mit \qty{75}{\W} Ausgangsleistung ist über eine Antennenleitung, die \qty{2,15}{\decibel} (Faktor $1,64$) Kabelverluste hat, an eine Dipol-Antenne angeschlossen. Welche EIRP wird von der Antenne maximal abgestrahlt?}{\qty{75}{\W}}
{\qty{123}{\W}}
{\qty{45,7}{\W}}
{\qty{60,6}{\W}}
\end{QQuestion}

}
\only<2>{
\begin{QQuestion}{EG506}{Ein Sender mit \qty{75}{\W} Ausgangsleistung ist über eine Antennenleitung, die \qty{2,15}{\decibel} (Faktor $1,64$) Kabelverluste hat, an eine Dipol-Antenne angeschlossen. Welche EIRP wird von der Antenne maximal abgestrahlt?}{\textbf{\textcolor{DARCgreen}{\qty{75}{\W}}}}
{\qty{123}{\W}}
{\qty{45,7}{\W}}
{\qty{60,6}{\W}}
\end{QQuestion}

}

\end{frame}

\begin{frame}\begin{itemize}
  \item Ist der Antennengewinn bezogen auf den Dipol angegeben, müssen wir den Gewinn des Dipols noch zusätzlich berücksichtigen, wenn nach der EIRP gefragt ist.
  \end{itemize}
\end{frame}

\begin{frame}
\only<1>{
\begin{QQuestion}{EG508}{Ein Sender mit \qty{5}{\W} Ausgangsleistung ist über eine Antennenleitung, die \qty{2}{\decibel} Kabelverluste hat, an eine Richtantenne mit \qty{5}{\decibel} Gewinn (auf den Dipol bezogen) angeschlossen. Welche EIRP wird von der Antenne abgestrahlt?}{\qty{9,98}{\W}}
{\qty{8,2}{\W}}
{\qty{41,2}{\W}}
{\qty{16,4}{\W}}
\end{QQuestion}

}
\only<2>{
\begin{QQuestion}{EG508}{Ein Sender mit \qty{5}{\W} Ausgangsleistung ist über eine Antennenleitung, die \qty{2}{\decibel} Kabelverluste hat, an eine Richtantenne mit \qty{5}{\decibel} Gewinn (auf den Dipol bezogen) angeschlossen. Welche EIRP wird von der Antenne abgestrahlt?}{\qty{9,98}{\W}}
{\qty{8,2}{\W}}
{\qty{41,2}{\W}}
{\textbf{\textcolor{DARCgreen}{\qty{16,4}{\W}}}}
\end{QQuestion}

}
\end{frame}

\begin{frame}
\frametitle{Lösungsweg}
\begin{itemize}
  \item gegeben: $P_{\mathrm{Sender}} = 5W$
  \item gegeben: $g_d = 5\mathrm{dB}$
  \item gegeben: $a = 2\mathrm{dB}$
  \item gesucht: $P_{\mathrm{EIRP}}$
  \end{itemize}
    \pause
    \begin{equation}\begin{split} \nonumber P_{\mathrm{EIRP}} &= P_{\mathrm{Sender}} \cdot 10^{\frac{g_d -- a + 2,15\mathrm{dB}}{10\mathrm{dB}}}\\ &= 5W \cdot 10^{\frac{5\mathrm{dB} -- 2\mathrm{dB} + 2,15\mathrm{dB}}{10\mathrm{dB}}}\\ &= 5W \cdot 3,27\\ &\approx 16,4W \end{split}\end{equation}



\end{frame}

\begin{frame}
\only<1>{
\begin{QQuestion}{EG509}{Ein Sender mit \qty{0,6}{\W} Ausgangsleistung ist über eine Antennenleitung, die \qty{1}{\decibel} Kabelverluste hat, an eine Richtantenne mit \qty{11}{\decibel}~Gewinn (auf Dipol bezogen) angeschlossen. Welche EIRP wird von der Antenne maximal abgestrahlt?}{\qty{12,7}{\W}}
{\qty{6,0}{\W}}
{\qty{7,8}{\W}}
{\qty{9,8}{\W}}
\end{QQuestion}

}
\only<2>{
\begin{QQuestion}{EG509}{Ein Sender mit \qty{0,6}{\W} Ausgangsleistung ist über eine Antennenleitung, die \qty{1}{\decibel} Kabelverluste hat, an eine Richtantenne mit \qty{11}{\decibel}~Gewinn (auf Dipol bezogen) angeschlossen. Welche EIRP wird von der Antenne maximal abgestrahlt?}{\qty{12,7}{\W}}
{\qty{6,0}{\W}}
{\qty{7,8}{\W}}
{\textbf{\textcolor{DARCgreen}{\qty{9,8}{\W}}}}
\end{QQuestion}

}

\end{frame}

\begin{frame}
\only<1>{
\begin{QQuestion}{EG510}{Ein Sender mit \qty{8,5}{\W} Ausgangsleistung ist über eine Antennenleitung, die \qty{1,5}{\decibel} Kabelverluste hat, an eine Antenne mit \qty{0}{\decibel} Gewinn (auf den Dipol bezogen) angeschlossen. Welche EIRP wird von der Antenne abgestrahlt?}{\qty{9,9}{\W}}
{\qty{19,7}{\W}}
{\qty{12,0}{\W}}
{\qty{6,0}{\W}}
\end{QQuestion}

}
\only<2>{
\begin{QQuestion}{EG510}{Ein Sender mit \qty{8,5}{\W} Ausgangsleistung ist über eine Antennenleitung, die \qty{1,5}{\decibel} Kabelverluste hat, an eine Antenne mit \qty{0}{\decibel} Gewinn (auf den Dipol bezogen) angeschlossen. Welche EIRP wird von der Antenne abgestrahlt?}{\textbf{\textcolor{DARCgreen}{\qty{9,9}{\W}}}}
{\qty{19,7}{\W}}
{\qty{12,0}{\W}}
{\qty{6,0}{\W}}
\end{QQuestion}

}

\end{frame}%ENDCONTENT
