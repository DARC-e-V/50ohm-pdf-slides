
\section{MUF und LUF}
\label{section:muf_luf_1}
\begin{frame}%STARTCONTENT

\frametitle{Maximal Usable Frequency (MUF)}
\begin{columns}
    \begin{column}{0.48\textwidth}
    \begin{itemize}
  \item Höchste zwischen zwei Orten verwendbare Frequenz
  \item Ist abhängig vom Abstrahlwinkel der Antenne
  \item Und der kritischen Frequenz der Ionosphäre
  \end{itemize}

    \end{column}
   \begin{column}{0.48\textwidth}
       
\begin{figure}
    \DARCimage{0.85\linewidth}{731include}
    \caption{\scriptsize Für den Amateurfunk relevante Schichten in der Atmosphäre}
    \label{e_atmosphaeren_schichten}
\end{figure}


   \end{column}
\end{columns}

\end{frame}

\begin{frame}
\frametitle{Berechnung der MUF}
$MUF \approx \dfrac{f_c}{sin(\alpha)}$

$\alpha$ ist der Abstrahlwinkel der Antenne zum Boden

$f_c$ ist die kritische Frequenz bei der senkrecht auf die Ionosphäre auftretende Funkstrahlen von den Regionen gebrochen werden $\rightarrow$ bei stärkerer Ionisation einer Region steigt die kritische Frequenz

\end{frame}

\begin{frame}
\only<1>{
\begin{QQuestion}{EH204}{Was bedeutet die \glqq MUF\grqq{} bei der Kurzwellenausbreitung?}{Kritische Grenzfrequenz}
{Niedrigste nutzbare Frequenz}
{Höchste nutzbare Frequenz}
{Mittlere Nutzfrequenz}
\end{QQuestion}

}
\only<2>{
\begin{QQuestion}{EH204}{Was bedeutet die \glqq MUF\grqq{} bei der Kurzwellenausbreitung?}{Kritische Grenzfrequenz}
{Niedrigste nutzbare Frequenz}
{\textbf{\textcolor{DARCgreen}{Höchste nutzbare Frequenz}}}
{Mittlere Nutzfrequenz}
\end{QQuestion}

}
\end{frame}

\begin{frame}
\only<1>{
\begin{QQuestion}{EH207}{Sie führen Funkbetrieb nahe der aktuell höchstmöglichen Frequenz (MUF) durch. Um den Funkbetrieb auf noch höheren Frequenzen fortsetzen zu können, muss die Ionisation der brechenden Region~...}{zunehmen.}
{abnehmen.}
{verschwinden.}
{unverändert bleiben.}
\end{QQuestion}

}
\only<2>{
\begin{QQuestion}{EH207}{Sie führen Funkbetrieb nahe der aktuell höchstmöglichen Frequenz (MUF) durch. Um den Funkbetrieb auf noch höheren Frequenzen fortsetzen zu können, muss die Ionisation der brechenden Region~...}{\textbf{\textcolor{DARCgreen}{zunehmen.}}}
{abnehmen.}
{verschwinden.}
{unverändert bleiben.}
\end{QQuestion}

}
\end{frame}

\begin{frame}
\only<1>{
\begin{QQuestion}{EH206}{Eine stärkere Ionisierung der F2-Region führt zu~...}{einer größeren Durchlässigkeit für die höheren Frequenzen.}
{einer stärkeren Absorption der höheren Frequenzen.}
{einer niedrigeren MUF.}
{einer höheren MUF.}
\end{QQuestion}

}
\only<2>{
\begin{QQuestion}{EH206}{Eine stärkere Ionisierung der F2-Region führt zu~...}{einer größeren Durchlässigkeit für die höheren Frequenzen.}
{einer stärkeren Absorption der höheren Frequenzen.}
{einer niedrigeren MUF.}
{\textbf{\textcolor{DARCgreen}{einer höheren MUF.}}}
\end{QQuestion}

}
\end{frame}

\begin{frame}
\frametitle{Lowest Usable Frequency (LUF)}
\begin{columns}
    \begin{column}{0.48\textwidth}
    \begin{itemize}
  \item Abhängig von der Ionisierung in der D-Schicht
  \item Je weniger Dämpfung in der D-Schicht, umso mehr tiefere Funkwellen können diese Schicht durchdringen und an den höheren Schichten reflektieren
  \end{itemize}

    \end{column}
   \begin{column}{0.48\textwidth}
       
\begin{figure}
    \DARCimage{0.85\linewidth}{731include}
    \caption{\scriptsize Für den Amateurfunk relevante Schichten in der Atmosphäre}
    \label{e_atmosphaeren_schichten}
\end{figure}


   \end{column}
\end{columns}

\end{frame}

\begin{frame}
\only<1>{
\begin{QQuestion}{EH209}{Die niedrigste brauchbare Frequenz (LUF) bei Raumwellenausbreitung zwischen zwei Orten hängt ab~...}{vom Ionisierungsgrad in der E-Region.}
{vom Abstrahlwinkel der Antenne.}
{vom Ionisierungsgrad in der D-Region.}
{von der Polarisation der Antenne.}
\end{QQuestion}

}
\only<2>{
\begin{QQuestion}{EH209}{Die niedrigste brauchbare Frequenz (LUF) bei Raumwellenausbreitung zwischen zwei Orten hängt ab~...}{vom Ionisierungsgrad in der E-Region.}
{vom Abstrahlwinkel der Antenne.}
{\textbf{\textcolor{DARCgreen}{vom Ionisierungsgrad in der D-Region.}}}
{von der Polarisation der Antenne.}
\end{QQuestion}

}
\end{frame}%ENDCONTENT
