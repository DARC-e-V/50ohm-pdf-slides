
\section{Kritische Frequenz}
\label{section:kritische_frequenz}
\begin{frame}%STARTCONTENT

\frametitle{Kritische Frequenz}
Wiederholung:

\begin{itemize}
  \item Bei \qty{90}{\degree} Abstrahlwinkel muss das Signal in der Ionosphäre eine \qty{180}{\degree}-Wendung vollziehen
  \item Kritische Frequenz f<sub>c</sub> bei der das Signal reflektiert wird
  \item MUF liefgt höher als f<sub>c</sub>, da in der Regel nicht senkrecht nach oben gesendet wird
  \end{itemize}

\end{frame}

\begin{frame}\begin{itemize}
  \item Kritische Frequenz ist je nach ionosphärische Region, dem Ort und der Zeit unterschiedlich
  \item Formelzeichen: f<sub>O</sub>
  \item Ergänzt durch die Schicht, z.B. f<sub>O</sub>F2
  \end{itemize}

\end{frame}

\begin{frame}
\only<1>{
\begin{QQuestion}{AH204}{Die kritische Frequenz der F2-Region (foF2) ist die~...}{niedrigste Frequenz, die bei senkrechter Abstrahlung von der F2-Region noch zur Erde zurückgeworfen wird.}
{höchste Frequenz, die bei senkrechter Abstrahlung von der F2-Region noch zur Erde zurückgeworfen wird.}
{höchste Frequenz, die bei waagerechter Abstrahlung von der F2-Region noch zur Erde zurückgeworfen wird.}
{niedrigste Frequenz, die bei waagerechter Abstrahlung von der F2-Region noch zur Erde zurückgeworfen wird.}
\end{QQuestion}

}
\only<2>{
\begin{QQuestion}{AH204}{Die kritische Frequenz der F2-Region (foF2) ist die~...}{niedrigste Frequenz, die bei senkrechter Abstrahlung von der F2-Region noch zur Erde zurückgeworfen wird.}
{\textbf{\textcolor{DARCgreen}{höchste Frequenz, die bei senkrechter Abstrahlung von der F2-Region noch zur Erde zurückgeworfen wird.}}}
{höchste Frequenz, die bei waagerechter Abstrahlung von der F2-Region noch zur Erde zurückgeworfen wird.}
{niedrigste Frequenz, die bei waagerechter Abstrahlung von der F2-Region noch zur Erde zurückgeworfen wird.}
\end{QQuestion}

}
\end{frame}

\begin{frame}
\only<1>{
\begin{QQuestion}{AH205}{Angenommen, die kritische Frequenz der F2-Region (foF2) liegt bei \qty{12}{\MHz}. Welche Aussage ist dann richtig? Bei Einstrahlung in die Ionosphäre unter einem Winkel von~...}{\qty{45}{\degree} liegt die niedrigste noch zur Erde zurückgeworfene Signalfrequenz bei \qty{12}{\MHz}.}
{\qty{90}{\degree} liegt die niedrigste noch zur Erde zurückgeworfene Signalfrequenz bei \qty{12}{\MHz}.}
{\qty{45}{\degree} liegt die höchste noch zur Erde zurückgeworfene Signalfrequenz bei \qty{12}{\MHz}}
{\qty{90}{\degree} liegt die höchste noch zur Erde zurückgeworfene Signalfrequenz bei \qty{12}{\MHz}.}
\end{QQuestion}

}
\only<2>{
\begin{QQuestion}{AH205}{Angenommen, die kritische Frequenz der F2-Region (foF2) liegt bei \qty{12}{\MHz}. Welche Aussage ist dann richtig? Bei Einstrahlung in die Ionosphäre unter einem Winkel von~...}{\qty{45}{\degree} liegt die niedrigste noch zur Erde zurückgeworfene Signalfrequenz bei \qty{12}{\MHz}.}
{\qty{90}{\degree} liegt die niedrigste noch zur Erde zurückgeworfene Signalfrequenz bei \qty{12}{\MHz}.}
{\qty{45}{\degree} liegt die höchste noch zur Erde zurückgeworfene Signalfrequenz bei \qty{12}{\MHz}}
{\textbf{\textcolor{DARCgreen}{\qty{90}{\degree} liegt die höchste noch zur Erde zurückgeworfene Signalfrequenz bei \qty{12}{\MHz}.}}}
\end{QQuestion}

}
\end{frame}%ENDCONTENT
