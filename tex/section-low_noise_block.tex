
\section{Low Noise Block (LNB)}
\label{section:low_noise_block}
\begin{frame}%STARTCONTENT

\only<1>{
\begin{PQuestion}{AF230}{Sie empfangen das Signal eines Satelliten auf \qty{10}{\GHz}. Die Kabellänge zwischen LNB und Empfänger beträgt \qty{20}{\m}. Warum ist die Kabeldämpfung trotz der hohen Empfangsfrequenz eher vernachlässigbar? }{Durch die Fernspeisespannung, die den LNB versorgt, sinkt die Kabeldämpfung.}
{Der LNB verstärkt das Empfangssignal und mischt dieses auf eine niedrigere Frequenz, auf der die Kabeldämpfung geringer ist. }
{Durch die Mischung des Empfangssignals mit der TCXO-Frequenz wird nur noch das Basisband übertragen. }
{Der LNB demoduliert das Signal. Die entstehende NF ist unempfindlich gegen Kabeldämpfung.}
{\DARCimage{1.0\linewidth}{470include}}\end{PQuestion}

}
\only<2>{
\begin{PQuestion}{AF230}{Sie empfangen das Signal eines Satelliten auf \qty{10}{\GHz}. Die Kabellänge zwischen LNB und Empfänger beträgt \qty{20}{\m}. Warum ist die Kabeldämpfung trotz der hohen Empfangsfrequenz eher vernachlässigbar? }{Durch die Fernspeisespannung, die den LNB versorgt, sinkt die Kabeldämpfung.}
{\textbf{\textcolor{DARCgreen}{Der LNB verstärkt das Empfangssignal und mischt dieses auf eine niedrigere Frequenz, auf der die Kabeldämpfung geringer ist. }}}
{Durch die Mischung des Empfangssignals mit der TCXO-Frequenz wird nur noch das Basisband übertragen. }
{Der LNB demoduliert das Signal. Die entstehende NF ist unempfindlich gegen Kabeldämpfung.}
{\DARCimage{1.0\linewidth}{470include}}\end{PQuestion}

}
\end{frame}

\begin{frame}
\only<1>{
\begin{PQuestion}{AF231}{Der LNB einer Satellitenempfangsanlage kann mit zwei unterschiedlichen Betriebsspannungen arbeiten. Was passiert, wenn die Versorgungsspannung am Bias-T im dargestellten Blockschaltbild auf \qty{18}{\V} erhöht wird?}{Der LNB schaltet auf einen anderen Satelliten um. }
{Der LNB schaltet die Polarisation um. }
{Der LNB wird durch Überspannung beschädigt. }
{Der LNB schaltet den Empfangsbereich um. }
{\DARCimage{1.0\linewidth}{470include}}\end{PQuestion}

}
\only<2>{
\begin{PQuestion}{AF231}{Der LNB einer Satellitenempfangsanlage kann mit zwei unterschiedlichen Betriebsspannungen arbeiten. Was passiert, wenn die Versorgungsspannung am Bias-T im dargestellten Blockschaltbild auf \qty{18}{\V} erhöht wird?}{Der LNB schaltet auf einen anderen Satelliten um. }
{\textbf{\textcolor{DARCgreen}{Der LNB schaltet die Polarisation um. }}}
{Der LNB wird durch Überspannung beschädigt. }
{Der LNB schaltet den Empfangsbereich um. }
{\DARCimage{1.0\linewidth}{470include}}\end{PQuestion}

}
\end{frame}%ENDCONTENT
