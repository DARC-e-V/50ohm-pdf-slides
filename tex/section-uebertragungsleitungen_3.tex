
\section{Übertragungsleitungen III}
\label{section:uebertragungsleitungen_3}
\begin{frame}%STARTCONTENT

\only<1>{
\begin{QQuestion}{AG312}{Bei einer symmetrischen Zweidrahtleitung ohne Gleichtaktanteil~...}{sind Spannung gegenüber Erde und Strom in beiden Leitern gleich groß und an jeder Stelle gleichphasig.}
{liegt einer der beiden Leiter auf Erdpotential.}
{gibt es keine Strom- und Spannungsverteilung auf der Leitung.}
{sind Spannung gegenüber Erde und Strom in beiden Leitern gleich groß und an jeder Stelle gegenphasig.}
\end{QQuestion}

}
\only<2>{
\begin{QQuestion}{AG312}{Bei einer symmetrischen Zweidrahtleitung ohne Gleichtaktanteil~...}{sind Spannung gegenüber Erde und Strom in beiden Leitern gleich groß und an jeder Stelle gleichphasig.}
{liegt einer der beiden Leiter auf Erdpotential.}
{gibt es keine Strom- und Spannungsverteilung auf der Leitung.}
{\textbf{\textcolor{DARCgreen}{sind Spannung gegenüber Erde und Strom in beiden Leitern gleich groß und an jeder Stelle gegenphasig.}}}
\end{QQuestion}

}
\end{frame}

\begin{frame}
\only<1>{
\begin{QQuestion}{AG301}{Um bei hohen Sendeleistungen auf den Kurzwellenbändern die Störwahrscheinlichkeit auf ein Mindestmaß zu begrenzen, sollte die für die Sendeantenne verwendete Speiseleitung innerhalb von Gebäuden~...}{kein ganzzahliges Vielfaches von $\lambda$/4 lang sein.}
{möglichst $\lambda$/4 lang sein.}
{geschirmt sein.}
{an keiner Stelle geerdet sein.}
\end{QQuestion}

}
\only<2>{
\begin{QQuestion}{AG301}{Um bei hohen Sendeleistungen auf den Kurzwellenbändern die Störwahrscheinlichkeit auf ein Mindestmaß zu begrenzen, sollte die für die Sendeantenne verwendete Speiseleitung innerhalb von Gebäuden~...}{kein ganzzahliges Vielfaches von $\lambda$/4 lang sein.}
{möglichst $\lambda$/4 lang sein.}
{\textbf{\textcolor{DARCgreen}{geschirmt sein.}}}
{an keiner Stelle geerdet sein.}
\end{QQuestion}

}
\end{frame}

\begin{frame}
\only<1>{
\begin{QQuestion}{AG303}{Welche Parameter beschreiben charakteristische Hochfrequenzeigenschaften eines Koaxialkabels?}{Biegeradius, Kabeldämpfung, Leitermaterial.}
{Wellenwiderstand, Kabeldämpfung, Verkürzungsfaktor.}
{Verkürzungsfaktor, Kabeldämpfung, Kabelfarbe.}
{Rückflußdämpfung, Dielektrizitätskonstante, Kabeldämpfung.}
\end{QQuestion}

}
\only<2>{
\begin{QQuestion}{AG303}{Welche Parameter beschreiben charakteristische Hochfrequenzeigenschaften eines Koaxialkabels?}{Biegeradius, Kabeldämpfung, Leitermaterial.}
{\textbf{\textcolor{DARCgreen}{Wellenwiderstand, Kabeldämpfung, Verkürzungsfaktor.}}}
{Verkürzungsfaktor, Kabeldämpfung, Kabelfarbe.}
{Rückflußdämpfung, Dielektrizitätskonstante, Kabeldämpfung.}
\end{QQuestion}

}
\end{frame}

\begin{frame}
\only<1>{
\begin{QQuestion}{AG314}{Die Ausbreitungsgeschwindigkeit in einem Koaxialkabel~...}{ist unbegrenzt.}
{ist höher als im Freiraum.}
{entspricht der Geschwindigkeit im Freiraum.}
{ist geringer als im Freiraum.}
\end{QQuestion}

}
\only<2>{
\begin{QQuestion}{AG314}{Die Ausbreitungsgeschwindigkeit in einem Koaxialkabel~...}{ist unbegrenzt.}
{ist höher als im Freiraum.}
{entspricht der Geschwindigkeit im Freiraum.}
{\textbf{\textcolor{DARCgreen}{ist geringer als im Freiraum.}}}
\end{QQuestion}

}
\end{frame}

\begin{frame}
\only<1>{
\begin{QQuestion}{AG302}{Welche Materialien werden für die Dielektriken gebräuchlicher Koaxkabel üblicherweise verwendet?}{PE-Schaum, Polystyrol, PTFE (Teflon).}
{Pertinax, Voll-PE, PE-Schaum.}
{PTFE (Teflon), Voll-PE, PE-Schaum.}
{Voll-PE, PE-Schaum, Epoxyd.}
\end{QQuestion}

}
\only<2>{
\begin{QQuestion}{AG302}{Welche Materialien werden für die Dielektriken gebräuchlicher Koaxkabel üblicherweise verwendet?}{PE-Schaum, Polystyrol, PTFE (Teflon).}
{Pertinax, Voll-PE, PE-Schaum.}
{\textbf{\textcolor{DARCgreen}{PTFE (Teflon), Voll-PE, PE-Schaum.}}}
{Voll-PE, PE-Schaum, Epoxyd.}
\end{QQuestion}

}
\end{frame}

\begin{frame}
\only<1>{
\begin{QQuestion}{AG317}{Welche mechanische Länge hat ein elektrisch $\lambda$/4 langes Koaxkabel mit Vollpolyethylenisolierung bei \qty{145}{\MHz}?}{\qty{34,2}{\cm}}
{\qty{51,7}{\cm}}
{\qty{103}{\cm}}
{\qty{17,1}{\cm}}
\end{QQuestion}

}
\only<2>{
\begin{QQuestion}{AG317}{Welche mechanische Länge hat ein elektrisch $\lambda$/4 langes Koaxkabel mit Vollpolyethylenisolierung bei \qty{145}{\MHz}?}{\textbf{\textcolor{DARCgreen}{\qty{34,2}{\cm}}}}
{\qty{51,7}{\cm}}
{\qty{103}{\cm}}
{\qty{17,1}{\cm}}
\end{QQuestion}

}
\end{frame}%ENDCONTENT
