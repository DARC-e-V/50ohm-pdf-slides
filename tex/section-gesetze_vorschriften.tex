
\section{Internationale Vereinbarungen, Gesetze und Vorschriften}
\label{section:gesetze_vorschriften}
\begin{frame}%STARTCONTENT

\frametitle{Internationale Vereinbarungen}
\begin{itemize}
  \item Die allgemeinen Regelungen der ITU Radio Regulations (RR) gelten auch für den Amateurfunkdienst.
  \item Hinzu kommen Empfehlungen der Europäischen Konferenz der Verwaltungen für Post und Telekommunikation (CEPT).
  \end{itemize}
\end{frame}

\begin{frame}
\frametitle{Gesetze und Vorschriften}
Die internationalen Vereinbarungen gelten nicht direkt für Funkamateure, sondern ihre Umsetzung in die nationalen Gesetze und Verordnungen eines Landes.

\end{frame}

\begin{frame}Deswegen ist in Deutschland maßgeblich:

\begin{itemize}
  \item Das Gesetz über den Amateurfunk (AFuG) bildet die Rechtsgrundlage für die Teilnahme am Amateurfunkdienst in Deutschland.
  \item Details regelt die Amateurfunkverordnung (AFuV).
  \item Die Aufgaben und Befugnisse, die aus dem AFuG und der AFuV erwachsen, nimmt die Bundesnetzagentur (BNetzA) wahr.
  \item Auch das Telekommunikationsgesetz (TKG) enthält Regelungen, die auf den Amateurfunkdienst anwendbar sind.
  \end{itemize}

\end{frame}

\begin{frame}
\only<1>{
\begin{QQuestion}{VA301}{Die allgemeinen Regelungen der Radio Regulations (RR) gelten~...}{lediglich hinsichtlich der Festlegung der Sendearten für den Amateurfunkdienst.}
{nicht für den Amateurfunkdienst.}
{lediglich hinsichtlich der Festlegung der Frequenzbereiche für den Amateurfunkdienst.}
{auch für den Amateurfunkdienst.}
\end{QQuestion}

}
\only<2>{
\begin{QQuestion}{VA301}{Die allgemeinen Regelungen der Radio Regulations (RR) gelten~...}{lediglich hinsichtlich der Festlegung der Sendearten für den Amateurfunkdienst.}
{nicht für den Amateurfunkdienst.}
{lediglich hinsichtlich der Festlegung der Frequenzbereiche für den Amateurfunkdienst.}
{\textbf{\textcolor{DARCgreen}{auch für den Amateurfunkdienst.}}}
\end{QQuestion}

}
\end{frame}

\begin{frame}
\only<1>{
\begin{QQuestion}{VC101}{Welches Gesetz bildet die Rechtsgrundlage und regelt die Voraussetzungen und die Bedingungen für die Teilnahme am Amateurfunkdienst in Deutschland?}{Das Gesetz über den Amateurfunk}
{Das Telekommunikationsgesetz}
{Das Gesetz über die Bereitstellung von Funkanlagen auf dem Markt}
{Das Gesetz über die elektromagnetische Verträglichkeit von Betriebsmitteln}
\end{QQuestion}

}
\only<2>{
\begin{QQuestion}{VC101}{Welches Gesetz bildet die Rechtsgrundlage und regelt die Voraussetzungen und die Bedingungen für die Teilnahme am Amateurfunkdienst in Deutschland?}{\textbf{\textcolor{DARCgreen}{Das Gesetz über den Amateurfunk}}}
{Das Telekommunikationsgesetz}
{Das Gesetz über die Bereitstellung von Funkanlagen auf dem Markt}
{Das Gesetz über die elektromagnetische Verträglichkeit von Betriebsmitteln}
\end{QQuestion}

}
\end{frame}

\begin{frame}
\only<1>{
\begin{QQuestion}{VC104}{Welche deutsche Behörde nimmt die Aufgaben und Befugnisse wahr, die sich aus dem Amateurfunkgesetz (AFuG) und der Amateurfunkverordnung (AFuV) ergeben?}{Die Bundesanstalt für den Digitalfunk der Behörden und Organisationen mit Sicherheitsaufgaben}
{Die Physikalisch-Technische Bundesanstalt}
{Die Bundesanstalt für Post und Telekommunikation}
{Die Bundesnetzagentur}
\end{QQuestion}

}
\only<2>{
\begin{QQuestion}{VC104}{Welche deutsche Behörde nimmt die Aufgaben und Befugnisse wahr, die sich aus dem Amateurfunkgesetz (AFuG) und der Amateurfunkverordnung (AFuV) ergeben?}{Die Bundesanstalt für den Digitalfunk der Behörden und Organisationen mit Sicherheitsaufgaben}
{Die Physikalisch-Technische Bundesanstalt}
{Die Bundesanstalt für Post und Telekommunikation}
{\textbf{\textcolor{DARCgreen}{Die Bundesnetzagentur}}}
\end{QQuestion}

}
\end{frame}

\begin{frame}
\only<1>{
\begin{QQuestion}{VE101}{Enthält das Telekommunikationsgesetz (TKG) Regelungen, die auf den Amateurfunkdienst anwendbar sind?}{Ja, einige Regelungen des TKG sind auch auf den Amateurfunkdienst anwendbar.}
{Nein, dafür gibt es das eigenständige Amateurfunkgesetz mit Amateurfunkverordnung.}
{Nein, der Amateurfunkdienst ist im TKG ausdrücklich ausgeschlossen.}
{Ja, alle Regelungen des TKG sind auch auf den Amateurfunkdienst anwendbar.}
\end{QQuestion}

}
\only<2>{
\begin{QQuestion}{VE101}{Enthält das Telekommunikationsgesetz (TKG) Regelungen, die auf den Amateurfunkdienst anwendbar sind?}{\textbf{\textcolor{DARCgreen}{Ja, einige Regelungen des TKG sind auch auf den Amateurfunkdienst anwendbar.}}}
{Nein, dafür gibt es das eigenständige Amateurfunkgesetz mit Amateurfunkverordnung.}
{Nein, der Amateurfunkdienst ist im TKG ausdrücklich ausgeschlossen.}
{Ja, alle Regelungen des TKG sind auch auf den Amateurfunkdienst anwendbar.}
\end{QQuestion}

}
\end{frame}%ENDCONTENT
