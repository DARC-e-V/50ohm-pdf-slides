
\section{Tote Zone II}
\label{section:tote_zone_2}
\begin{frame}%STARTCONTENT

\frametitle{Tote Zone}
\begin{columns}
    \begin{column}{0.48\textwidth}
    \begin{itemize}
  \item Je höher die Frequenz, desto größer ist der Radius der toten Zone
  \item Insbesondere auf den höheren Bändern kann es zur Fehlannahme einer freien Frequenz kommen
  \end{itemize}

    \end{column}
   \begin{column}{0.48\textwidth}
       
\begin{figure}
    \DARCimage{0.85\linewidth}{741include}
    \caption{\scriptsize Tote Zone}
    \label{a_tote_zone}
\end{figure}


   \end{column}
\end{columns}

\end{frame}

\begin{frame}
\only<1>{
\begin{QQuestion}{AH215}{Eine Aussendung auf dem \qty{20}{\m}-Band kann von der Funkstelle~A in einer Entfernung von \qty{1500}{\km}, nicht jedoch von der Funkstelle~B in \qty{60}{\km} Entfernung empfangen werden. Der Grund hierfür ist, dass~...}{zwei in verschiedenen ionosphärischen Regionen reflektierte Wellen mit auslöschender Phase bei Funkstelle~B eintreffen.}
{die Boden- und Raumwellen sich bei Funkstelle~B gegenseitig aufheben.}
{die Funkstelle B die Bodenwelle nicht mehr und die Raumwelle noch nicht empfangen kann.}
{bei Funkstelle~B der Mögel-Dellinger-Effekt aufgetreten ist.}
\end{QQuestion}

}
\only<2>{
\begin{QQuestion}{AH215}{Eine Aussendung auf dem \qty{20}{\m}-Band kann von der Funkstelle~A in einer Entfernung von \qty{1500}{\km}, nicht jedoch von der Funkstelle~B in \qty{60}{\km} Entfernung empfangen werden. Der Grund hierfür ist, dass~...}{zwei in verschiedenen ionosphärischen Regionen reflektierte Wellen mit auslöschender Phase bei Funkstelle~B eintreffen.}
{die Boden- und Raumwellen sich bei Funkstelle~B gegenseitig aufheben.}
{\textbf{\textcolor{DARCgreen}{die Funkstelle B die Bodenwelle nicht mehr und die Raumwelle noch nicht empfangen kann.}}}
{bei Funkstelle~B der Mögel-Dellinger-Effekt aufgetreten ist.}
\end{QQuestion}

}
\end{frame}%ENDCONTENT
