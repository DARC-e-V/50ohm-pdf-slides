
\section{Split-Verkehr}
\label{section:split_betrieb}
\begin{frame}%STARTCONTENT

\frametitle{Umgang mit dem Pile-Up 2}
\begin{itemize}
  \item Split-Verkehr ist die am Meisten genutzte Methode um vielen Anrufen eine Verbindung zu ermöglichen.
  \item Dabei empfängt die begehrte Station auf einer anderen Frequenz als sie sendet.
  \item Die CQ-rufende Station kündigt den Split-Verkehr mit Hinweis auf die Frequenz oder den Frequenzbereich an, in dem sie empfängt. Zum Beispiel \enquote{5 up} oder \enquote{split 14270 to 14280}.
  \end{itemize}

\end{frame}

\begin{frame}
\begin{figure}
    \DARCimage{0.85\linewidth}{672include}
    \caption{\scriptsize Split-Verkehr, Sendefrequenz der begehrten Station und Antwort-Frequenz, auf der sie empfängt}
    \label{n_split_verkehr_anruffrequenz}
\end{figure}


\begin{figure}
    \DARCimage{0.85\linewidth}{673include}
    \caption{\scriptsize Split-Verkehr, Sendefrequenz der begehrten Station und Frequenzbereich für Antworten, in dem sie empfängt}
    \label{n_split_verkehr_anrufbereich}
\end{figure}

\end{frame}

\begin{frame}
\only<1>{
\begin{QQuestion}{BE308}{Was ist \glqq Split-Verkehr\grqq{}?}{Nutzung unterschiedlicher Übertragungsverfahren in einem QSO}
{Verwenden von mehr als einem Funkgerät}
{Teilen einer Frequenz zwischen zwei Relaisfunkstellen}
{Senden und Empfangen auf unterschiedlichen Frequenzen}
\end{QQuestion}

}
\only<2>{
\begin{QQuestion}{BE308}{Was ist \glqq Split-Verkehr\grqq{}?}{Nutzung unterschiedlicher Übertragungsverfahren in einem QSO}
{Verwenden von mehr als einem Funkgerät}
{Teilen einer Frequenz zwischen zwei Relaisfunkstellen}
{\textbf{\textcolor{DARCgreen}{Senden und Empfangen auf unterschiedlichen Frequenzen}}}
\end{QQuestion}

}
\end{frame}

\begin{frame}
\only<1>{
\begin{QQuestion}{BE310}{Eine Station gibt am Ende ihres CQ-Rufes \glqq 5 up\grqq{}. Was bedeutet diese Angabe und was ist zu beachten?}{Die rufende Station behandelt meinen Anruf an 5ter Stelle. Ich muss also bei meinem Anruf 4 andere Funkverbindungen abwarten.}
{Die rufende Station hört 5~Minuten später auf ihrer eigenen Sendefrequenz. Ich muss also bei meinem Anruf 5~Minuten später senden und vorher prüfen, ob die Frequenz frei ist.}
{Die rufende Station sendet \qty{5}{\kHz} oberhalb ihrer eigenen Sendefrequenz. Ich muss also bei meinem Anruf \qty{5}{\kHz} höher empfangen und vorher prüfen, ob die Frequenz frei ist.}
{Die rufende Station hört \qty{5}{\kHz} oberhalb ihrer eigenen Sendefrequenz. Ich muss also bei meinem Anruf \qty{5}{\kHz} höher senden.}
\end{QQuestion}

}
\only<2>{
\begin{QQuestion}{BE310}{Eine Station gibt am Ende ihres CQ-Rufes \glqq 5 up\grqq{}. Was bedeutet diese Angabe und was ist zu beachten?}{Die rufende Station behandelt meinen Anruf an 5ter Stelle. Ich muss also bei meinem Anruf 4 andere Funkverbindungen abwarten.}
{Die rufende Station hört 5~Minuten später auf ihrer eigenen Sendefrequenz. Ich muss also bei meinem Anruf 5~Minuten später senden und vorher prüfen, ob die Frequenz frei ist.}
{Die rufende Station sendet \qty{5}{\kHz} oberhalb ihrer eigenen Sendefrequenz. Ich muss also bei meinem Anruf \qty{5}{\kHz} höher empfangen und vorher prüfen, ob die Frequenz frei ist.}
{\textbf{\textcolor{DARCgreen}{Die rufende Station hört \qty{5}{\kHz} oberhalb ihrer eigenen Sendefrequenz. Ich muss also bei meinem Anruf \qty{5}{\kHz} höher senden.}}}
\end{QQuestion}

}
\end{frame}

\begin{frame}
\only<1>{
\begin{QQuestion}{BE309}{Was bedeutet es, wenn eine begehrte Station CQ ruft und den Anruf mit \glqq split up 14270 to 14280\grqq{} beendet? Die Station~...}{wird im angegebenen Bereich mit einer Bandbreite von \qty{10}{\kHz} senden.}
{kündigt einen Wechsel ihrer Sendefrequenz in den angegebenen Bereich an.}
{bittet anrufende Stationen in dem angegebenen Bereich CW zu verwenden.}
{hört oberhalb ihrer Sendefrequenz auf wechselnden Frequenzen im angegebenen Bereich.}
\end{QQuestion}

}
\only<2>{
\begin{QQuestion}{BE309}{Was bedeutet es, wenn eine begehrte Station CQ ruft und den Anruf mit \glqq split up 14270 to 14280\grqq{} beendet? Die Station~...}{wird im angegebenen Bereich mit einer Bandbreite von \qty{10}{\kHz} senden.}
{kündigt einen Wechsel ihrer Sendefrequenz in den angegebenen Bereich an.}
{bittet anrufende Stationen in dem angegebenen Bereich CW zu verwenden.}
{\textbf{\textcolor{DARCgreen}{hört oberhalb ihrer Sendefrequenz auf wechselnden Frequenzen im angegebenen Bereich.}}}
\end{QQuestion}

}
\end{frame}

\begin{frame}
\only<1>{
\begin{QQuestion}{BE311}{Eine Station, die auf \qty{14205}{\kHz} CQ gerufen hat, sagt am Ende ihres Rufes \glqq tuning 290 to 300 up\grqq{}. Welche Frequenzen nutzen Sie, wenn Sie diese Station anrufen wollen?}{Ich rufe zwischen \num{14290} und \qty{14300}{\kHz} und höre auf \qty{14205}{\kHz}.}
{Ich rufe und höre zwischen \num{14290} und \qty{14300}{\kHz}.}
{Ich rufe auf \qty{14205}{\kHz} und höre zwischen \num{14290} und \qty{14300}{\kHz}.}
{Ich rufe auf \qty{14290}{\kHz} und höre auf \qty{14300}{\kHz}.}
\end{QQuestion}

}
\only<2>{
\begin{QQuestion}{BE311}{Eine Station, die auf \qty{14205}{\kHz} CQ gerufen hat, sagt am Ende ihres Rufes \glqq tuning 290 to 300 up\grqq{}. Welche Frequenzen nutzen Sie, wenn Sie diese Station anrufen wollen?}{\textbf{\textcolor{DARCgreen}{Ich rufe zwischen \num{14290} und \qty{14300}{\kHz} und höre auf \qty{14205}{\kHz}.}}}
{Ich rufe und höre zwischen \num{14290} und \qty{14300}{\kHz}.}
{Ich rufe auf \qty{14205}{\kHz} und höre zwischen \num{14290} und \qty{14300}{\kHz}.}
{Ich rufe auf \qty{14290}{\kHz} und höre auf \qty{14300}{\kHz}.}
\end{QQuestion}

}
\end{frame}%ENDCONTENT
