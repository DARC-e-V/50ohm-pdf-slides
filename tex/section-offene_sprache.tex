
\section{Offene Sprache}
\label{section:offene_sprache}
\begin{frame}%STARTCONTENT

\frametitle{Offene Sprache}
Im Amateurfunk darf nur offene Sprache verwendet werden.

\begin{itemize}
  \item Keine Verschleierungsverfahren wie geheime Codes
  \item Zulässig sind digitale Kodierungen, Morsezeichen und Abkürzungen
  \end{itemize}

\end{frame}

\begin{frame}
\only<1>{
\begin{QQuestion}{VD103}{Im Amateurfunkverkehr darf nur offene Sprache verwendet werden. Was gilt \underline{nicht} als offene Sprache und ist daher unzulässig?}{Digitale Übertragungsverfahren, die einen Decoder benötigen}
{Q-Gruppen und Amateurfunkabkürzungen}
{Sprachverschlüsselung zur Verschleierung des Inhalts}
{Morsetelegrafie und Fernschreiben}
\end{QQuestion}

}
\only<2>{
\begin{QQuestion}{VD103}{Im Amateurfunkverkehr darf nur offene Sprache verwendet werden. Was gilt \underline{nicht} als offene Sprache und ist daher unzulässig?}{Digitale Übertragungsverfahren, die einen Decoder benötigen}
{Q-Gruppen und Amateurfunkabkürzungen}
{\textbf{\textcolor{DARCgreen}{Sprachverschlüsselung zur Verschleierung des Inhalts}}}
{Morsetelegrafie und Fernschreiben}
\end{QQuestion}

}
\end{frame}%ENDCONTENT
