
\section{Persönliche Rufzeichen}
\label{section:persoenliche_rufzeichen}
\begin{frame}%STARTCONTENT
\begin{itemize}
  \item Am häufigsten vergeben
  \item In Deutschland gibt es drei Zulassungsklassen
  \end{itemize}
\begin{enumerate}
  \item[1] Klasse~N (Entry Level License)
  \item[2] Klasse~E (Novice)
  \item[3] Klasse~A (Advanced)
  \end{enumerate}
\begin{itemize}
  \item Die Klasse ist am Präfix und der Ziffer erkennbar
  \end{itemize}

\end{frame}

\begin{frame}
\begin{columns}
    \begin{column}{0.48\textwidth}
    \begin{table}
\begin{DARCtabular}{lX}
     Klasse  & Präfix und Ziffer   \\
     Klasse~N  & DN9   \\
     Klasse~E  & DO1 -- DO9   \\
     Klasse~A  & DB1 -- DD9   \\
      & DF1 -- DH9   \\
      & DJ1 -- DM9   \\
\end{DARCtabular}
\caption{Präfixe und Ziffern für personengebunde Rufzeichen}
\label{n_persoenliche_rufzeichen_praefixe}
\end{table}

    \end{column}
   \begin{column}{0.48\textwidth}
       \begin{table}
\begin{DARCtabular}{lX}
     Klasse  & Rufzeichen   \\
     Klasse~N  & DN9AAA, DN9BB   \\
     Klasse~E  & DO2AAA, DO2BB   \\
     Klasse~A  & DL3AAA, DL3BB   \\
\end{DARCtabular}
\caption{Beispiele für personengebunde Rufzeichen}
\label{n_persoenliche_rufzeichen_beispiele}
\end{table}

   \end{column}
\end{columns}

\end{frame}

\begin{frame}
\only<1>{
\begin{QQuestion}{BD105}{Zu welcher Rufzeichenart gehören Rufzeichen, die mit DN9 beginnen?}{Personengebundene Rufzeichen für Kurzwellenhörer}
{Personengebundene Rufzeichen der Klasse~A}
{Personengebundene Rufzeichen der Klasse~E}
{Personengebundene Rufzeichen der Klasse~N}
\end{QQuestion}

}
\only<2>{
\begin{QQuestion}{BD105}{Zu welcher Rufzeichenart gehören Rufzeichen, die mit DN9 beginnen?}{Personengebundene Rufzeichen für Kurzwellenhörer}
{Personengebundene Rufzeichen der Klasse~A}
{Personengebundene Rufzeichen der Klasse~E}
{\textbf{\textcolor{DARCgreen}{Personengebundene Rufzeichen der Klasse~N}}}
\end{QQuestion}

}
\end{frame}

\begin{frame}
\only<1>{
\begin{QQuestion}{BD106}{Zu welcher Rufzeichenart gehören Rufzeichen, die mit DO1 bis DO9 beginnen und ein zwei- oder dreistelliges Suffix haben? Personengebundene Rufzeichen der~...}{Klasse K}
{Klasse A}
{Klasse N}
{Klasse E}
\end{QQuestion}

}
\only<2>{
\begin{QQuestion}{BD106}{Zu welcher Rufzeichenart gehören Rufzeichen, die mit DO1 bis DO9 beginnen und ein zwei- oder dreistelliges Suffix haben? Personengebundene Rufzeichen der~...}{Klasse K}
{Klasse A}
{Klasse N}
{\textbf{\textcolor{DARCgreen}{Klasse E}}}
\end{QQuestion}

}
\end{frame}

\begin{frame}
\only<1>{
\begin{QQuestion}{BD104}{Zu welcher Rufzeichenart gehören Rufzeichen, die mit DL1 bis DL9 beginnen und ein zwei- oder dreistelliges Suffix haben? Personengebundene Rufzeichen der~...}{Klasse E}
{Klasse A}
{Klasse N}
{Klasse K}
\end{QQuestion}

}
\only<2>{
\begin{QQuestion}{BD104}{Zu welcher Rufzeichenart gehören Rufzeichen, die mit DL1 bis DL9 beginnen und ein zwei- oder dreistelliges Suffix haben? Personengebundene Rufzeichen der~...}{Klasse E}
{\textbf{\textcolor{DARCgreen}{Klasse A}}}
{Klasse N}
{Klasse K}
\end{QQuestion}

}
\end{frame}

\begin{frame}
\frametitle{Zulassung}
\begin{itemize}
  \item Nach bestandener Prüfung Antrag bei BNetzA stellen auf \emph{Zulassung zur Teilnahme am Amateurfunkdienst}
  \item Darauf kommt die Zulassungsurkunde mit persönlichem Rufzeichen
  \item Erst danach darf Funkbetrieb aufgenommen werden
  \item Die Zulassung ist nicht übertragbar
  \end{itemize}

\end{frame}

\begin{frame}
\only<1>{
\begin{QQuestion}{VC107}{Darf ein Funkamateur seine Amateurfunkzulassung vorübergehend einer anderen Person übertragen? Die Amateurfunkzulassung ist~...}{an die in der Zulassungsurkunde angegebene Person gebunden und nicht übertragbar.}
{nach vorheriger Anzeige bei der Bundesnetzagentur an Personen im gleichen Haushalt übertragbar.}
{übertragbar, wenn es sich bei der Person um einen Funkamateur mit erfolgreich abgelegter Prüfung handelt.}
{übertragbar, wenn es sich um ausländische Funkamateure handelt, die sich vorübergehend in Deutschland aufhalten.}
\end{QQuestion}

}
\only<2>{
\begin{QQuestion}{VC107}{Darf ein Funkamateur seine Amateurfunkzulassung vorübergehend einer anderen Person übertragen? Die Amateurfunkzulassung ist~...}{\textbf{\textcolor{DARCgreen}{an die in der Zulassungsurkunde angegebene Person gebunden und nicht übertragbar.}}}
{nach vorheriger Anzeige bei der Bundesnetzagentur an Personen im gleichen Haushalt übertragbar.}
{übertragbar, wenn es sich bei der Person um einen Funkamateur mit erfolgreich abgelegter Prüfung handelt.}
{übertragbar, wenn es sich um ausländische Funkamateure handelt, die sich vorübergehend in Deutschland aufhalten.}
\end{QQuestion}

}
\end{frame}

\begin{frame}
\frametitle{Wunschrufzeichen}
\begin{itemize}
  \item Auf dem Antrag können Wunschrufzeichen angegeben werden
  \item In der Rufzeichenliste der BNetzA (\textcolor{DARCblue}{\faLink~\href{https://50ohm.de/rzl}{50ohm.de/rzl}}) oder in der Webabfrage der BNetzA (\textcolor{DARCblue}{\faLink~\href{https://50ohm.de/rza}{50ohm.de/rza}}) nach freien Rufzeichen schauen
  \item Es gibt keinen Anspruch auf die Zuteilung eines bestimmten Rufzeichens
  \item Ohne Wunschrufzeichen wählt die BNetzA ein Rufzeichen aus
  \end{itemize}

\end{frame}

\begin{frame}
\frametitle{Auswahl von Rufzeichen}
\begin{itemize}
  \item Kurzschreibweise des Namens
  \item Initialen
  \item Gute Verständlichkeit beim Sprechen
  \item Einfache Morsetelegrafie
  \item Wortwitz
  \end{itemize}
\end{frame}

\begin{frame}
\only<1>{
\begin{QQuestion}{VD208}{Hat ein Funkamateur Anspruch auf Zuteilung eines bestimmten Rufzeichens?}{Ja, wenn es noch nicht vergeben ist.}
{Nein, es besteht kein Anspruch darauf.}
{Nein, es sei denn, er kann besondere persönliche Gründe geltend machen und das Rufzeichen ist frei.}
{Ja, wenn es ihm schon einmal zugeteilt war.}
\end{QQuestion}

}
\only<2>{
\begin{QQuestion}{VD208}{Hat ein Funkamateur Anspruch auf Zuteilung eines bestimmten Rufzeichens?}{Ja, wenn es noch nicht vergeben ist.}
{\textbf{\textcolor{DARCgreen}{Nein, es besteht kein Anspruch darauf.}}}
{Nein, es sei denn, er kann besondere persönliche Gründe geltend machen und das Rufzeichen ist frei.}
{Ja, wenn es ihm schon einmal zugeteilt war.}
\end{QQuestion}

}
\end{frame}

\begin{frame}
\frametitle{Änderung von Rufzeichen}
\begin{itemize}
  \item In der Regel ändert die BNetzA ein Rufzeichen nicht
  \item Bei Änderungen von Regularien oder Zulassungsklassen kann es notwendig werden
  \item Bei Stellung eines neuen Antrags auf Zulassung zum Amateurfunkdienst wird ein neues Rufzeichen vergeben
  \end{itemize}

\end{frame}

\begin{frame}
\only<1>{
\begin{QQuestion}{VC117}{Kann ein zugeteiltes Rufzeichen durch die Bundesnetzagentur geändert werden?}{Das zugeteilte Rufzeichen ist Eigentum des Funkamateurs, das durch die Bundesnetzagentur nicht geändert wird.}
{Bei Umzug in den Zuständigkeitsbereich einer anderen Außenstelle der Bundesnetzagentur erhält der Funkamateur eine neue Rufzeichenzuteilung.}
{Aus wichtigen Gründen, insbesondere bei Änderungen internationaler Vorgaben, kann das Rufzeichen geändert werden.}
{Bei Änderung der Anzeige zur Verordnung über das Nachweisverfahren zur Begrenzung elektromagnetischer Felder (BEMFV) erhält der Funkamateur ein anderes Rufzeichen.}
\end{QQuestion}

}
\only<2>{
\begin{QQuestion}{VC117}{Kann ein zugeteiltes Rufzeichen durch die Bundesnetzagentur geändert werden?}{Das zugeteilte Rufzeichen ist Eigentum des Funkamateurs, das durch die Bundesnetzagentur nicht geändert wird.}
{Bei Umzug in den Zuständigkeitsbereich einer anderen Außenstelle der Bundesnetzagentur erhält der Funkamateur eine neue Rufzeichenzuteilung.}
{\textbf{\textcolor{DARCgreen}{Aus wichtigen Gründen, insbesondere bei Änderungen internationaler Vorgaben, kann das Rufzeichen geändert werden.}}}
{Bei Änderung der Anzeige zur Verordnung über das Nachweisverfahren zur Begrenzung elektromagnetischer Felder (BEMFV) erhält der Funkamateur ein anderes Rufzeichen.}
\end{QQuestion}

}
\end{frame}%ENDCONTENT
