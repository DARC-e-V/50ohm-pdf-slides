
\section{Squelch}
\label{section:squelch}
\begin{frame}%STARTCONTENT

\begin{columns}
    \begin{column}{0.48\textwidth}
    \begin{itemize}
  \item Auf einer \enquote{leeren} Frequenz hört man Rauschen
  \item Bei FM ist das Rauschen besonders laut
  \item Mit der \emph{Rauschsperre} kann das Rauschen ausgeblendet werden
  \item Englisch \emph{Squelch} (SQL)
  \end{itemize}

    \end{column}
   \begin{column}{0.48\textwidth}
       
\begin{figure}
    \DARCimage{0.85\linewidth}{737include}
    \caption{\scriptsize Zeitlicher Verlauf der Amplitude auf einer Frequenz, zu sehen ist eine starke und eine schwache Sendung, drumherum keine Sendung (Rauschen), der Squelch blendet sowohl Rauschen als auch schwache Signale aus, wenn die Amplitude unter dem eingestellten Wert liegt.}
    \label{squelch}
\end{figure}


   \end{column}
\end{columns}

\end{frame}

\begin{frame}
\only<1>{
\begin{QQuestion}{NF302}{Was muss am Empfänger eingestellt werden, um bei FM das Grundrauschen auszublenden, wenn kein Nutzsignal empfangen wird?}{VOX}
{Squelch}
{RIT}
{Notchfilter}
\end{QQuestion}

}
\only<2>{
\begin{QQuestion}{NF302}{Was muss am Empfänger eingestellt werden, um bei FM das Grundrauschen auszublenden, wenn kein Nutzsignal empfangen wird?}{VOX}
{\textbf{\textcolor{DARCgreen}{Squelch}}}
{RIT}
{Notchfilter}
\end{QQuestion}

}
\end{frame}%ENDCONTENT
