
\section{Ohmsches Gesetz}
\label{section:ohmsches_gesetz}
\begin{frame}%STARTCONTENT
Kurze Wiederholung:

\begin{itemize}
  \item Elektrische Ladungen werden in Spannungesquellen getrennt, wodurch elektrische Spannung entsteht. Buchstabe $U$, Einheit Volt (V).
  \item Elektrische Spannung sorgt für elektrischen Stromfluss in geschlossenem Stromkreis. Buchstabe $I$, Einheit Ampere (A).
  \item Verbraucher üben in einem Stromkreis einen Widerstand aus und bremsen den Stromfluß. Buchstabe $R$, Einheit Ohm (Ω).
  \end{itemize}
\end{frame}

\begin{frame}
\only<1>{
\begin{QQuestion}{NA203}{Welche Einheit wird üblicherweise für den elektrische Widerstand verwendet?}{Volt (V)}
{Ohm ($\Omega$)}
{Ampere (A)}
{Watt (W)}
\end{QQuestion}

}
\only<2>{
\begin{QQuestion}{NA203}{Welche Einheit wird üblicherweise für den elektrische Widerstand verwendet?}{Volt (V)}
{\textbf{\textcolor{DARCgreen}{Ohm ($\Omega$)}}}
{Ampere (A)}
{Watt (W)}
\end{QQuestion}

}
\end{frame}

\begin{frame}
\frametitle{Zusammenhang}
\begin{columns}
    \begin{column}{0.48\textwidth}
    
\begin{figure}
    \DARCimage{0.85\linewidth}{664include}
    \caption{\scriptsize Stromkreis mit Batterie}
    \label{n_ohmsches_gesetz_stromkreis_mit_batterie}
\end{figure}


    \end{column}
   \begin{column}{0.48\textwidth}
       \begin{itemize}
  \item Spannung 10 V
  \item Strom 1 mA
  \end{itemize}

   \end{column}
\end{columns}

\end{frame}

\begin{frame}\begin{itemize}
  \item Bei 20 V erhöht sich der Strom auf 2 mA
  \item Bei 5 V verringert sich der Strom auf 0,5 mA
  \end{itemize}
    \pause
    $\dfrac{U}{I} = \dfrac{10 \ \text{V}}{0,001 \ \text{A}} = \dfrac{20 \ \text{V}}{0,002 \ \text{A}} = \dfrac{5 \ \text{V}}{0,0005 \ \text{A}} = 10000 \frac{\text{V}}{\text{A}}$
    \pause
    Proportionalität: $I$ ist proportional zu $U$ mit \emph{Proportionalitätsfaktor} 10000

\end{frame}

\begin{frame}
\frametitle{Widerstand}
\begin{itemize}
  \item Der Proportionalitätsfaktor von 10000 aus dem Beispiel ist der Widerstand $R$
  \item Einheit: $1 \ Ω = 1 \frac{\text{V}}{\text{A}}$
  \item Der Widerstand aus dem Beispiel beträgt 10000 Ω oder 10 kΩ
  \end{itemize}
\end{frame}

\begin{frame}
\frametitle{Ohmsches Gesetz}
Der Widerstand ist das \emph{Verhältnis von Spannung und Strom}

$ R = \dfrac{U}{I} $

\end{frame}

\begin{frame}
\only<1>{
\begin{PQuestion}{NB505}{Welcher Widerstandswert liegt vor?}{\qty{3,600}{\ohm}}
{\qty{40,000}{\ohm}}
{\qty{0,025}{\ohm}}
{\qty{0,200}{\ohm}}
{\DARCimage{0.75\linewidth}{508include}}\end{PQuestion}

}
\only<2>{
\begin{PQuestion}{NB505}{Welcher Widerstandswert liegt vor?}{\qty{3,600}{\ohm}}
{\textbf{\textcolor{DARCgreen}{\qty{40,000}{\ohm}}}}
{\qty{0,025}{\ohm}}
{\qty{0,200}{\ohm}}
{\DARCimage{0.75\linewidth}{508include}}\end{PQuestion}

}
\end{frame}

\begin{frame}
\frametitle{Formelumstellung}
\begin{columns}
    \begin{column}{0.48\textwidth}
    \begin{itemize}
  \item Spannung und Widerstand bekannt
  \item Strom unbekannt
  \end{itemize}
$ I = \dfrac{U}{R} $


    \end{column}
   \begin{column}{0.48\textwidth}
       \begin{itemize}
  \item Strom und Widerstand bekannt
  \item Spannung unbekannt
  \end{itemize}
$ U = R\cdot I $


   \end{column}
\end{columns}

\end{frame}

\begin{frame}
\only<1>{
\begin{PQuestion}{NB504}{Welche Spannung lässt einen Strom von \qty{90}{\mA} durch den Widerstand fließen?}{\qty{1,111}{\kV}}
{\qty{9,000}{\V}}
{\qty{9,000}{\kV}}
{\qty{1,111}{\V}}
{\DARCimage{0.75\linewidth}{507include}}\end{PQuestion}

}
\only<2>{
\begin{PQuestion}{NB504}{Welche Spannung lässt einen Strom von \qty{90}{\mA} durch den Widerstand fließen?}{\qty{1,111}{\kV}}
{\textbf{\textcolor{DARCgreen}{\qty{9,000}{\V}}}}
{\qty{9,000}{\kV}}
{\qty{1,111}{\V}}
{\DARCimage{0.75\linewidth}{507include}}\end{PQuestion}

}
\end{frame}

\begin{frame}
\only<1>{
\begin{QQuestion}{NB502}{Welcher der nachfolgenden Ausdrücke stellt den Zusammenhang zwischen Strom, Spannung und Widerstand korrekt dar?}{$R = U \cdot I$}
{$I = \dfrac{U}{R}$}
{$R = \dfrac{I}{U}$}
{$I =R \cdot U$}
\end{QQuestion}

}
\only<2>{
\begin{QQuestion}{NB502}{Welcher der nachfolgenden Ausdrücke stellt den Zusammenhang zwischen Strom, Spannung und Widerstand korrekt dar?}{$R = U \cdot I$}
{\textbf{\textcolor{DARCgreen}{$I = \dfrac{U}{R}$}}}
{$R = \dfrac{I}{U}$}
{$I =R \cdot U$}
\end{QQuestion}

}
\end{frame}

\begin{frame}
\only<1>{
\begin{QQuestion}{NB503}{Welcher der nachfolgenden Ausdrücke stellt den Zusammenhang zwischen Strom, Spannung und Widerstand korrekt dar?}{$R = \dfrac{U}{I}$}
{$R = U \cdot I$}
{$R = \dfrac{I}{U}$}
{$I =R \cdot U$}
\end{QQuestion}

}
\only<2>{
\begin{QQuestion}{NB503}{Welcher der nachfolgenden Ausdrücke stellt den Zusammenhang zwischen Strom, Spannung und Widerstand korrekt dar?}{\textbf{\textcolor{DARCgreen}{$R = \dfrac{U}{I}$}}}
{$R = U \cdot I$}
{$R = \dfrac{I}{U}$}
{$I =R \cdot U$}
\end{QQuestion}

}
\end{frame}

\begin{frame}
\only<1>{
\begin{QQuestion}{NB501}{Welcher der nachfolgenden Ausdrücke stellt den Zusammenhang zwischen Strom, Spannung und Widerstand korrekt dar?}{$R = U \cdot I$}
{$U = R \cdot I $}
{$R = \dfrac{I}{U}$}
{$I =R \cdot U$}
\end{QQuestion}

}
\only<2>{
\begin{QQuestion}{NB501}{Welcher der nachfolgenden Ausdrücke stellt den Zusammenhang zwischen Strom, Spannung und Widerstand korrekt dar?}{$R = U \cdot I$}
{\textbf{\textcolor{DARCgreen}{$U = R \cdot I $}}}
{$R = \dfrac{I}{U}$}
{$I =R \cdot U$}
\end{QQuestion}

}
\end{frame}%ENDCONTENT
