
\section{Sprachsignale}
\label{section:sprachsignale}
\begin{frame}%STARTCONTENT

\begin{figure}
    \DARCimage{0.85\linewidth}{742include}
    \caption{\scriptsize Menschliche Sprache im Amplitudenspektrum, links die tiefen und rechts die hohen Töne}
    \label{n_sprachspektrum}
\end{figure}

\end{frame}

\begin{frame}
\begin{columns}
    \begin{column}{0.48\textwidth}
    \begin{itemize}
  \item Sprechen wir in ein Mikrofon, dann wandelt es das Sprachsignal in elektrische Schwingungen um.
  \item Das Sprachsignal liegt nicht mehr als Schallwelle, sondern als elektrische Schwingung vor und kann im Funkgerät verarbeitet werden.
  \end{itemize}

    \end{column}
    \pause
    
   \begin{column}{0.48\textwidth}
       \begin{itemize}
  \item Die „Breite“ des Signals wird übrigens als \emph{Bandbreite} bezeichnet und in Hertz (Hz) angegeben.
  \item Angenommen es soll Sprache im Frequenzbereich von \qtyrange{300}{2700}{\hertz} übertragen werden.
  \item Die Bandbreite beträgt in diesem Falle \qty{2700}{\hertz} – \qty{300}{\hertz} = \qty{2400}{\hertz}
  \end{itemize}

   \end{column}
\end{columns}



\end{frame}

\begin{frame}
\only<1>{
\begin{QQuestion}{EA105}{Welche Einheit wird üblicherweise für die Bandbreite verwendet?}{Dezibel (dB)}
{Baud (Bd)}
{Bit pro Sekunde (Bit/s)}
{Hertz (Hz)}
\end{QQuestion}

}
\only<2>{
\begin{QQuestion}{EA105}{Welche Einheit wird üblicherweise für die Bandbreite verwendet?}{Dezibel (dB)}
{Baud (Bd)}
{Bit pro Sekunde (Bit/s)}
{\textbf{\textcolor{DARCgreen}{Hertz (Hz)}}}
\end{QQuestion}

}
\end{frame}%ENDCONTENT
