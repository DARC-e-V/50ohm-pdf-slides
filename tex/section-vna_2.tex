
\section{Vektorieller Netzwerkanalysator (VNA) II}
\label{section:vna_2}
\begin{frame}%STARTCONTENT

\only<1>{
\begin{QQuestion}{AI201}{Wie funktioniert ein vektorieller Netzwerkanalysator (VNA)? Ein HF-Generator erzeugt ein~...}{frequenzstabiles HF-Signal, mit dem  z.~B. ein Filter oder eine Antenne beaufschlagt wird. Aus der durch das Messobjekt entstehenden Fehlanpassung werden Dämpfungsverlauf oder Antennengewinn ermittelt.}
{frequenzstabiles HF-Signal, mit dem z.~B. ein Filter oder eine Antenne beaufschlagt wird. Die durch das angeschlossene Messobjekt erzeugten Strom- und Spannungsbäuche werden als Verläufe von z.~B. Impedanz und Phasenwinkel, Wirk- und Blindanteil oder dem Stehwellenverhältnis grafisch dargestellt.}
{frequenzveränderliches HF-Signal, mit dem z.~B. ein Filter oder eine Antenne beaufschlagt wird. Aus den durch das Messobjekt entstehenden Spannungseinbrüchen wird der Scheinwiderstand des Messobjektes ermittelt.}
{frequenzveränderliches HF-Signal, mit dem z.~B. ein Filter oder eine Antenne beaufschlagt wird. Die durch das angeschlossene Messobjekt veränderten Amplituden und Phasen des HF-Signals werden als Verläufe von z.~B. Impedanz und Phasenwinkel, Wirk- und Blindanteil oder dem Stehwellenverhältnis grafisch dargestellt.}
\end{QQuestion}

}
\only<2>{
\begin{QQuestion}{AI201}{Wie funktioniert ein vektorieller Netzwerkanalysator (VNA)? Ein HF-Generator erzeugt ein~...}{frequenzstabiles HF-Signal, mit dem  z.~B. ein Filter oder eine Antenne beaufschlagt wird. Aus der durch das Messobjekt entstehenden Fehlanpassung werden Dämpfungsverlauf oder Antennengewinn ermittelt.}
{frequenzstabiles HF-Signal, mit dem z.~B. ein Filter oder eine Antenne beaufschlagt wird. Die durch das angeschlossene Messobjekt erzeugten Strom- und Spannungsbäuche werden als Verläufe von z.~B. Impedanz und Phasenwinkel, Wirk- und Blindanteil oder dem Stehwellenverhältnis grafisch dargestellt.}
{frequenzveränderliches HF-Signal, mit dem z.~B. ein Filter oder eine Antenne beaufschlagt wird. Aus den durch das Messobjekt entstehenden Spannungseinbrüchen wird der Scheinwiderstand des Messobjektes ermittelt.}
{\textbf{\textcolor{DARCgreen}{frequenzveränderliches HF-Signal, mit dem z.~B. ein Filter oder eine Antenne beaufschlagt wird. Die durch das angeschlossene Messobjekt veränderten Amplituden und Phasen des HF-Signals werden als Verläufe von z.~B. Impedanz und Phasenwinkel, Wirk- und Blindanteil oder dem Stehwellenverhältnis grafisch dargestellt.}}}
\end{QQuestion}

}
\end{frame}

\begin{frame}
\only<1>{
\begin{QQuestion}{AI202}{Welches dieser Messgeräte ist für die Ermittlung der Resonanzfrequenz eines Traps, der für einen Dipol genutzt werden soll, am besten geeignet?}{Ein Frequenzmessgerät}
{Eine SWR-Messbrücke}
{Ein vektorieller Netzwerk Analysator}
{Ein Resonanzwellenmesser}
\end{QQuestion}

}
\only<2>{
\begin{QQuestion}{AI202}{Welches dieser Messgeräte ist für die Ermittlung der Resonanzfrequenz eines Traps, der für einen Dipol genutzt werden soll, am besten geeignet?}{Ein Frequenzmessgerät}
{Eine SWR-Messbrücke}
{\textbf{\textcolor{DARCgreen}{Ein vektorieller Netzwerk Analysator}}}
{Ein Resonanzwellenmesser}
\end{QQuestion}

}
\end{frame}

\begin{frame}
\only<1>{
\begin{QQuestion}{AI203}{Die Resonanzfrequenz eines abgestimmten HF-Kreises kann mit einem~...}{vektoriellen Netzwerkanalysator (VNA) überprüft werden.}
{Gleichspannungsmessgerät überprüft werden.}
{digitalen Frequenzmessgerät überprüft werden.}
{Ohmmeter überprüft werden.}
\end{QQuestion}

}
\only<2>{
\begin{QQuestion}{AI203}{Die Resonanzfrequenz eines abgestimmten HF-Kreises kann mit einem~...}{\textbf{\textcolor{DARCgreen}{vektoriellen Netzwerkanalysator (VNA) überprüft werden.}}}
{Gleichspannungsmessgerät überprüft werden.}
{digitalen Frequenzmessgerät überprüft werden.}
{Ohmmeter überprüft werden.}
\end{QQuestion}

}
\end{frame}

\begin{frame}
\only<1>{
\begin{QQuestion}{AI204}{Sie haben einen vektoriellen Netzwerkanalysator (VNA) an den Speisepunkt ihrer Kurzwellenantenne angeschlossen. Das Gerät zeigt R = \qty{54}{\ohm} und jX = \qty{-12}{\ohm} an. Was bedeutet das Messergebnis?}{Der ohmsche Anteil der Antennenimpedanz beträgt \qty{54}{\ohm}, der Blindanteil beträgt \qty{12}{\ohm} und ist induktiv.}
{Die Impedanz der Antenne beträgt \qty{66}{\ohm}. Es entsteht eine große induktive Fehlanpassung.}
{Der ohmsche Widerstand der Antennenimpedanz beträgt \qty{54}{\ohm}, der Blindanteil beträgt \qty{12}{\ohm} und ist kapazitiv.}
{Die Antenne ist wegen ihres großen Blindwiderstandes nur zum Empfang, nicht jedoch zum Senden geeignet.}
\end{QQuestion}

}
\only<2>{
\begin{QQuestion}{AI204}{Sie haben einen vektoriellen Netzwerkanalysator (VNA) an den Speisepunkt ihrer Kurzwellenantenne angeschlossen. Das Gerät zeigt R = \qty{54}{\ohm} und jX = \qty{-12}{\ohm} an. Was bedeutet das Messergebnis?}{Der ohmsche Anteil der Antennenimpedanz beträgt \qty{54}{\ohm}, der Blindanteil beträgt \qty{12}{\ohm} und ist induktiv.}
{Die Impedanz der Antenne beträgt \qty{66}{\ohm}. Es entsteht eine große induktive Fehlanpassung.}
{\textbf{\textcolor{DARCgreen}{Der ohmsche Widerstand der Antennenimpedanz beträgt \qty{54}{\ohm}, der Blindanteil beträgt \qty{12}{\ohm} und ist kapazitiv.}}}
{Die Antenne ist wegen ihres großen Blindwiderstandes nur zum Empfang, nicht jedoch zum Senden geeignet.}
\end{QQuestion}

}
\end{frame}

\begin{frame}
\only<1>{
\begin{QQuestion}{AI205}{Sie haben einen vektoriellen Netzwerkanalysator (VNA), der auf den VHF-Bereich eingestellt ist, an den Speisepunkt ihrer VHF-Antenne angeschlossen. Das Gerät zeigt R~=~\qty{50}{\ohm}~und~jX~=~\qty{0}{\ohm} an. Was erkennen Sie aus diesen Werten?}{Der fehlende Blindanteil~(jX) deutet darauf hin, dass die Antenne defekt ist.}
{Die Antenne ist für den Betrieb an einen VHF-Sender mit \qty{50}{\ohm} Ausgangsimpedanz gut angepasst.}
{Die Antenne ist für den Betrieb an einem Sender mit \qty{50}{\ohm} Ausgangsimpedanz schlecht angepasst, da der erforderliche Blindanteil~(jX) von \qty{50}{\ohm} fehlt.}
{Die Antenne ist wegen des fehlenden Blindwiderstandanteils nur zum Empfang, nicht jedoch zum Senden geeignet.}
\end{QQuestion}

}
\only<2>{
\begin{QQuestion}{AI205}{Sie haben einen vektoriellen Netzwerkanalysator (VNA), der auf den VHF-Bereich eingestellt ist, an den Speisepunkt ihrer VHF-Antenne angeschlossen. Das Gerät zeigt R~=~\qty{50}{\ohm}~und~jX~=~\qty{0}{\ohm} an. Was erkennen Sie aus diesen Werten?}{Der fehlende Blindanteil~(jX) deutet darauf hin, dass die Antenne defekt ist.}
{\textbf{\textcolor{DARCgreen}{Die Antenne ist für den Betrieb an einen VHF-Sender mit \qty{50}{\ohm} Ausgangsimpedanz gut angepasst.}}}
{Die Antenne ist für den Betrieb an einem Sender mit \qty{50}{\ohm} Ausgangsimpedanz schlecht angepasst, da der erforderliche Blindanteil~(jX) von \qty{50}{\ohm} fehlt.}
{Die Antenne ist wegen des fehlenden Blindwiderstandanteils nur zum Empfang, nicht jedoch zum Senden geeignet.}
\end{QQuestion}

}
\end{frame}

\begin{frame}
\only<1>{
\begin{QQuestion}{AI206}{Sie haben einen vektoriellen Netzwerkanalysator (VNA) an den Speisepunkt Ihrer Kurzwellenantenne angeschlossen. Das Gerät zeigt R = \qty{54}{\ohm} und jX = \qty{+12}{\ohm} an. Was bedeutet das Messergebnis?}{Der ohmsche Anteil der Antennenimpedanz beträgt \qty{54}{\ohm}, der Blindanteil beträgt \qty{12}{\ohm} und ist induktiv.}
{Die Impedanz der Antenne beträgt \qty{66}{\ohm}. Es entsteht eine große induktive Fehlanpassung.}
{Der ohmsche Widerstand der Antennenimpedanz beträgt \qty{54}{\ohm}, der Blindanteil beträgt \qty{12}{\ohm} und ist kapazitiv.}
{Die Antenne ist wegen ihres großen Blindwiderstandes nur zum Empfang, nicht jedoch zum Senden geeignet.}
\end{QQuestion}

}
\only<2>{
\begin{QQuestion}{AI206}{Sie haben einen vektoriellen Netzwerkanalysator (VNA) an den Speisepunkt Ihrer Kurzwellenantenne angeschlossen. Das Gerät zeigt R = \qty{54}{\ohm} und jX = \qty{+12}{\ohm} an. Was bedeutet das Messergebnis?}{\textbf{\textcolor{DARCgreen}{Der ohmsche Anteil der Antennenimpedanz beträgt \qty{54}{\ohm}, der Blindanteil beträgt \qty{12}{\ohm} und ist induktiv.}}}
{Die Impedanz der Antenne beträgt \qty{66}{\ohm}. Es entsteht eine große induktive Fehlanpassung.}
{Der ohmsche Widerstand der Antennenimpedanz beträgt \qty{54}{\ohm}, der Blindanteil beträgt \qty{12}{\ohm} und ist kapazitiv.}
{Die Antenne ist wegen ihres großen Blindwiderstandes nur zum Empfang, nicht jedoch zum Senden geeignet.}
\end{QQuestion}

}
\end{frame}

\begin{frame}
\only<1>{
\begin{PQuestion}{AI207}{Sie haben einen vektoriellen Netzwerkanalysator (VNA) an einen selbstgebauten Halbwellendipol angeschlossen und messen den dargestellten Resonanzverlauf. Was müssen Sie tun, um diese Antenne auf das \qty{80}{\metre}-Band abzustimmen? }{Sie fügen in beide Strahlerhälften jeweils einen \qty{50}{\ohm}~Widerstand ein}
{Sie verlängern beide Enden gleichmäßig.}
{Sie fügen in beide Strahlerhälften jeweils eine Induktivität ein.}
{Sie verkürzen beide Enden gleichmäßig.}
{\DARCimage{1.0\linewidth}{526include}}\end{PQuestion}

}
\only<2>{
\begin{PQuestion}{AI207}{Sie haben einen vektoriellen Netzwerkanalysator (VNA) an einen selbstgebauten Halbwellendipol angeschlossen und messen den dargestellten Resonanzverlauf. Was müssen Sie tun, um diese Antenne auf das \qty{80}{\metre}-Band abzustimmen? }{Sie fügen in beide Strahlerhälften jeweils einen \qty{50}{\ohm}~Widerstand ein}
{Sie verlängern beide Enden gleichmäßig.}
{Sie fügen in beide Strahlerhälften jeweils eine Induktivität ein.}
{\textbf{\textcolor{DARCgreen}{Sie verkürzen beide Enden gleichmäßig.}}}
{\DARCimage{1.0\linewidth}{526include}}\end{PQuestion}

}
\end{frame}

\begin{frame}
\only<1>{
\begin{PQuestion}{AI208}{Sie haben einen vektoriellen Netzwerkanalysator (VNA) an einen selbstgebauten Halbwellendipol angeschlossen und messen den dargestellten Resonanzverlauf. Was müssen Sie tun, um diese Antenne auf das \qty{80}{\metre}-Band abzustimmen? }{Sie verlängern beide Drahtenden gleichmäßig.}
{Sie verkürzen beide Drahtenden gleichmäßig.}
{Sie fügen in beide Strahlerhälften jeweils eine Kapazität ein.}
{Sie fügen eine Mantelwellensperre ein.}
{\DARCimage{1.0\linewidth}{527include}}\end{PQuestion}

}
\only<2>{
\begin{PQuestion}{AI208}{Sie haben einen vektoriellen Netzwerkanalysator (VNA) an einen selbstgebauten Halbwellendipol angeschlossen und messen den dargestellten Resonanzverlauf. Was müssen Sie tun, um diese Antenne auf das \qty{80}{\metre}-Band abzustimmen? }{\textbf{\textcolor{DARCgreen}{Sie verlängern beide Drahtenden gleichmäßig.}}}
{Sie verkürzen beide Drahtenden gleichmäßig.}
{Sie fügen in beide Strahlerhälften jeweils eine Kapazität ein.}
{Sie fügen eine Mantelwellensperre ein.}
{\DARCimage{1.0\linewidth}{527include}}\end{PQuestion}

}
\end{frame}%ENDCONTENT
