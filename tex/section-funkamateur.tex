
\section{Funkamateur}
\label{section:funkamateur}
\begin{frame}%STARTCONTENT

\frametitle{Funkamateur}
\begin{itemize}
  \item Funkamateure sind ordnungsgemäß befugte Personen, die sich ausschließlich mit persönlichem Ziel und ohne finanzielle Interessen für Funktechnik interessieren.
  \end{itemize}
\begin{itemize}
  \item Im Sinne des Amateurfunkgesetzes (AFuG) ist man Funkamateur, wenn man ein Amateurfunkzeugnis oder eine harmonisierte Amateurfunk-Prüfungsbescheinigung hat.
  \end{itemize}
\end{frame}

\begin{frame}Finanzielle Interessen sind mit dem Amateurfunk nicht vereinbar.

Beispiel: Funkamateure, die gemeinsam bei einem Handwerksbetrieb arbeiten, dürfen Ihre Arbeit nicht über das örtliche Amateurfunkrelais koordinieren.

\end{frame}

\begin{frame}
\only<1>{
\begin{QQuestion}{VA104}{Welche Aussage über Funkamateure enthält die Begriffsbestimmung des Amateurfunkdienstes in den Radio Regulations (RR)?}{Funkamateure sind ordnungsgemäß befugte Personen, die sich ausschließlich mit persönlichem Ziel und ohne finanzielle Interessen für Funktechnik interessieren.}
{Keine, da der Amateurfunkdienst in den Radio Regulations (RR) nicht definiert ist.}
{Funkamateure sind die Inhaber einer Prüfungsbescheinigung und befassen sich mit der Funktechnik aus gewerblich-wirtschaftlichem Interesse.}
{Funkamateure dürfen nur Mitteilungen von geringer Bedeutung übertragen, die es nicht rechtfertigen, öffentliche Telekommunikationsdienste in Anspruch zu nehmen.}
\end{QQuestion}

}
\only<2>{
\begin{QQuestion}{VA104}{Welche Aussage über Funkamateure enthält die Begriffsbestimmung des Amateurfunkdienstes in den Radio Regulations (RR)?}{\textbf{\textcolor{DARCgreen}{Funkamateure sind ordnungsgemäß befugte Personen, die sich ausschließlich mit persönlichem Ziel und ohne finanzielle Interessen für Funktechnik interessieren.}}}
{Keine, da der Amateurfunkdienst in den Radio Regulations (RR) nicht definiert ist.}
{Funkamateure sind die Inhaber einer Prüfungsbescheinigung und befassen sich mit der Funktechnik aus gewerblich-wirtschaftlichem Interesse.}
{Funkamateure dürfen nur Mitteilungen von geringer Bedeutung übertragen, die es nicht rechtfertigen, öffentliche Telekommunikationsdienste in Anspruch zu nehmen.}
\end{QQuestion}

}
\end{frame}

\begin{frame}
\only<1>{
\begin{QQuestion}{VC105}{Welches der nachfolgend genannten Dokumente benötigt man, um Funkamateur im Sinne des Amateurfunkgesetzes (AFuG) zu sein?}{Einen gültigen Personalausweis oder Reisepass, aus dem hervorgeht, dass man seinen Wohnsitz in der Bundesrepublik hat}
{Ein Führungszeugnis aus dem hervorgeht, dass man nicht vorbestraft ist}
{Ein Amateurfunkzeugnis oder eine harmonisierte Amateurfunk-Prüfungsbescheinigung}
{Eine Bescheinigung darüber, dass man erfolgreich am Ausbildungsfunkverkehr in der Bundesrepublik Deutschland teilgenommen hat}
\end{QQuestion}

}
\only<2>{
\begin{QQuestion}{VC105}{Welches der nachfolgend genannten Dokumente benötigt man, um Funkamateur im Sinne des Amateurfunkgesetzes (AFuG) zu sein?}{Einen gültigen Personalausweis oder Reisepass, aus dem hervorgeht, dass man seinen Wohnsitz in der Bundesrepublik hat}
{Ein Führungszeugnis aus dem hervorgeht, dass man nicht vorbestraft ist}
{\textbf{\textcolor{DARCgreen}{Ein Amateurfunkzeugnis oder eine harmonisierte Amateurfunk-Prüfungsbescheinigung}}}
{Eine Bescheinigung darüber, dass man erfolgreich am Ausbildungsfunkverkehr in der Bundesrepublik Deutschland teilgenommen hat}
\end{QQuestion}

}
\end{frame}

\begin{frame}
\only<1>{
\begin{QQuestion}{VC113}{Nach dem Amateurfunkgesetz ist ein Funkamateur der Inhaber eines Amateurfunkzeugnisses oder einer harmonisierten Prüfungsbescheinigung, der sich aus persönlicher Neigung und \underline{nicht}~...}{aus gewerblich-wirtschaftlichem Interesse mit dem Amateurfunkdienst befasst.}
{aus politischem oder religiösem Interesse mit dem Amateurfunkdienst befasst.}
{aus technisch-wissenschaftlichem Interesse mit dem Amateurfunkdienst befasst.}
{aus sozialem oder kulturellem Interesse mit dem Amateurfunkdienst befasst.}
\end{QQuestion}

}
\only<2>{
\begin{QQuestion}{VC113}{Nach dem Amateurfunkgesetz ist ein Funkamateur der Inhaber eines Amateurfunkzeugnisses oder einer harmonisierten Prüfungsbescheinigung, der sich aus persönlicher Neigung und \underline{nicht}~...}{\textbf{\textcolor{DARCgreen}{aus gewerblich-wirtschaftlichem Interesse mit dem Amateurfunkdienst befasst.}}}
{aus politischem oder religiösem Interesse mit dem Amateurfunkdienst befasst.}
{aus technisch-wissenschaftlichem Interesse mit dem Amateurfunkdienst befasst.}
{aus sozialem oder kulturellem Interesse mit dem Amateurfunkdienst befasst.}
\end{QQuestion}

}
\end{frame}%ENDCONTENT
