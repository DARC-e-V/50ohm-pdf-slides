
\section{Scatter}
\label{section:scatter}
\begin{frame}%STARTCONTENT
\begin{itemize}
  \item \emph{Scatter}: Besondere Formen der Reflexion und Streuung eines Funksignals
  \item Damit können größere Entfernungen überbrückt werden
  \end{itemize}
\end{frame}

\begin{frame}
\frametitle{Regenscatter}
\begin{itemize}
  \item Englisch \emph{Rainscatter}
  \item Streuung an Regentropfen in alle Richtungen (Rayleigh-Streuung)
  \item Tropfengröße muss zur Wellenlänge passen: 6- und 3-cm-Band
  \item Antenne wird auf Regenwolke gehalten
  \item Rauer Ton in SSB- und CW-Signalen (ähnlich Aurora)
  \end{itemize}

\end{frame}

\begin{frame}
\only<1>{
\begin{QQuestion}{AH311}{Um welche Art von Überreichweiten handelt es sich bei Regenscatter (Rainscatter)?}{Streuungen von Mikrowellen, insbesondere im \qty{23}{\cm}-Band, an Regentropfen und Hagelkörnern.}
{Reflexionen in den VHF- und UHF-Bereichen an größeren Regentropfen.}
{Streuungen von Mikrowellen, insbesondere im \qty{3}{\cm}-Band, an Regen- und Gewitterwolken.}
{Reflexionen im \qty{13}{\cm}-Band bei Eisregen.}
\end{QQuestion}

}
\only<2>{
\begin{QQuestion}{AH311}{Um welche Art von Überreichweiten handelt es sich bei Regenscatter (Rainscatter)?}{Streuungen von Mikrowellen, insbesondere im \qty{23}{\cm}-Band, an Regentropfen und Hagelkörnern.}
{Reflexionen in den VHF- und UHF-Bereichen an größeren Regentropfen.}
{\textbf{\textcolor{DARCgreen}{Streuungen von Mikrowellen, insbesondere im \qty{3}{\cm}-Band, an Regen- und Gewitterwolken.}}}
{Reflexionen im \qty{13}{\cm}-Band bei Eisregen.}
\end{QQuestion}

}
\end{frame}

\begin{frame}
\frametitle{Backscatter}
\begin{itemize}
  \item Brechung der Raumwelle zurück zum Empfänger
  \item Vor allem während der Dämmerung
  \item Starke und schnell schwankende Signalstärke (\emph{Flatterfading}, flutter fading)
  \end{itemize}
\end{frame}

\begin{frame}
\only<1>{
\begin{QQuestion}{AH223}{Was ist für ein \glqq Backscatter-Signal\grqq{} charakteristisch?}{hohe Signalstärken}
{Pfeif- und Knattergeräusche}
{schnelle, unregelmäßige Feldstärkeschwankungen (Flatterfading)}
{breitbandiges Rauschen}
\end{QQuestion}

}
\only<2>{
\begin{QQuestion}{AH223}{Was ist für ein \glqq Backscatter-Signal\grqq{} charakteristisch?}{hohe Signalstärken}
{Pfeif- und Knattergeräusche}
{\textbf{\textcolor{DARCgreen}{schnelle, unregelmäßige Feldstärkeschwankungen (Flatterfading)}}}
{breitbandiges Rauschen}
\end{QQuestion}

}
\end{frame}

\begin{frame}
\frametitle{Aircraft-Scatter}
\begin{itemize}
  \item Reflexion (also eigentlich kein Scatter) von VHF, UHF und SHF an Flugzeugen
  \item Flugzeug muss auf Verbindungslinie zwischen Sender und Empfänger sein
  \item Recht kurze Verbindung aufgrund der schnellen Bewegung des Flugzeugs
  \end{itemize}

\end{frame}

\begin{frame}
\only<1>{
\begin{QQuestion}{AH310}{Was versteht man unter Aircraft-Scatter (AS)?}{Überhorizontverbindungen im VHF- und UHF-Bereich durch Reflexionen an Funkfeuern.}
{Das Beobachten des Funkverkehrs von Flugzeugen mit Hilfe von Amateurfunkgeräten und Antennen.}
{Überhorizontverbindungen im VHF-, UHF- und SHF-Bereich durch Reflexion an Flugzeugen.}
{Betrieb einer Amateurfunkstelle an Bord eines Flugzeuges.}
\end{QQuestion}

}
\only<2>{
\begin{QQuestion}{AH310}{Was versteht man unter Aircraft-Scatter (AS)?}{Überhorizontverbindungen im VHF- und UHF-Bereich durch Reflexionen an Funkfeuern.}
{Das Beobachten des Funkverkehrs von Flugzeugen mit Hilfe von Amateurfunkgeräten und Antennen.}
{\textbf{\textcolor{DARCgreen}{Überhorizontverbindungen im VHF-, UHF- und SHF-Bereich durch Reflexion an Flugzeugen.}}}
{Betrieb einer Amateurfunkstelle an Bord eines Flugzeuges.}
\end{QQuestion}

}
\end{frame}%ENDCONTENT
