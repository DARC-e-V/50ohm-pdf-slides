
\section{Verstärkerklassen}
\label{section:verstaerker_klasse}
\begin{frame}%STARTCONTENT

\only<1>{
\begin{PQuestion}{AD416}{Das folgende Bild zeigt eine idealisierte Steuerkennlinie eines Transistors mit vier eingezeichneten Arbeitspunkten $\text{AP}_1$ bis $\text{AP}_4$.  Welcher Arbeitspunkt ist welcher Verstärkerbetriebsart zuzuordnen?}{$\text{AP}_1$ ist kein geeigneter Verstärkerarbeitspunkt, $\text{AP}_2$ entspricht A-Betrieb, $\text{AP}_3$ entspricht B-Betrieb, $\text{AP}_4$ entspricht C-Betrieb.}
{$\text{AP}_1$ ist kein geeigneter Verstärkerarbeitspunkt, $\text{AP}_2$ entspricht C-Betrieb, $\text{AP}_3$ entspricht B-Betrieb, $\text{AP}_4$ entspricht A-Betrieb.}
{$\text{AP}_1$ entspricht C-Betrieb, $\text{AP}_2$ entspricht B-Betrieb, $\text{AP}_3$ entspricht AB-Betrieb, $\text{AP}_4$ entspricht A-Betrieb.}
{$\text{AP}_1$ entspricht A-Betrieb, $\text{AP}_2$ entspricht AB-Betrieb, $\text{AP}_3$ entspricht B-Betrieb, $\text{AP}_4$ entspricht C-Betrieb.}
{\DARCimage{0.75\linewidth}{377include}}\end{PQuestion}

}
\only<2>{
\begin{PQuestion}{AD416}{Das folgende Bild zeigt eine idealisierte Steuerkennlinie eines Transistors mit vier eingezeichneten Arbeitspunkten $\text{AP}_1$ bis $\text{AP}_4$.  Welcher Arbeitspunkt ist welcher Verstärkerbetriebsart zuzuordnen?}{$\text{AP}_1$ ist kein geeigneter Verstärkerarbeitspunkt, $\text{AP}_2$ entspricht A-Betrieb, $\text{AP}_3$ entspricht B-Betrieb, $\text{AP}_4$ entspricht C-Betrieb.}
{$\text{AP}_1$ ist kein geeigneter Verstärkerarbeitspunkt, $\text{AP}_2$ entspricht C-Betrieb, $\text{AP}_3$ entspricht B-Betrieb, $\text{AP}_4$ entspricht A-Betrieb.}
{\textbf{\textcolor{DARCgreen}{$\text{AP}_1$ entspricht C-Betrieb, $\text{AP}_2$ entspricht B-Betrieb, $\text{AP}_3$ entspricht AB-Betrieb, $\text{AP}_4$ entspricht A-Betrieb.}}}
{$\text{AP}_1$ entspricht A-Betrieb, $\text{AP}_2$ entspricht AB-Betrieb, $\text{AP}_3$ entspricht B-Betrieb, $\text{AP}_4$ entspricht C-Betrieb.}
{\DARCimage{0.75\linewidth}{377include}}\end{PQuestion}

}
\end{frame}

\begin{frame}
\only<1>{
\begin{QQuestion}{AD419}{Welche Merkmale hat ein HF-Leistungsverstärker im A-Betrieb?}{Wirkungsgrad \qtyrange{80}{87}{\percent}, hoher Oberschwingungsanteil, der Ruhestrom ist null.}
{Wirkungsgrad bis zu \qty{70}{\percent}, geringer Oberschwingungsanteil, geringer bis mittlerer Ruhestrom.}
{Wirkungsgrad bis zu \qty{80}{\percent}, geringer Oberschwingungsanteil, sehr geringer Ruhestrom.}
{Wirkungsgrad ca. \qty{40}{\percent}, sehr geringer Oberschwingungsanteil, hoher Ruhestrom.}
\end{QQuestion}

}
\only<2>{
\begin{QQuestion}{AD419}{Welche Merkmale hat ein HF-Leistungsverstärker im A-Betrieb?}{Wirkungsgrad \qtyrange{80}{87}{\percent}, hoher Oberschwingungsanteil, der Ruhestrom ist null.}
{Wirkungsgrad bis zu \qty{70}{\percent}, geringer Oberschwingungsanteil, geringer bis mittlerer Ruhestrom.}
{Wirkungsgrad bis zu \qty{80}{\percent}, geringer Oberschwingungsanteil, sehr geringer Ruhestrom.}
{\textbf{\textcolor{DARCgreen}{Wirkungsgrad ca. \qty{40}{\percent}, sehr geringer Oberschwingungsanteil, hoher Ruhestrom.}}}
\end{QQuestion}

}
\end{frame}

\begin{frame}
\only<1>{
\begin{QQuestion}{AD420}{Welche Merkmale hat ein HF-Leistungsverstärker im B-Betrieb?}{Wirkungsgrad ca. \qty{40}{\percent}, sehr geringer Oberschwingungsanteil, hoher Ruhestrom.}
{Wirkungsgrad bis zu \qty{70}{\percent}, geringer Oberschwingungsanteil, geringer bis mittlerer Ruhestrom.}
{Wirkungsgrad bis zu \qty{80}{\percent}, geringer Oberschwingungsanteil, sehr geringer Ruhestrom.}
{Wirkungsgrad \qtyrange{80}{87}{\percent}, hoher Oberschwingungsanteil, der Ruhestrom ist null.}
\end{QQuestion}

}
\only<2>{
\begin{QQuestion}{AD420}{Welche Merkmale hat ein HF-Leistungsverstärker im B-Betrieb?}{Wirkungsgrad ca. \qty{40}{\percent}, sehr geringer Oberschwingungsanteil, hoher Ruhestrom.}
{Wirkungsgrad bis zu \qty{70}{\percent}, geringer Oberschwingungsanteil, geringer bis mittlerer Ruhestrom.}
{\textbf{\textcolor{DARCgreen}{Wirkungsgrad bis zu \qty{80}{\percent}, geringer Oberschwingungsanteil, sehr geringer Ruhestrom.}}}
{Wirkungsgrad \qtyrange{80}{87}{\percent}, hoher Oberschwingungsanteil, der Ruhestrom ist null.}
\end{QQuestion}

}
\end{frame}

\begin{frame}
\only<1>{
\begin{QQuestion}{AD421}{Welche Merkmale hat ein HF-Leistungsverstärker im C-Betrieb?}{Wirkungsgrad ca. \qty{40}{\percent}, sehr geringer Oberschwingungsanteil, hoher Ruhestrom.}
{Wirkungsgrad bis zu \qty{70}{\percent}, geringer Oberschwingungsanteil, geringer bis mittlerer Ruhestrom.}
{Wirkungsgrad bis zu \qty{80}{\percent}, geringer Oberschwingungsanteil, sehr geringer Ruhestrom.}
{Wirkungsgrad \qtyrange{80}{87}{\percent}, hoher Oberschwingungsanteil, der Ruhestrom ist null.}
\end{QQuestion}

}
\only<2>{
\begin{QQuestion}{AD421}{Welche Merkmale hat ein HF-Leistungsverstärker im C-Betrieb?}{Wirkungsgrad ca. \qty{40}{\percent}, sehr geringer Oberschwingungsanteil, hoher Ruhestrom.}
{Wirkungsgrad bis zu \qty{70}{\percent}, geringer Oberschwingungsanteil, geringer bis mittlerer Ruhestrom.}
{Wirkungsgrad bis zu \qty{80}{\percent}, geringer Oberschwingungsanteil, sehr geringer Ruhestrom.}
{\textbf{\textcolor{DARCgreen}{Wirkungsgrad \qtyrange{80}{87}{\percent}, hoher Oberschwingungsanteil, der Ruhestrom ist null.}}}
\end{QQuestion}

}
\end{frame}

\begin{frame}
\only<1>{
\begin{QQuestion}{AD424}{Ein HF-Leistungsverstärker im A-Betrieb wird mit einer Drainspannung von \qty{50}{\volt} und einem Drainstrom von \qty{2}{\ampere} betrieben. Wie hoch ist die zu erwartende Ausgangsleistung des Verstärkers?}{$\approx$ \qty{85}{\W}}
{$\approx$ \qty{40}{\W}}
{$\approx$ \qty{60}{\W}}
{$\approx$ \qty{75}{\W}}
\end{QQuestion}

}
\only<2>{
\begin{QQuestion}{AD424}{Ein HF-Leistungsverstärker im A-Betrieb wird mit einer Drainspannung von \qty{50}{\volt} und einem Drainstrom von \qty{2}{\ampere} betrieben. Wie hoch ist die zu erwartende Ausgangsleistung des Verstärkers?}{$\approx$ \qty{85}{\W}}
{\textbf{\textcolor{DARCgreen}{$\approx$ \qty{40}{\W}}}}
{$\approx$ \qty{60}{\W}}
{$\approx$ \qty{75}{\W}}
\end{QQuestion}

}
\end{frame}

\begin{frame}
\frametitle{Lösungsweg}
\begin{itemize}
  \item gegeben: $U=50V$
  \item gegeben: $I = 2A$
  \item gegeben: $\eta_A \approx 40\%$
  \item gesucht: $P_{ab}$
  \end{itemize}
    \pause
    $P_{zu} = U \cdot I = 50V \cdot 2A = 100W$
    \pause
    $\eta_A = \frac{P_{ab}}{P_{zu}} \Rightarrow P_{ab} = \eta_A \cdot P_{zu} = 0,4 \cdot 100W = 40W$



\end{frame}

\begin{frame}
\only<1>{
\begin{QQuestion}{AD425}{Ein HF-Leistungsverstärker im C-Betrieb wird mit einer Drainspannung von \qty{50}{\V} und einem Drainstrom von \qty{2}{\A} betrieben. Wie hoch ist die zu erwartende Ausgangsleistung des Verstärkers?}{$\approx$ \qty{85}{\W}}
{$\approx$ \qty{70}{\W}}
{$\approx$ \qty{60}{\W}}
{$\approx$ \qty{40}{\W}}
\end{QQuestion}

}
\only<2>{
\begin{QQuestion}{AD425}{Ein HF-Leistungsverstärker im C-Betrieb wird mit einer Drainspannung von \qty{50}{\V} und einem Drainstrom von \qty{2}{\A} betrieben. Wie hoch ist die zu erwartende Ausgangsleistung des Verstärkers?}{\textbf{\textcolor{DARCgreen}{$\approx$ \qty{85}{\W}}}}
{$\approx$ \qty{70}{\W}}
{$\approx$ \qty{60}{\W}}
{$\approx$ \qty{40}{\W}}
\end{QQuestion}

}
\end{frame}

\begin{frame}
\frametitle{Lösungsweg}
\begin{itemize}
  \item gegeben: $U=50V$
  \item gegeben: $I = 2A$
  \item gegeben: $\eta_C \approx 85\%$
  \item gesucht: $P_{ab}$
  \end{itemize}
    \pause
    $P_{zu} = U \cdot I = 50V \cdot 2A = 100W$
    \pause
    $\eta_C = \frac{P_{ab}}{P_{zu}} \Rightarrow P_{ab} = \eta_C \cdot P_{zu} = 0,85 \cdot 100W = 85W$



\end{frame}

\begin{frame}
\only<1>{
\begin{QQuestion}{AD418}{In welcher Größenordnung liegt der Ruhestrom eines HF-Leistungsverstärkers im C-Betrieb?}{Bei etwa \qtyrange{70}{80}{\percent} des Stromes bei Nennleistung}
{Bei etwa \qtyrange{10}{20}{\percent} des Stromes bei Nennleistung}
{Bei null Ampere}
{Bei fast \qty{100}{\percent} des Stromes bei Nennleistung}
\end{QQuestion}

}
\only<2>{
\begin{QQuestion}{AD418}{In welcher Größenordnung liegt der Ruhestrom eines HF-Leistungsverstärkers im C-Betrieb?}{Bei etwa \qtyrange{70}{80}{\percent} des Stromes bei Nennleistung}
{Bei etwa \qtyrange{10}{20}{\percent} des Stromes bei Nennleistung}
{\textbf{\textcolor{DARCgreen}{Bei null Ampere}}}
{Bei fast \qty{100}{\percent} des Stromes bei Nennleistung}
\end{QQuestion}

}
\end{frame}

\begin{frame}
\only<1>{
\begin{QQuestion}{AD417}{Wie verhält sich der Kollektorstrom eines NPN-Transistors in einer HF-Verstärkerstufe im B-Betrieb, wenn die Basis-Emitterspannung erhöht wird?}{Er bleibt konstant.}
{Er verringert sich geringfügig.}
{Er nimmt erheblich zu.}
{Er nimmt erheblich ab.}
\end{QQuestion}

}
\only<2>{
\begin{QQuestion}{AD417}{Wie verhält sich der Kollektorstrom eines NPN-Transistors in einer HF-Verstärkerstufe im B-Betrieb, wenn die Basis-Emitterspannung erhöht wird?}{Er bleibt konstant.}
{Er verringert sich geringfügig.}
{\textbf{\textcolor{DARCgreen}{Er nimmt erheblich zu.}}}
{Er nimmt erheblich ab.}
\end{QQuestion}

}
\end{frame}

\begin{frame}
\only<1>{
\begin{QQuestion}{AD422}{In welchem Arbeitspunkt kann ein HF-Leistungsverstärker für einen SSB-Sender betrieben werden?}{A-, AB- oder B-Betrieb}
{AB-, B- oder C-Betrieb}
{B- oder C-Betrieb}
{A-, AB-, B- oder C-Betrieb}
\end{QQuestion}

}
\only<2>{
\begin{QQuestion}{AD422}{In welchem Arbeitspunkt kann ein HF-Leistungsverstärker für einen SSB-Sender betrieben werden?}{\textbf{\textcolor{DARCgreen}{A-, AB- oder B-Betrieb}}}
{AB-, B- oder C-Betrieb}
{B- oder C-Betrieb}
{A-, AB-, B- oder C-Betrieb}
\end{QQuestion}

}
\end{frame}

\begin{frame}
\only<1>{
\begin{QQuestion}{AJ218}{In welcher Arbeitspunkteinstellung darf die Endstufe eines SSB-Senders \underline{nicht} betrieben werden?}{A-Betrieb}
{C-Betrieb}
{B-Betrieb}
{AB-Betrieb}
\end{QQuestion}

}
\only<2>{
\begin{QQuestion}{AJ218}{In welcher Arbeitspunkteinstellung darf die Endstufe eines SSB-Senders \underline{nicht} betrieben werden?}{A-Betrieb}
{\textbf{\textcolor{DARCgreen}{C-Betrieb}}}
{B-Betrieb}
{AB-Betrieb}
\end{QQuestion}

}
\end{frame}

\begin{frame}
\only<1>{
\begin{QQuestion}{AD423}{Wenn ein linearer HF-Leistungsverstärker im AB-Betrieb durch ein SSB-Signal übersteuert wird, führt dies zu~...}{Splatter auf benachbarten Frequenzen.}
{parasitären Schwingungen des Verstärkers.}
{Frequenzsprüngen in der Sendefrequenz.}
{Chirp im Sendesignal.}
\end{QQuestion}

}
\only<2>{
\begin{QQuestion}{AD423}{Wenn ein linearer HF-Leistungsverstärker im AB-Betrieb durch ein SSB-Signal übersteuert wird, führt dies zu~...}{\textbf{\textcolor{DARCgreen}{Splatter auf benachbarten Frequenzen.}}}
{parasitären Schwingungen des Verstärkers.}
{Frequenzsprüngen in der Sendefrequenz.}
{Chirp im Sendesignal.}
\end{QQuestion}

}
\end{frame}

\begin{frame}
\only<1>{
\begin{QQuestion}{AF402}{Welcher Arbeitspunkt der Leistungsverstärkerstufe eines Senders erzeugt grundsätzlich den größten Oberschwingungsanteil?}{A-Betrieb}
{B-Betrieb}
{AB-Betrieb}
{C-Betrieb}
\end{QQuestion}

}
\only<2>{
\begin{QQuestion}{AF402}{Welcher Arbeitspunkt der Leistungsverstärkerstufe eines Senders erzeugt grundsätzlich den größten Oberschwingungsanteil?}{A-Betrieb}
{B-Betrieb}
{AB-Betrieb}
{\textbf{\textcolor{DARCgreen}{C-Betrieb}}}
\end{QQuestion}

}
\end{frame}

\begin{frame}
\only<1>{
\begin{QQuestion}{AF403}{Welche Maßnahmen sind für Ausgangsanpassschaltung und Ausgangsfilter eines HF-Verstärkers im C-Betrieb vorzunehmen? Beide müssen...}{in einem gut abschirmenden Metallgehäuse untergebracht werden.}
{in einem gut isolierten Kunststoffgehäuse untergebracht werden. }
{vor dem Verstärker eingebaut werden.}
{direkt an der Antenne befestigt werden.}
\end{QQuestion}

}
\only<2>{
\begin{QQuestion}{AF403}{Welche Maßnahmen sind für Ausgangsanpassschaltung und Ausgangsfilter eines HF-Verstärkers im C-Betrieb vorzunehmen? Beide müssen...}{\textbf{\textcolor{DARCgreen}{in einem gut abschirmenden Metallgehäuse untergebracht werden.}}}
{in einem gut isolierten Kunststoffgehäuse untergebracht werden. }
{vor dem Verstärker eingebaut werden.}
{direkt an der Antenne befestigt werden.}
\end{QQuestion}

}
\end{frame}%ENDCONTENT
