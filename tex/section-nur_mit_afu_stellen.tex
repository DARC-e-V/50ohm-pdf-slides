
\section{Funkverkehr nur mit Funkamateuren}
\label{section:nur_mit_afu_stellen}
\begin{frame}%STARTCONTENT

\frametitle{Funkverkehr nur mit Funkamateuren}
Eine Amateurfunkstation darf nur andere Amateurfunkstationen kontaktieren.

Es ist unzulässig, mit Funkstellen anderer Funkdienste zu funken.

\end{frame}

\begin{frame}
\only<1>{
\begin{QQuestion}{VC111}{Mit welchen Funkstellen darf der Funkamateur Funkverkehr abwickeln?}{Mit allen Funkstellen, die auf den Amateurfunkbändern tätig sind}
{Ausschließlich mit anderen Amateurfunkstellen}
{Mit anderen Amateurfunkstellen und Funkstellen der Behörden und Organisationen mit Sicherheitsaufgaben (BOS)}
{Mit anderen Amateurfunkstellen und Funkstellen des Flug- und/oder Seefunkdienstes}
\end{QQuestion}

}
\only<2>{
\begin{QQuestion}{VC111}{Mit welchen Funkstellen darf der Funkamateur Funkverkehr abwickeln?}{Mit allen Funkstellen, die auf den Amateurfunkbändern tätig sind}
{\textbf{\textcolor{DARCgreen}{Ausschließlich mit anderen Amateurfunkstellen}}}
{Mit anderen Amateurfunkstellen und Funkstellen der Behörden und Organisationen mit Sicherheitsaufgaben (BOS)}
{Mit anderen Amateurfunkstellen und Funkstellen des Flug- und/oder Seefunkdienstes}
\end{QQuestion}

}
\end{frame}

\begin{frame}
\only<1>{
\begin{QQuestion}{VD703}{Unter welchen Voraussetzungen darf ein Funkamateur mit seinem Amateurfunkgerät Funkverkehr im CB-Funk-Bereich durchführen?}{Der Funkamateur darf mit seiner Amateurfunkstelle unter keinen Umständen im CB-Funk-Bereich senden.}
{Wenn das Amateurfunkgerät vom Funkamateur so eingestellt wurde, dass die technischen Vorschriften für CB-Funkgeräte eingehalten werden}
{Wenn eine Genehmigung zum Betrieb von CB-Funkgeräten vorliegt}
{Wenn die Sendeleistung auf \qty{4}{\W} ERP bei FM und AM bzw. \qty{12}{\W} PEP bei SSB begrenzt wird}
\end{QQuestion}

}
\only<2>{
\begin{QQuestion}{VD703}{Unter welchen Voraussetzungen darf ein Funkamateur mit seinem Amateurfunkgerät Funkverkehr im CB-Funk-Bereich durchführen?}{\textbf{\textcolor{DARCgreen}{Der Funkamateur darf mit seiner Amateurfunkstelle unter keinen Umständen im CB-Funk-Bereich senden.}}}
{Wenn das Amateurfunkgerät vom Funkamateur so eingestellt wurde, dass die technischen Vorschriften für CB-Funkgeräte eingehalten werden}
{Wenn eine Genehmigung zum Betrieb von CB-Funkgeräten vorliegt}
{Wenn die Sendeleistung auf \qty{4}{\W} ERP bei FM und AM bzw. \qty{12}{\W} PEP bei SSB begrenzt wird}
\end{QQuestion}

}
\end{frame}

\begin{frame}
\frametitle{Nachrichtenübermittlung}
Darüber hinaus ist es auch grundsätzlich unzulässig, Nachrichten von oder an Nicht-Funkamateure zu übermitteln.

Die einzige Ausnahme sind Not- und Katastrophenfälle. Dann ist es erlaubt Nachrichten von und an Nicht-Funkamateure zu senden.

\end{frame}

\begin{frame}
\only<1>{
\begin{QQuestion}{VC112}{Darf ein Funkamateur Nachrichten, die nicht den Amateurfunkdienst betreffen, für und an Dritte übermitteln?}{Ja, jederzeit}
{Nein, unter keinen Umständen}
{Nur in Not- und Katastrophenfällen}
{Nur nach Aufforderung durch die zuständige Außenstelle der Bundesnetzagentur}
\end{QQuestion}

}
\only<2>{
\begin{QQuestion}{VC112}{Darf ein Funkamateur Nachrichten, die nicht den Amateurfunkdienst betreffen, für und an Dritte übermitteln?}{Ja, jederzeit}
{Nein, unter keinen Umständen}
{\textbf{\textcolor{DARCgreen}{Nur in Not- und Katastrophenfällen}}}
{Nur nach Aufforderung durch die zuständige Außenstelle der Bundesnetzagentur}
\end{QQuestion}

}
\end{frame}%ENDCONTENT
