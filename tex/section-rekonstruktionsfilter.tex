
\section{Rekonstruktionsfilter}
\label{section:rekonstruktionsfilter}
\begin{frame}%STARTCONTENT

\only<1>{
\begin{QQuestion}{AF624}{Welcher Filtertyp ist als Rekonstruktionsfilter geeignet und wo ist das Filter zu platzieren?}{Tiefpassfilter nach dem D/A-Umsetzer}
{Hochpassfilter nach dem D/A-Umsetzer}
{Tiefpassfilter vor dem A/D-Umsetzer}
{Hochpassfilter vor dem A/D-Umsetzer}
\end{QQuestion}

}
\only<2>{
\begin{QQuestion}{AF624}{Welcher Filtertyp ist als Rekonstruktionsfilter geeignet und wo ist das Filter zu platzieren?}{\textbf{\textcolor{DARCgreen}{Tiefpassfilter nach dem D/A-Umsetzer}}}
{Hochpassfilter nach dem D/A-Umsetzer}
{Tiefpassfilter vor dem A/D-Umsetzer}
{Hochpassfilter vor dem A/D-Umsetzer}
\end{QQuestion}

}
\end{frame}

\begin{frame}
\only<1>{
\begin{question2x2}{AF625}{Sie wollen ein Sprachsignal mit einer Abtastrate von $f_{\symup{A}}$ = 8000 Samples je Sekunde rekonstruieren. Nach dem D/A-Umsetzer soll ein Rekonstruktionsfilter eingesetzt werden. Welcher Amplitudengang ist für das Filter am besten geeignet?}{\DARCimage{1.0\linewidth}{401include}}
{\DARCimage{1.0\linewidth}{655include}}
{\DARCimage{1.0\linewidth}{402include}}
{\DARCimage{1.0\linewidth}{403include}}
\end{question2x2}

}
\only<2>{
\begin{question2x2}{AF625}{Sie wollen ein Sprachsignal mit einer Abtastrate von $f_{\symup{A}}$ = 8000 Samples je Sekunde rekonstruieren. Nach dem D/A-Umsetzer soll ein Rekonstruktionsfilter eingesetzt werden. Welcher Amplitudengang ist für das Filter am besten geeignet?}{\DARCimage{1.0\linewidth}{401include}}
{\textbf{\textcolor{DARCgreen}{\DARCimage{1.0\linewidth}{655include}}}}
{\DARCimage{1.0\linewidth}{402include}}
{\DARCimage{1.0\linewidth}{403include}}
\end{question2x2}

}
\end{frame}%ENDCONTENT
