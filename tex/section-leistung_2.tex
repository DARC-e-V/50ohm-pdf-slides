
\section{Leistung II}
\label{section:leistung_2}
\begin{frame}%STARTCONTENT

\frametitle{Leistungsberechnung}
Wir kennen bereits

$P = U\cdot I = \dfrac{U^2}{R} = I^2\cdot R$
\begin{columns}
    \begin{column}{0.48\textwidth}
    Nach U umgestellt:

$U = \dfrac{P}{I} = \sqrt{P \cdot R}$


    \end{column}
   \begin{column}{0.48\textwidth}
       Nach I umgestellt:

$I = \dfrac{P}{U} = \sqrt{\dfrac{P}{R}}$


   \end{column}
\end{columns}

\end{frame}

\begin{frame}
\only<1>{
\begin{QQuestion}{EB505}{In welcher Antwort sind alle dargestellten Zusammenhänge zwischen Strom, Spannung, Widerstand und Leistung richtig?}{$I = \dfrac{\sqrt{P}}{R};\quad U = \sqrt{P}\cdot R$}
{$I = \sqrt{P\cdot R};\quad U = \sqrt{\dfrac{P}{R}}$}
{$I = \sqrt{\dfrac{R}{P}};\quad U = \sqrt{P\cdot R}$}
{$I = \sqrt{\dfrac{P}{R}};\quad U = \sqrt{P\cdot R}$}
\end{QQuestion}

}
\only<2>{
\begin{QQuestion}{EB505}{In welcher Antwort sind alle dargestellten Zusammenhänge zwischen Strom, Spannung, Widerstand und Leistung richtig?}{$I = \dfrac{\sqrt{P}}{R};\quad U = \sqrt{P}\cdot R$}
{$I = \sqrt{P\cdot R};\quad U = \sqrt{\dfrac{P}{R}}$}
{$I = \sqrt{\dfrac{R}{P}};\quad U = \sqrt{P\cdot R}$}
{\textbf{\textcolor{DARCgreen}{$I = \sqrt{\dfrac{P}{R}};\quad U = \sqrt{P\cdot R}$}}}
\end{QQuestion}

}
\end{frame}

\begin{frame}
\only<1>{
\begin{QQuestion}{EB506}{In welcher Antwort sind alle dargestellten Zusammenhänge zwischen Widerstand, Leistung, Spannung und Strom richtig?}{$R = \dfrac{U^2}{P};\quad R = \dfrac{P}{I^2}$}
{$R = U^2\cdot I;\quad R = \dfrac{P}{I^2}$}
{$R = \dfrac{P}{U^2};\quad R = P\cdot I^2$}
{$R = \dfrac{U^2}{P};\quad R = P\cdot I^2$}
\end{QQuestion}

}
\only<2>{
\begin{QQuestion}{EB506}{In welcher Antwort sind alle dargestellten Zusammenhänge zwischen Widerstand, Leistung, Spannung und Strom richtig?}{\textbf{\textcolor{DARCgreen}{$R = \dfrac{U^2}{P};\quad R = \dfrac{P}{I^2}$}}}
{$R = U^2\cdot I;\quad R = \dfrac{P}{I^2}$}
{$R = \dfrac{P}{U^2};\quad R = P\cdot I^2$}
{$R = \dfrac{U^2}{P};\quad R = P\cdot I^2$}
\end{QQuestion}

}
\end{frame}

\begin{frame}
\only<1>{
\begin{QQuestion}{EB504}{An einem Widerstand $R$ wird die elektrische Leistung $P$ in Wärme umgesetzt. Sie kennen die Größen $P$ und $R$. Nach welcher der Formeln können Sie die Spannung ermitteln, die an dem Widerstand $R$ anliegt?}{$U = \dfrac{P}{R}$}
{$U = R\cdot P$}
{$U = \sqrt{\dfrac{P}{R}}$}
{$U = \sqrt{P\cdot R}$}
\end{QQuestion}

}
\only<2>{
\begin{QQuestion}{EB504}{An einem Widerstand $R$ wird die elektrische Leistung $P$ in Wärme umgesetzt. Sie kennen die Größen $P$ und $R$. Nach welcher der Formeln können Sie die Spannung ermitteln, die an dem Widerstand $R$ anliegt?}{$U = \dfrac{P}{R}$}
{$U = R\cdot P$}
{$U = \sqrt{\dfrac{P}{R}}$}
{\textbf{\textcolor{DARCgreen}{$U = \sqrt{P\cdot R}$}}}
\end{QQuestion}

}
\end{frame}

\begin{frame}
\only<1>{
\begin{QQuestion}{EB507}{Der Effektivwert der Spannung an einer künstlichen \qty{50}{\ohm}-Antenne wird mit \qty{100}{\V} gemessen. Die Leistung an der Last beträgt~...}{\qty{100}{\W}.}
{\qty{50}{\W}.}
{\qty{200}{\W}.}
{\qty{400}{\W}.}
\end{QQuestion}

}
\only<2>{
\begin{QQuestion}{EB507}{Der Effektivwert der Spannung an einer künstlichen \qty{50}{\ohm}-Antenne wird mit \qty{100}{\V} gemessen. Die Leistung an der Last beträgt~...}{\qty{100}{\W}.}
{\qty{50}{\W}.}
{\textbf{\textcolor{DARCgreen}{\qty{200}{\W}.}}}
{\qty{400}{\W}.}
\end{QQuestion}

}
\end{frame}

\begin{frame}
\only<1>{
\begin{QQuestion}{EB508}{Wieviel Leistung wird an einer künstlichen \qty{50}{\ohm}-Antenne umgesetzt, wenn ein effektiver Strom von \qty{2}{\A} fließt?}{\qty{25}{\W}}
{\qty{100}{\W}}
{\qty{200}{\W}}
{\qty{250}{\W}}
\end{QQuestion}

}
\only<2>{
\begin{QQuestion}{EB508}{Wieviel Leistung wird an einer künstlichen \qty{50}{\ohm}-Antenne umgesetzt, wenn ein effektiver Strom von \qty{2}{\A} fließt?}{\qty{25}{\W}}
{\qty{100}{\W}}
{\textbf{\textcolor{DARCgreen}{\qty{200}{\W}}}}
{\qty{250}{\W}}
\end{QQuestion}

}
\end{frame}

\begin{frame}
\only<1>{
\begin{QQuestion}{EB509}{Für welche Leistung muss ein \qty{100}{\ohm}-Widerstand mindestens ausgelegt sein, wenn an ihm \qty{10}{\V} abfallen sollen?}{\qty{10,0}{\W}}
{\qty{1,00}{\W}}
{\qty{0,01}{\W}}
{\qty{0,10}{\W}}
\end{QQuestion}

}
\only<2>{
\begin{QQuestion}{EB509}{Für welche Leistung muss ein \qty{100}{\ohm}-Widerstand mindestens ausgelegt sein, wenn an ihm \qty{10}{\V} abfallen sollen?}{\qty{10,0}{\W}}
{\textbf{\textcolor{DARCgreen}{\qty{1,00}{\W}}}}
{\qty{0,01}{\W}}
{\qty{0,10}{\W}}
\end{QQuestion}

}
\end{frame}

\begin{frame}
\only<1>{
\begin{QQuestion}{EB510}{Ein Widerstand von \qty{10}{\kohm} hat eine maximale Spannungsfestigkeit von \qty{700}{\V} und eine maximale Belastbarkeit von \qty{1}{\W}. Welche Gleichspannung darf höchstens an den Widerstand angelegt werden, um ihn im spezifizierten Bereich zu betreiben?}{\qty{775}{\V}}
{\qty{0,01}{\kV}}
{\qty{0,7}{\kV}}
{\qty{100}{\V}}
\end{QQuestion}

}
\only<2>{
\begin{QQuestion}{EB510}{Ein Widerstand von \qty{10}{\kohm} hat eine maximale Spannungsfestigkeit von \qty{700}{\V} und eine maximale Belastbarkeit von \qty{1}{\W}. Welche Gleichspannung darf höchstens an den Widerstand angelegt werden, um ihn im spezifizierten Bereich zu betreiben?}{\qty{775}{\V}}
{\qty{0,01}{\kV}}
{\qty{0,7}{\kV}}
{\textbf{\textcolor{DARCgreen}{\qty{100}{\V}}}}
\end{QQuestion}

}
\end{frame}

\begin{frame}
\only<1>{
\begin{QQuestion}{EB511}{Ein Widerstand von \qty{100}{\kohm} hat eine maximale Spannungsfestigkeit von \qty{1000}{\V} und eine maximale Belastbarkeit von \qty{6}{\W}. Welche Gleichspannung darf höchstens an den Widerstand angelegt werden ohne ihn zu überlasten?}{\qty{100}{\V}}
{\qty{775}{\V}}
{\qty{0,07}{\kV}}
{\qty{1,00}{\kV}}
\end{QQuestion}

}
\only<2>{
\begin{QQuestion}{EB511}{Ein Widerstand von \qty{100}{\kohm} hat eine maximale Spannungsfestigkeit von \qty{1000}{\V} und eine maximale Belastbarkeit von \qty{6}{\W}. Welche Gleichspannung darf höchstens an den Widerstand angelegt werden ohne ihn zu überlasten?}{\qty{100}{\V}}
{\textbf{\textcolor{DARCgreen}{\qty{775}{\V}}}}
{\qty{0,07}{\kV}}
{\qty{1,00}{\kV}}
\end{QQuestion}

}
\end{frame}

\begin{frame}
\only<1>{
\begin{QQuestion}{EB512}{Ein Widerstand von \qty{120}{\ohm} hat eine Belastbarkeit von \qty{23,0}{\W}. Welcher Strom darf höchstens durch den Widerstand fließen, damit er nicht überlastet wird?}{\qty{2,28}{\A}}
{\qty{192}{\mA}}
{\qty{43,7}{\mA}}
{\qty{438}{\mA}}
\end{QQuestion}

}
\only<2>{
\begin{QQuestion}{EB512}{Ein Widerstand von \qty{120}{\ohm} hat eine Belastbarkeit von \qty{23,0}{\W}. Welcher Strom darf höchstens durch den Widerstand fließen, damit er nicht überlastet wird?}{\qty{2,28}{\A}}
{\qty{192}{\mA}}
{\qty{43,7}{\mA}}
{\textbf{\textcolor{DARCgreen}{\qty{438}{\mA}}}}
\end{QQuestion}

}
\end{frame}

\begin{frame}
\frametitle{Leistung bei Wechselspannung}
\begin{itemize}
  \item Bei Wechselspannungen muss mit dem Effektivwert gerechnet werden
  \end{itemize}
\end{frame}

\begin{frame}
\only<1>{
\begin{QQuestion}{EB503}{Gelten die Formeln für die Leistung an einem rein ohmschen Widerstand auch bei Wechselspannung?}{Nein, da die periodische Änderung von Strom und Spannung dann vernachlässigt wird.}
{Ja, wenn mit den Effektivwerten gerechnet wird.}
{Ja, wenn mit den Spitzenwerten gerechnet wird.}
{Nein, da die Blindleistung nicht berücksichtigt wird.}
\end{QQuestion}

}
\only<2>{
\begin{QQuestion}{EB503}{Gelten die Formeln für die Leistung an einem rein ohmschen Widerstand auch bei Wechselspannung?}{Nein, da die periodische Änderung von Strom und Spannung dann vernachlässigt wird.}
{\textbf{\textcolor{DARCgreen}{Ja, wenn mit den Effektivwerten gerechnet wird.}}}
{Ja, wenn mit den Spitzenwerten gerechnet wird.}
{Nein, da die Blindleistung nicht berücksichtigt wird.}
\end{QQuestion}

}
\end{frame}

\begin{frame}
\only<1>{
\begin{QQuestion}{EB513}{Ein Oszilloskop zeigt einen sinusförmigen Spitze-Spitze-Wert von \qty{25}{\V} an einem \qty{1000}{\ohm} Widerstand an. Der Effektivstrom durch den Widerstand beträgt~...}{\qty{25}{\mA}.}
{\qty{12,5}{\mA}.}
{\qty{8,8}{\mA}.}
{\qty{40}{\A}.}
\end{QQuestion}

}
\only<2>{
\begin{QQuestion}{EB513}{Ein Oszilloskop zeigt einen sinusförmigen Spitze-Spitze-Wert von \qty{25}{\V} an einem \qty{1000}{\ohm} Widerstand an. Der Effektivstrom durch den Widerstand beträgt~...}{\qty{25}{\mA}.}
{\qty{12,5}{\mA}.}
{\textbf{\textcolor{DARCgreen}{\qty{8,8}{\mA}.}}}
{\qty{40}{\A}.}
\end{QQuestion}

}

\end{frame}

\begin{frame}
\frametitle{PEP}
\begin{itemize}
  \item \emph{Peak Envelope Power} ist die Spitzenleistung eines Senders
  \item Leistung bei der höchsten Spitze einer Hochfrequenzschwingung
  \end{itemize}

\end{frame}

\begin{frame}
\only<1>{
\begin{QQuestion}{EB501}{Die Spitzenleistung eines Senders (PEP) ist~...}{die unmittelbar nach dem Senderausgang messbare Leistung über die Spitzen der Periode einer durchschnittlichen Hochfrequenzschwingung, bevor Zusatzgeräte (z.~B. Anpassgeräte) durchlaufen werden.}
{die Leistung, die der Sender unter normalen Betriebsbedingungen während einer Periode der Hochfrequenzschwingung bei der höchsten Spitze der Modulationshüllkurve durchschnittlich an einen reellen Abschlusswiderstand abgeben kann.}
{die durchschnittliche Leistung, die ein Sender unter normalen Betriebsbedingungen an die Antennenspeiseleitung während eines Zeitintervalls abgibt, das im Verhältnis zur Periode der tiefsten Modulationsfrequenz ausreichend lang ist.}
{das Produkt aus der Leistung, die unmittelbar der Antenne zugeführt wird, und ihrem Gewinnfaktor in einer Richtung, bezogen auf den Halbwellendipol.}
\end{QQuestion}

}
\only<2>{
\begin{QQuestion}{EB501}{Die Spitzenleistung eines Senders (PEP) ist~...}{die unmittelbar nach dem Senderausgang messbare Leistung über die Spitzen der Periode einer durchschnittlichen Hochfrequenzschwingung, bevor Zusatzgeräte (z.~B. Anpassgeräte) durchlaufen werden.}
{\textbf{\textcolor{DARCgreen}{die Leistung, die der Sender unter normalen Betriebsbedingungen während einer Periode der Hochfrequenzschwingung bei der höchsten Spitze der Modulationshüllkurve durchschnittlich an einen reellen Abschlusswiderstand abgeben kann.}}}
{die durchschnittliche Leistung, die ein Sender unter normalen Betriebsbedingungen an die Antennenspeiseleitung während eines Zeitintervalls abgibt, das im Verhältnis zur Periode der tiefsten Modulationsfrequenz ausreichend lang ist.}
{das Produkt aus der Leistung, die unmittelbar der Antenne zugeführt wird, und ihrem Gewinnfaktor in einer Richtung, bezogen auf den Halbwellendipol.}
\end{QQuestion}

}
\end{frame}

\begin{frame}
\frametitle{Mittlere Leistung}
\begin{itemize}
  \item Durchschnittliche Leistung eines Senders
  \item Beschreibung ergibt zu einem späteren Zeitpunkt mehr Sinn, wenn Hüllkurven durchgesprochen wurden
  \end{itemize}
\end{frame}

\begin{frame}
\only<1>{
\begin{QQuestion}{EB502}{Die mittlere Leistung eines Senders ist~...}{das Produkt aus der Leistung, die unmittelbar der Antenne zugeführt wird, und ihrem Gewinnfaktor in einer Richtung, bezogen auf den Halbwellendipol.}
{die unmittelbar nach dem Senderausgang messbare Leistung über die Spitzen der Periode einer durchschnittlichen Hochfrequenzschwingung, bevor Zusatzgeräte (z.~B. Anpassgeräte) durchlaufen werden.}
{die durchschnittliche Leistung, die ein Sender unter normalen Betriebsbedingungen während einer Periode der Hochfrequenzschwingung bei der höchsten Spitze der Modulationshüllkurve der Antennenspeiseleitung zuführt.}
{die durchschnittliche Leistung, die ein Sender unter normalen Betriebsbedingungen an die Antennenspeiseleitung während eines Zeitintervalls abgibt, das im Verhältnis zur Periode der tiefsten Modulationsfrequenz ausreichend lang ist.}
\end{QQuestion}

}
\only<2>{
\begin{QQuestion}{EB502}{Die mittlere Leistung eines Senders ist~...}{das Produkt aus der Leistung, die unmittelbar der Antenne zugeführt wird, und ihrem Gewinnfaktor in einer Richtung, bezogen auf den Halbwellendipol.}
{die unmittelbar nach dem Senderausgang messbare Leistung über die Spitzen der Periode einer durchschnittlichen Hochfrequenzschwingung, bevor Zusatzgeräte (z.~B. Anpassgeräte) durchlaufen werden.}
{die durchschnittliche Leistung, die ein Sender unter normalen Betriebsbedingungen während einer Periode der Hochfrequenzschwingung bei der höchsten Spitze der Modulationshüllkurve der Antennenspeiseleitung zuführt.}
{\textbf{\textcolor{DARCgreen}{die durchschnittliche Leistung, die ein Sender unter normalen Betriebsbedingungen an die Antennenspeiseleitung während eines Zeitintervalls abgibt, das im Verhältnis zur Periode der tiefsten Modulationsfrequenz ausreichend lang ist.}}}
\end{QQuestion}

}
\end{frame}%ENDCONTENT
