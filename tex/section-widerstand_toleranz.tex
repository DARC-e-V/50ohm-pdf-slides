
\section{Widerstandstoleranzen}
\label{section:widerstand_toleranz}
\begin{frame}%STARTCONTENT
\begin{itemize}
  \item Einfache Prozentrechnung
  \item Korrektur nach unten und oben vom angegebenen Widerstandswert
  \end{itemize}
\end{frame}

\begin{frame}
\only<1>{
\begin{QQuestion}{EC112}{Ein Widerstand hat eine Toleranz von \qty{10}{\percent}. Bei einem nominalen Widerstandswert von \qty{5,6}{\kohm} liegt der tatsächliche Wert zwischen~...}{\qtyrange{4760}{6440}{\ohm}.}
{\qtyrange{5040}{6160}{\ohm}.}
{\qtyrange{4,7}{6,8}{\kohm}.}
{\qtyrange{5,2}{6,3}{\kohm}.}
\end{QQuestion}

}
\only<2>{
\begin{QQuestion}{EC112}{Ein Widerstand hat eine Toleranz von \qty{10}{\percent}. Bei einem nominalen Widerstandswert von \qty{5,6}{\kohm} liegt der tatsächliche Wert zwischen~...}{\qtyrange{4760}{6440}{\ohm}.}
{\textbf{\textcolor{DARCgreen}{\qtyrange{5040}{6160}{\ohm}.}}}
{\qtyrange{4,7}{6,8}{\kohm}.}
{\qtyrange{5,2}{6,3}{\kohm}.}
\end{QQuestion}

}
\end{frame}

\begin{frame}
\only<1>{
\begin{QQuestion}{EC113}{Die Farbringe grün, blau und rot sowie ein silberner auf einem Widerstand mit 4 Farbringen bedeuten einen Widerstandswert zwischen~...}{\qtyrange{5240}{6360}{\ohm}.}
{\qtyrange{4760}{6440}{\ohm}.}
{\qtyrange{4760}{6840}{\ohm}.}
{\qtyrange{5040}{6160}{\ohm}.}
\end{QQuestion}

}
\only<2>{
\begin{QQuestion}{EC113}{Die Farbringe grün, blau und rot sowie ein silberner auf einem Widerstand mit 4 Farbringen bedeuten einen Widerstandswert zwischen~...}{\qtyrange{5240}{6360}{\ohm}.}
{\qtyrange{4760}{6440}{\ohm}.}
{\qtyrange{4760}{6840}{\ohm}.}
{\textbf{\textcolor{DARCgreen}{\qtyrange{5040}{6160}{\ohm}.}}}
\end{QQuestion}

}
\end{frame}%ENDCONTENT
