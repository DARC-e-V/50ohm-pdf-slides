
\section{Amplitudenmodulation (AM)}
\label{section:am}
\begin{frame}%STARTCONTENT

\begin{columns}
    \begin{column}{0.48\textwidth}
    
\begin{figure}
    \DARCimage{0.85\linewidth}{716include}
    \caption{\scriptsize Signal eines AM-Rundfunksenders (Sprache / Musik)}
    \label{n_Wasserfall0}
\end{figure}


    \end{column}
   \begin{column}{0.48\textwidth}
       \begin{itemize}
  \item Modulationssignal wird durch Änderung der Amplitude auf den Träger aufmoduliert
  \item Frequenz des Trägers bleibt unverändert
  \item Änderung der Amplitude ändert die Form des Trägers $\rightarrow$ entspricht nicht mehr einer Sinusschwingung
  \item Zusätzliche Frequenzen heißen \emph{Seitenbänder}
  \end{itemize}

   \end{column}
\end{columns}

\end{frame}

\begin{frame}
\begin{columns}
    \begin{column}{0.48\textwidth}
    \begin{itemize}
  \item In den Seitenbändern steckt die übertragene Information, also z.\,B. die Sprache
  \item Die von AM belegte \emph{Bandbreite} ist doppelt so hoch wie die höchste Frequenz des Modulationssignals
  \end{itemize}

    \end{column}
   \begin{column}{0.48\textwidth}
       
\begin{figure}
    \DARCimage{0.85\linewidth}{476include}
    \caption{\scriptsize Symbolische Darstellung eines amplitudenmodulierten Signals mit Träger und Seitenbändern}
    \label{n_seitenband}
\end{figure}


   \end{column}
\end{columns}

\end{frame}

\begin{frame}
\only<1>{
\begin{QQuestion}{NE202}{Welche Aussage zur Amplitudenmodulation ist richtig? Durch das Informationssignal~...}{wird die Amplitude des Trägers beeinflusst. Die Frequenz des Trägers bleibt dabei konstant.}
{wird die Frequenz des Trägers beeinflusst. Die Amplitude des Trägers bleibt dabei konstant.}
{werden gleichzeitig Amplitude und Frequenz des Trägers beeinflusst.}
{werden nacheinander Amplitude und Frequenz des Trägers beeinflusst.}
\end{QQuestion}

}
\only<2>{
\begin{QQuestion}{NE202}{Welche Aussage zur Amplitudenmodulation ist richtig? Durch das Informationssignal~...}{\textbf{\textcolor{DARCgreen}{wird die Amplitude des Trägers beeinflusst. Die Frequenz des Trägers bleibt dabei konstant.}}}
{wird die Frequenz des Trägers beeinflusst. Die Amplitude des Trägers bleibt dabei konstant.}
{werden gleichzeitig Amplitude und Frequenz des Trägers beeinflusst.}
{werden nacheinander Amplitude und Frequenz des Trägers beeinflusst.}
\end{QQuestion}

}
\end{frame}

\begin{frame}
\only<1>{
\begin{PQuestion}{NE206}{Welche spektrale Darstellung ergibt sich für die Modulationsart AM bei diesem  Audiospektrum?}{\DARCimage{1.0\linewidth}{477include}}
{\DARCimage{1.0\linewidth}{476include}}
{\DARCimage{1.0\linewidth}{478include}}
{\DARCimage{1.0\linewidth}{479include}}
{\DARCimage{1.0\linewidth}{475include}}\end{PQuestion}

}
\only<2>{
\begin{PQuestion}{NE206}{Welche spektrale Darstellung ergibt sich für die Modulationsart AM bei diesem  Audiospektrum?}{\DARCimage{1.0\linewidth}{477include}}
{\textbf{\textcolor{DARCgreen}{\DARCimage{1.0\linewidth}{476include}}}}
{\DARCimage{1.0\linewidth}{478include}}
{\DARCimage{1.0\linewidth}{479include}}
{\DARCimage{1.0\linewidth}{475include}}\end{PQuestion}

}
\end{frame}%ENDCONTENT
