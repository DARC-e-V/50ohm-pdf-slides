
\section{Impedanztransformation}
\label{section:impedanztransformation}
\begin{frame}%STARTCONTENT

\only<1>{
\begin{QQuestion}{AG412}{Eine Halbwellen-Übertragungsleitung ist an einem Ende mit \qty{50}{\ohm} abgeschlossen. Wie groß ist die Eingangsimpedanz am anderen Ende dieser Leitung?}{\qty{200}{\ohm}}
{\qty{25}{\ohm}}
{\qty{100}{\ohm}}
{\qty{50}{\ohm}}
\end{QQuestion}

}
\only<2>{
\begin{QQuestion}{AG412}{Eine Halbwellen-Übertragungsleitung ist an einem Ende mit \qty{50}{\ohm} abgeschlossen. Wie groß ist die Eingangsimpedanz am anderen Ende dieser Leitung?}{\qty{200}{\ohm}}
{\qty{25}{\ohm}}
{\qty{100}{\ohm}}
{\textbf{\textcolor{DARCgreen}{\qty{50}{\ohm}}}}
\end{QQuestion}

}
\end{frame}

\begin{frame}
\only<1>{
\begin{QQuestion}{AG416}{Ein Halbwellendipol hat bei seiner Resonanzfrequenz am Einspeisepunkt eine Impedanz von \qty{70}{\ohm}. Er wird über ein $\lambda$/2-langes \qty{300}{\ohm}-Flachbandkabel gespeist. Wie groß ist die Impedanz am Eingang der Speiseleitung?}{\qty{300}{\ohm}.}
{\qty{185}{\ohm}.}
{\qty{70}{\ohm}.}
{\qty{370}{\ohm}.}
\end{QQuestion}

}
\only<2>{
\begin{QQuestion}{AG416}{Ein Halbwellendipol hat bei seiner Resonanzfrequenz am Einspeisepunkt eine Impedanz von \qty{70}{\ohm}. Er wird über ein $\lambda$/2-langes \qty{300}{\ohm}-Flachbandkabel gespeist. Wie groß ist die Impedanz am Eingang der Speiseleitung?}{\qty{300}{\ohm}.}
{\qty{185}{\ohm}.}
{\textbf{\textcolor{DARCgreen}{\qty{70}{\ohm}.}}}
{\qty{370}{\ohm}.}
\end{QQuestion}

}
\end{frame}

\begin{frame}
\only<1>{
\begin{PQuestion}{AG413}{Einem Halbwellendipol wird die Sendeleistung über eine abgestimmte $\lambda$/2-Speiseleitung zugeführt. Wie hoch ist die Impedanz $Z_1$ am Einspeisepunkt des Dipols? Und wie hoch ist die Impedanz $Z_2$ am Anfang der Speiseleitung?}{$Z_1$ ist hochohmig und $Z_2$ niederohmig.}
{$Z_1$ und $Z_2$ sind hochohmig.}
{$Z_1$ ist niederohmig und $Z_2$ hochohmig.}
{$Z_1$ und $Z_2$ sind niederohmig.}
{\DARCimage{1.0\linewidth}{312include}}\end{PQuestion}

}
\only<2>{
\begin{PQuestion}{AG413}{Einem Halbwellendipol wird die Sendeleistung über eine abgestimmte $\lambda$/2-Speiseleitung zugeführt. Wie hoch ist die Impedanz $Z_1$ am Einspeisepunkt des Dipols? Und wie hoch ist die Impedanz $Z_2$ am Anfang der Speiseleitung?}{$Z_1$ ist hochohmig und $Z_2$ niederohmig.}
{$Z_1$ und $Z_2$ sind hochohmig.}
{$Z_1$ ist niederohmig und $Z_2$ hochohmig.}
{\textbf{\textcolor{DARCgreen}{$Z_1$ und $Z_2$ sind niederohmig.}}}
{\DARCimage{1.0\linewidth}{312include}}\end{PQuestion}

}
\end{frame}

\begin{frame}
\only<1>{
\begin{PQuestion}{AG414}{Einem Ganzwellendipol wird die Sendeleistung über eine abgestimmte $\lambda$/2-Speiseleitung zugeführt. Wie hoch ist die Impedanz $Z_1$ am Einspeisepunkt des Dipols und wie hoch ist die Impedanz $Z_2$ am Anfang der Speiseleitung?}{$Z_1$ ist niederohmig und $Z_2$ hochohmig.}
{$Z_1$ und $Z_2$ sind hochohmig.}
{$Z_1$ und $Z_2$ sind niederohmig.}
{$Z_1$ ist hochohmig und $Z_2$ niederohmig.}
{\DARCimage{1.0\linewidth}{312include}}\end{PQuestion}

}
\only<2>{
\begin{PQuestion}{AG414}{Einem Ganzwellendipol wird die Sendeleistung über eine abgestimmte $\lambda$/2-Speiseleitung zugeführt. Wie hoch ist die Impedanz $Z_1$ am Einspeisepunkt des Dipols und wie hoch ist die Impedanz $Z_2$ am Anfang der Speiseleitung?}{$Z_1$ ist niederohmig und $Z_2$ hochohmig.}
{\textbf{\textcolor{DARCgreen}{$Z_1$ und $Z_2$ sind hochohmig.}}}
{$Z_1$ und $Z_2$ sind niederohmig.}
{$Z_1$ ist hochohmig und $Z_2$ niederohmig.}
{\DARCimage{1.0\linewidth}{312include}}\end{PQuestion}

}
\end{frame}

\begin{frame}
\only<1>{
\begin{PQuestion}{AG415}{Einem Ganzwellendipol wird die Sendeleistung über eine abgestimmte $\lambda$/4-Speiseleitung zugeführt. Wie hoch ist die Impedanz $Z_1$ am Einspeisepunkt des Dipols und wie hoch ist die Impedanz $Z_2$ am Anfang der Speiseleitung?}{$Z_1$ ist hochohmig und $Z_2$ niederohmig.}
{$Z_1$ und $Z_2$ sind hochohmig.}
{$Z_1$ und $Z_2$ sind niederohmig.}
{$Z_1$ ist niederohmig und $Z_2$ hochohmig.}
{\DARCimage{1.0\linewidth}{312include}}\end{PQuestion}

}
\only<2>{
\begin{PQuestion}{AG415}{Einem Ganzwellendipol wird die Sendeleistung über eine abgestimmte $\lambda$/4-Speiseleitung zugeführt. Wie hoch ist die Impedanz $Z_1$ am Einspeisepunkt des Dipols und wie hoch ist die Impedanz $Z_2$ am Anfang der Speiseleitung?}{\textbf{\textcolor{DARCgreen}{$Z_1$ ist hochohmig und $Z_2$ niederohmig.}}}
{$Z_1$ und $Z_2$ sind hochohmig.}
{$Z_1$ und $Z_2$ sind niederohmig.}
{$Z_1$ ist niederohmig und $Z_2$ hochohmig.}
{\DARCimage{1.0\linewidth}{312include}}\end{PQuestion}

}
\end{frame}

\begin{frame}
\only<1>{
\begin{QQuestion}{AG417}{Ein Dipol mit einem Fußpunktwiderstand von \qty{60}{\ohm} soll über eine $\lambda$/4-Transformationsleitung mit einem \qty{240}{\ohm}-Flachbandkabel gespeist werden. Welchen Wellenwiderstand muss die Transformationsleitung haben?}{\qty{232}{\ohm}}
{\qty{150}{\ohm}}
{\qty{120}{\ohm}}
{\qty{300}{\ohm}}
\end{QQuestion}

}
\only<2>{
\begin{QQuestion}{AG417}{Ein Dipol mit einem Fußpunktwiderstand von \qty{60}{\ohm} soll über eine $\lambda$/4-Transformationsleitung mit einem \qty{240}{\ohm}-Flachbandkabel gespeist werden. Welchen Wellenwiderstand muss die Transformationsleitung haben?}{\qty{232}{\ohm}}
{\qty{150}{\ohm}}
{\textbf{\textcolor{DARCgreen}{\qty{120}{\ohm}}}}
{\qty{300}{\ohm}}
\end{QQuestion}

}
\end{frame}

\begin{frame}
\frametitle{Lösungsweg}
\begin{itemize}
  \item gegeben: $Z_A = 60Ω$
  \item gegeben: $Z_E = 240Ω$
  \item gesucht: $Z$
  \end{itemize}
    \pause
    $Z = \sqrt{Z_E \cdot Z_A} = \sqrt{240Ω \cdot 60Ω} = 120Ω$



\end{frame}

\begin{frame}
\only<1>{
\begin{QQuestion}{AG418}{Ein Faltdipol mit einem Fußpunktwiderstand von \qty{240}{\ohm} soll mit einer Hühnerleiter gespeist werden, deren Wellenwiderstand \qty{600}{\ohm} beträgt. Zur Anpassung soll ein $\lambda$/4 langes Stück Hühnerleiter mit einem anderen Wellenwiderstand verwendet werden. Welchen Wellenwiderstand muss die Transformationsleitung haben?}{\qty{380}{\ohm}}
{\qty{420}{\ohm}}
{\qty{840}{\ohm}}
{\qty{240}{\ohm}}
\end{QQuestion}

}
\only<2>{
\begin{QQuestion}{AG418}{Ein Faltdipol mit einem Fußpunktwiderstand von \qty{240}{\ohm} soll mit einer Hühnerleiter gespeist werden, deren Wellenwiderstand \qty{600}{\ohm} beträgt. Zur Anpassung soll ein $\lambda$/4 langes Stück Hühnerleiter mit einem anderen Wellenwiderstand verwendet werden. Welchen Wellenwiderstand muss die Transformationsleitung haben?}{\textbf{\textcolor{DARCgreen}{\qty{380}{\ohm}}}}
{\qty{420}{\ohm}}
{\qty{840}{\ohm}}
{\qty{240}{\ohm}}
\end{QQuestion}

}
\end{frame}

\begin{frame}
\frametitle{Lösungsweg}
\begin{itemize}
  \item gegeben: $Z_A = 240Ω$
  \item gegeben: $Z_E = 600Ω$
  \item gesucht: $Z$
  \end{itemize}
    \pause
    $Z = \sqrt{Z_E \cdot Z_A} = \sqrt{600Ω \cdot 240Ω} = 380Ω$



\end{frame}

\begin{frame}
\only<1>{
\begin{PQuestion}{AG406}{Worum handelt es sich bei dieser Schaltung? Es handelt sich um~...}{einen abstimmbaren Sperrkreis zur Entkopplung der Antenne  vom Sender.}
{einen regelbaren Bandpass mit veränderbarer Bandbreite zur Kompensation der Auskoppelverluste.}
{ein Pi-Filter zur Impedanztransformation und Verbesserung der Unterdrückung von Oberwellen.}
{einen Saugkreis, der die zweite Harmonische unterdrückt und so den Wirkungsgrad der Verstärkerstufe erhöht.}
{\DARCimage{0.75\linewidth}{425include}}\end{PQuestion}

}
\only<2>{
\begin{PQuestion}{AG406}{Worum handelt es sich bei dieser Schaltung? Es handelt sich um~...}{einen abstimmbaren Sperrkreis zur Entkopplung der Antenne  vom Sender.}
{einen regelbaren Bandpass mit veränderbarer Bandbreite zur Kompensation der Auskoppelverluste.}
{\textbf{\textcolor{DARCgreen}{ein Pi-Filter zur Impedanztransformation und Verbesserung der Unterdrückung von Oberwellen.}}}
{einen Saugkreis, der die zweite Harmonische unterdrückt und so den Wirkungsgrad der Verstärkerstufe erhöht.}
{\DARCimage{0.75\linewidth}{425include}}\end{PQuestion}

}
\end{frame}%ENDCONTENT
