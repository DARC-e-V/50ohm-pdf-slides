
\section{Spannungsgesteuerter Oszillator (VCO)}
\label{section:oszillator_vco}
\begin{frame}%STARTCONTENT

\only<1>{
\begin{QQuestion}{AD601}{Was versteht man unter einem VCO? Ein VCO ist ein~...}{quarzstabilisierter Referenzoszillator.}
{Oszillator, der mittels eines Drehkondensators abgestimmt wird.}
{spannungsgesteuerter Oszillator.}
{variabler Quarzoszillator.}
\end{QQuestion}

}
\only<2>{
\begin{QQuestion}{AD601}{Was versteht man unter einem VCO? Ein VCO ist ein~...}{quarzstabilisierter Referenzoszillator.}
{Oszillator, der mittels eines Drehkondensators abgestimmt wird.}
{\textbf{\textcolor{DARCgreen}{spannungsgesteuerter Oszillator.}}}
{variabler Quarzoszillator.}
\end{QQuestion}

}
\end{frame}

\begin{frame}
\only<1>{
\begin{QQuestion}{AD611}{Wenn HF-Signale unerwünscht auf einen VFO zurückkoppeln, kann dies zu~...}{Frequenzsynthese führen.}
{Frequenzinstabilität führen.}
{Gegenkopplung führen.}
{Mehrwegeausbreitung führen.}
\end{QQuestion}

}
\only<2>{
\begin{QQuestion}{AD611}{Wenn HF-Signale unerwünscht auf einen VFO zurückkoppeln, kann dies zu~...}{Frequenzsynthese führen.}
{\textbf{\textcolor{DARCgreen}{Frequenzinstabilität führen.}}}
{Gegenkopplung führen.}
{Mehrwegeausbreitung führen.}
\end{QQuestion}

}
\end{frame}%ENDCONTENT
