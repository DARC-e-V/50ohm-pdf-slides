
\section{Modulatoren}
\label{section:modulatoren}
\begin{frame}%STARTCONTENT

\only<1>{
\begin{PQuestion}{AD503}{Bei dieser Schaltung ist der mit X bezeichnete Anschluss~...}{der Ausgang für das NF-Signal.}
{der Ausgang für eine Regelspannung.}
{der Ausgang für das Oszillatorsignal.}
{der Ausgang für das ZF-Signal.}
{\DARCimage{1.0\linewidth}{142include}}\end{PQuestion}

}
\only<2>{
\begin{PQuestion}{AD503}{Bei dieser Schaltung ist der mit X bezeichnete Anschluss~...}{der Ausgang für das NF-Signal.}
{\textbf{\textcolor{DARCgreen}{der Ausgang für eine Regelspannung.}}}
{der Ausgang für das Oszillatorsignal.}
{der Ausgang für das ZF-Signal.}
{\DARCimage{1.0\linewidth}{142include}}\end{PQuestion}

}
\end{frame}

\begin{frame}
\only<1>{
\begin{PQuestion}{AD507}{Bei dieser Schaltung handelt es sich um einen~...}{LSB-Modulator.}
{USB-Modulator.}
{FM-Modulator.}
{AM-Modulator.}
{\DARCimage{1.0\linewidth}{772include}}\end{PQuestion}

}
\only<2>{
\begin{PQuestion}{AD507}{Bei dieser Schaltung handelt es sich um einen~...}{LSB-Modulator.}
{USB-Modulator.}
{FM-Modulator.}
{\textbf{\textcolor{DARCgreen}{AM-Modulator.}}}
{\DARCimage{1.0\linewidth}{772include}}\end{PQuestion}

}
\end{frame}

\begin{frame}
\only<1>{
\begin{PQuestion}{AD508}{Bei dieser Schaltung handelt es sich um einen Modulator zur Erzeugung von~...}{AM-Signalen mit unterdrücktem Träger.}
{phasenmodulierten Signalen.}
{frequenzmodulierten Signalen.}
{AM-Signalen.}
{\DARCimage{1.0\linewidth}{155include}}\end{PQuestion}

}
\only<2>{
\begin{PQuestion}{AD508}{Bei dieser Schaltung handelt es sich um einen Modulator zur Erzeugung von~...}{AM-Signalen mit unterdrücktem Träger.}
{phasenmodulierten Signalen.}
{\textbf{\textcolor{DARCgreen}{frequenzmodulierten Signalen.}}}
{AM-Signalen.}
{\DARCimage{1.0\linewidth}{155include}}\end{PQuestion}

}
\end{frame}

\begin{frame}
\only<1>{
\begin{PQuestion}{AD509}{Was ermöglicht die abgebildete Schaltung?}{Die Erzeugung von Phasenmodulation}
{Die HF-Pegelbegrenzung und HF-Pegeleinstellung bei FM-Funkgeräten}
{Die Erzeugung von Amplitudenmodulation}
{Die Hubbegrenzung und Hubeinstellung bei FM-Funkgeräten}
{\DARCimage{1.0\linewidth}{44include}}\end{PQuestion}

}
\only<2>{
\begin{PQuestion}{AD509}{Was ermöglicht die abgebildete Schaltung?}{Die Erzeugung von Phasenmodulation}
{Die HF-Pegelbegrenzung und HF-Pegeleinstellung bei FM-Funkgeräten}
{Die Erzeugung von Amplitudenmodulation}
{\textbf{\textcolor{DARCgreen}{Die Hubbegrenzung und Hubeinstellung bei FM-Funkgeräten}}}
{\DARCimage{1.0\linewidth}{44include}}\end{PQuestion}

}
\end{frame}

\begin{frame}
\only<1>{
\begin{QQuestion}{AE206}{Welche Baugruppe sollte für die analoge Erzeugung eines unterdrückten Zweiseitenband-Trägersignals verwendet werden?}{Demodulator}
{Quarzfilter}
{Bandfilter}
{Balancemischer}
\end{QQuestion}

}
\only<2>{
\begin{QQuestion}{AE206}{Welche Baugruppe sollte für die analoge Erzeugung eines unterdrückten Zweiseitenband-Trägersignals verwendet werden?}{Demodulator}
{Quarzfilter}
{Bandfilter}
{\textbf{\textcolor{DARCgreen}{Balancemischer}}}
\end{QQuestion}

}
\end{frame}

\begin{frame}
\only<1>{
\begin{QQuestion}{AF302}{Welcher Mischertyp ist am besten geeignet, um ein Doppelseitenbandsignal mit unterdrücktem Träger zu erzeugen?}{Ein Mischer mit einem einzelnen FET}
{Ein Balancemischer}
{Ein Mischer mit einer Varaktordiode}
{Ein quarzgesteuerter Mischer}
\end{QQuestion}

}
\only<2>{
\begin{QQuestion}{AF302}{Welcher Mischertyp ist am besten geeignet, um ein Doppelseitenbandsignal mit unterdrücktem Träger zu erzeugen?}{Ein Mischer mit einem einzelnen FET}
{\textbf{\textcolor{DARCgreen}{Ein Balancemischer}}}
{Ein Mischer mit einer Varaktordiode}
{Ein quarzgesteuerter Mischer}
\end{QQuestion}

}
\end{frame}

\begin{frame}
\only<1>{
\begin{QQuestion}{AF303}{Wie kann mit analoger Technologie ein SSB-Signal erzeugt werden?}{In einem Balancemodulator wird ein Zweiseitenband-Signal erzeugt. Das Seitenbandfilter selektiert ein Seitenband heraus.}
{In einem Balancemodulator wird ein Zweiseitenband-Signal erzeugt. Ein auf die Trägerfrequenz abgestimmter Saugkreis filtert den Träger aus.}
{In einem Balancemodulator wird ein Zweiseitenband-Signal erzeugt. Ein auf die Trägerfrequenz abgestimmter Sperrkreis filtert den Träger aus.}
{In einem Balancemodulator wird ein Zweiseitenband-Signal erzeugt. In einem Frequenzteiler wird ein Seitenband abgespalten.}
\end{QQuestion}

}
\only<2>{
\begin{QQuestion}{AF303}{Wie kann mit analoger Technologie ein SSB-Signal erzeugt werden?}{\textbf{\textcolor{DARCgreen}{In einem Balancemodulator wird ein Zweiseitenband-Signal erzeugt. Das Seitenbandfilter selektiert ein Seitenband heraus.}}}
{In einem Balancemodulator wird ein Zweiseitenband-Signal erzeugt. Ein auf die Trägerfrequenz abgestimmter Saugkreis filtert den Träger aus.}
{In einem Balancemodulator wird ein Zweiseitenband-Signal erzeugt. Ein auf die Trägerfrequenz abgestimmter Sperrkreis filtert den Träger aus.}
{In einem Balancemodulator wird ein Zweiseitenband-Signal erzeugt. In einem Frequenzteiler wird ein Seitenband abgespalten.}
\end{QQuestion}

}
\end{frame}

\begin{frame}
\only<1>{
\begin{QQuestion}{AF304}{Bei üblichen analogen Methoden zur Aufbereitung eines SSB-Signals werden~...}{der Träger unterdrückt und ein Seitenband hinzugesetzt.}
{der Träger hinzugesetzt und ein Seitenband ausgefiltert.}
{der Träger unterdrückt und ein Seitenband ausgefiltert.}
{der Träger unterdrückt und beide Seitenbänder ausgefiltert.}
\end{QQuestion}

}
\only<2>{
\begin{QQuestion}{AF304}{Bei üblichen analogen Methoden zur Aufbereitung eines SSB-Signals werden~...}{der Träger unterdrückt und ein Seitenband hinzugesetzt.}
{der Träger hinzugesetzt und ein Seitenband ausgefiltert.}
{\textbf{\textcolor{DARCgreen}{der Träger unterdrückt und ein Seitenband ausgefiltert.}}}
{der Träger unterdrückt und beide Seitenbänder ausgefiltert.}
\end{QQuestion}

}
\end{frame}

\begin{frame}
\only<1>{
\begin{PQuestion}{AF305}{Dieses Blockschaltbild zeigt einen SSB-Sender. Die Stufe bei \glqq?\grqq{} ist ein...}{ZF-Notchfilter zur Unterdrückung des unerwünschten Seitenbands.}
{RC-Hochpass zur Unterdrückung des unteren Seitenbands.}
{RL-Tiefpass zur Unterdrückung des oberen Seitenbands.}
{Quarzfilter als Bandpass für das gewünschte Seitenband.}
{\DARCimage{1.0\linewidth}{98include}}\end{PQuestion}

}
\only<2>{
\begin{PQuestion}{AF305}{Dieses Blockschaltbild zeigt einen SSB-Sender. Die Stufe bei \glqq?\grqq{} ist ein...}{ZF-Notchfilter zur Unterdrückung des unerwünschten Seitenbands.}
{RC-Hochpass zur Unterdrückung des unteren Seitenbands.}
{RL-Tiefpass zur Unterdrückung des oberen Seitenbands.}
{\textbf{\textcolor{DARCgreen}{Quarzfilter als Bandpass für das gewünschte Seitenband.}}}
{\DARCimage{1.0\linewidth}{98include}}\end{PQuestion}

}
\end{frame}

\begin{frame}
\only<1>{
\begin{PQuestion}{AF306}{Welches Schaltungsteil ist in der folgenden Blockschaltung am Ausgang des NF-Verstärkers angeschlossen?}{symmetrisches Filter}
{Balancemischer}
{Dynamikkompressor}
{DSB-Filter}
{\DARCimage{1.0\linewidth}{500include}}\end{PQuestion}

}
\only<2>{
\begin{PQuestion}{AF306}{Welches Schaltungsteil ist in der folgenden Blockschaltung am Ausgang des NF-Verstärkers angeschlossen?}{symmetrisches Filter}
{\textbf{\textcolor{DARCgreen}{Balancemischer}}}
{Dynamikkompressor}
{DSB-Filter}
{\DARCimage{1.0\linewidth}{500include}}\end{PQuestion}

}
\end{frame}

\begin{frame}
\only<1>{
\begin{PQuestion}{AF307}{Die folgende Blockschaltung zeigt eine SSB-Aufbereitung mit einem \qty{9}{\MHz}-Quarzfilter. Welche Frequenz wird in der Schalterstellung USB mit der NF gemischt?}{\qty{9,0000}{\MHz}}
{\qty{8,9970}{\MHz}}
{\qty{8,9985}{\MHz}}
{\qty{9,0030}{\MHz}}
{\DARCimage{1.0\linewidth}{39include}}\end{PQuestion}

}
\only<2>{
\begin{PQuestion}{AF307}{Die folgende Blockschaltung zeigt eine SSB-Aufbereitung mit einem \qty{9}{\MHz}-Quarzfilter. Welche Frequenz wird in der Schalterstellung USB mit der NF gemischt?}{\qty{9,0000}{\MHz}}
{\qty{8,9970}{\MHz}}
{\textbf{\textcolor{DARCgreen}{\qty{8,9985}{\MHz}}}}
{\qty{9,0030}{\MHz}}
{\DARCimage{1.0\linewidth}{39include}}\end{PQuestion}

}
\end{frame}

\begin{frame}
\frametitle{Lösungsweg}
\begin{itemize}
  \item gegeben: $f_Q = 9MHz$
  \item gegeben: $f_{LSB} = 9,0015MHz$
  \item gesucht: $f_{USB}$
  \end{itemize}
    \pause
    $f_{USB} = f_Q -- (f_{LSB} -- f_Q) = 9MHz -- (9,0015MHz -- 9MHz) = 9MHz -- 0,0015MHz =8,9985MHz$



\end{frame}

\begin{frame}
\only<1>{
\begin{PQuestion}{AF308}{Bei dieser Schaltung handelt es sich um einen Modulator zur Erzeugung von~...}{AM-Signalen mit unterdrücktem Träger.}
{phasenmodulierten Signalen.}
{frequenzmodulierten Signalen.}
{LSB-Signalen.}
{\DARCimage{1.0\linewidth}{759include}}\end{PQuestion}

}
\only<2>{
\begin{PQuestion}{AF308}{Bei dieser Schaltung handelt es sich um einen Modulator zur Erzeugung von~...}{\textbf{\textcolor{DARCgreen}{AM-Signalen mit unterdrücktem Träger.}}}
{phasenmodulierten Signalen.}
{frequenzmodulierten Signalen.}
{LSB-Signalen.}
{\DARCimage{1.0\linewidth}{759include}}\end{PQuestion}

}
\end{frame}

\begin{frame}
\only<1>{
\begin{PQuestion}{AF309}{Wozu dienen $R_1$ und $C_1$ bei dieser Schaltung?  }{Sie dienen zur Einstellung des Modulationsgrades des erzeugten DSB-Signals.}
{Sie dienen zum Ausgleich von Frequenzgangs- und Laufzeitunterschieden.}
{Sie dienen zur Einstellung des Frequenzhubes mit Hilfe der ersten Trägernullstelle.}
{Sie dienen zur Einstellung der Trägerunterdrückung nach Betrag und Phase.}
{\DARCimage{1.0\linewidth}{762include}}\end{PQuestion}

}
\only<2>{
\begin{PQuestion}{AF309}{Wozu dienen $R_1$ und $C_1$ bei dieser Schaltung?  }{Sie dienen zur Einstellung des Modulationsgrades des erzeugten DSB-Signals.}
{Sie dienen zum Ausgleich von Frequenzgangs- und Laufzeitunterschieden.}
{Sie dienen zur Einstellung des Frequenzhubes mit Hilfe der ersten Trägernullstelle.}
{\textbf{\textcolor{DARCgreen}{Sie dienen zur Einstellung der Trägerunterdrückung nach Betrag und Phase.}}}
{\DARCimage{1.0\linewidth}{762include}}\end{PQuestion}

}
\end{frame}

\begin{frame}
\only<1>{
\begin{PQuestion}{AF310}{Dieser Schaltungsauszug ist Teil eines Senders. Welche Funktion hat die Diode?}{Sie stabilisiert die Betriebsspannung für den Oszillator, um diesen von der Stromversorgung der anderen Stufen zu entkoppeln.}
{Sie beeinflusst die Resonanzfrequenz des Schwingkreises in Abhängigkeit des NF-Spannungsverlaufs und moduliert so die Oszillatorfrequenz.}
{Sie begrenzt die Amplituden des Eingangssignals und vermeidet so die Übersteuerung der Oszillatorstufe.}
{Sie dient zur Erzeugung von Amplitudenmodulation in Abhängigkeit von den Frequenzen im Basisband.}
{\DARCimage{1.0\linewidth}{158include}}\end{PQuestion}

}
\only<2>{
\begin{PQuestion}{AF310}{Dieser Schaltungsauszug ist Teil eines Senders. Welche Funktion hat die Diode?}{Sie stabilisiert die Betriebsspannung für den Oszillator, um diesen von der Stromversorgung der anderen Stufen zu entkoppeln.}
{\textbf{\textcolor{DARCgreen}{Sie beeinflusst die Resonanzfrequenz des Schwingkreises in Abhängigkeit des NF-Spannungsverlaufs und moduliert so die Oszillatorfrequenz.}}}
{Sie begrenzt die Amplituden des Eingangssignals und vermeidet so die Übersteuerung der Oszillatorstufe.}
{Sie dient zur Erzeugung von Amplitudenmodulation in Abhängigkeit von den Frequenzen im Basisband.}
{\DARCimage{1.0\linewidth}{158include}}\end{PQuestion}

}
\end{frame}

\begin{frame}
\only<1>{
\begin{QQuestion}{AD510}{Welche Signale stehen am Ausgang eines symmetrisch eingestellten Balancemischers an?}{Der vollständige Träger}
{Viele Mischprodukte}
{Der verringerte Träger und ein Seitenband}
{Die zwei Seitenbänder}
\end{QQuestion}

}
\only<2>{
\begin{QQuestion}{AD510}{Welche Signale stehen am Ausgang eines symmetrisch eingestellten Balancemischers an?}{Der vollständige Träger}
{Viele Mischprodukte}
{Der verringerte Träger und ein Seitenband}
{\textbf{\textcolor{DARCgreen}{Die zwei Seitenbänder}}}
\end{QQuestion}

}
\end{frame}%ENDCONTENT
