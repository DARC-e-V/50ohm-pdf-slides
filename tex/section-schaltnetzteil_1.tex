
\section{Schaltnetzteil I}
\label{section:schaltnetzteil_1}
\begin{frame}%STARTCONTENT
\begin{itemize}
  \item Das im vorigen Kapitel vorgestellte Netzteil hat den Nachteil eines hohen Gewichts durch den Transformator und einen schlechten Wirkungsgrad aufgrund von Verlusten bei der Konstanthaltung der Ausgangsspannung.
  \item Schaltnetzteile bringen die Eingangsspannung auf eine höhere Frequenz, wodurch kleinere Transformatoren eingesetzt werden können und bieten effizientere Wege die Ausgangsspannung konstant zu halten.
  \end{itemize}
\end{frame}

\begin{frame}Details auch hier im Klasse~A Kurs, wir konzentrieren uns auf die positiven Eigenschaften:

\begin{itemize}
  \item \emph{Hoher Wirkungsgrad}
  \item \emph{Geringes Gewicht}
  \item \emph{Geringes Volumen}
  \end{itemize}
\end{frame}

\begin{frame}
\only<1>{
\begin{QQuestion}{ED302}{Welche Eigenschaften hat ein Schaltnetzteil?}{Hoher Wirkungsgrad, geringes Gewicht, geringes Volumen.}
{Niedriger Wirkungsgrad, geringes Gewicht, geringes Volumen.}
{Hoher Wirkungsgrad, hohes Gewicht, geringes Volumen.}
{Hoher Wirkungsgrad, geringes Gewicht, großes Volumen.}
\end{QQuestion}

}
\only<2>{
\begin{QQuestion}{ED302}{Welche Eigenschaften hat ein Schaltnetzteil?}{\textbf{\textcolor{DARCgreen}{Hoher Wirkungsgrad, geringes Gewicht, geringes Volumen.}}}
{Niedriger Wirkungsgrad, geringes Gewicht, geringes Volumen.}
{Hoher Wirkungsgrad, hohes Gewicht, geringes Volumen.}
{Hoher Wirkungsgrad, geringes Gewicht, großes Volumen.}
\end{QQuestion}

}
\end{frame}

\begin{frame}Aber: Wo Licht ist, ist auch Schatten.

\begin{itemize}
  \item Durch die hohen Frequenzen kann es zu \emph{hochfrequenten Störungen} kommen, die besonders im Kurzwellenbereich stören.
  \item Bei für den Amateurfunk konzipierten Netzgeräten ist das inzwischen kein Problem mehr, bei Netzgeräten in der Unterhaltungselektronik schon.
  \end{itemize}
\end{frame}

\begin{frame}
\only<1>{
\begin{QQuestion}{ED303}{Welches ist der Hauptnachteil eines Schaltnetzteils ?}{Ein Schaltnetzteil kann keine so hohen Ströme abgeben.}
{Ein Schaltnetzteil hat einen niedrigen Wirkungsgrad.}
{Ein Schaltnetzteil kann hochfrequente Störungen erzeugen.}
{Ein Schaltnetzteil hat hohe Verluste.}
\end{QQuestion}

}
\only<2>{
\begin{QQuestion}{ED303}{Welches ist der Hauptnachteil eines Schaltnetzteils ?}{Ein Schaltnetzteil kann keine so hohen Ströme abgeben.}
{Ein Schaltnetzteil hat einen niedrigen Wirkungsgrad.}
{\textbf{\textcolor{DARCgreen}{Ein Schaltnetzteil kann hochfrequente Störungen erzeugen.}}}
{Ein Schaltnetzteil hat hohe Verluste.}
\end{QQuestion}

}
\end{frame}%ENDCONTENT
