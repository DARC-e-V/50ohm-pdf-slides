
\section{Near Vertical Incidence Skywave (NVIS)}
\label{section:nvis}
\begin{frame}%STARTCONTENT

\only<1>{
\begin{QQuestion}{AG125}{Welche Antennen sind für NVIS-Ausbreitung (Near Vertical Incident Skywave), wie sie für Notfunk-Verbindungen im KW-Bereich benutzt werden, gut geeignet?}{Eine Vertikalantenne einer Gesamtlänge zwischen \num{0,5} und \num{0,625} (5/8) Wellenlängen über gutem Radialnetz.}
{Mit Drähten aufgebauter horizontaler Faltdipol in möglichst genau \num{0,8} Wellenlängen Höhe über Grund.}
{Als \glqq Inverted-V\grqq{} aufgespannte Drähte mit einem Speisepunkt in mindestens einer Wellenlänge Höhe über Grund.}
{Horizontal aufgespannte Drähte in einer Höhe von höchstens \num{0,25} Wellenlängen über Grund.}
\end{QQuestion}

}
\only<2>{
\begin{QQuestion}{AG125}{Welche Antennen sind für NVIS-Ausbreitung (Near Vertical Incident Skywave), wie sie für Notfunk-Verbindungen im KW-Bereich benutzt werden, gut geeignet?}{Eine Vertikalantenne einer Gesamtlänge zwischen \num{0,5} und \num{0,625} (5/8) Wellenlängen über gutem Radialnetz.}
{Mit Drähten aufgebauter horizontaler Faltdipol in möglichst genau \num{0,8} Wellenlängen Höhe über Grund.}
{Als \glqq Inverted-V\grqq{} aufgespannte Drähte mit einem Speisepunkt in mindestens einer Wellenlänge Höhe über Grund.}
{\textbf{\textcolor{DARCgreen}{Horizontal aufgespannte Drähte in einer Höhe von höchstens \num{0,25} Wellenlängen über Grund.}}}
\end{QQuestion}

}
\end{frame}

\begin{frame}
\only<1>{
\begin{QQuestion}{AG224}{Welche Eigenschaften besitzt eine in geringer Höhe aufgebaute, auf Kurzwelle betriebene NVIS-Antenne (Near Vertical Incident Skywave)?}{Sie vergrößert durch ihre flache Abstrahlung den Bereich der Bodenwelle.}
{Sie ermöglicht durch annähernd senkrechte Abstrahlung eine Raumwellenausbreitung ohne tote Zone um den Sendeort herum.}
{Ihre senkrechte Abstrahlung bringt die D-Region zum Verschwinden, so dass die Tagesdämpfung über dem Sendeort lokal aufgehoben wird.}
{Sie erzeugt mit ihrer Reflexion am nahen Erdboden eine zirkular polarisierte Abstrahlung, die Fading reduziert.}
\end{QQuestion}

}
\only<2>{
\begin{QQuestion}{AG224}{Welche Eigenschaften besitzt eine in geringer Höhe aufgebaute, auf Kurzwelle betriebene NVIS-Antenne (Near Vertical Incident Skywave)?}{Sie vergrößert durch ihre flache Abstrahlung den Bereich der Bodenwelle.}
{\textbf{\textcolor{DARCgreen}{Sie ermöglicht durch annähernd senkrechte Abstrahlung eine Raumwellenausbreitung ohne tote Zone um den Sendeort herum.}}}
{Ihre senkrechte Abstrahlung bringt die D-Region zum Verschwinden, so dass die Tagesdämpfung über dem Sendeort lokal aufgehoben wird.}
{Sie erzeugt mit ihrer Reflexion am nahen Erdboden eine zirkular polarisierte Abstrahlung, die Fading reduziert.}
\end{QQuestion}

}
\end{frame}%ENDCONTENT
