
\section{Bandbreite II}
\label{section:bandbreite_2}
\begin{frame}%STARTCONTENT
\begin{itemize}
  \item Bandbreite eines Signals beschreibt die Differenz zwischen maximaler und minimaler Sendefrequenz einer Aussendung
  \item Die Bandbreite wird in Hertz (Hz) gemessen
  \end{itemize}
\end{frame}

\begin{frame}
\only<1>{
\begin{QQuestion}{EA105}{Welche Einheit wird üblicherweise für die Bandbreite verwendet?}{Dezibel (dB)}
{Baud (Bd)}
{Bit pro Sekunde (Bit/s)}
{Hertz (Hz)}
\end{QQuestion}

}
\only<2>{
\begin{QQuestion}{EA105}{Welche Einheit wird üblicherweise für die Bandbreite verwendet?}{Dezibel (dB)}
{Baud (Bd)}
{Bit pro Sekunde (Bit/s)}
{\textbf{\textcolor{DARCgreen}{Hertz (Hz)}}}
\end{QQuestion}

}
\end{frame}%ENDCONTENT
