
\section{Aufenthalt im Strahlengang}
\label{section:strahlengang_aufenthalt}
\begin{frame}%STARTCONTENT
\begin{itemize}
  \item Insbesondere im Mikrowellenbereich werden Parabol- oder Helixantenen verwendet
  \item Hoher Antennengewinn
  \item Aus wenig Eingangsleistung wird eine hohe Strahlungsleistung
  \item \qty{20}{\dB} sind üblich $\rightarrow$ \qty{1}{\watt} Sendeleistung werden \qty{100}{\watt} Strahlungsleistung
  \end{itemize}

\end{frame}

\begin{frame}\begin{itemize}
  \item Hohe elektromagnetische Felder in der Strahlungskeule
  \item Gefahr für Körper, insbesondere Augen, Gehirn und Hoden
  \item Kann zu Erkrankungen dieser Organe führen
  \item Die Strahlung ist nicht direkt zu spüren
  \item \emph{Der Aufenthalt im direkten Strahlengang von Sendeantennen ist zu vermeiden!}
  \end{itemize}
\end{frame}

\begin{frame}
\only<1>{
\begin{QQuestion}{EK201}{Was ist aus Sicherheitsgründen besonders beim Umgang mit Mikrowellen zu beachten?}{Es ist eine Kopfbedeckung aus Abschirmfolie (z. B. aus Aluminium) zu tragen.}
{Der Duty-Cycle des Senders sollte \qty{50}{\percent} nicht überschreiten.}
{Ein Aufenthalt im direkten Strahlengang von Sendeantennen ist zu vermeiden.}
{Zur Einhaltung des Personenschutzes muss EMV-Schutzkleidung getragen werden.}
\end{QQuestion}

}
\only<2>{
\begin{QQuestion}{EK201}{Was ist aus Sicherheitsgründen besonders beim Umgang mit Mikrowellen zu beachten?}{Es ist eine Kopfbedeckung aus Abschirmfolie (z. B. aus Aluminium) zu tragen.}
{Der Duty-Cycle des Senders sollte \qty{50}{\percent} nicht überschreiten.}
{\textbf{\textcolor{DARCgreen}{Ein Aufenthalt im direkten Strahlengang von Sendeantennen ist zu vermeiden.}}}
{Zur Einhaltung des Personenschutzes muss EMV-Schutzkleidung getragen werden.}
\end{QQuestion}

}

\end{frame}%ENDCONTENT
