
\section{Mögel-Dellinger-Effekt}
\label{section:moegel_dellinger_effekt}
\begin{frame}%STARTCONTENT
\begin{itemize}
  \item Sonneneruptionen mit Plasma-Flares ionisieren die D-Region
  \item Hohe Dämpfung der Raumwelle bis \qty{300}{\mega\hertz}
  \item Totaler Ausfall der Raumwelle für wenige Minuten bis Stunden möglich
  \item Kann nur tagsüber auftreten
  \item Besonders stark bei Sonnenfleckenmaximum
  \end{itemize}

\end{frame}

\begin{frame}
\only<1>{
\begin{QQuestion}{EH214}{Ein plötzlicher Anstieg der Intensitäten von UV- und Röntgenstrahlung nach einem Flare (Energieausbruch auf der Sonne) führt zu erhöhter Ionisierung der D-Region und damit zu zeitweiligem Ausfall der Raumwellenausbreitung auf der Kurzwelle. Diese Erscheinung bezeichnet man als~...}{sporadische E-Ausbreitung.}
{Mögel-Dellinger-Effekt.}
{kritischer Schwund.}
{Aurora-Effekt.}
\end{QQuestion}

}
\only<2>{
\begin{QQuestion}{EH214}{Ein plötzlicher Anstieg der Intensitäten von UV- und Röntgenstrahlung nach einem Flare (Energieausbruch auf der Sonne) führt zu erhöhter Ionisierung der D-Region und damit zu zeitweiligem Ausfall der Raumwellenausbreitung auf der Kurzwelle. Diese Erscheinung bezeichnet man als~...}{sporadische E-Ausbreitung.}
{\textbf{\textcolor{DARCgreen}{Mögel-Dellinger-Effekt.}}}
{kritischer Schwund.}
{Aurora-Effekt.}
\end{QQuestion}

}
\end{frame}

\begin{frame}
\only<1>{
\begin{QQuestion}{EH215}{Welche Auswirkung hat der Mögel-Dellinger-Effekt auf die Ausbreitung von Kurzwellen?}{Das Übersprechen der Modulation eines starken Senders auf andere, über die Ionosphäre übertragene HF-Signale.}
{Den zeitlich begrenzten Schwund durch Mehrwegeausbreitung in der Ionosphäre.}
{Die zeitlich begrenzt auftretende Verzerrung der Modulation.}
{Den zeitlich begrenzten Ausfall der Raumwellenausbreitung.}
\end{QQuestion}

}
\only<2>{
\begin{QQuestion}{EH215}{Welche Auswirkung hat der Mögel-Dellinger-Effekt auf die Ausbreitung von Kurzwellen?}{Das Übersprechen der Modulation eines starken Senders auf andere, über die Ionosphäre übertragene HF-Signale.}
{Den zeitlich begrenzten Schwund durch Mehrwegeausbreitung in der Ionosphäre.}
{Die zeitlich begrenzt auftretende Verzerrung der Modulation.}
{\textbf{\textcolor{DARCgreen}{Den zeitlich begrenzten Ausfall der Raumwellenausbreitung.}}}
\end{QQuestion}

}
\end{frame}%ENDCONTENT
