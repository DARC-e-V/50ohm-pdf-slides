
\section{Nicht-sinusförmige Signale}
\label{section:nicht_sinus_signale}
\begin{frame}%STARTCONTENT

\only<1>{
\begin{PQuestion}{AB403}{Eine periodische Schwingung, die wie das folgende Signal aussieht, besteht~...}{aus der Grundschwingung mit ganzzahligen Vielfachen dieser Frequenz (Oberschwingungen).}
{aus der Grundschwingung und Teilen dieser Frequenz (Unterschwingungen).}
{aus der Grundschwingung ohne weitere Frequenzen.}
{aus der Grundschwingung mit zufälligen Frequenzschwankungen.}
{\DARCimage{1.0\linewidth}{595include}}\end{PQuestion}

}
\only<2>{
\begin{PQuestion}{AB403}{Eine periodische Schwingung, die wie das folgende Signal aussieht, besteht~...}{\textbf{\textcolor{DARCgreen}{aus der Grundschwingung mit ganzzahligen Vielfachen dieser Frequenz (Oberschwingungen).}}}
{aus der Grundschwingung und Teilen dieser Frequenz (Unterschwingungen).}
{aus der Grundschwingung ohne weitere Frequenzen.}
{aus der Grundschwingung mit zufälligen Frequenzschwankungen.}
{\DARCimage{1.0\linewidth}{595include}}\end{PQuestion}

}
\end{frame}

\begin{frame}
\only<1>{
\begin{QQuestion}{AB401}{Was sind Harmonische?}{Harmonische sind ausschließlich die ungeradzahligen (1, 3, 5,~...) Vielfachen einer Frequenz.}
{Harmonische sind die ganzzahligen (1, 2, 3,~...) Teile einer Frequenz.}
{Harmonische sind die ganzzahligen (1, 2, 3,~...) Vielfachen einer Frequenz.}
{Harmonische sind ausschließlich die geradzahligen (2, 4, 6,~...) Teile einer Frequenz.}
\end{QQuestion}

}
\only<2>{
\begin{QQuestion}{AB401}{Was sind Harmonische?}{Harmonische sind ausschließlich die ungeradzahligen (1, 3, 5,~...) Vielfachen einer Frequenz.}
{Harmonische sind die ganzzahligen (1, 2, 3,~...) Teile einer Frequenz.}
{\textbf{\textcolor{DARCgreen}{Harmonische sind die ganzzahligen (1, 2, 3,~...) Vielfachen einer Frequenz.}}}
{Harmonische sind ausschließlich die geradzahligen (2, 4, 6,~...) Teile einer Frequenz.}
\end{QQuestion}

}
\end{frame}

\begin{frame}
\only<1>{
\begin{QQuestion}{AB402}{Die dritte Oberwelle entspricht~...}{der dritten Harmonischen.}
{der vierten Harmonischen.}
{der zweiten Harmonischen.}
{der zweiten ungeradzahligen Harmonischen.}
\end{QQuestion}

}
\only<2>{
\begin{QQuestion}{AB402}{Die dritte Oberwelle entspricht~...}{der dritten Harmonischen.}
{\textbf{\textcolor{DARCgreen}{der vierten Harmonischen.}}}
{der zweiten Harmonischen.}
{der zweiten ungeradzahligen Harmonischen.}
\end{QQuestion}

}
\end{frame}

\begin{frame}
\only<1>{
\begin{QQuestion}{AI615}{Mit welchem Messgerät kann man das Vorhandensein von Harmonischen nachweisen?}{Spektrumanalysator}
{Stehwellenmessgerät}
{Vektorieller Netzwerkanalysator (VNA)}
{Frequenzzähler}
\end{QQuestion}

}
\only<2>{
\begin{QQuestion}{AI615}{Mit welchem Messgerät kann man das Vorhandensein von Harmonischen nachweisen?}{\textbf{\textcolor{DARCgreen}{Spektrumanalysator}}}
{Stehwellenmessgerät}
{Vektorieller Netzwerkanalysator (VNA)}
{Frequenzzähler}
\end{QQuestion}

}
\end{frame}

\begin{frame}
\only<1>{
\begin{QQuestion}{AI614}{Mit welchem der folgenden Messinstrumente können die Amplituden der Harmonischen eines Signals gemessen werden? Sie können gemessen werden mit einem~...}{Breitbandpegelmesser.}
{Frequenzzähler.}
{Spektrumanalysator.}
{Multimeter.}
\end{QQuestion}

}
\only<2>{
\begin{QQuestion}{AI614}{Mit welchem der folgenden Messinstrumente können die Amplituden der Harmonischen eines Signals gemessen werden? Sie können gemessen werden mit einem~...}{Breitbandpegelmesser.}
{Frequenzzähler.}
{\textbf{\textcolor{DARCgreen}{Spektrumanalysator.}}}
{Multimeter.}
\end{QQuestion}

}
\end{frame}

\begin{frame}
\only<1>{
\begin{QQuestion}{AJ201}{Die zweite Harmonische der Frequenz \qty{3,730}{\MHz} befindet sich auf~...}{\qty{11,190}{\MHz}.}
{\qty{1,865}{\MHz}.}
{\qty{7,460}{\MHz}.}
{\qty{5,730}{\MHz}.}
\end{QQuestion}

}
\only<2>{
\begin{QQuestion}{AJ201}{Die zweite Harmonische der Frequenz \qty{3,730}{\MHz} befindet sich auf~...}{\qty{11,190}{\MHz}.}
{\qty{1,865}{\MHz}.}
{\textbf{\textcolor{DARCgreen}{\qty{7,460}{\MHz}.}}}
{\qty{5,730}{\MHz}.}
\end{QQuestion}

}
\end{frame}

\begin{frame}
\frametitle{Lösungsweg}
\begin{itemize}
  \item gegeben: $f = 3,730MHz$
  \item gesucht: $f$ der 2. Harmonischen
  \end{itemize}
    \pause
    $2 \cdot f = 2 \cdot 3,730MHz = 7,460MHz$



\end{frame}

\begin{frame}
\only<1>{
\begin{QQuestion}{AJ205}{Die zweite ungeradzahlige Harmonische der Frequenz \qty{144,690}{\MHz} ist~...}{\qty{723,450}{\MHz}.}
{\qty{289,380}{\MHz}.}
{\qty{145,000}{\MHz}.}
{\qty{434,070}{\MHz}.}
\end{QQuestion}

}
\only<2>{
\begin{QQuestion}{AJ205}{Die zweite ungeradzahlige Harmonische der Frequenz \qty{144,690}{\MHz} ist~...}{\qty{723,450}{\MHz}.}
{\qty{289,380}{\MHz}.}
{\qty{145,000}{\MHz}.}
{\textbf{\textcolor{DARCgreen}{\qty{434,070}{\MHz}.}}}
\end{QQuestion}

}
\end{frame}

\begin{frame}
\frametitle{Lösungsweg}
\begin{itemize}
  \item gegeben: $f = 144,690MHz$
  \item gesucht: $f$ als 2. ungeradzahlige Harmonische
  \end{itemize}
    \pause
    \begin{enumerate}
  \item[2] ungeradzahlige Harmonische = 3. Harmonische
  \end{enumerate}
$3 \cdot f = 3 \cdot 144,690MHz = 434,070MHz$



\end{frame}

\begin{frame}
\only<1>{
\begin{QQuestion}{AJ202}{Auf welche Frequenz müsste ein Empfänger eingestellt werden, um die dritte Harmonische einer nahen \qty{7,050}{\MHz}-Aussendung erkennen zu können?}{\qty{21,150}{\MHz}}
{\qty{14,100}{\MHz}}
{\qty{35,250}{\MHz}}
{\qty{28,200}{\MHz}}
\end{QQuestion}

}
\only<2>{
\begin{QQuestion}{AJ202}{Auf welche Frequenz müsste ein Empfänger eingestellt werden, um die dritte Harmonische einer nahen \qty{7,050}{\MHz}-Aussendung erkennen zu können?}{\textbf{\textcolor{DARCgreen}{\qty{21,150}{\MHz}}}}
{\qty{14,100}{\MHz}}
{\qty{35,250}{\MHz}}
{\qty{28,200}{\MHz}}
\end{QQuestion}

}
\end{frame}

\begin{frame}
\frametitle{Lösungsweg}
\begin{itemize}
  \item gegeben: $f = 7,050MHz$
  \item gesucht: $f$ als 3. Harmonische
  \end{itemize}
    \pause
    $3 \cdot f = 3 \cdot 7,050MHz = 21,150MHz$



\end{frame}

\begin{frame}
\only<1>{
\begin{QQuestion}{AJ206}{Auf welchen Frequenzen kann ein \qty{144,300}{\MHz} SSB-Sendesignal Störungen verursachen?}{\qty{432,900}{\MHz} und \qty{1298,700}{\MHz}}
{\qty{433,900}{\MHz} und \qty{1296,700}{\MHz}}
{\qty{438,900}{\MHz} und \qty{1290,700}{\MHz}}
{\qty{434,900}{\MHz} und \qty{1298,700}{\MHz}}
\end{QQuestion}

}
\only<2>{
\begin{QQuestion}{AJ206}{Auf welchen Frequenzen kann ein \qty{144,300}{\MHz} SSB-Sendesignal Störungen verursachen?}{\textbf{\textcolor{DARCgreen}{\qty{432,900}{\MHz} und \qty{1298,700}{\MHz}}}}
{\qty{433,900}{\MHz} und \qty{1296,700}{\MHz}}
{\qty{438,900}{\MHz} und \qty{1290,700}{\MHz}}
{\qty{434,900}{\MHz} und \qty{1298,700}{\MHz}}
\end{QQuestion}

}
\end{frame}

\begin{frame}
\frametitle{Lösungsweg}
\begin{itemize}
  \item gegeben: $f = 144,300MHz$
  \item gesucht: mehrere Harmonische
  \end{itemize}
    \pause
    $$\begin{align}\notag 2 \cdot 144,300MHz &= 288,600MHz\\ \notag 3 \cdot 144,300MHz &= \bold{432,900MHz}\\ \notag &\vdots\\ \notag 9 \cdot 144,300MHz &= \bold{1298,700MHz}\end{align}$$



\end{frame}%ENDCONTENT
